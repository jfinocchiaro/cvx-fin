\documentclass[12pt]{article}
\usepackage[utf8]{inputenc}
\usepackage{mathtools, amsmath, amssymb, graphicx, verbatim, amsthm}
%\usepackage[thmmarks, thref, amsthm]{ntheorem}
\usepackage{color}
\usepackage{wrapfig}
\usepackage{subcaption}
\usepackage[colorinlistoftodos,textsize=tiny]{todonotes} % need xargs for below
%\usepackage{accents}
\usepackage{bbm}

\newcommand{\Comments}{0}
\newcommand{\mynote}[2]{\ifnum\Comments=1\textcolor{#1}{#2}\fi}
\newcommand{\mytodo}[2]{\ifnum\Comments=1%
  \todo[linecolor=#1!80!black,backgroundcolor=#1,bordercolor=#1!80!black]{#2}\fi}
\newcommand{\raf}[1]{\mynote{green}{[RF: #1]}}
\newcommand{\raft}[1]{\mytodo{green!20!white}{RF: #1}}
\newcommand{\jessie}[1]{\mynote{purple}{[JF: #1]}}
\newcommand{\jessiet}[1]{\mytodo{purple!20!white}{JF: #1}}
\ifnum\Comments=1               % fix margins for todonotes
  \setlength{\marginparwidth}{1in} \fi


\newcommand{\reals}{\mathbb{R}}
\newcommand{\posreals}{\reals_{>0}}%{\reals_{++}}
\newcommand{\dz}{\frac{d}{dz}}
\newcommand{\dx}{\frac{d}{dx}}
\newcommand{\dr}{\frac{d}{dr}}
\newcommand{\du}{\frac{d}{du}}

\newcommand{\conv}{\mathrm{conv}}

%m upper and lower bounds
\newcommand{\mup}{\overline{m}}
\newcommand{\mlow}{\underline{m}}


% alphabetical order, by convention
\newcommand{\E}{\mathbb{E}}
\newcommand{\F}{\mathcal{F}}
\newcommand{\I}{\mathcal{I}}
\renewcommand{\P}{\mathcal{P}}
\newcommand{\R}{\mathcal{R}}
\newcommand{\Y}{\mathcal{Y}}
\renewcommand{\P}{\mathcal{P}}

\newcommand{\inter}[1]{\mathring{#1}}%\mathrm{int}(#1)}
%\newcommand{\expectedv}[3]{\overline{#1}(#2,#3)}
\newcommand{\expectedv}[3]{\E_{Y\sim{#3}} {#1}(#2,Y)}

\DeclareMathOperator*{\argmax}{arg\,max}
\DeclareMathOperator*{\argmin}{arg\,min}
\DeclareMathOperator*{\arginf}{arg\,inf}
\DeclareMathOperator*{\sgn}{sgn}

\newtheorem{theorem}{Theorem}
\newtheorem{lemma}{Lemma}
\newtheorem{proposition}{Proposition}
\newtheorem{claim}{Claim}
\newtheorem{corollary}{Corollary}

\theoremstyle{definition}
\newtheorem{definition}{Definition}
\newtheorem{condition}{Condition}
\newtheorem{case}{Case}
%\theorempostwork{\setcounter{case}{0}}

%%% Seriously COLT, seriously??
% \let\oldciteyear\citeyear
% \renewcommand{\citeyear}[1]{(\oldciteyear{#1})}


\title{1-d Convex Elicitation Notes}
\date{\today}

\begin{document}

\maketitle

\section{Directly elicitable 1-d properties with convex losses}

Suppose we have a loss $\ell(r,y)$ which, for each $y$, is convex in $r$.
This induces a loss $\ell(r;p) = \E_{y\sim p} \ell(r,y)$ which is also convex in $r$ for each fixed $p$.

Define $\Gamma^p = \argmin_r \ell(r;p)$ to be the set of optimal reports for $p$, and note that it is a \textbf{set}.\footnote{I'm using slightly nonstandard notation here because I'm worried about multivalued properties.}
Define $\Gamma^r = \{p : r \in \Gamma^p\}$ to be the level set of $r$ as usual.

\begin{lemma}[Monotonicity] \label{lemma:monot}
  Consider a convex one-dimensional loss with $r \in \argmin_{z} \ell(z;p)$ and $r' \not\in \argmin_z \ell(z;p)$.
  If $r < r' \leq r''$ then
    \[ \ell(r';p) \leq \ell(r'';p) . \]
  The same holds if $r > r' \geq r''$.
\end{lemma}
\begin{proof}
  TODO
\end{proof}

I want to use this to claim that the set of distributions $P$ that $\Gamma$ maps to a convex set $R$ is convex.
In other words, in 1-d, the interval of reports $[a,b]$ is mapped to by a convex set of $p$.

\begin{lemma}[Order-preserving]
  TODO
\end{lemma}
\begin{proof}
%  Let $R \subseteq \text{image}(\Gamma)$ be a convex set, i.e. an interval of the real line.
%  We want to show that the following set of probability distributions is convex:
%    \[ \Gamma^{-1}(R) = \{p: \Gamma(p) \in R \} . \]
%  Let $r,r' \in R$; let $p \in \Gamma^{-1}(r)$ and $p' \in \Gamma^{-1}(r')$.
%  We have to show that $\Gamma(\alpha p + (1-\alpha)p') \in R$.
%
%  \begin{align*}
%    \Gamma(\alpha p + (1-\alpha)p')
%      &= \argmin_z \ell(z; \alpha p + (1-\alpha)p')  \\
%      &= \argmin_z \alpha \ell(z; p) + (1-\alpha) \ell(z; p') .
%  \end{align*}
%  Now use monotonicity: assume without loss of generality that $z < r$.
%  Then
\end{proof}


\end{document}
%%% Local Variables:
%%% mode: latex
%%% TeX-master: t
%%% End:
