\documentclass[12pt]{article}
\usepackage[utf8]{inputenc}
\usepackage{mathtools, amsmath, amssymb, graphicx, verbatim, amsthm}
%\usepackage[thmmarks, thref, amsthm]{ntheorem}
\usepackage{color}
\usepackage{wrapfig}
\usepackage{subcaption}
\usepackage[colorinlistoftodos,textsize=tiny]{todonotes} % need xargs for below
%\usepackage{accents}
\usepackage{bbm}

\newcommand{\Comments}{0}
\newcommand{\mynote}[2]{\ifnum\Comments=1\textcolor{#1}{#2}\fi}
\newcommand{\mytodo}[2]{\ifnum\Comments=1%
  \todo[linecolor=#1!80!black,backgroundcolor=#1,bordercolor=#1!80!black]{#2}\fi}
\newcommand{\raf}[1]{\mynote{green}{[RF: #1]}}
\newcommand{\raft}[1]{\mytodo{green!20!white}{RF: #1}}
\newcommand{\jessie}[1]{\mynote{purple}{[JF: #1]}}
\newcommand{\jessiet}[1]{\mytodo{purple!20!white}{JF: #1}}
\ifnum\Comments=1               % fix margins for todonotes
  \setlength{\marginparwidth}{1in} \fi


\newcommand{\reals}{\mathbb{R}}
\newcommand{\posreals}{\reals_{>0}}%{\reals_{++}}
\newcommand{\dz}{\frac{d}{dz}}
\newcommand{\dx}{\frac{d}{dx}}
\newcommand{\dr}{\frac{d}{dr}}
\newcommand{\du}{\frac{d}{du}}

\newcommand{\conv}{\mathrm{conv}}

%m upper and lower bounds
\newcommand{\mup}{\overline{m}}
\newcommand{\mlow}{\underline{m}}


% alphabetical order, by convention
\newcommand{\E}{\mathbb{E}}
\newcommand{\F}{\mathcal{F}}
\newcommand{\I}{\mathcal{I}}
\renewcommand{\P}{\mathcal{P}}
\newcommand{\R}{\mathcal{R}}
\newcommand{\Y}{\mathcal{Y}}
\renewcommand{\P}{\mathcal{P}}

\newcommand{\inter}[1]{\mathring{#1}}%\mathrm{int}(#1)}
%\newcommand{\expectedv}[3]{\overline{#1}(#2,#3)}
\newcommand{\expectedv}[3]{\E_{Y\sim{#3}} {#1}(#2,Y)}

\DeclareMathOperator*{\argmax}{arg\,max}
\DeclareMathOperator*{\argmin}{arg\,min}
\DeclareMathOperator*{\arginf}{arg\,inf}
\DeclareMathOperator*{\sgn}{sgn}

\newtheorem{theorem}{Theorem}
\newtheorem{lemma}{Lemma}
\newtheorem{proposition}{Proposition}
\newtheorem{claim}{Claim}
\newtheorem{corollary}{Corollary}

\theoremstyle{definition}
\newtheorem{definition}{Definition}
\newtheorem{condition}{Condition}
\newtheorem{case}{Case}
%\theorempostwork{\setcounter{case}{0}}

%%% Seriously COLT, seriously??
% \let\oldciteyear\citeyear
% \renewcommand{\citeyear}[1]{(\oldciteyear{#1})}


\title{1-d Convex Elicitation Notes}
\author{Bo?}
\date{\today}

\begin{document}

\maketitle

\section{Directly elicitable 1-d properties with convex losses}

Suppose we have a loss $\ell(r,y)$ which, for each $y$, is convex in $r$.
This induces a loss $\ell(r;p) = \E_{y\sim p} \ell(r,y)$ which is also convex in $r$ for each fixed $p$.
It induces a property where $\Gamma: \Delta_{\Y} \to 2^{\R}$ is
  \[ \Gamma(p) = \argmin_r \ell(r;p) \]
and write $\Gamma_r = \{p : r \in \Gamma(p) \}$.

Let $A_{\Gamma} = \{\Gamma_r : r \in \R\}$ be the collection of level sets and $B_{\Gamma} = \{\Gamma(p) : p \in \Delta_{\Y}\}$ be the collection of report sets. (For a single-valued property, $B_{\Gamma} = \{\{r\} : r \in \R\}$.

%\paragraph{(Non)-example.}
%Consider the mode on four outcomes $\{1,2,3,4\}$.
%It can in a sense be $1$-elicited, by letting $\R = [1,4]$ and loss function $\ell(r,y) = \begin{cases} 0 & y = \text{round}(r)  \\ 1  & \text{otherwise} \end{cases}$.
%
\begin{lemma}[Convex mapping, $1$d] \label{lemma:convex-map}
  Suppose $\Gamma$ is elicited by a one-dimensional convex loss function.
  Let $A \subseteq A_{\Gamma}$ be a collection of level sets that glue together into a convex set, i.e. $P = \Cup_{a \in A} a$ is convex.

  Then $\Gamma(P) = \cup_{p \in P} \Gamma(p)$ is convex.
\end{lemma}

\begin{lemma}[Monotonicity] \label{lemma:monot}
  Consider a convex one-dimensional loss with $r \in \argmin_{z} \ell(z;p)$ and $r' \not\in \argmin_z \ell(z;p)$.
  If $r < r' \leq r''$ then
    \[ \ell(r';p) \leq \ell(r'';p) . \]
  The same holds if $r > r' \geq r''$.
\end{lemma}
\begin{proof}
  Since $\ell(r,\cdot)$ is a convex function in $r$, we know by Jensen's inequality that $\ell(\lambda r + (1-\lambda)r''; p) \leq \lambda \ell(r; p) + (1-\lambda) \ell(r''; p)$ for $\lambda \in [0,1]$.
  As we know $r < r' \leq r''$, we can write $r'$ as a convex combination of $r$ and $r''$: that is, $r' = \lambda r + (1-\lambda) r''$.
  By substitution, we then observe $\ell(r'; p) \leq \lambda \ell(r; p) + (1-\lambda) \ell(r''; p) < \lambda \ell(r'; p) + (1-\lambda) \ell(r''; p)$ for $\lambda \in (0,1]$, as $\ell(r;p) < \ell(r';p)$.
  With some simple manipulation, we then observe $\ell(r';p) \leq \ell(r'';p)$
  
\end{proof}

I want to use this to claim that the set of distributions $P$ that $\Gamma$ maps to a convex set $R$ is convex.
In other words, in 1-d, the interval of reports $[a,b]$ is mapped to by a convex set of $p$.

\begin{lemma}[Order-preserving]
  TODO
\end{lemma}
\begin{proof}
%  Let $R \subseteq \text{image}(\Gamma)$ be a convex set, i.e. an interval of the real line.
%  We want to show that the following set of probability distributions is convex:
%    \[ \Gamma^{-1}(R) = \{p: \Gamma(p) \in R \} . \]
%  Let $r,r' \in R$; let $p \in \Gamma^{-1}(r)$ and $p' \in \Gamma^{-1}(r')$.
%  We have to show that $\Gamma(\alpha p + (1-\alpha)p') \in R$.
%
%  \begin{align*}
%    \Gamma(\alpha p + (1-\alpha)p')
%      &= \argmin_z \ell(z; \alpha p + (1-\alpha)p')  \\
%      &= \argmin_z \alpha \ell(z; p) + (1-\alpha) \ell(z; p') .
%  \end{align*}
%  Now use monotonicity: assume without loss of generality that $z < r$.
%  Then
\end{proof}


\end{document}
%%% Local Variables:
%%% mode: latex
%%% TeX-master: t
%%% End:
