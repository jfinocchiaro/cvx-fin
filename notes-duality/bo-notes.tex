\documentclass[12pt]{article}
\usepackage{amsmath,amsthm,amsfonts,amssymb}

\newtheorem{theorem}{Theorem}
\newtheorem{claim}{Claim}
\theoremstyle{definition}
\newtheorem{definition}{Definition}

\DeclareMathOperator*{\argmin}{\textrm{argmin}}
\DeclareMathOperator*{\argmax}{\textrm{argmax}}
\DeclareMathOperator*{\E}{\mathbb{E}}

\newcommand{\reals}{\mathbb{R}}

\begin{document}

%\begin{definition}
%  Given a polytope $\{p \in \Delta_n : B p \geq \vec{0}\}$ where $B \in \reals^{k \times n}$, a \emph{$d$-representation} is a collection of $k' \geq k$ unit vectors in $\reals^d$ satisfying the following.
%  Forming the matrix $V \in \reals^{k' \times d}$ whose rows are the unit vectors, we must have for each $y$ that the polytope $\{ w : Vw \leq B_y \}$ is nonempty (where $B_y$ is the $y$th column).
%\end{definition}
%[Comment: we can show $k' \geq k$ is necessary, conjecture $k'=k$ is sufficient.
%
%%\begin{definition}
%%  Say a $d$-representation $V$ of a polytope $\{p : Bp \geq \vec{0}\}$ is \emph{accurate} if for all $p$ in the polytope, $0$ is in the $p$-Minkowski combination:
%%  \[ 0 \in \sum_y p_y \{w : Vw \leq B_y \} . \]
%%\end{definition}
%
%\begin{theorem}
%%  If an elicitable finite property is $d$-embeddable, then each of its level sets (1) has a $d$-representation that (2) is accurate.
%  If an elicitable finite property is $d$-embeddable, then each of its level sets has a $d$-representation.
%\end{theorem}
%\begin{proof}
%  Let $\ell_y: \reals^d \to \reals$ be the $y$th loss function.
%  Write $\ell(r;p) = \sum_y p_y \ell_y(r)$ for the expected loss under report $r$.
%  In particular, take $r$ to be the embedding point for a given level set.
%  In this case, $r$ is optimal for belief $p$ if and only if $0 \in \partial \ell(r;p)$ where the subgradients are taken with respect to $r$ only.
%  Therefore, the level set is exactly those $p$ such that $0 \in \partial \ell(r;p)$.
%  We now construct these constraints from the loss functions.
%
%  Because $\ell_y$ is a polyhedral convex function, its subgradient set $\partial \ell_y(r)$ is a polytope and can be described as follows.
%  For any vector $v \in \reals^d$, the \emph{face exposed by $v$} is the set $F_{vy} = \argmax_{w \in \partial \ell_y(r)} \langle v, w \rangle$ with corresponding maximum value $b_{vy} = \langle v, w \rangle$ where $w \in F_{vy}$.
%  Observe that for the polytope $\alpha \partial \ell_y(r)$ for positive $\alpha$, the vector $v$ exposes the face $\alpha F_{vy}$ with maximum value $\alpha b_{vy}$.
%
%  Now, the subgradient set of a weighted sum is the weighted Minkowski sum of the subgradient sets:
%    \[ \partial \ell(r;p) = \sum_y p_y \partial \ell_y(r) \]
%  So any element $w \in \partial \ell(r;p)$ can be written $w = \sum_y p_y w_y$ for $w_y \in \partial \ell_y(r)$.
%  We know $\langle v , w_y \rangle \leq b_{vy}$, so $\langle v, w \rangle \leq \sum_y p_y b_{vy}$.
%  It follows that $\partial \ell(r;p)$ is the set of points $w$ satisfying that for all $v$, $\langle v,w \rangle \leq \sum_y p_y b_{vy}$.
%  In particular therefore, $0 \in \partial \ell(r;p)$ if and only if $p$ is in the level set if and only if $0 \leq \sum_y p_y b_{vy}$ for all $v$.
%  It is a polytope, so we can collect a finite set of $k$ vectors $v_1,\dots,v_k$ and form a matrix $B$ where $B_{iy} = b_{v_iy}$, and the level set is $\{p : B p \geq 0 \}$.
%  Collect $v_1,\dots,v_k$ as the rows of a matrix $V$.
%  For each $y$, the polytope $\partial \ell_y(r)$ satisfies the constraints $\langle v_i,w\rangle \leq B_{iw}$, i.e. is contained in $P_y = \{ w : Vw \leq B_y\}$.
%  So that polytope is nonempty.
%  So $V$ is a $d$-representation of the level set.
%%  Furthermore, if $0 \in \partial \ell(r;p) = \sum_y p_y \partial \ell_y(r)$, then $0 \in \sum_y p_y P_y$, so $V$ is accurate.
%\end{proof}
%
%We use this to prove lower bounds as follows:
%\begin{enumerate}
%  \item We're given a finite property.
%  \item We pick any level set and write it as $\{ p : Bp \geq \vec{0}\}$ for a matrix $B \in \reals^{k \times n}$.
%  \item We show there is no set of $k$ vectors $v_1,\dots,v_k$ that can form a $d$-representation.
%  \item In particular, we show that however you pick $v_1,\dots,v_k$, for one of the outcomes $y$, the polytope $\{w : Vw \leq B_y \}$ is empty (infeasible).
%%  \item Or, if it is possible that $d$-representations exist, we show that none of them can possibly be accuate.
%\end{enumerate}
%
%
%\break

\section{2019-01-18}

\begin{definition}
  Call a matrix $V \in \reals^{k \times d}$ and a column $b \in \reals^k$ a \emph{linear system}; say it is \emph{feasible} if the polyhedron $\{w : Vw \leq b\}$ is nonempty and say it is \emph{spanning} if every constraint is tight, i.e. for all $j=1,\dots,k$ there exists $w$ with $Vw \leq b$ and $\langle V_j, w \rangle = b_j$.
\end{definition}

\begin{definition}
  Given a polytope $\{p : Bp \geq 0\}$ with $B \in \reals^{k \times n}$, a \emph{$d$-representation} is a collection of $k$ unit vectors in $d$ dimensions, $V \in \reals^{k \times d}$, satisfying that the linear systems $(V, B_y)_{y=1}^n$ are all (1) feasible and furthermore (2) spanning.
\end{definition}

\begin{theorem}
  If an elicitable finite property is $d$-embeddable, then each full-dimensional level set has a $d$-representation.
\end{theorem}
\begin{proof}[Proof/sketch]
  Suppose the level set is $\{p : Bp \geq 0\}$ with $B \in \reals^{k \times n}$.
  
  If $d$-embeddable then for the associated embedding point $r$, there is a set of $d$ polytopes $\{\partial \ell_y(r)\}_{y=1}^d$ such that $p$ is in the level set if and only if
  \[ 0 \in \partial \ell(r;p) = \sum_y p_y \partial \ell_y(r) . \]
  Let
    \[ H_y(v) = \max_{w \in \partial \ell_y(r)} \langle v , w \rangle . \]
  So $w \in \partial \ell(r;p)$ if and only if, for all $v$,
    \[ \langle v, w\rangle \leq \sum_y p_y H_y(v) . \]
  In particular, $p$ is in the level set if and only if, for all $v$,
    \[ 0 \leq \sum_y p_y H_y(v) . \]
  Which we know equals the set $p$ such that for $j=1,\dots,k$,
    \[ 0 \leq \sum_y p_y B_{jy} . \]
  These two sets are equal.
  If it is full-dimensional, then there exists a minimal set of $k$ unit vectors $v_1,\dots,v_k$ such that $H_y(v_j) = B_{jy}$.
  Collect them as rows of $V \in \reals^{k \times d}$.
  We have by the above that each subgradient set $T_r = \partial \ell_y(r)$ is contained in $\{w: Vw \leq B_y\}$.
  Since $T_r$ is contained in it, the linear system must be feasible.
  We also have that each inequality is achieved for some $w \in T_r$, i.e. for some $w \in T_r$ (and hence in the linear system's polyhedron), $V_jw = B_{jy}$.
  So the system is spanning as well.
\end{proof}

This necessary condition can guide upper bounds as well as lower bounds, because it says we can search over collections of $k$ unit vectors where the level set is the intersection of $k$ halfspaces.
Once we pick a candidate set $V$, we immediately obtain the linear systems $(V,B_y)_{y=1}^n$ and can automatically check that they are feasible and spanning.
If they are spanning, we know the loss function's subgradient polyhedra must be subsets of these polyhedra and also spanning, so the search is quite constrained.

\end{document}
