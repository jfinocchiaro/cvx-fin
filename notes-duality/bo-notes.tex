\documentclass[12pt]{article}
\usepackage{amsmath,amsthm,amsfonts,amssymb}

\DeclareMathOperator*{\argmin}{\textrm{argmin}}
\DeclareMathOperator*{\argmax}{\textrm{argmax}}
\DeclareMathOperator*{\E}{\mathbb{E}}

\newcommand{\reals}{\mathbb{R}}

\begin{document}

We have an elicitable finite property on $n$ outcomes.
Can there be a $d$-embedding?

Imagine so and let $\ell_y: \reals^d \to \reals$ be the $y$th loss function.
Write $\ell_y(r;p) = \sum_y p_y \ell_y(r)$ for the expected loss under report $r$.

A point $r$ is optimal for $p$ if and only if $0 \in \partial \ell_y(r;p)$ where the subgradients are taken with respect to $r$, fixing $p$.
The subgradient set of a weighted sum is the weighted Minkowski sum of the subgradient sets:
  \[ \partial \ell_y(r;p) = \sum_y p_y \partial \ell_y(r) \]
In other words, $0 \in \partial \ell_y(r;p) \iff \exists \{w_y \in \partial \ell_y(r) \} \text{~s.t.~} \sum_y p_y w_y = 0$.
Interestingly, for this $r$, for all full-support $p$ the Minkowski combination has the same combinatorial shape.

Now, $\partial \ell_y(r;p)$ is a polytope and can be described via linear inequalities as follows.
Take any unit vector $v \in \reals^d$; the \emph{faces exposed} by $v$ for each $y$ are the sets $F_{vy} = \argmax_{w \in \partial \ell_y(r)} \langle v, w \rangle$ and they have corresponding maximum values $f_{vy} = \langle v,w \rangle$ for any $w \in F_{vy}$.

So the polytope $\partial \ell_y(r;p)$ is the set of points $w$ satisfying that for all $v$, $\langle v, w \rangle \leq \sum_y p_y f_{vy}$.
This can be reduced to a finite number of inequalities because each unique inequality is given by a selection of faces of the polytopes $\partial \ell_y(r)$ for each $y$.
Further, it can actually be reduced to vectors that uniquely give an inequality (if multiple $v$ give the same inequality, then all of them can be eliminated) because these describe the facets of the polytope rather than lower-dimensional faces.

\paragraph{Summary.}
If the property embeds in $d$ dimensions, then for each $r$ we have a polytope described by $\{f_{vy} : v,y\}$ for a finite set of $v$.
We have that $r$ is optimal for $p$ if and only if $\vec{0}$ is in this polytope, in other words, if for all $v$, $0 \leq \sum_y p_y f_{vy}$.
Let $F_r$ be the matrix where each row is the tuple of $f_{v1}, \dots, f_{vn}$, but for convenience negate all entries; then $r$ is optimal for $p$ if and only if $F_r p \leq \vec{0}$.
In particular, the set of $p$ for which $r$ is optimal is exactly this polytope.

Okay, meanwhile, for any level set of the property $S$, it is a cell in the power diagram, i.e. a polytope described by some set of linear inequalities $Bp \leq \vec{0}$ for $B \in \reals^n \times \reals^k$.

\paragraph{Putting this to use.}
If $r$ is an embedding point for a level set, then $r$ is optimal if and only if $p$ is in the level set, so the polytopes $F_r p \leq \vec{0}$ and $Bp \leq \vec{0}$ must be identical.


\end{document}
