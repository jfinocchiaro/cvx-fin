\documentclass[11pt]{colt2019}
\usepackage[utf8]{inputenc}
\usepackage{mathtools, amsmath, amssymb, graphicx, verbatim}
%\usepackage[thmmarks, thref, amsthm]{ntheorem}
\usepackage{color}
\usepackage{wrapfig}
\usepackage{subcaption}
\usepackage[colorinlistoftodos,textsize=tiny]{todonotes} % need xargs for below
%\usepackage{accents}
\usepackage{bbm}
\usepackage{xspace}

\newcommand{\Comments}{1}
\newcommand{\mynote}[2]{\ifnum\Comments=1\textcolor{#1}{#2}\fi}
\newcommand{\mytodo}[2]{\ifnum\Comments=1%
  \todo[linecolor=#1!80!black,backgroundcolor=#1,bordercolor=#1!80!black]{#2}\fi}
\newcommand{\raf}[1]{\mynote{green}{[RF: #1]}}
\newcommand{\raft}[1]{\mytodo{green!20!white}{RF: #1}}
\newcommand{\jessie}[1]{\mynote{purple}{[JF: #1]}}
\newcommand{\jessiet}[1]{\mytodo{purple!20!white}{JF: #1}}
\ifnum\Comments=1               % fix margins for todonotes
  \setlength{\marginparwidth}{1in}
\fi


\newcommand{\reals}{\mathbb{R}}
\newcommand{\posreals}{\reals_{>0}}%{\reals_{++}}

% alphabetical order, by convention
\newcommand{\D}{\mathcal{D}}
\newcommand{\E}{\mathbb{E}}
\newcommand{\F}{\mathcal{F}}
\newcommand{\I}{\mathcal{I}}
\renewcommand{\P}{\mathcal{P}}
\newcommand{\R}{\mathcal{R}}
\newcommand{\X}{\mathcal{X}}
\newcommand{\Y}{\mathcal{Y}}


\newcommand{\inter}[1]{\mathring{#1}}%\mathrm{int}(#1)}
%\newcommand{\expectedv}[3]{\overline{#1}(#2,#3)}
\newcommand{\expectedv}[3]{\E_{Y\sim{#3}} {#1}(#2,Y)}
\newcommand{\toto}{\rightrightarrows}
\newcommand{\trim}{\mathrm{trim}}
\newcommand{\fplc}{finite-piecewise-linear and convex\xspace} %xspace for use in text
\newcommand{\conv}{\mathrm{conv}}
\newcommand{\ones}{\mathbbm{1}}

\DeclareMathOperator*{\argmax}{arg\,max}
\DeclareMathOperator*{\argmin}{arg\,min}
\DeclareMathOperator*{\arginf}{arg\,inf}
\DeclareMathOperator*{\sgn}{sgn}

%\newtheorem{theorem}{Theorem}
%\newtheorem{lemma}{Lemma}
%\newtheorem{proposition}{Proposition}
%\newtheorem{definition}{Definition}
%\newtheorem{corollary}{Corollary}
%\newtheorem{conjecture}{Conjecture}


\title{Finite Property Convex Elicitation Paper}
\author{Jessie + Raf + Bo}

\begin{document}

\maketitle

\begin{abstract}
  Convex surrogates are sweet.
  Given a loss for a classification-like problem, there are two natural approaches to design convex surrogates.
  First, one may attempt to map each prediction to a low-dimensional vector, and try to find a convex loss in that space with the right calibration.
  Second, one may simply try to find a surrogate within the class of piecewise-linear convex, or polyhedral, losses.
  We show an equivalence between these two approaches, and \raf{more stuff}.
  We show that every loss with a finite number of predictions has a convex surrogate in the above sense using one fewer dimension that the number of outcomes, and give a full characterization of the losses needing only $d$ dimensions for such a surrogate.
  We then apply this characterization to show novel lower bounds for abstain loss, demonstrating the power of our techniques over \raf{check this} alternatives such as feasible subspace dimension.
\end{abstract}

\section{Notation and Definitions}

All of this is up for negotiation...

...except we really should say ``polyhedral'' losses, since that is (a) already standard, and (b) captures the entire phrase ``finite piecewise-linear convex''.

\section{Polyhedral Losses and Embeddings}

\raf{This is missing a bunch of stuff: equivalence of possible definitions, existence of a (weak) link function, etc}

\subsection{Dimension 1: real-valued losses}

\begin{theorem}
  Every polyhedral $L : \reals \to \reals^\Y$ embeds a finite orderable property.
\end{theorem}

\begin{theorem}\label{thm:fplc-orderable}
  A finite property is convex embeddable in $1$ dimension if and only if it is orderable.
  Moreover, this remains true when restricting to polyhedral losses.
\end{theorem}

\subsection{Higher dimensions}

\begin{theorem}
  Every polyhedral loss embeds a finite elicitable property.
\end{theorem}

\section{Embedding Dimension}

\subsection{A universal upper bound}
\raf{Note: does not require the property to be finite, or even well-behaved.  We should also note that this upper bound is not really obvious (it would be if we did not insist on convexity of the loss...)}
\begin{theorem}
  Every elicitable property is $(|\Y|-1)$-embeddable.
\end{theorem}

\subsection{Characterization}

\raf{The full characterization:}
\begin{theorem}
  A finite property $\Gamma:\R\toto\P$ is $d$-embeddable if and only if there exists an embedding $\varphi: \R \to \reals^d$ and polytopes $T(r,y) \subseteq \reals^d$ for all $r\in\R, y\in\Y$, such that the following two conditions hold:
  \begin{enumerate}
  \item For all $r\in\R, p\in\P$, we have $r\in \Gamma(p) \iff 0 \in \sum_{y\in\Y} p_y T(r,y)$, where summation is the Minkowski sum.
  \item For all $y\in\Y$ there exists some convex function $L_y$ such that for all $r\in\R$ we have $T(r,y) = \partial L_y(\varphi(r))$.
  \end{enumerate}
\end{theorem}

\subsection{Useful necessary conditions}

\begin{corollary}
  \raf{Bo's polytope stuff}
\end{corollary}

\begin{corollary}
  \raf{reduction to 1 dimension}
\end{corollary}

\section{Application: Abstain Loss}

\begin{theorem}
  Abstain with $\alpha=1/2$ and $|\Y|=5$ is not $2$-embeddable.
\end{theorem}

\begin{theorem}
  Abstain with $\alpha > 1/2$ and $|\Y|=7$ is not $2$-embeddable.
\end{theorem}

\begin{conjecture}
  Abstain with $\alpha=1/2$ is $(\log_2 |\Y|)$-embeddable, but no lower.
\end{conjecture}

\section{Conclusion and Future Work}

Let $\mathrm{elic}_{embed}$ denote the elicitation complexity with respect to embeddings, $\mathrm{elic}_{Pcvx}$ the complexity with respect to polyhedral convex losses, and $\mathrm{elic}_{cvx}$ the complexity with respect to arbitrary convex losses.

\begin{conjecture}
  $\mathrm{elic}_{embed}(\Gamma) = \mathrm{elic}_{Pcvx}(\Gamma) = \mathrm{elic}_{cvx}(\Gamma)$ for all finite, elicitable properties $\Gamma$.
\end{conjecture}

\subsection*{Acknowledgements}
We thank Peter Bartlett for several discussions early on, which led to a proof of \raf{1-d reduction} among other insights.

\end{document}
%%% Local Variables:
%%% mode: latex
%%% TeX-master: t
%%% End:
