\documentclass[anon,12pt]{colt2019}
\usepackage[utf8]{inputenc}
\usepackage{mathtools, amsmath, amssymb, graphicx, verbatim}
%\usepackage[thmmarks, thref, amsthm]{ntheorem}
\usepackage{color}
\usepackage{wrapfig}
\usepackage{subcaption}
\usepackage[colorinlistoftodos,textsize=tiny]{todonotes} % need xargs for below
%\usepackage{accents}
\usepackage{bbm}
\usepackage{xspace}

\newcommand{\Comments}{1}
\newcommand{\mynote}[2]{\ifnum\Comments=1\textcolor{#1}{#2}\fi}
\newcommand{\mytodo}[2]{\ifnum\Comments=1%
  \todo[linecolor=#1!80!black,backgroundcolor=#1,bordercolor=#1!80!black]{#2}\fi}
\newcommand{\raf}[1]{\mynote{green}{[RF: #1]}}
\newcommand{\raft}[1]{\mytodo{green!20!white}{RF: #1}}
\newcommand{\jessie}[1]{\mynote{purple}{[JF: #1]}}
\newcommand{\jessiet}[1]{\mytodo{purple!20!white}{JF: #1}}
\newcommand{\bo}[1]{\mynote{blue}{[Bo: #1]}}
\newcommand{\botodo}[1]{\mytodo{blue!20!white}{[Bo: #1]}}
\ifnum\Comments=1               % fix margins for todonotes
  \setlength{\marginparwidth}{1in}
\fi


\newcommand{\reals}{\mathbb{R}}
\newcommand{\posreals}{\reals_{>0}}%{\reals_{++}}

\newcommand{\prop}[1]{\Gamma[#1]}
\newcommand{\eliccts}{\mathrm{elic}_\mathrm{cts}}
\newcommand{\eliccvx}{\mathrm{elic}_\mathrm{cvx}}
\newcommand{\elicpoly}{\mathrm{elic}_\mathrm{pcvx}}
\newcommand{\elicembed}{\mathrm{elic}_\mathrm{embed}}

\newcommand{\cell}{\mathrm{cell}}

\newcommand{\simplex}{\Delta_\Y}

% alphabetical order, by convention
\newcommand{\D}{\mathcal{D}}
\newcommand{\E}{\mathbb{E}}
\newcommand{\F}{\mathcal{F}}
\newcommand{\I}{\mathcal{I}}
\newcommand{\R}{\mathcal{R}}
\newcommand{\X}{\mathcal{X}}
\newcommand{\Y}{\mathcal{Y}}


\newcommand{\inter}[1]{\mathring{#1}}%\mathrm{int}(#1)}
%\newcommand{\expectedv}[3]{\overline{#1}(#2,#3)}
\newcommand{\expectedv}[3]{\E_{Y\sim{#3}} {#1}(#2,Y)}
\newcommand{\toto}{\rightrightarrows}
\newcommand{\strip}{\mathrm{strip}}
\newcommand{\trim}{\mathrm{trim}}
\newcommand{\fplc}{finite-piecewise-linear and convex\xspace} %xspace for use in text
\newcommand{\conv}{\mathrm{conv}}
\newcommand{\ones}{\mathbbm{1}}

\DeclareMathOperator*{\argmax}{arg\,max}
\DeclareMathOperator*{\argmin}{arg\,min}
\DeclareMathOperator*{\arginf}{arg\,inf}
\DeclareMathOperator*{\sgn}{sgn}

%\newtheorem{theorem}{Theorem}
%\newtheorem{lemma}{Lemma}
%\newtheorem{proposition}{Proposition}
%\newtheorem{definition}{Definition}
%\newtheorem{corollary}{Corollary}
%\newtheorem{conjecture}{Conjecture}


\title{Convex Surrogates via Polyhedral Losses}
\coltauthor{%
 \Name{Jessie} \Email{email}\\
 \addr CU
 \AND
 \Name{Raf} \Email{email}\\
 \addr CU
 \AND
 \Name{Bo}  \Email{email}\\
 \addr MSFT
}

\begin{document}

\maketitle

\begin{abstract}
  Convex surrogates are sweet.
  Given a loss for a classification-like problem, there are two natural approaches to design convex surrogates.
  First, one may attempt to map each prediction to a low-dimensional vector, and try to find a convex loss in that space with the right calibration.
  Second, one may simply try to find a surrogate within the class of piecewise-linear convex, or polyhedral, losses.
  We show an equivalence between these two approaches, and \raf{more stuff}.
  We show that every loss with a finite number of predictions has a convex surrogate in the above sense using one fewer dimension that the number of outcomes, and give a full characterization of the losses needing only $d$ dimensions for such a surrogate.
  We then apply this characterization to show novel lower bounds for abstain loss, demonstrating the power of our techniques over \raf{check this} alternatives such as feasible subspace dimension.
\end{abstract}
\begin{keywords}%
  information elicitation, proper scoring rules, surrogate loss functions
\end{keywords}


\section{Setting and Background}

\subsection{Notation}
\begin{itemize}
  \item $\Y$ is the finite outcome space, and $n := |\Y|$.
  \item $\simplex\subseteq\reals^\Y$ is the set of distributions over $\Y$, thought of as vectors of probabilities.
  \item For an outcome $y$, we denote $p_y$ as the probability of outcome $y$ being observed in nature.
  \item If $\gamma = \prop{\ell}$, then $\gamma:\simplex \toto \R$ is the property elicited by the loss $\ell: \R \to \reals^\Y$.
  \item $d$ is the dimension of the embedding space
  \item $L:\reals^d \to \reals^\Y_+$ is a loss taking a report $u \in \reals^d$ and mapping the loss over each outcome in $\Y$. \raf{Changed to $\reals^\Y_+$ below.}
  \item $\Gamma: \simplex \toto \reals^d$ is the set-valued property mapping a probability distribution to a report $u \in \reals^d$.
  \item $\Gamma_u = \{ p \in \simplex : u \in \Gamma(p) \}$
  \item $\psi$ is the link function (see below)
  \item $\varphi:\R \to \reals^d$ is the embedding function
  \item $\trim(\Gamma)$ and $\strip(\Gamma)$
  \item $\eliccvx$, $\elicembed$, and $\elicpoly$
  \item $V$ and $B(P)$ as we get ready for later.
\end{itemize}

\subsection{Properties and Losses}
\begin{definition}
  Property $\Gamma:\simplex\toto\R$

  Level set $\Gamma_r$

\raf{Should bake non-degenerate into the definition: insist that $\Gamma(p) \neq \emptyset$ for all $p$.}

  Redundant, finite \raf{I think to avoid a TON of extra mentions of ``non-redundant'' we should take finite to mean finite and non-redundant}
\end{definition}

\begin{definition}
  \raf{I realized that (a) it is common to restrict losses to be non-negative, and (b) it solves some annoying issues where we have to say ``if $L$ is bounded below...'' nicely.  So unless any objections, let's say $L \geq 0$.}

  A loss $L:\R\to\reals^\Y_+$, elicits a property $\Gamma:\simplex \toto \R$ if, for all $p \in \simplex$ and $r \in \Gamma(p)$, we observe
  \begin{equation}
   = \argmin_r \E_{Y\sim p}[L(r,Y)]
  \end{equation}
  \jessie{Do we want regular elicitation vs essentially elicits...}
\end{definition}

\begin{definition}
  Polyhedral function, loss
\end{definition}

\subsection{Link Functions}

\begin{definition}
  Let properties $\gamma:\simplex\toto\R$ and $\Gamma:\simplex\toto\R'$ be given.
  A \emph{link function} is a map $\psi:\R'\to\R$.
  We say that a link $\psi$ is:
  \begin{itemize}
  \item \emph{Calibrated} if for all $p\in\simplex$, $r'\in \Gamma(p) \implies \psi(r') \in \gamma(p)$; in other words, if for all $r'\in\R'$, $\Gamma_{r'} \subseteq \gamma_{\psi(r')}$;
  \item \emph{Separated}, with respect to a loss $L$ eliciting $\Gamma$, if for all $p \in \simplex$,
  \begin{align*}
  \inf_{u \in \R'; \psi(u) \not \in \gamma(p)} \E_{Y\sim p}[L(u, Y)] > \inf_{u \in \R'}\E_{Y\sim p}[L(u, Y)]~.
  \end{align*}
  \item \emph{Exhaustive} if for all $p\in\simplex$, $\gamma(p) = \psi(\Gamma(p)) := \{\psi(r') : r'\in\Gamma(p)\}$.
  \end{itemize}
\end{definition}

It is also interesting to contrast the definitions above with the following.
\begin{definition}
  A \emph{strong link} is a set-valued function $\psi:\R'\toto\R$ such that for all $p\in\simplex$ and $r'\in\R'$, we have $r'\in\Gamma(p) \implies \gamma(p) = \psi(r')$.
\end{definition}

\raf{We should note here that a discussion like this has not yet appeared in the property elicitation literature, since most papers address direct elicitation, where no link is needed, or single-valued properties in the case of indirect elicitation.  Cite examples for indirect: Frongillo--Kash (NIPS), Fissler--Ziegel (AoS), mode paper, others?}

\raf{Discussion about these definitions using 0-1 loss, hinge, and logistic.  Namely: sign is calibrated for hinge and logistic.  If sign is interpreted as set-valued (0 maps to $\{1,-1\}$, then it is strong for logistic.  If not, sign is not exhaustive for logistic.  Sign is exhaustive for hinge.  It is also separated, but something like $\psi(u) =
  \begin{cases}
    -1 & u \leq -1\\
    1 & u > -1
  \end{cases}$
  would not be, even though it would be calibrated.}


\subsection{Elicitation Complexity}

\begin{definition}
  A loss \emph{indirectly elicits} a property $\gamma$ if it elicits a property with a calibrated link to $\gamma$.
\end{definition}

\raf{Notation up for discussion}
$\eliccvx$ and $\elicpoly$; forward reference $\elicembed$

\subsection{Important Examples}
\raf{Mode, abstain, others?  Mention what is known, especially $\elicpoly($abstain$_{1/2}) \leq \log_2 n$.}

\section{Polyhedral Losses and Embeddings}

\raf{Glue needed...}
\begin{definition}
  A property $\Gamma : \simplex \toto \reals^d$ \emph{embeds} a property $\gamma : \simplex \toto \R$ if there exists some injective embdedding $\varphi:\R\to\reals^d$ such that for all $p\in\simplex,r\in\R$ we have $r \in \gamma(p) \iff \varphi(r) \in \Gamma(p)$.
  Similarly, we say a loss $L:\reals^d\to\reals^\Y$ embeds $\gamma$ if $\prop{L}$ embeds $\gamma$.
\end{definition}

\begin{definition}
  We say $\gamma$ is \emph{$d$-embeddable}, or just \emph{embeddable}, if such an $L$ exists and is convex.
  The \emph{embedding dimension} of a finite property $\gamma$, written $\elicembed(\gamma)$, is the minimum $d\geq 0$ such that $\gamma$ is $d$-embeddable.
\end{definition}

\begin{definition}\label{def:trim}
  Let $\Gamma:\simplex \toto\R'$ be an elicitable property.
  We define $\trim(\Gamma)$ as the set of maximal level sets of $\Gamma$.
  Formally, $\trim(\Gamma) = \{\Gamma_u : u \in \R'\} \setminus \{ \Gamma_u : \exists w \in \R' \text{ s.t. } \Gamma_u \subsetneq \Gamma_w \}$.
\end{definition}
\raf{NOTE, maybe for later: we need to show that the union of trim is the simplex}

Take note that the unlabeled property $\trim(\Gamma)$ is nonredundant, meaning that for any $\theta \in \trim(\Gamma)$, there is no level set $\theta' \in \trim(\Gamma)$ such that $\theta \subset \theta'$.

\raf{Explain why this is a useful concept in general, and for convex elic}
Looking at the $\trim$ of a property allows us to relate the finite property of interest to the subtlely different surrogate property that we actually elicit when embedding the original property into $\reals^d$.
In the surrogate property, there is often an infinite set of ``hidden'' level sets corresponding to the reports in the convex hull of embedded points whose level sets have nonempty intersection.
Taking the $\trim$ thus allows us to remove labels so that we can relate the optimal sets of original reports and their embedded points, and then we remove the ``hidden'' level sets that are only optimal when we are indifferent between a subset of the embedded points.

\begin{lemma}\label{lem:finite-full-dim}
  Let $\gamma$ be a finite (non-redundant) property elicited by a loss $L$.
  Then the negative Bayes risk $G$ of $L$ is polyhedral, and the level sets of $\gamma$ are the projections of the facets of the epigraph of $G$ onto $\simplex$, and thus form a power diagram.
  In particular, the level sets $\gamma$ are full-dimensional in $\simplex$ (i.e.,\ of dimension $n-1$).
\end{lemma}

\begin{lemma}\label{lem:loss-restrict}
  Let $L$ elicit $\Gamma:\simplex\toto\R_1$, and let $\R_2\subseteq\R_1$ such that $\Gamma(p) \cap \R_2 \neq \emptyset$ for all $p\in\simplex$.
  Then $L|_{\R_2}$ ($L$ restricted to $\R_2$) elicits $\gamma:\simplex\toto\R_2$ defined by $\gamma(p) = \Gamma(p)\cap \R_2$.
\end{lemma}
\begin{proof}
  Let $p\in\simplex$ be fixed throughout.
  First let $r \in \gamma(p) = \Gamma(p) \cap \R_2$.
  Then $r \in \Gamma(p) = \argmin_{u\in\R_1} p\cdot L(u)$, so as $r\in\R_2$ we have in particular $r \in \argmin_{u\in\R_2} p\cdot L(u)$.
  For the other direction, suppose $r \in \argmin_{u\in\R_2} p\cdot L(u)$.
  By our assumption, we must have some $r^* \in \Gamma(p) \cap \R_2$.
  On the one hand, $r^*\in\Gamma(p) = \argmin_{u\in\R_1} p\cdot L(u)$.
  On the other, as $r^* \in \R_2$, we certainly have $r^* \in \argmin_{u\in\R_2} p\cdot L(u)$.
  But now we must have $p\cdot L(r) = p\cdot L(r^*)$, and thus $r \in \argmin_{u\in\R_1} p\cdot L(u) = \Gamma(p)$ as well.
  We now see $r \in \Gamma(p) \cap \R_2$.
\end{proof}

\begin{proposition}\label{prop:embed-trim}
  Let $\Gamma:\simplex\toto\reals^d$ be an elicitable property.
  The following are equivalent:
  \begin{enumerate}
  \item $\Gamma$ embeds a finite property $\gamma:\simplex \toto \R$.
    \jessiet{Should ``nonredundant'' be added to the moreover as well?-- No, because you could have a redundant, finite property so that these are true.}\raft{Our definition of finite implies non-redundant; the proposition would not be true for redundant $\gamma$, since $\trim(\Gamma)\neq\gamma$.}
  \item $\trim(\Gamma)$ is a finite set, and $\cup\,\trim(\Gamma) = \simplex$.
  \item There is a finite set of full-dimensional level sets $\Theta$ of $\Gamma$, and $\cup\,\Theta = \simplex$.
  \end{enumerate}
  Moreover, when any of the above hold, $\{\gamma_r : r\in\R\} = \trim(\Gamma) = \Theta$, and $\gamma$ is elicitable.
\end{proposition}
\begin{proof}
  Let $L$ elicit $\Gamma$.

  1 $\Rightarrow$ 2:
  By the embedding condition, taking $\R_1 = \reals^d$ and $\R_2 = \varphi(\R)$ satisfies the conditions of Lemma~\ref{lem:loss-restrict}: for all $p\in\simplex$, as $\gamma(p) \neq \emptyset$ by definition, we have some $r\in\gamma(p)$ and thus some $\varphi(r) \in \Gamma(p)$.
  Let $G(p) := -\min_{u\in\reals^d} p\cdot L(u)$ be the negative Bayes risk of $L$, which is convex, and $G_{\R}$ that of $L|_{\varphi(\R)}$.
  By the above, we actually have $G = G_\R$: the minimum is always achieved by some $\varphi(r) \in \varphi(\R)$.
  As $\gamma$ is finite, $G$ is polyhedral.
  Moreover, the projection of the epigraph of $G$ onto $\simplex$ forms a power diagram, with the facets projecting onto the level sets of $\gamma$ (the cells of the power diagram).
  \raf{Could use some justification for the previous two sentences.}
  As $L$ elicits $\Gamma$, for all $u\in\reals^d$, the hyperplane $p\mapsto p\cdot L(u)$ is a supporting hyperplane of the epigraph of $G$ at $(p,G(p))$ if and only if $u\in\Gamma(p)$.
  This supporting hyperplane exposes some face $F$ of the epigraph of $G$, which must be contained in some facet $F'$.
  Thus, the projection of $F$, which is $\Gamma_u$, must be contained in the projection of $F'$, which is a level set of $\gamma$.
  We conclude that $\Gamma_u \subseteq \gamma_r$ for some $r\in\R$.
  Hence, $\trim(\Gamma) = \{\gamma_r : r\in\R\}$, which is finite, and unions to $\simplex$.


  %%% The below is basically the argument from the Lemma, for this specific case.
  % We will first show $L'$ elicits $\gamma$, where $L':\R\to\reals^\Y_+$ is defined by $L'(r) = L(\varphi(r))$.
  % From the definition of embedding, for all $p\in\simplex$ we have $r \in \gamma(p) \iff \varphi(r) \in \Gamma(p) = \argmin_{u\in\reals^d} p\cdot L(u)$.
  % Thus, $r\in\gamma(p)$ implies $r \in \argmin_{r'\in\R} p\cdot L(\varphi(r')) = \argmin_{r'\in\R} p\cdot L'(r')$.
  % For the other direction, let $r \in \argmin_{r'\in\R} p\cdot L'(r')$, and let $r'\in\gamma(p)$ (where use $\gamma(p) \neq \emptyset$).
  % By the above, we have $r'\in\argmin_{r'\in\R} p\cdot L'(r')$ as well, so we must have $p\cdot L'(r) = p\cdot L'(r')$, and therefore $p\cdot L(\varphi(r)) = p\cdot L(\varphi(r'))$.
  % But we also have $\varphi(r') \in \argmin_{u\in\reals^d} p\cdot L(u)$, and therefore $\varphi(r) \in \argmin_{u\in\reals^d} p\cdot L(u) = \Gamma(p)$.
  % By the definition of embedding, we now have $r \in \gamma(p)$, as was to be shown.

  % for all the given that $\Gamma$ embeds the finite property $\gamma$, we show that $\trim(\Gamma) = \{\Gamma_u : u \in \varphi(\R) \} = \{\gamma_r : r \in \R\}$.
  % By elicitability, each level set $\Gamma_u$ corresponds to a subgradient of the expected score function, call it $G(p) := -\langle p, L(\Gamma(p))\rangle$ for any choice from $\Gamma$.
  % We claim $G$ is polyhedral. $\gamma$ is finite (and by definition non-redundant), so each level set $\gamma_r = \Gamma_{\phi(r)}$ is full-dimensional\bo{need to prove this separately in a lemma}.
  % So the corresponding subgradient is tangent to $G$ above a full-dimensional level set, and this is true for every $\gamma_r$.
  % Every $p$ is in one of these level sets and so is below some facet.
  % This implies that $G$ is polyhedral with facets exactly corresponding to $\{\gamma_r : r \in \R\} = \{\Gamma_u : u \in \varphi(\R)\}$.
  % Finally, for any $u \not\in \varphi(\R)$, we have\bo{need to cite a lemma statement, per discussion with Raf} that $\Gamma_u$ lies below some face of the epigraph of $G$, hence is contained in the set lying below some facet, which is one of the $\Gamma_{\varphi(r)}$.
  % So the maximal level sets of $\Gamma$ are exactly those lying below facets, which are exactly the level sets of $\gamma$.
  % This gives 2, and the fact that the level sets of $\gamma$ are full-dimensional and union to $\Delta_{\Y}$ gives 3.

  % Clearly, $\trim(\Gamma) \supseteq \{\Gamma_u : u \in \varphi(\R)\}$ as we are simply restricting the number of level sets we are indexing over.
%  Now, consider some $\theta \in \trim(\Gamma)$ and the report $u$ so that $\theta = \Gamma_u$.
%  If $u \in \varphi(\R)$, then clearly we are done.
%  If $u \not\in \varphi(\R)$ but there is some report $w \in \varphi(\R)$ so that $\Gamma_u = \Gamma_w$, then we are again done since $\trim(\Gamma)$ is nonredundant.
%  If there is no such $w \in \R$, then we claim $\Gamma_u \not \in \trim(\Gamma)$.
%  To see this, consider that if $\Gamma_u$ is full-dimensional, it must be a relabeling of another report by Lemma~\ref{lem:unique-opt-on-inter} in the Appendix.
%  If $\Gamma_u$ is not full-dimensional, it must be a proper subset of some full-dimensional level set by Lemma~\ref{lem:trim-subsets-not-full-dimensional}, and therefore not in $\trim(\Gamma)$.
%%\jessiet{skimmed over this, but follows from the convexity of the scoring rule.  My thought for the paper is to put that proof in the appendix,and just have a footnote or side note here with a reference to the proof in the appendix.}
%  Thus, we conclude $\trim(\Gamma) \subseteq \{\Gamma_u : u \in \varphi(\R)\}$.
%  Since the sets are equal and the latter is finite, we conclude that if $\Gamma$ embeds the finite property $\gamma$, then $\trim(\Gamma)$ is finite.

%  For $2 \implies 3$, since each level set $\theta \in \trim(\Gamma)$ is full-dimensional by Lemma~\ref{lem:trim-full-dim}, suppose for now that $\Theta := \trim(\Gamma)$.
%  %As $\Gamma$ is nondegenerate, for every $p \in \simplex$, there must be some report $u$ corresponding to the level set such that $p \in \Gamma_u$.
%  Since $\Gamma$ is nondegenerate, for all $p \in \simplex$, there is a level set $\theta \in \trim(\Gamma)$ such that $p \in \theta$, since the level sets of elicitable properties are closed and convex, and only level sets that are not full-dimensional level sets are removed in taking $\trim(\Gamma)$.
%  If, for some $p \in \simplex$, the level set $\Gamma_u$ corresponding to $p$ was removed, it would be because there is a level set corresponding to $w \in \reals^d$ such that $\Gamma_u \subset \Gamma_w$, but then $p \in \Gamma_w$.

  2 $\Rightarrow$ 3: let $\R = \{u_1,\ldots,u_k\} \subseteq\reals^d$ be a set of distinct reports such that $\trim(\Gamma) = \{\Gamma_{u_1},\ldots,\Gamma_{u_k}\}$.
  Now as $\cup\,\trim(\Gamma) = \simplex$, for any $p\in\simplex$, we have $p\in\Gamma_{u_i}$ for some $u_i\in\R$, and thus $\Gamma(p) \cap \R \neq \emptyset$.
  We now satisfy the conditions of Lemma~\ref{lem:loss-restrict} with $\R_1 = \reals^d$ and $\R_2 = \R$.
  The property $\gamma:p\mapsto\Gamma(p)\cap\R$ is non-redundant by the definition of $\trim$, finite, and elicitable.
  Now from Lemma~\ref{lem:finite-full-dim}, the level sets $\Theta = \{\gamma_r:r\in\R\}$ are full-dimensional, and union to $\simplex$.
  Statement 3 then follows from the fact that $\gamma_r = \Gamma_r$ for all $r\in\R$.

  %%% Again, this is subsumed by the Lemma
  % Let $\R = \{u_1,\ldots,u_k\}$; we show that $\gamma:\simplex\toto\R$, $\gamma(p) = \{u_i : p\in\Gamma_{u_i}\} = \Gamma(p) \cap \R$, is elicitable.
  % Define $L':\R\to\reals^\Y_+$ by $L'(u_i) = L(u_i)$.
  % Clearly, if $u_i \in \gamma(p)$, then $u_i \in \Gamma(p) = \argmin_{u\in\reals^d} p\cdot L(u)$, and thus $u_i \in \argmin_{u\in\R} p\cdot L'(u)$.
  % For the other direction, suppose $u_i \in \argmin_{u\in\R} p\cdot L'(u)$.
  % For any $u' \in \Gamma(p)$, by the definition of $\trim$, we have some $u_j\in\R$ for which $u_j\in\Gamma(p) = \argmin_{u\in\reals^d} p\cdot L(u)$ as well.
  % But now we must have $p\cdot L(u_i) = p\cdot L(u_j)$, and thus $u_i \in \Gamma(p)$ as well.
  % As $u_i\in\R$, we have $u_i \in \Gamma(p)\cap \R = \gamma(p)$, as desired.


  \raf{Wow, kudos to you two for this proof.  Much simpler than I would have guessed!}

  3 $\Rightarrow$ 1: let $\Theta = \{\theta_1,\ldots,\theta_k\}$.
  For all $i\in\{1,\ldots,k\}$ let $u_i\in\reals^d$ such that $\Gamma_{u_i} = \theta_i$.
  Now define $\gamma:\simplex\toto\{1,\ldots,k\}$ by $\gamma(p) = \{i : p\in\theta_i\}$, which is non-degenerate as $\cup\,\Theta = \simplex$.
  By construction, we have $\gamma_i = \theta_i = \Gamma_{u_i}$ for all $i$, so letting $\varphi(i) = u_i$ we satisfy the definition of embedding, namely statement 1.
\end{proof}

\begin{proposition}\label{prop:embed-link}
  Let $L$ elicit $\Gamma:\simplex\toto\reals^d$ which embeds a finite property $\gamma$.
  Then there is a calibrated link from $\Gamma$ to $\gamma$, which can be taken to be separated if $L$ is polyhedral.
\end{proposition}
\begin{proof}
  Let $\gamma:\simplex\toto\R$.
  Proposition~\ref{prop:embed-trim} gives us that $\trim(\Gamma) = \{\gamma_r : r\in\R\}$.
  We conclud that for any $u\in\reals^d$, there is some $r\in\R$ such that $\Gamma_u \subseteq \gamma_r$.
  The link $\psi : \reals^d \to \R$ which encodes these choices is calibrated.

  Now assume $L$ is polyhedral.

  \jessiet{Hole jere: $j$-faces of the function.}
  For the second part of the statement, in order for $\psi$ to be separated, for $\epsilon > 0$, for each embedded point $u \in \reals^d$ and every $v \in B(\epsilon, u)$, we must observe $\psi(v) = \psi(u)$.
  If the loss $L$ is polyhedral, we want to show that we can construct a $\psi$ so that the above holds.

  First, consider that each of the embedded points in $u \in \varphi(\R)$ must have ball $B(\epsilon, u)$ centered on $u$ so that no other $w \in \varphi(\R)$ is in $B(\epsilon, u)$.
  If this is not true, then \jessie{...?} \jessiet{I'm not sure how this messses with things, but I would think the level sets corresponding to the two points would differ on a set of zero measure, if at all, or both be in the $\arginf$of the loss...?}

  Now for any $u \in \varphi(\R)$ and $v \in B(\epsilon, u)$, if $\psi(v)$ does not map to $\psi(u)$, then there must be some other $w \in \varphi(\R)$ so that $\psi(v) = \psi(w)$.
  Since $L$ is polyhedral, we know that for $p \in \simplex$ such that $v \in \Gamma(p)$, then since $u$ and $v$ are sufficiently close, $\E_p L(w,Y) = \E_p L(v,Y) \leq \E_p L(u,Y)$.
  However, if the inequality was strict, then we contradict the polyhedral structure of $L$, since $v$ is not an embedded point.
  Therefore, $v \in \Gamma(p) \implies u \in \Gamma(p)$, and we can alter the link function so that $\psi(v) = \psi(u)$.
%If no such $\psi$ existed, then for some $p$ such that $u \in \arginf_{z \in \R} \E_p L(z,Y)$, we know there is a $\delta > 0$ such that $\E_p L(v,Y) - \E_p L(u,Y) < \delta$ by continuity of $L$.
%Additionally, if $\psi$ was not selective, then we know $\psi(v) = \psi(w)$ for some embedded point $w \neq u$.
%Since the losses of $u$ and $v$ are not bounded about away from each other and $\psi(v) = \psi(w)$, then the loss $L$ must be constant on $\conv(\{u,v,w\})$, otherwise we would either contradict elicitability of $\Gamma$ or the construction of the calibrated link $\psi$.
%Therefore, we can reassign $\psi(v) = \psi(u)$ for any $v\in B(\epsilon, u)$, and such a link additionally separated.
\end{proof}

\begin{theorem}
  Let $\gamma$ be a finite property.
  The following are equivalent.
  \begin{enumerate}
  \item $\gamma$ is elicitable.
  \item $\gamma$ is embeddable.
  \item $\gamma$ is embeddable via a polyhedral loss.
  \end{enumerate}
\end{theorem}
\begin{proof}
  We trivially have $3\Rightarrow 2$, and by Proposition~\ref{prop:embed-trim} \raf{only if it's updated like I mentioned} we also have $2\Rightarrow 1$.
  It thus suffices to show $1\Rightarrow 3$, which is a corollary of a stronger result given as Theorem \raf{fill in} in Appendix \raf{fill in}, stating that every non-redundant elicitable property, even if not finite, is embeddable in $d =n-1$ dimensions.
  \raf{NOTE: still need to check the math on the conjugate for the general case, since the expected score function is not necessarily closed for general properties.  Not high priority.}
  (Recall that $n = |\Y|$.)
  The construction uses the convex conjugate of the expected score function for $\gamma$, which is polyhedral when $\gamma$ is finite.
\end{proof}

\begin{definition}
  Let $\Gamma:\simplex\toto\R$ and $\Gamma':\simplex\toto\R'$.
  Then $\Gamma'$ \emph{refines} $\Gamma$ if for all $r'\in\R'$ we have $\Gamma'_{r'} \subseteq \Gamma_r$ for some $r\in\R$.
  That is, the cells of $\Gamma'$ are all contained in cells of $\Gamma$.
\end{definition}

\begin{theorem}\label{thm:polyhedral-embed}
  Every polyhedral loss embeds a finite elicitable property.
  Moreover, a polyhedral loss indirectly elicits a finite elicitable property $\gamma$ if and only if it embeds a property which refines $\gamma$.
\end{theorem}
\begin{proof}
  \raf{This proof is kind of awesome.}

  Let $L:\reals^d\to\reals_+$ be a polyhedral loss.
  For all $p$, let $P(p)$ be the epigraph of the convex function $u\mapsto p\cdot L(u)$.
  From Lemma~\ref{lem:polyhedral-pd-same}, we have that the power diagram induced by the projection of $P(p)$ onto $\reals^d$ is constant whenever $p\in\inter\simplex$.
  Let $q\in\inter\simplex$ be the uniform distribution on $\Y$, and $V_\Y$ be the set of vertices of $P(q)$ projected onto $\reals^d$.
  By the above, this set is the same had we replaced $q$ by any $p\in\inter\simplex$.

  Now let $\Gamma := \Gamma[L]$.
  We claim for all $p\in\inter\simplex$, that $\Gamma(p) \cap V_\Y \neq \emptyset$.
  To see this, let $u \in \Gamma(p)$, and $u' = (u,p\cdot L(u)) \in P(p)$.
  The optimality of $u$ is equivalent to $u$ being contained in the face exposed by the normal $(0,\ldots,0,-1)\in\reals^{d+1}$, which is a face of $P(p)$.
  Let $v'\in\reals^{d+1}$ be a vertex on such a face, and $v\in V_\Y$ its projection onto $\reals^d$.
  Then $v$ is also optimal, and therefore $v\in\Gamma(p)$.

  Now consider $\Y'\subset \Y$.
  Applying the above argument on distributions $p$ with support exactly $\Y'$, we have a similar guarantee: a finite set $V_{\Y'}$ such that for all $p$ with support exactly $\Y'$, we have $\Gamma(p) \cap V_{\Y'} \neq \emptyset$.
  (When $\Y' = \{y\}$ is a singleton, we simply take the projected vertices of $L(\cdot)_y$.)

  Thus, taking $V = \bigcup_{\Y'\subseteq\Y} V_{\Y'}$, we have for all $p\in\simplex$ that $\Gamma(p) \cap V \neq \emptyset$.
  This implies that $\trim(\Gamma) \subseteq \{\Gamma_v : v\in V\}$, which is finite, so Proposition~\ref{prop:embed-trim} now gives the conclusion.

  For the second part, \raf{just need to show that the link can only glue level sets together; this gets a little tricky with the boundaries of the cells, since you have choices, but }
  From Lemma~\ref{lem:pbar}
\end{proof}

\raf{Transition: wait, what about dimension?}

\section{Embeddings via Real-Valued Losses}

The following is a minor modification of a similar definition of~\citet{lambert2018elicitation}.
\begin{definition}
  A finite property $\gamma:\simplex\toto\R$ is \emph{orderable} if there is an enumeration $\R = \{r_1,\ldots,r_k\}$ such that for all $i\in\{1,\ldots,k-1\}$, the intersection $\gamma_{r_i} \cap \gamma_{r_{i+1}}$ is a hyperplane intersected with $\simplex$.
\end{definition}

\citet{lambert2018elicitation} shows that a finite property has a \emph{order-sensitive} loss, meaning one assigning higher loss to ``farther'' reports (with respect to a given order), if and only if it is orderable.
We will see that, in fact, a finite property is 1-embeddable if and only if it is orderable.
Both the forward direction and its converse will make use of a related definition: we say a property is \emph{monotone} if, roughly speaking, it can be defined in terms of (possibly infinitely many) hyperplanes whose normal vectors have coefficients which are monotone in the report value.

First, the forward direction.

\begin{proposition}\label{prop:orderable-embed}
  Every orderable property $\gamma$ is 1-embeddable.
\end{proposition}
\begin{proof}
  Let $\R = \{r_1,\ldots,r_k\}$ be the report space for $\gamma$.
  From Lemma~\ref{lem:prop-L-monotone}, stating that finite properties are orderable if and only if they are monotone, we have functions $a:\R\to\reals^\Y, b:\R\to\reals^\Y$ satisfying the following: (i) $\gamma_{r_i} = \{p\in\simplex : a(r_i) \cdot p \leq 0 \leq b(r_i) \cdot p\}$ for all $i\in\{1,\ldots,k\}$, and (ii) $a(r_i) \leq b(r_i) = a(r_{i+1}) \leq b(r_{i+1})$ for all $i \in \{1,\ldots,k-1\}$.
  We now define $\varphi(r_i) = i \in \reals$, and define $L$ as follows:
  \begin{equation*}
    L(u)_y =
    \begin{cases}
      a(r_1) u & u < 1 \\
      c_i + b(r_i) u & u \in [i,i+1), i\in\{1,\ldots,k\} \\
      c_k + b(r_k) u & u \geq k+1
    \end{cases}~,
  \end{equation*}
  where the $c_i$ are chosen to make $L(u)_y$ continuous, namely, $c_i = \sum_{j=1}^{i} a(r_j)_y$.
  (Recall that $b(r_i) = a(r_{i+1})$.)
  By the condition (ii) above, $L(\cdot)_y$ is convex for all $y\in\Y$, and moreover, its subgradients at $\varphi(\R) = \{1,\ldots,k\}$ are given by
  $\partial L(i)_y = [a(r_i)_y,b(r_i)_y]$.
  By condition (i), for all $i\in\{1,\ldots,k\}$ and $p\in\simplex$ we now have
  \begin{align*}
    i \in \prop{L}(p)
    &\iff 0 \in \partial (p \cdot L(i)) \\
    &\iff 0 \in \sum_{y\in\Y} p_y \partial L(i)_y \\
    &\iff 0 \in \sum_{y\in\Y} p_y [a(r_i)_y,b(r_i)_y] \\
    &\iff a(r_i) \cdot p \leq 0 \leq b(r_i) \cdot p \\
    &\iff r_i \in \gamma(p)~,
  \end{align*}
  where the sums are Minkowski sums.
\end{proof}

The converse of Proposition~\ref{prop:orderable-embed}, that every 1-embeddable property is orderable, follows from a much broader statement, given below as Proposition~\ref{prop:indirect-orderable}: finite elicitable properties which are indirectly elicited by 1-dimensional convex losses must be orderable.
Together with Proposition~\ref{prop:embed-link}, which shows that embeddings give rise to calibrated links, we have in particular that 1-embeddable finite properties are orderable.

\begin{proposition}\label{prop:indirect-orderable}
  If convex $L : \reals \to \reals^\Y$ indirectly elicits a finite elicitable property $\gamma$, then $\gamma$ is orderable.
\end{proposition}

The proof of Proposition~\ref{prop:indirect-orderable} follows from Lemma~\ref{lem:prop-L-monotone} in Appendix~\ref{app:dimension-1}, which states that the property elicited by any convex loss must be monotone in the sense above.
(The normal vectors are constructed in terms of the subgradients of the loss, thus guaranteeing monotonicity of their coefficients.)

As a corollary of Proposition~\ref{prop:indirect-orderable} and Theorem~\ref{thm:polyhedral-embed}, we now also see that polyhedral losses embed orderable properties.

\begin{corollary}\label{cor:embed-orderable}
  Every polyhedral $L : \reals \to \reals^\Y$ embeds an orderable property.
\end{corollary}

The central conclusion from this section is the following theorem, which states that several elicitation complexity classes coincide in dimension 1.
The proof is given in Appendix~\ref{app:dimension-1}, which only needs to show the equivalence of statements 1 and 5, as we have already established the equivalence of the first four statements: $1 \implies 2,3,4$ from Proposition~\ref{prop:orderable-embed}, as the proof gives a loss which is polyhedral (and convex), and Proposition~\ref{prop:indirect-orderable} shows $2,3,4 \implies 1$, as each of these cases would give a convex loss indirectly eliciting a finite elicitable property.
\begin{theorem}\label{thm:1d-tfae}
  Let $\gamma$ be a finite elicitable property.
  The following are equivalent.
  \begin{enumerate}
  \item $\gamma$ is orderable.
  \item $\elicembed(\gamma)=1$. ($\gamma$ is 1-embeddable.)
  \item $\elicpoly(\gamma)=1$. ($\gamma$ is indirectly elicitable via a polyhedral loss $L:\reals\to\reals^\Y$.)
  \item $\eliccvx(\gamma)=1$. ($\gamma$ is indirectly elicitable via a convex loss $L:\reals\to\reals^\Y$.)
  \item $\eliccts(\gamma)=1$. ($\gamma$ is indirectly elicitable via a real-valued continuous non-locally-constant property.)
  \end{enumerate}
\end{theorem}

\raf{Note the relationship to~\cite{finocchiaro2018convex}, and that this is nearly a corollary, as that work almost shows $5\implies 4$, but it needs slightly more assumptions than we need.  Instead, we give a direct proof.}

\section{Higher-Dimensional Embeddings}\label{sec:general-dimensions}

Orderability is a useful characterization of 1-embeddable properties, in the sense that the orderability of a property is easily tested.
We seek a similar condition for higher dimensions.
Toward this end, let us decompose orderability into two conditions which will lend themselves to generalization.
A property $\gamma$ is orderable if and only if there exist closed intervals $[a(r)_y,b(r)_y] \subseteq \reals$ for all $r\in\R, y\in\Y$, such that the following two conditions hold:
\begin{enumerate}
\item (Optimality) For all $r\in\R, p\in\simplex$, we have $r\in \gamma(p) \iff a(r)\cdot p \leq 0 \leq b(r)\cdot p$.
\item (Monotonicity) There exists an ordering $\R=\{r_1,\ldots,r_k\}$ such that for all $y\in\Y$, $i\in\{1,\ldots,k-1\}$, we have $b(r_i) = a(r_{i+1})$.
\end{enumerate}
Thinking of the intervals $[a(r)_y,b(r)_y]$ as subgradients of a convex function $L_y$, the first condition states that the report $r$ is optimal exactly when it should be, and the second states that the intervals are indeed subgradients of some convex function.

Generalizing the above to higher dimensions, we obtain the following.
\begin{theorem}
  A finite property $\gamma:\simplex\toto\R$ is $d$-embeddable if and only if there exist polytopes $T(r,y) \subseteq \reals^d$ for all $r\in\R, y\in\Y$, such that the following two conditions hold:
  \begin{enumerate}
  \item (Optimality) For all $r\in\R, p\in\simplex$, we have $r\in \gamma(p) \iff 0 \in \sum_{y\in\Y} p_y T(r,y)$, where summation is the Minkowski sum.
  \item (Monotonicity) There exists an embedding $\varphi: \R \to \reals^d$ and set of convex functions $\{L_y:\reals^d\to\reals\}_{y\in\Y}$ such that for all $y\in\Y, r\in\R$ we have $T(r,y) = \partial L_y(\varphi(r))$.
  \end{enumerate}
\end{theorem}

\bo{Added the below blurb}
\raf{I love it.  This would be a great place to discuss the relationship to FSD, which is ``even more local'' in the sense that it does not even look at a whole cell at once, just one piece of the boundary.  We can then say that the additional power is strong enough to give novel lower bounds where FSD remains silent.  (In particular, abstain looks 1d / orderable when zoomed in as far as FSD, but one can conclude it is not orderable using condition 1 alone.)}
This characterization decomposes embeddability into \emph{local} and \emph{global} necessary and sufficient conditions.
The local conditions at each embedding point turn out to imply very strong restrictions on the structure of possible polyhedral losses, regardless of any other structure in the property or where the embedding points are placed.
Meanwhile, the global condition says that, once each local embedding point's demands are satisfied, we must be able to place them all in $\reals^d$ in a coherent manner that simultaneously respects all of their requirements.

\bo{I think we should write separately the result that the expected loss function only has vertices above the embedding points (um, sort of - WLOG?). I think we've known it for so long we forgot that it's a really interesting and also a necessary and sufficient condition. Does it also imply that each individual loss function's vertices can only be at embedding points?}
\raf{This is one of those statements that is probably true in some sense but we're not sure what sense.  There could be vertices that are just never optimal, and it's hard to know exactly how to rule them out.}

\subsection{Useful necessary conditions}

Each embedding point's optimality condition turns out to imply limitations on \emph{each} of the $n$ loss functions.
This can be enough to rule out embeddability only by considering a single level set, as we develop next.

Recall that a \emph{halfspace-representation (H-rep)} of a closed polyhedral set $S$ in $d$ dimensions is a pair $(V,b)$ with $V \in \reals^{k \times d}$ and $b \in \reals^k$ such that $S = \{ w : Vw \leq b \}$.
We say an H-rep is \emph{tight} if for all $i=1,\dots,k$, there exists $w \in S$ such that $\langle V_i , w \rangle = b_i$.
\jessie{Imagine two planes in $\reals^3$ (constructing the H-rep) intersect on the line $\ell$ and a third plane only intersects the H-rep on $\ell$, but doesn't make the convex polytope any smaller.  Are we still ok with this being considered tight?  (I think I'm saying if each face is not full-dimensional?)}

\begin{definition}\label{def:d-construction}
  Given a polytope $\{p \in \Delta_{\Y} : Bp \geq 0\}$ for $B \in \reals^{k \times n}$, a \emph{$d$-construction} of it is a $V \in \reals^{k \times d}$, each row a unit vector, such that, for each column $B_i$ of $B$, the pair $(V,B_i)$ is a tight halfspace representation.
\end{definition}

\begin{corollary}\label{cor:d-embed-implies-d-construction}
  If a property is $d$-embeddable, then each of its level sets has a $d$-construction.
  \bo{Is it possible that some uniqueness of the construction is also true?}\jessie{Up to rotation, scaling, and translation, possible but I wouldn't think so.  ex: Abstain(1/2) on 3 outsomes can be embedded in $\reals^2$ with both an equilateral and isosceles triangle}
\end{corollary}
\begin{proof}
  Suppose the level set is represented as $\Gamma_r = \{p : Bp \geq 0\}$ with $B \in \reals^{k \times n}$ for the minimum $k$ (i.e. the number of facets of the polytope).
  Let $u \in \reals^d$ be its embedding point.
  The convex expected loss is minimized at $u$ if and only if it has the zero vector as a subgradient at $u$, i.e.
    \[ p \in \Gamma_r \qquad \iff \qquad 0 \in \partial \langle L(u), p \rangle . \]
  Subgradient sets satisfy the following two properties: $\partial(f_1 + f_2) = \partial f_1 + \partial f_2$ where the right side is the Minkowski sum of the subgradient sets (i.e. the set of sums of all pairs); and $\partial(\alpha f_1) = \alpha \partial f_1$.
  So
    \[ \partial \langle p , L(u) \rangle = \sum_{y \in \Y} p_y \partial L(u)_y \]
  where the right side is a Minkowski sum of subgradient sets.
  We now consider the halfspaces defining this set.
  Define the $y$th support function as $H_y(v) = \max_{w \in \partial L(u)_y} \langle w, v \rangle$ to obtain that $w \in \partial \langle p, L(u) \rangle$ if and only if, for all unit vectors $v$,
   \[ \langle v, w \rangle \leq \sum_{y \in \Y} p_y H_y(v) . \]
  In particular, the embedding point is optimal for $p$ if and only if this is satisfied for $w=0$.
  So, letting $B_i$ be the $i$th column of $B$, we have
    \[ \Gamma_r = \{p : \sum_y p_y B_y \geq 0\} = \{p : \sum_y p_y H_y(v) \geq 0 ~ (\forall v, \|v\|=1)\} . \]
  \bo{This is the part I'm still not perfectly happy with}
  These sets are equal and full-dimensional (as the property is finite and non-redundant and $B$ is minimal).
  So there exist $k$ unit vectors $v_1,\dots,v_k \in \reals^d$ such that $H_y(v_i) = B_{iy}$.
  Collected as rows of $V$, we obtain that each $(V,B_y)$ is a tight halfspace representation, as it contains $\partial L(u)_y$ and $H_y(v_i) = B_{iy}$, meaning there exists $w \in \partial L(u)_y$ such that $\langle w, v_i \rangle = B_{iy}$.
\end{proof}
This provides the following recipe for both constructing losses and proving lower bounds.
First, take any level set and represent it as $\{p : Bp \geq 0\}$.
Now if $B$ has $k$ rows, search for a collection of $k$ unit vectors in $\reals^d$.
For any candidate $V$ (whose rows are the unit vectors), a computer program can automatically verify whether the representations $\{(V,B_y)\}_{y \in \Y}$ are tight.
In particular, if any $(V,B_y)$ is infeasible (i.e. there does not exist any $w$ satisfying all constraints $\langle v_i, w \rangle \leq B_{iy}$), then $V$ is not a $d$-construction.
If $(V,B_y)$ is feasible but some constraints are redundant, i.e. it is not tight, then again we do not have a $d$-construction.

If a $d$-construction $V$ is found, one also obtains that the subgradient sets $\partial L(u)_y$ must ``span'' the respective polytopes $\{w : Vw \leq B_y\}$, i.e. must be contained within them and must make each constraint tight.
Repeating this process for each level set and associated embedding point gives a relatively tight picture of what any possible loss must look like.

A lower bound can be proven by showing, for any level set, that no collection of $k$ unit vectors can form a $d$-construction.
We will give examples of this in $2$ dimensions next.

\raf{Paragraph explaining two other useful necessary conditions: WMON, and reduction to 1 dimension}

\subsection{Applications: Abstain Loss and 0-1 Loss \raf{or something}}

\begin{theorem}[\bo{cite}]
  The mode with $|\Y|=n$ is not $(n-2)$-embeddable.
\end{theorem}
\begin{proof}
  Call the outcomes $\Y = \{1,\dots,n\}$.
  We will pick the level set for the report $r=1$ and show it has no $d$-representation.
  The level set can be represented as $\{p : p_1 \geq p_i ~ (\forall i \neq 1)\}$, or the set of $p$ such that $Bp \geq 0$ with
  \[ B = \begin{pmatrix}
       1  & -1 &  0 & \cdots & 0  \\
       1  &  0 & -1 & \cdots & 0  \\
          &    &    & \vdots &    \\
       1  &  0 &  0 & \cdots & -1
     \end{pmatrix} .
  \]
  Here $B$ has $n-1$ rows, one for each $i \neq 1$.

  A $d$-representation is therefore a $V$ with $n-1$ rows, each a unit vector $v_i$ in $\reals^d$.
  Suppose $d \leq n-2$, i.e. there are at least $d+1$ rows.
  Now consider the halfspace representation given by the second column of $B$, namely $T = \{ w : Vw \leq (-1, 0, \dots, 0)^{\intercal}\}$.
  The first $d+1$ rows are linearly dependent in $\reals^d$.
  Therefore, we can write the $d+1$st row as $v_{d+1} = \alpha_1 v_1 + \cdots + \alpha_d v_d$.
  So if $w \in T$, then $\langle v_1,w \rangle \leq -1$ and $\langle v_i,w \rangle \leq 0$ for all $i=2,\dots,d$.
  Taking the weighted combination, we obtain $\langle v_{d+1}, w \rangle \leq -\alpha_1$.
  If $\alpha_1$ is positive, this implies there is slack in the $d+1$st constraint, so the halfspace representation cannot be tight.
  If $\alpha_1$ is nonpositive, then we pick some other pair of vectors $i \neq j$ rather than $1,d+1$, such that $v_j = \sum_{i'\neq j} \alpha_{i'} v_{i'}$ with $\alpha_i > 0$.
  \bo{I claim there must exist two such vectors in this collection ... need to prove it though!}
  Choosing the $i$th column of $B$ as a constraint gives the contradiction as above.
\end{proof}
To recall what is happening inside this proof: we chose the first level set, which gave us a matrix $B$.
The first column was all-$1$, which corresponds to a constraint on the subgradient set of the loss on outcome $1$ for report $1$, which is not an interesting constraint.
The other constraints gave a contradiction if $d$ is too small.
\bo{More to say?}

\begin{theorem}
  Abstain with $\alpha=1/2$ and $|\Y|=5$ is not $2$-embeddable.
\end{theorem}
\begin{proof}
\jessie{Taking a crack at it- will iterate again tomorrow.}
By Corollary~\ref{cor:d-embed-implies-d-construction}, we want to show there is no $2$-construction for the abstain cell $B^{(abs)} := \ones - 2D$, where $\ones$ is the $5\times 5$ matrix of all entries being $1$, and $D$ being the matrix with $1$ on the diagonal and $0$ elsewhere.

In order to show there is no $V\in \reals^{5\times 2}$ so that for all $i \in \{1, 2, \ldots, 5\}$ that $(V, B_i)$ is a tight H-rep, following from Definition~\ref{def:d-construction}.
In other words, we show there is no $V \in \reals^{5\times 2}$ and $w \in \reals^2$ so that $\{w : Vw = B_i\}$ is nonempty.

Consider $V = [v_1, \ldots, v_5]^T$ and, without loss of generality, $v_1 = [0,1]$.
In order for the H-rep to be tight, then for all $v_j$ for $j > 1$, $\langle v_j, w \rangle = 1$ and $w = [x, -1]$.
In order for $\{w : Vw = B_i \}$ to be nonempty for all $i$, we assert that we must have some other $v_j$ (say, $v_2$) so that $v_2 := -v_1 = [0, -1]$.
\jessie{I don't like this justification, but it's late and I know what I'm \emph{trying} to say... kind of.}
This is observed since the H-rep, otherwise, would not be tight, since a line in $\reals^2$ cannot be (non-redundantly) bounded by $4$ points.
In fact, a line in $\reals^2$ cannot be nonredundantly bounded by $3$ points.
This suggests that there is no set of normals $V \in \reals^{5\times 2}$ so that $\{w : Vw = B_1\}$ is tight for all $v_j$ and $B_1$.

Therefore, since there is no $2$-construction of abstain(1/2) on $5$ outcomes, we conclude it is not $2$-embeddable.
\jessie{Weird to think of it this way after drawing so many pictures, but this might be more elegant if it's close.}
\end{proof}


\begin{theorem}
  Abstain with $\alpha > 1/2$ and $|\Y|=7$ is not $2$-embeddable.
\end{theorem}

\begin{conjecture}
  Abstain with $\alpha=1/2$ is $(\log_2 |\Y|)$-embeddable, but no lower.
\end{conjecture}

\section{Conclusion and Future Work}

Let $\mathrm{elic}_{embed}$ denote the elicitation complexity with respect to embeddings, $\mathrm{elic}_{Pcvx}$ the complexity with respect to polyhedral convex losses, and $\mathrm{elic}_{cvx}$ the complexity with respect to arbitrary convex losses.

\begin{conjecture}
  $\mathrm{elic}_{embed}(\Gamma) = \mathrm{elic}_{Pcvx}(\Gamma) = \mathrm{elic}_{cvx}(\Gamma)$ for all finite, elicitable properties $\Gamma$.
\end{conjecture}


%%%%%%%%%%%%%%%%%%%%%%%%%%%%%%%%%%%%%%%%%%%%%%%%%%%%%%%%%%%%%%%
\section{Conclusion and Directions}  \label{sec:conclusion}
\paragraph{Summary.}
This work is part of a broader research program to understand convex surrogates through the lens of property elicitation: the relationship between finite losses and convex surrogates, the link functions connecting them, and the properties they elicit.
We seek a general theory that, given a property, can prescribe when and how to construct convex surrogate losses that elicit it, and specifically, determine the minimum dimension required.
Even more broadly, one could replace ``convex'' by any notion of ``nice'' surrogate desired.

This work formalized the \emph{embedding} approach where labels are identified with points in $\reals^d$.
We saw \bo{in theorem ??} that this approach is tightly connected to the use of polyhedral (i.e. piecewise linear convex) loss functions.
We also saw that it is a general technique that can be used for any finite elicitable property.

We then investigated the \emph{dimensionality} of $\reals^d$ required for such embeddings, giving a characterization in terms of the structure of the property in the simplex.
This gave a complete understanding of the one-dimensional case, and a complete characterization albeit weaker understanding in higher dimensions.
%RF got here
The two key conditions are an optimality condition that relates the structure of each level set to existence of polytopes in $\reals^d$ satisfying certain conditions; and a monotonicity condition relating such polytopes for different level sets.
This yields new lower bounds in particular for the abstain loss.

\paragraph{Directions.}
There are several direct open questions involving embedding dimension.
It would be interesting and perhaps practically useful to develop further techniques for upper bounds: automatically constructing embeddings in $\reals^d$ and associated polyhedral losses from a given property, with $d$ as small as possible.
Another direction is additional lower bound techniques, or further development of our necessary conditions.

This paper also suggests an agenda of defining more nuanced classes of surrogate losses and studying their properties.
For example, a tangential topic in this work was the characteristics of a ``good'' link function; formalizing and exploring this question is an exciting direction.
We would also like to move toward a full understanding of the differences between these classes.
For example, how does embedding dimension compare in general to convex elicitation diemsnion (the dimensionality $d$ required of \emph{any} convex surrogate loss)?
These questions have both theoretical interest and potential practical significance.

\subsection*{Acknowledgements}
We thank Peter Bartlett for several discussions early on, which led to a proof of \raf{1-d reduction} among other insights.
\raf{Almost forgot: should ack Arpit too of course!}

\bibliography{diss,extra}

%%%%%%%%%%%%%%%%%%%%%%%%%%%%%%%%%%%%%%%%%%%%%%%%%%%%%%%%%%%%%%%
\appendix

\section{Considering unlabeled properties}
  Note that when $\Gamma$ is elicitable, there is a convex score $G:\simplex \to \reals$ representing the Bayes Risk (i.e. $G(p)$ is the minimal expected loss over all reports at $p$.)
  There is a set $\D \subseteq \partial_p G$ so that there is a bijection $\phi:\D \to \R$.
  Therefore, the subgradients of $G$ correspond to the property value at the distribution $p$.

  The next two proofs take advantage of this duality correspondence between the subgradients of the convex score $G$ of an elicitable property and the property values.


\begin{lemma}\label{lem:trim-subsets-not-full-dimensional}
Let $\Gamma:\simplex \toto \R'$ be an elicitable property.
For any $u,w \in \R'$, if $\Gamma_u \subsetneq \Gamma_w$, then $\Gamma_u$ is not full dimensional.
\end{lemma}
\begin{proof}
  If there is a full-dimensional level set $\Gamma_u \subsetneq \Gamma_w$, then the corresponding scoring rule $G$ must not be differentiable (i.e. $|\partial_p G(p)| > 1$) for all $p \in \Gamma_u$.
%  (This is true because $\phi^{-1}(u) \neq \phi^{-1}(w)$ as the level sets are not equal.)
  However, closed, convex, full-dimensional level sets have positive Lebesgue measure in $\reals^{n-1}$,\jessiet{I don't know if we need a citation/proof for this assertion...} and thus contradicts the convexity of $G$, as convex functions are differentiable at all but a countable set of points.
  This in turn contradicts the elicitability of $\Gamma$, and so, we conclude that $\Gamma_u$ is not full-dimensional.
\end{proof}

\begin{lemma}\label{lem:unique-opt-on-inter}
  Let $\Gamma:\simplex \toto \R'$ be an elicitable property such that $\trim(\Gamma)$ is finite.
  For all $u \in \R'$, if $p \in \inter{\Gamma_u}$, then $\Gamma(p) = \{u\}$.
\end{lemma}
\begin{proof}
\jessiet{Come back and formalize}
  We'll use the relationship between nonredundant properties and subgradients of the expected score $G$.
  Now consider the following convex analysis fact: for any convex function and any two point which share a subgradient, the function is flat between those points.
  In particular, $G$ is flat on $\Gamma_u$, and therefore $G$ must be differentiable on the interior of $\Gamma_u$.
  Therefore, $G$ must only have one subgradient on $\inter{\Gamma_u}$.
  \bigskip
  \jessie{Formalization below}
  \hrule
  Consider that $\trim(\Gamma)$ is nonredundant, and $p \in \inter{\Gamma_u}$ only if $\Gamma_u \in \trim(\Gamma)$ by Lemma~\ref{lem:trim-full-dim}, as a closed, convex set is full-dimensional if and only if it has nonempty interior. \jessiet{Sanity check here?}
  Since $\Gamma$ is elicitable, the score $G$ must be convex.
  For $w \in \R'$, if $\Gamma_w \subseteq \Gamma_u$, then we contradict the nonredundancy of $\trim(\Gamma)$, so we only worry about the case where $\Gamma_w \not \subseteq \Gamma_u$..

  Suppose there is such a level set.
  There must be another level set $v$ so that $\Gamma_w \cap \inter{\Gamma_v}$ is nonempty, since level sets of elicitable properties are closed.
  Therefore, for $q \in \Gamma_w$, we observe $\Gamma(q) \subseteq \{u,w\}$ or $\{v,w\}$.
  More importantly- for $q \in \Gamma_w$, $\Gamma(q)$ is not constant.

  Let's say there is a $q \in \Gamma_w \cap \inter{\Gamma_u}$ and $q' \in \Gamma_w \cap \inter{\Gamma_v}$.
  Additionally, $\conv(\{q,q'\}) \subseteq \Gamma_w$, and observe that $\conv(\{q,q'\})$ is not countable.
  Since there is a bijection from $\Gamma(p)$ to $\partial_p G(p)$, we then note that $|\partial_pG(p)| > 1$ on an uncountable set, and $\partial_p G(p)$ is not constant on $\conv(\{q,q'\})$.
  This contradicts the convexity of $G$, and in turn, the elicitability of $\Gamma$ and $\trim(\Gamma)$.

  Therefore, for all $p \in \inter{\Gamma_u}$, we must observe $\Gamma(p) = \{u\}$.

\end{proof}

\begin{corollary}\label{cor:cant-span-interiors}
  If $\Gamma:\simplex \toto \R'$ is a nonredundant, finite elicitable property, no level set spans the interior of two other level sets.
  Formally, there is no $w \in \R'$ so that for any $p \in \inter{\Gamma_u}$ and $q \in \inter{\Gamma_v}$, $p,q \in \Gamma_w$ for reports $u,v \in \R'$ with $u \neq v$.
\end{corollary}
\jessiet{I can write something for a proof of this if you want, but it follows from the previous statement pretty quickly.}

\begin{lemma}\label{lem:trim-full-dim}
  Consider a nondegenerate, elicitable property $\Gamma$ with finitely many full-dimensional level sets.
  Every $\theta \in \Theta := \trim(\Gamma)$ is full-dimensional.
\end{lemma}
\begin{proof}
  Consider an arbitrary level set $\theta$.
  If $\theta$ is not full-dimensional, we want to show that it must be the intersection of some full-dimensional level sets in $\trim(\Gamma)$.\jessiet{Why?  Add more details about why the level set can't span two full-dimensional level sets}
  If it is a proper subset of the level sets of which it is the intersection, and so it is not in the $\trim$ as it is in the subtracted set.

  Now suppose that a level set $\theta$ was not full-dimensional and not a proper subset of any one level set.
  It must then span the interior multiple level sets, as level sets of elicitable properties are closed.
  By Corollary~\ref{cor:cant-span-interiors}, we contradict the elicitability of $\Gamma$.
  Thus, any $\theta$ that is not full-dimensional must be the intersection of a finite set of full-dimensional level sets $\theta_i \in \trim(\Gamma)$.
  As trimming a property removes such redundant reports, we then observe that $\theta \not \in \trim(\Gamma)$.
  Therefore, each $\theta \in \trim(\Gamma)$ must be full-dimensional.
\end{proof}



%\begin{lemma}\label{lem:finite-nonred-fdls}
%Let $\Gamma:\simplex \toto \R'$ be a nonredundant, finite elicitable property.
%Each level set $\Gamma_u \in \{\Gamma_u : u \in \R' \}$ is full-dimensional.
%\end{lemma}
%\begin{proof}
%Let us consider 3 cases: first, we cannot observe a level set is a subset of or equal to another by nonredundancy of the property.
%Now, if one level set spans the interiors of two different level sets, we will see that this property is not elicitable.
%Since we assume $\Gamma$ is elicitable, we know the score $G$ is convex over $\simplex$.

%Now, suppose there are two distributions $q,q'$ such that $q \in \inter{\Gamma_u}$ and $q' \in \inter{\Gamma_{u'}}$ for $u \neq u'$.
%Additionally, if there is a report $w$ such that $\conv(\{q,q'\}) \subseteq \Gamma_w$, we say that the level set $\Gamma_w$ spans the interior of two level sets.

%First, consider that if $q \in \inter{\Gamma_u}$ and $q' \in \inter{\Gamma_{u'}}$ are two nonredundant points in the set $S$ such that $\Gamma_w = \conv(S)$, then there are distributions $p := q + \epsilon(q-q')$ and $p' = q' + \epsilon(q'-q)$ that are both in the interior of their respective level sets and $\Gamma(p) \cap \Gamma(p') = \emptyset$.
%By \jessie{Raf and Ian 2018} Lemma 10, then for $\lambda \in (0,1)$, the score $G(\lambda p + (1-\lambda) p') < \lambda G(p) + (1-\lambda) G(p')$.
%However, if $q$ and $q'$ share a subgradient, then the score $G$ is flat on $\conv(\{q,q'\})$, which is a subset of $\conv(\{p,p'\})$.
%However, this contradicts \jessie{Raf and Ian 2018} Lemma 10, as the score $G$ is flat (and therefore the strict inequality is actually an equality) on $(a,b) \subseteq (0,1)$.

%Now, if the level set $\Gamma_w$ is described as the convex hull of distributions $q \not \in \inter{\Gamma_u}$ or $q' \not \in \inter{\Gamma_{u'}}$, such distributions $p$ and $p'$ do not exist in the respective level sets.
%\jessie{.... other argument}

%Therefore, for the elicitable property $\Gamma$, there cannot be a level set $\Gamma_w$ that spans the interiors of two other level set.s

%\end{proof}

\section{To Sort}

Several definitions from \citet{aurenhammer1987power}.
\begin{definition}
  A \emph{cell complex} in $\reals^d$ is a set $C$ of faces (of dimension $0,\ldots,d$) which (i) union to $\reals^d$, (ii) have pairwise disjoint relative interiors, and (iii) any nonempty intersection of faces $F,F'$ in $C$ is a face of $F$ and $F'$ and an element of $C$.
\end{definition}

\begin{definition}
  Given sites $s_1,\ldots,s_k\in\reals^d$ and weights $w_1,\ldots,w_k \geq 0$, the corresponding \emph{power diagram} is the cell complex given by
  \begin{equation}
    \label{eq:pd}
    \cell(s_i) = \{ x \in\reals^d : \forall j \in \{1,\ldots,k\} \, \|x - s_i\|^2 - w_i \leq \|x - s_j\| - w_j\}~.
  \end{equation}
\end{definition}

\begin{definition}
  A cell complex $C$ in $\reals^d$ is \emph{affinely equivalent} to a (convex) polyhedron $P \subseteq \reals^{d+1}$ if $C$ is a (linear) projection of the faces of $P$.
\end{definition}

\citet{aurenhammer1987power} shows that a cell complex is affinely equivalent to a convex polyhedron if and only if it is a power diagram.
In particular, one can consider the epigraph of a polyhedral convex function on $\reals^d$ and the projection down to $\reals^d$; in this case we call the resulting power diagram \emph{induced} by the convex function.
We extend Aurenhammer's result to a weighted sum of convex functions, showing that the induced power diagram is the same for any choice of strictly positive weights.

\begin{lemma}\label{lem:polyhedral-pd-same}
  Let $f_1,\ldots,f_k:\reals^d\to\reals$ be polyhedral convex functions.
  The power diagram induced by $\sum_{i=1}^k p_i f_i$ is the same for all $p \in \inter\simplex$.
\end{lemma}
\begin{proof}
  For any convex function $g$ with epigraph $P$, the proof of~\citet[Theorem 4]{aurenhammer1987power} shows that the power diagram induced by $g$ is determined by the facets of $P$.
  Let $F$ be a facet of $P$, and $F'$ its projection down to $\reals^d$.
  It follows that $g|_{F'}$ is affine, and thus $g$ is differentiable on $\inter F'$ with constant derivative $d\in\reals^d$.
  Conversely, for any subgradient $d'$ of $g$, the set of points $\{x\in\reals^d : d'\in\partial g(x)\}$ is the projection of a face of $P$; we conclude that $F = \{(x,g(x))\in\reals^{d+1} : d\in\partial g(x)\}$ and $F' = \{x\in\reals^d : d\in\partial g(x)\}$.

  Now let $f := \sum_{i=1}^k f_i$ with epigraph $P$, and $f' := \sum_{i=1}^k p_i f_i$ with epigraph $P'$.
  By \raf{Rockafellar}, $f,f'$ are polyhedral.
  We now show that $f$ is differentiable whenever $f'$ is differentiable:
  \begin{align*}
    \partial f(x) = \{d\}
    &\iff \sum_{i=1}^k \partial f_i(x) = \{d\} \\
    &\iff \forall i\in\{1,\ldots,k\}, \; \partial f_i(x) = \{d_i\} \\
    &\iff \forall i\in\{1,\ldots,k\}, \; \partial p_i f_i(x) = \{p_id_i\} \\
    &\iff \sum_{i=1}^k \partial p_if_i(x) = \left\{\sum_{i=1}^k p_id_i\right\} \\
    &\iff \partial f'(x) = \left\{\sum_{i=1}^k p_id_i\right\}~.
  \end{align*}
  From the above observations, every facet of $P$ is determined by the derivative of $f$ at any point in the interior of its projection, and vice versa.
  Letting $x$ be such a point in the interior, we now see that the facet of $P'$ containing $(x,f'(x))$ has the same projection, namely $\{x'\in\reals^d : \nabla f(x) \in \partial f(x')\} = \{x'\in\reals^d : \nabla f'(x) \in \partial f'(x')\}$.
  Thus, the power diagrams induced by $f$ and $f'$ are the same.
  The conclusion follows from the obversation that the above held for any strictly positive weights $p$, and $f$ was fixed.
\end{proof}

\begin{theorem}
  Let $\gamma:\simplex\toto\R$ be an elicitable finite property.
  Then $\gamma$ is $(n-1)$-embeddable, where $n=|\Y|$.
\end{theorem}
\begin{proof}
  Let $S:\R\to\reals^\Y$ be a scoring rule (negative loss) eliciting $\gamma$, and let $G:\simplex\to\reals$ be its expected score function (negative Bayes risk), given by $G(p) = \max_{r\in\R} p \cdot S(r)$.
  By Lemma~\ref{lem:finite-full-dim}, $G$ is polyhedral and therefore closed, and by~\citet[Theorem 19.2]{rockafellar1997convex}, its convex conjugate $C : \reals^\Y \to \reals$ is also polyhedral and closed.
  
  By \raf{results with Ian} we know that for some $\D \subseteq \partial G(\simplex) \subseteq \reals^\Y$ we have some bijection $\varphi : \R \to \D$ such that $\Gamma(p) = \varphi^{-1}(\D \cap \partial G(p))$.
  \raft{In other words, $\Gamma$ is a relabeling of subgradients of $G$}
  Thus, we can write
  \begin{equation}
    \label{eq:1}
    r \in \Gamma(p) \iff \varphi(r) \in \partial G(p) \iff p \in \partial C(\varphi(r))~,
  \end{equation}
  where the final equivalence follows from convex duality.
  \raft{Of course, we need to be careful here; this only holds if $G$ is proper and lower semi-continuous, but those should both hold for us.}

  Now define $L(\hat r,y) = C(\hat r) - \hat r \cdot \ones_y$ on the report space $\hat \R = \conv(\D)$;
  we will show that $L$ essentially elicits $\Gamma$.
  As $L$ is convex, we may write the optimality condition as
  \begin{equation}
    \label{eq:2}
    \hat r \in \Gamma_L(p) \iff \hat r \in \argmax_{\hat r'} \E_pL(\hat r',Y) \iff 0 \in \partial \E_pL(\hat r,Y)~,
  \end{equation}
  where of course the subgradient $\partial$ is with respect to $\hat r$.
  Plugging in our definition of $L$, we have
  \begin{align*}
    0 \in \partial \E_pL(\hat r,Y)
    & \iff 0 \in \partial \E_p \left(C(\hat r) - \hat r \cdot \ones_Y\right)
    \\
    & \iff 0 \in \partial \left(C(\hat r) - \hat r \cdot p\right)
    \\
    & \iff 0 \in \partial C(\hat r) - \{p\}
    \\
    & \iff p \in \partial C(\hat r)~.
  \end{align*}
  Putting this together with eq.~\eqref{eq:1}, we have for any $\hat r \in \D$ that
  \begin{equation}
    \label{eq:3}
    \hat r \in \Gamma_L(p) \iff p \in \partial C(\hat r) \iff \varphi^{-1}(\hat r) \in \Gamma(p)~,
  \end{equation}
  which implies that $L$ essentially elicits $\Gamma$ for this $\varphi$.
  (The trivially induced map $\varphi:\R\to\hat\R$ is injective as $\varphi:\R\to\D$ was a bijection.)
  \raf{If we assume $\Gamma$ is non-redundant, I believe we also have $\Gamma = \trim(\Gamma_L)$ using $\D$ is the report set.}
\end{proof}

\section{Dimension 1}\label{app:dimension-1}

\begin{definition}\label{def:monotone-prop}
  A property $\Gamma:\simplex\to\R$ is \emph{monotone} if there are maps $a:\R\to\reals^\Y$, $b:\R\to\reals^\Y$ and a total ordering $<$ of $\R$ such that the following two conditions hold.
  \begin{enumerate}
  \item For all $r\in\R$, we have $\Gamma_r = \{p\in\simplex : a(r) \cdot p \leq 0 \leq b(r) \cdot p\}$.
  \item For all $r < r'$, we have $a(r) \leq b(r) \leq a(r') \leq b(r')$ (component-wise).
  % \item For all $r,r'\in\R$ and $p\in\Gamma_{r'}\setminus\Gamma_r$, we have $b(r) \cdot p < 0 \implies r' > r$ and $a(r) \cdot p > 0 \implies r' < r$.
  \end{enumerate}
\end{definition}

\begin{lemma}\label{lem:orderable-monotone}
  A finite property is orderable if and only if it is monotone.
\end{lemma}
\begin{proof}
  Let $\gamma:\simplex\toto\R$ be finite and monotone.
  Then we can use the total ordering of $\R$ to write $\R = \{r_1,\ldots,r_k\}$ such that $r_i < r_{i+1}$ for all $i \in \{1,\ldots,k-1\}$.
  We now have $\gamma_{r_i} \cap \gamma_{r_{i+1}} = \{p\in\simplex : a(r_{i+1}) \cdot p \leq 0 \leq b(r_i) \cdot p\}$.
  If this intersection is empty, then there must be some $p$ with $b(r_i) \cdot p < 0$ and $a(r_{i+1}) \cdot p > 0$; by monotonicity, no earlier or later reports can be in $\gamma(p)$, so we see that $\gamma(p) = \emptyset$, a contradiction.
  Thus the intersection is nonempty, and as we also know $b(r_i) \leq a(r_{i+1})$ we conclude $b(r_i) = a(r_{i+1})$, and the intersection is the hyperplane defined by $b(r_i) = a(r_{i+1})$.

  For the converse, let $\gamma:\simplex\toto\R$ be finite and orderable.
  From~\cite[Theorem 4]{lambert2018elicitation}, we have positively-oriented normals $v_i\in\reals^\Y$ for all $i \in \{1,\ldots,k-1\}$ such that $\gamma_{r_i} \cap \gamma_{r_{i+1}} = \{p\in\simplex : v_i\cdot p = 0\}$, and moreover, for all $i \in \{2,\ldots,k-1\}$, we have $\gamma_{r_i} = \{p\in\simplex : v_{i-1} \cdot p \leq 0 \leq v_i \cdot p\}$, while $\gamma_{r_1} = \{p\in\simplex : 0 \leq v_1 \cdot p\}$ and $\gamma_{r_k} = \{p\in\simplex : v_{k-1} \cdot p \leq 0\}$.
  From the positive orientation of the $v_i$, we have for all $p\in\simplex$ that $\sgn(v_i\cdot p)$ is monotone in $i$.
  In particular, it must be that for all $y$, $\sgn((v_i)_y)$ is monotone in $i$, taking the distribution with all weight on outcome $y$.
  % (Similarly, if $\gamma(\ones_y) = \{r_j,r_{j+1}\}$, then $(v_i)_y = v_i \cdot \ones_y < 0$ for $i < j$, $(v_j)_y = 0$, and $(v_i)_y > 0$ for $i > j$.)
  \raf{This observation could use a better proof.}

  For all $i\in\{2,\ldots,k-1\}$, we wish to find $\alpha_i \geq 0$ such that $v_{i-1} \leq \alpha_i v_i$ (component-wise).
  To this end, fix $i$ and let $v = v_{i-1}, w = v_{i}, \alpha=\alpha_i$.
  By the above, there is no $y\in\Y$ with $v_y > 0 > w_y$, so the condition $v \leq \alpha w$ is equivalent to the following:
  \begin{enumerate}
  \item[(i)] For all $y$ with $v_y,w_y < 0$, $\alpha \leq v_y/w_y$.
  \item[(ii)] For all $y$ with $v_y,w_y > 0$, $\alpha \geq v_y/w_y$.
  \end{enumerate}
  Defining $\Y^- = \{y\in\Y: v_i,w_y < 0\}, \Y^+ = \{y\in\Y: v_i,w_y > 0\}$, these conditions are in turn equivalent to $\min_{y\in\Y^-} v_y/w_y \geq \max_{y\in\Y^+} v_y/w_y$, with the usual conventions $\min \emptyset = \infty,\; \max \emptyset = -\infty$.
  Suppose this inequality were not satisfied.
  Then we would have $y \in \Y^-,y'\in\Y^+$ such that $0 < v_y/w_y < w_{y'}/v_y$, which would in turn imply $|v_y|/v_{y'} < |w_y| / w_{y'}$.
  Letting $c = \tfrac 1 2 \left(|w_y| / w_{y'} + |v_y|/v_{y'}\right)$ and taking $p$ to be the distribution with weight $1/(1+c)$ on $y$ and $c/(1+c)$ on $y'$, we see that
  \begin{align*}
    v \cdot p &= \frac 1 {1+c} \left(v_y + \tfrac 1 2 (|w_y| / w_{y'} + |v_y|/v_{y'})v_{y'}\right) > \frac 1 {1+c} \left(v_y + (|v_y|/v_{y'})v_{y'}\right) = 0
    \\
    w \cdot p &= \frac 1 {1+c} \left(w_y + \tfrac 1 2 (|w_y| / w_{y'} + |v_y|/v_{y'})v_{y'}\right) < \frac 1 {1+c} \left(w_y + (|w_y|/w_{y'})w_{y'}\right) = 0~,
  \end{align*}
  thus violating the observation that $\sgn(v_i\cdot p)$ is monotone in $i$ (recall that $v = v_{i-1}, w = v_{i}$).

  We can now construct the $a,b$ in Definition~\ref{def:monotone-prop}.
  Letting $\alpha_i = \tfrac 1 2 \left(\min_{y\in\Y^-(i)} (v_{i-1})_y/(v_{i})_y + \max_{y\in\Y^+(i)} (v_{i-1})_y/(v_{i})_y\right)$, the preceding argument shows that $v_{i-1} \leq \alpha_i v_{i}$ for all $i \in \{2,\ldots,k-1\}$.
  We now set $b(r_i) = a(r_{i+1}) = (\prod_{j=2}^i \alpha_j) v_i$ for $i\in\{1,\ldots,k-1\}$, as well as $a(r_1) = -\max_{y\in\Y} |(v_1)_y|\ones$ and $b(r_k) = \max_{y\in\Y} |a(r_k)_y|\ones$, where $\ones\in\reals^\Y$ denotes the all-ones vector.

  The above establishes the second condition of monotone properties in Definition~\ref{def:monotone-prop}.
  To see the first condition, note that we have already established it for $i\in\{2,\ldots,k-1\}$.
  For $i=1,k$, we merely observe that $a(r_1) \cdot p \leq 0$ and $b(r_k) \cdot p \geq 0$ for all $p\in\simplex$.
\end{proof}

\begin{lemma}\label{lem:prop-L-monotone}
  For any convex loss $L : \reals \to \reals^\Y$, $\prop{L}$ is monotone.
\end{lemma}
\begin{proof}
  Let $a,b$ be defined by $a(r)_y = \partial_- L(r)_y$ and $b(r) = \partial_+ L(r)_y$, that is, the left and right derivatives of $L(\cdot)_y$ at $r$, respectively.
  Then $\partial L(r)_y = [a(r)_y,b(r)_y]$.
  We now have $r \in \prop{L}(p) \iff 0 \in \partial p\cdot L(r) \iff a(r)\cdot p \leq 0 \leq b(r) \cdot p$, showing the first condition.
  The second condition follows as the subgradients of $L$ are monotone functions.
  \raf{More detail and some Rockafellar cites}
\end{proof}

\newcommand{\Pbar}{\overline P}
\begin{lemma}\label{lem:pbar}
  Let $\gamma:\simplex\toto\R$ be a finite elicitable property, and suppose there is a calibrated link $\psi$ from an elicitable $\Gamma$ to $\gamma$.
  For each $r\in\R$, define $P_r = \bigcup_{u\in\psi^{-1}(r)} \Gamma_u \subseteq \simplex$, and let $\Pbar_r$ denote the closure of the convex hull of $P_r$.
  Then $\gamma_r = \Pbar_r$ for all $r\in\R$.
\end{lemma}
\begin{proof}
  \raf{I bet this could be shorter.  The $\Pbar$ sets are contained in level sets of $\gamma$ and union to the simplex -- so from power diagram stuff, they should be exactly the cells.}
  As $P_r \subseteq \gamma_r$ by the definition of calibration, and $\gamma_r$ is closed and convex, we must have $\Pbar_r \subseteq \gamma_r$.
  Furthermore, again by calibration of $\psi$, we must have $\bigcup_{r\in\R} P_r = \bigcup_{u\in\reals} \Gamma_u = \simplex$, and thus $\bigcup_{r\in\R} \Pbar_r = \simplex$ as well.

  Suppose for a contradiction that $\gamma_r \neq \Pbar_r$ for some $r\in\R$.
  From Lemma~\ref{lem:finite-full-dim}, $\gamma_r$ has nonempty interior, so we must have some $p\in\inter\gamma_r \setminus \Pbar_r$.
  But as $\bigcup_{r'\in\R} \Pbar_{r'} = \simplex$, we then have some $r'\neq r$ with $p\in\Pbar_{r'} \subseteq \gamma_{r'}$, a contradiction.
  \raf{Need a cite for this too.}
  Hence, for all $r\in\R$ we have $\gamma_r = \Pbar_r$.
\end{proof}

We now prove the remaining results.
%Proposition~\ref{prop:indirect-orderable}, which states the following: if convex $L : \reals \to \reals^\Y$ indirectly elicits a finite elicitable property $\gamma$, then $\gamma$ is orderable.
\begin{proof}[of Proposition~\ref{prop:indirect-orderable}]
  Let $\gamma:\simplex\toto\R$.
  From Lemma~\ref{lem:prop-L-monotone}, $\Gamma := \prop{L}$ is monotone.
  Let $\psi:\reals\to\R$ be the calibrated link from $\Gamma$ to $\gamma$.
  From Lemma~\ref{lem:pbar}, we have $\Pbar_r = \gamma_r$ for all $r\in\R$, where $\Pbar_r$ is the closure of the convex hull of $\bigcup_{u\in\psi^{-1}(r)} \Gamma_u$.

  As $\Gamma$ is monotone, we must have $a,b : \R\to\reals^\Y$ such that $\Pbar_r = \{p\in\simplex : a(r) \cdot p \leq 0 \leq b(r) \cdot p\}$.
  (Take $a(r)_y = \inf_{u\in\psi^{-1}(r)} a(u)_y$ and $b(r)_y = \sup_{u\in\psi^{-1}(r)} b(u)_y$.)
  Now taking $p_r\in\inter\gamma_r$ and picking $u_r \in \Gamma(p)$, we order $\R = \{r_1,\ldots,r_k\}$ so that $u_{r_i} < u_{r_{i+1}}$ for all $i\in\{1,\ldots,k-1\}$.
  (The $u_{r_i}$ must all be distinct, as we chose $p_r$ so that $\gamma(p_r) = \{r\}$, so $\psi(u_{r_i}) = r_i$ for all $i$.)

  Let $i\in\{1,\ldots,k-1\}$.
  By monotonicity of $\Gamma$, we must have $a(r_i) \leq b(r_i) \leq a(r_{i+1}) \leq b(r_{i+1})$.
  As $\bigcup_{r\in\R} \Pbar_r = \bigcup_{r\in\R} \gamma_r = \simplex$, we must therefore have $b(r_i) = a(r_{i+1})$.
  Finally, we conclude $\gamma_{r_i} \cap \gamma_{r_{i+1}} = \{p\in\simplex : b(r_i) \cdot p = 0\}$.
  As these statements hold for all $i\in\{1,\ldots,k-1\}$, $\gamma$ is orderable.
\end{proof}

\begin{proof}[of Theorem~\ref{thm:1d-tfae}]
  As remarked before the theorem statement, it remains only to show $1 \iff 5$.
  For the forward direction,
  \raf{Construction: same as the polyhedral construction, but linear interpolation}

  For the converse, let $\Gamma:\simplex\to\reals$ be a continuous, non-locally-constant, elicitable property.
  Then by~\cite{lambert2018elicitation,steinwart2014elicitation}, we have some $a(u)\in\reals^\Y$ for all $u\in\Gamma(\simplex)$ such that $\Gamma_u = \{p\in\simplex : a(u) \cdot p = 0\}$.

  % From Lemma~\ref{lem:pbar}, we have $\Pbar_r = \gamma_r$ for all $r\in\R$, where $\Pbar_r$ is the closure of the convex hull of $\bigcup_{u\in\psi^{-1}(r)} \Gamma_u$.
  By the continuity of $\Gamma$, for all $r\in\R$ we must have some open interval $I_r = (a_r,b_r)$ such that $\Gamma^{-1}(\inter\gamma_r) = I_r$.
  % Moreover, we must have $\psi(I_r) = \{r\}$ by calibration of $\psi$.
  \raf{I feel like we are so close here...}

\end{proof}



\end{document}
%%% Local Variables:
%%% mode: latex
%%% TeX-master: t
%%% End:
