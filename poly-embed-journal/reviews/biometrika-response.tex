\documentclass[a4paper]{article}
\usepackage[margin=1in]{geometry}
\usepackage{amsmath,amssymb}
\usepackage{xcolor}

% WHERE HAS THIS BEEN ALL MY LIFE
\usepackage{xr}
\externaldocument{elic-complex-journal}

\usepackage[colorinlistoftodos,textsize=tiny]{todonotes} % need xargs for below

\newcommand{\Comments}{1}
\definecolor{gray}{gray}{0.5}
\definecolor{lightred}{rgb}{1,0.6,0.6}
\definecolor{darkgreen}{rgb}{0,0.5,0}
\newcommand{\mynote}[2]{\ifnum\Comments=1\textcolor{#1}{#2}\fi}
\newcommand{\mytodo}[2]{\ifnum\Comments=1%
  \todo[linecolor=#1!80!black,backgroundcolor=#1,bordercolor=#1!80!black]{#2}\fi}

\newcommand{\ian}[1]{\mynote{blue}{[IK: #1]}}
\newcommand{\iant}[1]{\mytodo{blue!20!white}{IK: #1}}
\newcommand{\raf}[1]{\mynote{darkgreen}{[RF: #1]}}
\newcommand{\raft}[1]{\mytodo{green!20!white}{RF: #1}}
%\newcommand{\rafbtw}[1]{\mynote{gray}{[BTW: #1]}}
\newcommand{\btw}[1]{\mytodo{gray!20!white}{#1}}
\newcommand{\future}[1]{\mytodo{red!20!white}{\textcolor{gray!50!black}{#1}}}
\newcommand{\appendixfodder}[1]{\mynote{gray}{[APPENDIX?: #1]}}

\newcommand{\kl}{\textnormal{KL}}
\newcommand{\relint}{\textnormal{relint}}
\newcommand{\interior}{\mathrm{int}\,}
\newcommand{\tr}{\top}
\newcommand{\id}{\mathrm{id}}
\newcommand{\Var}{\mathrm{Var}}

\newcommand{\A}{\mathcal{A}}
\newcommand{\C}{\mathcal{C}}
\newcommand{\D}{\mathcal{D}}
\newcommand{\E}{\mathbb{E}}
\renewcommand{\H}{\mathcal{H}}
\newcommand{\I}{\mathcal{I}}
\newcommand{\fin}{{\mathrm{fin}}}
\newcommand{\N}{\mathbb{N}}
\renewcommand{\P}{\mathcal{P}}
\newcommand{\gaussians}{\mathcal{G}_\mathrm{mix}}
\newcommand{\Pquant}{\mathcal{P}_\mathrm{q}}
\newcommand{\Q}{\mathbb{Q}}
\newcommand{\Qc}{\mathcal{Q}}
\newcommand{\R}{\mathcal{R}}
\let\oldS\S 
\renewcommand{\S}{\mbox{\oldS\hspace{-0.5mm}}}
\newcommand{\V}{\mathcal{V}}
\newcommand{\W}{\mathcal{W}}
\newcommand{\X}{\mathcal{X}}
\newcommand{\Y}{\mathcal{Y}}

\newcommand{\convhull}{\mathrm{conv}}
\newcommand{\conv}{\convhull}
\newcommand{\ext}{\mathrm{ext}}
\newcommand{\countinf}{\infty}%\boldsymbol{\omega}}

\newcommand{\toto}{\rightrightarrows}
\newcommand{\nondiff}{\mathsf{nondiff}}
\newcommand{\lsc}{l.s.c.}
\newcommand{\dom}{\mathrm{dom}}
\newcommand{\defeq}{\doteq}
%\newcommand{\ones}{\mathbbm{1}}
\newcommand{\ones}{\mathds{1}}  % to kill type 3 fonts
\newcommand{\abs}[1]{\left\lvert #1 \right\rvert}
\newcommand{\sgn}{\mathrm{sgn}}
\newcommand{\im}{\mathop{\mathrm{im}}}
\newcommand{\spn}{\mathop{\mathrm{span}}}
\newcommand{\affspn}{\mathop{\mathrm{affinespan}}}

\def\reals{\mathbb{R}}
\def\integers{\mathbb{Z}}
\def\extreals{\mathbb{\overline{R}}}

\newcommand{\argmin}{\mathop{\mathrm{argmin}}}
\newcommand{\argmax}{\mathop{\mathrm{argmax}}}
\newcommand{\arginf}{\mathop{\mathrm{arginf}}}
\newcommand{\argsup}{\mathop{\mathrm{argsup}}}

\newcommand{\inprod}[1]{\left\langle #1 \right\rangle}

\newcommand{\elic}{\mathsf{elic}}
\newcommand{\elici}{\elic_\ID}
\newcommand{\elicifin}{\elic_{\I^\fin}}
\newcommand{\iden}{\mathsf{iden}}
\newcommand{\idprop}{\Gamma_{\id}}
\newcommand{\EL}{\mathcal{E}}
\newcommand{\ID}{\mathcal{I}}
\newcommand{\ES}{\mathrm{ES}}
\newcommand{\lbar}{\underline{L}}
\newcommand{\vspan}{\mathrm{span}}
\newcommand{\affhull}{\mathrm{affhull}}
\newcommand{\affdim}{\mathrm{affdim}}
\newcommand{\codim}{\mathrm{codim}}


%%% Local Variables:
%%% mode: latex
%%% TeX-master: "elic-complex-journal"
%%% End:


\usepackage[colorinlistoftodos,textsize=tiny]{todonotes} % need xargs for below

%\newcommand{\Comments}{1}
%\definecolor{gray}{gray}{0.5}
%\definecolor{lightred}{rgb}{1,0.6,0.6}
%\definecolor{darkgreen}{rgb}{0,0.5,0}
%\newcommand{\mynote}[2]{\ifnum\Comments=1\textcolor{#1}{#2}\fi}
%\newcommand{\mytodo}[2]{\ifnum\Comments=1\todo[linecolor=#1!80!black,backgroundcolor=#1,bordercolor=#1!80!black]{#2}\fi}
%\newcommand{\ian}[1]{\mynote{blue}{[IK: #1]}}
%\newcommand{\iant}[1]{\mytodo{blue!20!white}{IK: #1}}
%\newcommand{\raf}[1]{\mynote{darkgreen}{[RF: #1]}}
%\newcommand{\raft}[1]{\mytodo{green!20!white}{RF: #1}}

%\newcommand{\response}[1]{\textcolor{blue!50!black}{Response: #1}}
\newenvironment{response}{\color{blue!50!black}}{\color{black}}

\begin{document}

\begin{center}
{\Large Response to Reviews}
\end{center}

We thank the editor and reviewers for their feedback.  While the instructions said only a brief response was needed, we have elected to provide a full letter because we discovered we needed a stronger condition than had actually been assumed.  Essentially, the error was that in the proofs of Lemma 8 and Corollary 7 it was mistakenly assumed that the span of a level set of an identifiable property and the kernel of the identification function for that level set should have the same codimension, when actually the codimension of the former could be unboundedly larger than that of the latter.  A way to avoid this, which we have adopted, is to assume that 0 is in the interior of the image of the identification function, an assumption inspired by Assumption V1 of Fissler and Ziegel (2016) and similar in spirit to our existing conditions.

We first discuss the changes needed to implement this correction, and then address the changes to address other comments, all of which were minor.\\
\\
Regards,\\
Rafael Frongillo and Ian Kash

\subsection*{Changes to Address the Issue}

\begin{enumerate}

\item A new condition, Condition 1, which is inspired by Assumption V1 of Fissler and Ziegel (2016), appears in Section 2.4.  As this replaces the role played by the former Conditions 3 and 4 they have been removed.

\item The former Lemma 8 has been replaced with Lemma 1, which has been promoted to the main text and appears in Section 4.1.  It captures how Condition 1 is used to provide a lower bound of $k$.

\item Corollary 7 has been revised to apply Condition 1 rather than the former Conditions 3 and 4, to get the tighter bound of $k+1$.

\item The supplement now contains two new technical lemmas, Lemma 10 and Lemma 14, encapsulating the relevant facts of linear algebra.  They are used in the new proofs of Lemma 1 and Corollary 7.

\item The assumption that Condition 1 holds has been propagated as needed, most notably to Proposition 3, but also to Corollaries 1, 4, and 5 and Propositions 7 and 8.  The condition is checked for particular properties and families of distributions in the proofs of Corollary 1 (omitted), Corollary 3, Corollary 4, Corollary 5,  Lemma 5, and Lemma 6.  The only place where checking this was non-trivial is when the property is an expectile and the set of distributions contains all mixtures of Gaussians.  See Statement 2 in B.4.
\end{enumerate}


\subsection*{Detailed response to Review 1}

\begin{enumerate}
\item After Definition 2: to clarify the concept of weak elicitation, it might be good to mention
that a constant function L elicits any property , and hence it is meaningless to talk about
weak elicitability (at least in the context of this paper). The reason is that elicitability
(i.e., elicitation without specifying the loss) is an important notion throughout, and the reader
should be warned that weak elicitability is nothing similar. (Of course, I acknowledge that
weak elicitation, i.e., specifying the loss, is an interesting notion.)

The paragraph after the definition now makes this point explicit.

\item Some section titles are not consistently capitalized: compare subsection titles in Sections 4-5 and the others.

Section title capitalization should now be consistent.

\item I noticed that equation numbers in the appendices are abused. For instance, (1) appears for
three different equations in the main text and in the appendices. I suggest to use (B.1) and
(D.1) for the one in the appendices.

This reset appears to be part of the Biometrika class, so we have left it as is.

\item p24, l34: it seems that there is an extra ) at the end of the equation.

Fixed.

\item The authors mentioned that they proposed the new quantity variantile. I needed to say that
I recently got to know that this quantity was used earlier by some other researchers under the
term variancile (arguably also a natural choice of terminology). The authors can search the
word variancile on Google, and it should appear. However, as I checked today (Feb 2020),
it seems that there is no reliable source to cite for this concept (I guess the authors will find
the same). I am fine if the authors choose to leave it as it is, but I feel that to be scientifically
correct, I needed to let the authors know of this.

Thank you for sharing this information.  We now acknowledge this unpublished work when introducing the variantile.

\end{enumerate}

\subsection*{Detailed response to Associate Editor}

\begin{enumerate}

\item (i) brackets should be [\{(..)\}] not (((..)))
(ii) no italicised text; try and avoid text in brackets and abbreviations (e.g. I think the paper would read better without the abbreviation ERM)
(iii) citations should be author (year)
(iv) no footnotes
(v) If you use subsections within a section there should be no text before the first subsection (e.g. Section 2, 3 etc.) You do not need any overview text at the start of a section.
(vi) the Discussion section cannot have any recap of the work in the paper.

All of these have been changed to conform to the journal style.  We eliminated most parentheticals and abbreviations, including ERM.  However, we did keep a small number of abbreviations where we believed the concept particularly well known by the abbreviation, for example CVaR and SVM and it would be surprising to readers familiar with the concept not to include it.  We also kept using $\{\}$ when defining a set, which appears to be consistent with prior papers published in the journal.  Finally, we dropped the notation for properties using $\{\}$ when the order did not matter, since it became ambiguous and confusing in some cases.  Instead we now have a comment at the start of 4.2 that the order is not relevant.

\item The paper, and in particular the discussion, is very long compared to the norm for the journal. The revised paper should be at most 20 pages excluding the online supplement (which would include the appendices). This should be possible by tightening the text in places. Please separate out the main paper and the supplement in the next revision.

The revised paper is now 20 pages long, through a combination of text tightening and moving selected material to the supplement.

\end{enumerate}

\end{document}


%%% Local Variables:
%%% mode: latex
%%% TeX-master: t
%%% End:
