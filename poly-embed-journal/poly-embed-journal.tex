\documentclass[11pt]{article}
\usepackage[numbers, compress, sort]{natbib}
\usepackage[margin=1.1in]{geometry}

\usepackage{float}
\usepackage[utf8]{inputenc} % allow utf-8 input
\usepackage[T1]{fontenc}    % use 8-bit T1 fonts
\usepackage{lmodern}
\usepackage{hyperref}       % hyperlinks  %[implicit=false, bookmarks=false]
\usepackage{url}            % simple URL typesetting
\usepackage{booktabs}       % professional-quality tables
\usepackage{amsfonts}       % blackboard math symbols
\usepackage{nicefrac}       % compact symbols for 1/2, etc.
\usepackage{microtype}      % microtypography
\usepackage[normalem]{ulem}
\usepackage{thm-restate}
\usepackage{array}
\newcolumntype{R}{>{$}r<{$}} % math-mode version of "l" column type
\newcolumntype{C}{>{$}c<{$}} % math-mode version of "c" column type
\newcommand\numberthis{\addtocounter{equation}{1}\tag{\theequation}}

\usepackage{mathtools, amsmath, amssymb, graphicx, verbatim, amsthm}
%\usepackage[thmmarks, thref, amsthm]{ntheorem}
\usepackage{color}
\definecolor{darkblue}{rgb}{0.0,0.0,0.2}
\definecolor{darkgreen}{rgb}{0.0,0.3,0.0}
\hypersetup{colorlinks,breaklinks,
            linkcolor=darkblue,urlcolor=darkblue,
            anchorcolor=darkblue,citecolor=darkblue}
\usepackage{wrapfig}
\usepackage[font=small]{caption}
\usepackage{subcaption}
\usepackage[colorinlistoftodos,textsize=tiny]{todonotes} % need xargs for below
%\usepackage{accents}
\usepackage{bbm}
\usepackage{xspace}

\usetikzlibrary{calc}
\newcommand{\Comments}{1}
\newcommand{\mynote}[2]{\ifnum\Comments=1\textcolor{#1}{#2}\fi}
\newcommand{\mytodo}[2]{\ifnum\Comments=1%
  \todo[linecolor=#1!80!black,backgroundcolor=#1,bordercolor=#1!80!black]{#2}\fi}
\newcommand{\raf}[1]{\mynote{darkgreen}{[RF: #1]}}
\newcommand{\raft}[1]{\mytodo{green!20!white}{RF: #1}}
\newcommand{\jessie}[1]{\mynote{teal}{[JF: #1]}}
\newcommand{\jessiet}[1]{\mytodo{teal!20!white}{JF: #1}}
\newcommand{\proposedadd}[1]{\mynote{orange}{#1}}
\newcommand{\bo}[1]{\mynote{blue}{[Bo: #1]}}
\newcommand{\botodo}[1]{\mytodo{blue!20!white}{[Bo: #1]}}
\newcommand{\btw}[1]{\mytodo{gray!20!white}{BTW: #1}}%TURN OFF FOR NOW \mytodo{gray}{#1}}
\ifnum\Comments=1               % fix margins for todonotes
  \setlength{\marginparwidth}{1in}
\fi


\newcommand{\reals}{\mathbb{R}}
\newcommand{\posreals}{\reals_{>0}}%{\reals_{++}}
\newcommand{\dom}{\mathrm{dom}}

\newcommand{\prop}[1]{\mathrm{prop}[#1]}
\newcommand{\eliccts}{\mathrm{elic}_\mathrm{cts}}
\newcommand{\eliccvx}{\mathrm{elic}_\mathrm{cvx}}
\newcommand{\elicpoly}{\mathrm{elic}_\mathrm{pcvx}}
\newcommand{\elicembed}{\mathrm{elic}_\mathrm{embed}}
\newcommand{\affhull}{\mathrm{affhull}}
\newcommand{\card}{\mathrm{card}}

\newcommand{\cell}{\mathrm{cell}}

\newcommand{\abstain}[1]{\mathrm{abstain}_{#1}}
\newcommand{\mode}{\mathrm{mode}}

\newcommand{\simplex}{\Delta_\Y}

% alphabetical order, by convention
\newcommand{\A}{\mathcal{A}}
\newcommand{\C}{\mathcal{C}}
\newcommand{\D}{\mathcal{D}}
\newcommand{\E}{\mathbb{E}}
\newcommand{\F}{\mathcal{F}}
\renewcommand{\H}{\mathcal{H}}
\newcommand{\I}{\mathcal{I}}
\renewcommand{\L}{\mathcal{L}}
\newcommand{\N}{\mathcal{N}}
\newcommand{\OP}{\mathcal{OP}}
\newcommand{\R}{\mathcal{R}}
\newcommand{\Sc}{\mathcal{S}}
\newcommand{\U}{\mathcal{U}}
\newcommand{\V}{\mathcal{V}}
\newcommand{\X}{\mathcal{X}}
\newcommand{\Y}{\mathcal{Y}}


\newcommand{\risk}[1]{\underline{#1}}
\newcommand{\inprod}[2]{\langle #1, #2 \rangle}%\mathrm{int}(#1)}
\newcommand{\relint}{\mathrm{relint}}
%\newcommand{\inter}[1]{\mathring{#1}}%\mathrm{int}(#1)}
\newcommand{\inter}{\mathrm{inter}}
%\newcommand{\expectedv}[3]{\overline{#1}(#2,#3)}
\newcommand{\expectedv}[3]{\E_{Y\sim{#3}} {#1}(#2,Y)}
\newcommand{\toto}{\rightrightarrows}
\newcommand{\strip}{\mathrm{strip}}
\newcommand{\trim}{\mathrm{trim}}
\newcommand{\red}{\mathrm{red}}
\newcommand{\trimred}{\mathrm{trim}_\red}
\newcommand{\trimcover}{\mathrm{trim}}
\newcommand{\fplc}{finite-piecewise-linear and convex\xspace} %xspace for use in text
\newcommand{\conv}{\mathrm{conv}}
\newcommand{\epi}{\mathrm{epi}}
\newcommand{\indopp}{\bar{\mathbbm{1}}}
\newcommand{\ones}{\mathbbm{1}}
\DeclarePairedDelimiter\ceil{\lceil}{\rceil}

\newcommand{\Ind}[1]{\ones\{#1\}}

\newcommand{\hinge}{L_{\mathrm{hinge}}}
\newcommand{\ellzo}{\ell_{\text{0-1}}}
\newcommand{\ellabs}[1]{\ell_{#1}}
\newcommand{\elltopk}{\ell^{\text{top-$k$}}}
\newcommand{\elltop}[1]{\ell^{\text{top-$#1$}}}
\newcommand{\emb}{{\tt e}}

\DeclareMathOperator*{\argmax}{arg\,max}
\DeclareMathOperator*{\argmin}{arg\,min}
\DeclareMathOperator*{\sgn}{sgn}

\newtheorem{theorem}{Theorem}
\newtheorem{lemma}{Lemma}
\newtheorem{proposition}{Proposition}
\newtheorem{corollary}{Corollary}
\newtheorem{conjecture}{Conjecture}
\newtheorem{claim}{Claim}

\newtheorem{definition}{Definition}
\newtheorem{remark}{Remark}
\newtheorem{assumption}{Assumption}


\title{An Embedding Framework for Consistent Polyhedral Surrogates}
%\title{Consistent Polyhedral Surrogates via Embeddings}
%\title{Convex Surrogates via Polyhedral Losses}
\author{%
 Jessie Finocchiaro \\
 \texttt{jefi8453@colorado.edu}\\
 CU Boulder
 \and
 Rafael Frongillo\\
 \texttt{raf@colorado.edu}\\
 CU Boulder
 \and
 Bo Waggoner\\
 \texttt{bwag@colorado.edu}\\
 CU Boulder
}

\begin{document}

\maketitle

\begin{abstract}
We formalize and study the natural approach of designing convex surrogate loss functions via embeddings for problems such as classification, ranking, or structured prediction. 
In this approach, one embeds each of the finitely many predictions (e.g. classes) as a point in $\reals^d$, assigns the original loss values to these points, and ``convexifies'' the loss in some way to obtain a surrogate.
We prove that this approach is equivalent, in a strong sense, to working with polyhedral (piecewise-linear convex) losses.
Moreover, given any polyhedral loss $L$, we give a construction of a link function through which $L$ is a consistent surrogate for the loss it embeds.
Our framework yields succinct proofs of consistency or inconsistency of various polyhedral surrogates in the literature, and for further reveals the discrete problems that inconsistent surrogates are consistent for.
We also show \raf{Newer stuff}
\end{abstract}

%\begin{keywords}%
%  property elicitation, proper scoring rules, surrogate loss functions, embeddings
%\end{keywords}

\section{Introduction}\label{sec:intro}

%\raf{MOVED FROM SEC 2; maybe incorporate?}
%Often, for computational or other reasons, one is interested in a property $\gamma$ such as the mode, but wishes to elicit it by minimizing some other ``surrogate'' loss $L$ over some other space such as $\reals^d$, then mapping the result back to $\R$ using a link function.
%We now formalize this procedure.
%
%\jessie{ TODO Items 01 Mar 2021
%	\begin{itemize}
%		\item \sout{Reorg sections: \S 3 (Embeddings, matching risks, etc.), \S 5 (Calibrated link), Examples, \S 4 (Polyhedral losses: I.E. implies consistency) OR Main results (w/o proof), Examples, \S 3, \S 5, \S 4}
%		\item \sout{Flesh out and expand on Ordered Partition Example}
%		\item add paragraph of discussion on the requirements of polyhedral losses
%		\item Check proof and statement of Theorem~\ref{thm:poly-ie-implies-consistent} - Jessie, 17 Mar 21: took a look to brush up; Theorem~\ref{lem:ie-iff-embeds-refinement} can use a look too.
%		\item Try to make section~\ref{sec:poly-loss-embed} self-contained, moving up results from appendix 
%		\begin{itemize}
%		\item \sout{for now, move everything up except PDs}
%		\item see if you can isolate what exactly we need from PDs and state it as a new lemma (which is just about Bayes risks) \raf{*}
%		\item rethink proofs of Prop 1 and 2, see if we resolve the redundancy  \raf{*}
%		\end{itemize}
%		\item Make Cor 1 a Theorem: "For L polyhedral, ell discrete, L ind elic prop[ell] <==> L consistent wrt ell"
%\end{itemize}
%TODO Items summer 2020
%	\begin{itemize}
%		\item Formalize reverse implication proof of Proposition 1 with projections into the affine hull and back
%		\item Mention how our focus on polyhedral risks and general links extends DKR Section 3.1 (probably after Prop 1)
%		\item Replace Proposition 2 with Lemma 7?
%		\item Theorem~\ref{thm:discrete-loss-poly-embeddable}: move from n-dimensional construction to n-1 dimensional construction (or add n-1 dim construction in appendix if it gets too hairy?)
%	\end{itemize}}

In supervised learning, one tries to learn a hypothesis which fits some labeled data, as judged by a target loss function.
Unfortunately, minimizing the target loss directly is typically computationally intractable, especially for discrete prediction tasks like classification, ranking, and structured prediction.
Instead, one typically minimizes a surrogate loss which is convex and therefore efficiently minimized.
Given a surrogate hypothesis, a link function then translates back to the target problem.
This general approach, called surrogate risk minimization, is ubiquitous in supervised machine learning algorithms.
\botodo{This is pretty inaccessible to nonexperts. I think we should consider a rewrite that expands this intro and makes it much more accessible. For example what is a surrogate? It would help many readers a lot to spell things out more.}
\raft{I added this paragraph (based on our rates paper) and then transitioned into the next}

% Convex surrogate losses are a central building block in machine learning for finite prediction problems such as classification and structured prediction tasks.
A growing body of work seeks to design and analyze convex surrogates for particular target loss functions, and more broadly, understand the best empirical risk minimization bounds that can be found for a surrogate, for which consistency is a necessary condition.
For example, recent work has developed tools to bound the prediction dimension of the surrogate, meaning the dimension of the range of the surrogate hypothesis~\cite{frongillo2015elicitation,  ramaswamy2016convex}.
Yet in some cases these bounds are far from tight, such as for \emph{abstain loss} (classification with an abstain option)~\citep{bartlett2008classification,yuan2010classification,ramaswamy2016convex,ramaswamy2018consistent,zhang2018reject}.
Furthermore, the kinds of strategies available for constructing surrogates, and their relative power, are not well understood.

We augment this literature by studying a particularly natural approach for finding convex surrogates, wherein one ``embeds'' a discrete loss.
Specifically, we say a convex surrogate $L$ embeds a discrete loss $\ell$ if there is an injective embedding from the discrete reports (predictions) to a vector space such that (i) the original loss values are recovered, and (ii) a report is $\ell$-optimal if and only if the embedded report is $L$-optimal.
If this embedding can be extended to a calibrated link function, which roughly maps approximately $L$-optimal reports to $\ell$-optimal reports, then consistency follows~\citep{agarwal2015consistent}.
Common examples of this general construction include hinge loss as a surrogate for 0-1 loss and the abstain surrogate mentioned above~\citep{ramaswamy2018consistent}.


% OLD INTRO PARAGRAPH:
% Using tools from property elicitation, we show a tight relationship between such embeddings and the class of polyhedral (piecewise-linear convex) loss functions.
% In particular, by focusing on Bayes risks, we show that every discrete loss is embedded by some polyhedral loss, and every polyhedral loss function embeds some discrete loss.
% Moreover, we show that any polyhedral loss gives rise to a calibrated link function to the loss it embeds,
% thus giving a very general framework to construct consistent convex surrogates for arbitrary losses.


We prove that such an embedding scheme is intimately related to the class of polyhedral (piecewise-linear and convex) loss functions.
In particular, every discrete loss is embedded by a polyhedral surrogate.
Moreover, such an embedding gives rise to calibrated link function.
Our proofs give explicit constructions for the surrogate (\S~\ref{sec:poly-loss-embed}) and link (\S~\ref{sec:calibration}) given a discrete loss.

\begin{restatable}{theorem}{embedpolyinformal}\label{thm:embed-poly-informal}
  Every discrete loss $\ell$ is embedded by some polyhedral loss $L$, and every polyhedral loss $L$ embeds some discrete loss $\ell$.
\end{restatable}

\begin{restatable}{theorem}{linkinformal}\label{thm:link-informal}
  Given any polyhedral loss $L$, let $\ell$ be the discrete loss it embeds. There exists a link function $\psi$ such that $(L,\psi)$ is calibrated with respect to $\ell$.
\end{restatable}

To better understand existing polyhedral surrogates, we provide tools to find the discrete losses they embed (Proposition~\ref{prop:representative-embeds-restriction}).
In short, if one can identify a finite \emph{representative set} $\Sc$ of reports for a surrogate $L$, meaning $\Sc$ always contains an $L$-optimal report for any label distribution, then $L$ embeds $L|_\Sc$, the loss given by $L$ restricting to $\Sc$.
We apply these tools to several polyhedral surrogates which have been proposed recently (\S~\ref{sec:applications}), showing in some cases that they are inconsistent for the desired target, but are consistent for a different target: the discrete loss they embed.

Underpinning our results are several observations which formalize the idea that polyhedral losses ``behave like'' discrete losses.
For example, discrete losses have polyhedral Bayes risks (as the minimum of finitely many linear functions), as do polyhedral losses (Lemma~\ref{lem:polyhedral-range-gamma}).
As a consequence, polyhedral losses always have finite representative sets, and restricting the loss to any such set is an embedding.

We also provide several observations beyond what is needed to prove our main results, which we view as conceptual contributions (\S~\ref{sec:min-rep-sets},~\ref{sec:poly-ie-consistency}).
Using tools from property elicitation, we show an equivalence between minumum reprosentative sets and ``non-redundancy'', wherein no report is dominated by another.
We further show that, while the minimum representative set is not always unique, the loss values associated with it are unique, giving rise to a natural ``trim'' operation on losses.
Finally, using our main results, we show the following result: when restricting to the class of polyhedral surrogates, indirect elicitation is both necessary and sufficient for consistency (Theorem~\ref{thm:poly-ie-implies-consistent}).

Taken together, we view our contribution as both conceptual and practical.
We uncover the remarkable structure of polyhedral surrogates, deepening our understanding of the relationship between surrogate and discrete target losses.
This structure leads to a powerful new framework to design and analyze surrogate losses, which we apply to several examples.
We hope our framework will inspire new research, and we conclude with several exciting directions for future work.

% \jessie{Take out this paragraph?}
%We go on to initiate a study of \emph{embedding dimension}: the minimal dimension for a given discrete loss to be so embedded.
%Our results give a complete characterization in the case of dimension 1; that is, real-valued surrogates.
%We find that a discrete loss can be embedded in the real line if and only if it is order-sensitive, meaning there is an ordering of the predictions such that the loss increases as one moves away from the correct prediction.
%We show further that order-sensitivity characterizes the existence of \emph{any} real-valued convex surrogate, implying that embeddings lose no generality.
%We conclude with preliminary results in higher dimensions and other future directions.

\paragraph{Related works.}
The literature on convex surrogates focuses mainly on smooth surrogate losses~\citep{crammer2001algorithmic,bartlett2006convexity,bartlett2008classification, duchi2018multiclass, williamson2016composite, reid2010composite,menon2019multilabel,zhang2020convex,bao2020calibrated}.
Nevertheless, nonsmooth losses, such as the polyhedral losses we consider, have been proposed and studied for a variety of classification-like problems~\citep{yang2018consistency,yu2018lovasz,lapin2015top}.
Moreover,~\citet{zhang2020bayes} describe the impact of the hypothesis class has on consistency, and when consistency relative to the hypothesis class differs from Bayes consistency; the latter is what we describe in this paper when we say ``consistency.''

A notable addition to this literature is~\citet{ramaswamy2018consistent}, who argue that nonsmooth losses may enable dimension reduction of the prediction space (range of the surrogate hypothesis) relative to smooth losses, illustrating this conjecture with a surrogate for \emph{abstain loss} needing only $\log(n)$ dimensions for $n$ labels, whereas the best known smooth loss needs $n-1$ dimensions.
Their surrogate is a natural example of an embedding (cf.~\S~\ref{sec:abstain}), and serves as inspiration for our work.


While property elicitation has by now an extensive literature~\citep{savage1971elicitation,osband1985information-eliciting,lambert2008eliciting,gneiting2011making,steinwart2014elicitation,frongillo2015vector-valued,fissler2016higher,lambert2018elicitation}, these works are mostly concerned with point estimation problems.
Literature directly connecting property elicitation to consistency is sparse.
However,~\citet{agarwal2015consistent} consider single-valued properties in finite outcome settings, whereas finite properties elicited by general convex losses are necessarily set-valued.
\citet{finocchiaro2021unifying} additionally relates indirect property elicitation to consistency when one is given either a target loss or property in both discrete and continuous prediction settings, assuming surrogates attain their infimum in expectation over all distributions over the outcomes.

% \citet{agarwal2015consistent} and~\citet{ramaswamy2016convex} study the existence of convex surrogates; the latter proves lower bounds on dimensionality of hypothesis spaces for e.g.~0-1 loss via their notion of feasible subspace dimension.
%   These works are generally not directly comparable to our results, as they do not consider the embedding method.
%   However, we will mention relevant results inline.


\section{Setting}
\label{sec:setting}

For discrete prediction problems like classification, due to hardness of directly optimizing a given discrete loss, many machine learning algorithms minimize a surrogate loss function with better optimization qualities, e.g., convexity.
Of course, to show that this surrogate loss successfully addresses the original problem, one needs to establish consistency, which depends crucially on the choice of link function that maps surrogate reports (predictions) to original reports.
After introducing notation, and terminology from property elicitation, we study \emph{calibration} (Definition~\ref{def:calibrated}), which is equivalent to consistency in finite outcome settings~\citep{bartlett2006convexity,tewari2007consistency,ramaswamy2016convex} and depends solely on the conditional distribution over $\Y$.
Consistency is a prerequisite to obtain empirical risk minimization bounds, motivating our firm requirement of constructing consistent surrogates.
\jessiet{Should this last sentence move up a bit?}

\subsection{Notation and Losses}
\label{sec:notation-losses}

Let $\Y$ be a finite label space, and throughout let $n=|\Y|$.
The set of probability distributions on $\Y$ is denoted $\simplex\subseteq\reals^{\Y}_+$, represented as vectors of probabilities (requiring $\|p\|_1 = 1$).
We write $p_y$ for the probability of outcome $y \in \Y$ drawn from $p \in \simplex$.
While supervised algorithms learn hypothesis functions $h:\X \to \R$ that minimize expected loss over data representations and labels, one can abstract away the hypothesis function for simplicity and reason about generic reports.
We first discuss this conditional setting, with just labels $\Y$ and no features $\X$, and show in \S~\ref{subsec:calibration-links} how these notions relate to the usual $\X\times\Y$ setting considering features.
\botodo{I don't expect the previous sentence to make sense to most readers. Suggest we add some explanation or detail -- or defer this comment to a bit later.}
\jessiet{Better?}


\bo{Suggest: ``...loss function $L:\R\to\reals^\Y_+$...'' then transition to target and surrogate}
\raf{Moved: `` which maps a report (prediction) $r$ from a finite set $\R$ to the vector of loss values $L(r) = (L(r)_y)_{y\in\Y}$ for each possible outcome $y\in\Y$''}
We write the corresponding expected loss when $Y \sim p$ as $\inprod{p}{\ell(r)}$ and $\inprod{p}{L(u)}$.
The \emph{Bayes risk} of a loss $L:\reals^d\to\reals^\Y_+$ is the function $\risk{L}:\simplex\to\reals_+$ given by $\risk{L}(p) := \inf_{u\in\reals^d} \inprod{p}{L(u)}$; naturally for discrete losses we write $\risk{\ell}$ with the infimum over $\R$.
When restricting the domain of a loss $L$ from $\R$ to $\R'$, we write $L|_{\R'}$.

We assume that a given discrete prediction problem, such as classification, is given in the form of a discrete \emph{target loss}, which we denote $\ell:\R\to\reals^\Y_+$, where we assume $\R$ is a finite set.
%We will assume throughout that the given discrete loss is \emph{non-redundant}, meaning every report is uniquely optimal (minimizes expected loss) for some distribution $p\in\simplex$.
Similarly, surrogate losses will be written $L:\reals^d\to\reals^\Y_+$, typically with reports written $u\in\reals^d$.
%\btw{Cut generic loss -- from neurips 2019 lol}%(A generic loss will be written $L:\R\to\reals^\Y$.)

%\jessie{In general, we shall denote discrete reports by $r \in \R$, and real-valued reports by $u \in \U$.}
For example, 0-1 loss is a discrete loss with $\R = \Y = \{-1,1\}$
given by $\ellzo(r)_y = \Ind{r \neq y}$, with Bayes risk $\risk{\ellzo}(p) = 1-\max_{y\in\Y} p_y$.
Two important surrogates for $\ellzo$ are hinge loss $\hinge(u)_y = (1-yu)_+$, where $(x)_+ = \max(x,0)$, and logistic loss $L(u)_y = \log(1+\exp(-yu))$ for $u\in\reals$.
See Figure~\ref{fig:bayes-risks-01} for a visualization of the Bayes risks of 0-1, Hinge, and Logistic losses, respectively.

Most of the surrogates $L$ we consider will be \emph{polyhedral}, meaning piecewise linear and convex; we therefore briefly recall the relevant definitions.
In $\reals^d$, a \emph{polyhedral set} or \emph{polyhedron} is the intersection of a finite number of closed halfspaces.
A \emph{polytope} is a bounded polyhedral set.
A convex function $f:\reals^d\to\reals$ is \emph{polyhedral} if its epigraph is polyhedral, or equivalently, if it can be written as a pointwise maximum of a finite set of affine functions~\citep{rockafellar1997convex}.
%
\begin{definition}[Polyhedral loss]
  A loss $L: \reals^d \to \reals^{\Y}_+$ is \emph{polyhedral} if $L(u)_y$ is a polyhedral convex function of $u$ for each $y\in\Y$.
\end{definition}
%
For example, hinge loss is polyhedral, whereas logistic loss is not.
To motivate our focus on polyhedral losses, we echo~\citet[Section~1.2]{ramaswamy2018consistent}, who note that smooth surrogates often encode much more information than necessary, and in these cases non-smooth surrogates are the best candidates to achieve a low prediction dimension $d$.

\subsection{Property Elicitation}
\label{sec:property-elicitation}

To make headway, we will appeal to concepts and results from the property elicitation literature, which elevates the \emph{property}, or map from distributions to optimal reports, as a central object to study in its own right.
In our case, this map will often be multivalued, meaning a single distribution could yield multiple optimal reports.
(For example, when $p=(1/2,1/2)$, both $r=1$ and $r=-1$ optimize 0-1 loss.)
To this end, we will use double arrow notation to mean a mapping to all nonempty subsets, so that $\Gamma: \simplex \toto \R$ is shorthand for $\Gamma: \simplex \to 2^{\R} \setminus \{\emptyset\}$.
%See the discussion following Definition~\ref{def:elicits} for notational conventions.% regarding $\R$, $\Gamma$, $\gamma$, $L$, $\ell$, etc.

\begin{definition}[Property, level set]\label{def:property}
  A \emph{property} is a function $\Gamma:\simplex\toto\R$.
  The \emph{level set} of $\Gamma$ for report $r$ is the set $\Gamma_r := \{p \in \simplex : r \in \Gamma(p)\}$.
\end{definition}

Intuitively, $\Gamma(p)$ is the set of reports which should be optimal for a given distribution $p$, and $\Gamma_r$ is the set of distributions for which the report $r$ should be optimal.
By optimal, we mean minimizing an associated loss function in expectation over $p$, which we formalize shortly.
Note that our definitions align such that discrete losses elicit finite properties (those with finite range). %; both are non-redundant in the correct senses.
For example, the \emph{mode} is the %finite
property $\mode(p) = \argmax_{y\in\Y} p_y$, and captures the set of optimal reports for 0-1 loss: for each distribution over the labels, one should report the most likely label.
In this case we say 0-1 loss \emph{elicits} the mode, as we formalize below.
% This terminology comes from the information elicitation \jessiet{information vs property... do we care?}\raft{we could also say economics literature; not ``property'' though since that doesn't offer an explanation of the word ``elicits''} literature~\citep{savage1971elicitation,osband1985information-eliciting,lambert2008eliciting}, in which a report $r$ is elicited from some forecaster by scoring her with a loss on the observed outcome $y$.

\begin{definition}[Elicits]
  \label{def:elicits}
  A loss $L:\R\to\reals^\Y_+$, \emph{elicits} a property $\Gamma:\simplex \toto \R$ if
  \begin{equation}
    \forall p\in\simplex,\;\;\;\Gamma(p) = \argmin_{r \in \R} \inprod{p}{L(r)}~.
  \end{equation}
  As $L$ elicits a unique property, we write $\prop{L}$ to refer to the property elicited by a loss $L$.
\end{definition}

For finite properties (those with $|\R|<\infty$) and discrete losses, we will use lowercase notation $\gamma$ and $\ell$, respectively, with reports $r\in\R$; for surrogate properties and losses we use $\Gamma$ and $L$, with reports $u\in\reals^d$.
% For finite properties (those with $|\R|<\infty$) and discrete losses, we will use lowercase notation $\gamma:\simplex\toto\R$ and $\ell:\R\toto\reals^\Y_+$, respectively, with reports $r\in\R$; for surrogate properties and losses we use $\Gamma:\simplex\toto\reals^d$ and $L:\reals^d\toto\reals^\Y_+$, with reports $u\in\reals^d$.
For general properties and losses, we will also use $\Gamma$ and $L$, as above.

% Often we will work with two properties simultaneously, a finite property $\gamma:\simplex\toto\R$, and its embedded version $\Gamma:\simplex\toto\reals^d$.
% In this case, we write the embedded reports as $u\in\reals^d$, reserving $r\in\R$ for the finite property.

\subsection{Calibration and Links}
\label{subsec:calibration-links}


To assess whether a surrogate and link function align with the original loss, we turn to the common condition of \emph{calibration}.
Roughly, a surrogate and link are calibrated if the best possible expected loss achieved by linking to an incorrect report is strictly suboptimal, which require that the excess loss of some report is bounded by (a constant times) the excess loss of the linked report.

% \begin{definition}[Consistency]\label{def:consistency}
%   A surrogate $L:\reals^d\to\reals^\Y_+$ and link $\psi:\reals^d\to\R$ is \emph{consistent} with a discrete loss $\ell:\R\to\reals^\Y_+$ if \raf{statement about distributions on $\X\times\Y$, limits of hypotheses, and Bayes opt hypotheses}
% \end{definition}

% \begin{definition}[Separated Link]\label{def:links}
%   Let discrete loss $\ell:\R\to\reals^\Y_+$ and surrogate $L:\reals^d\to\reals^\Y_+$ be given.
%   A \emph{link function} is a map $\psi:\reals^d\to\R$.
%   We say that a link $\psi$ is \emph{$\delta$-separated} for some $\delta > 0$ if for all $p \in \simplex$ and $u\in\reals^d$, we have
%   \begin{align*}
%     \inprod{p}{L(u)} - \inf_{u' \in \reals^d} \inprod{p}{L(u')} \geq \delta\left(\inprod{p}{\ell(\psi(u))} - \min_{r \in \R} \inprod{p}{\ell(r)}\right)~.
%   \end{align*}
%   A link is \emph{separated} if it is $\delta$-separated for some $\delta>0$.
% \end{definition}

\begin{definition}
  \label{def:calibrated}
  Let discrete loss $\ell:\R\to\reals^\Y_+$, proposed surrogate $L:\reals^d\to\reals^\Y_+$, and link function $\psi:\reals^d\to\R$ be given.
  We say $(L,\psi)$ is \emph{calibrated} with respect to $\ell$ if
for all $p \in \simplex$,
  \begin{equation}
    \label{eq:calibrated}
  \inf_{u \in \reals^d : \psi(u) \not\in \gamma(p)} \inprod{p}{L(u)} > \inf_{u \in \reals^d} \inprod{p}{L(u)}~.
  \end{equation}
  If $(L, \psi)$ is calibrated with respect to $\ell$, we call $\psi$ a \emph{calibrated link.}
\end{definition}
It is well-known in finite-outcome settings that calibration is equivalent to \emph{consistency}, in the following sense (cf.~\citep{bartlett2006convexity,zhang2004statistical,agarwal2015consistent}).
Suppose we have the feature space $\X$ and label space $\Y$.
Let $L^*$ be the best possible expected $L$-loss achieved by any hypothesis $H:\X\to\reals^d$, and $\ell^*$ the best expected $\ell$-loss for any hypothesis $h:\X\to\R$, respectively.
If $(L,\psi)$ is consistent, then for any sequence of surrogate hypotheses $H_1,H_2,\ldots$ whose $L$-loss limits to $L^*$, the $\ell$-loss of $\psi\circ H_1,\psi \circ H_2, \ldots$ limits to $\ell^*$ for all data distributions $D \in \Delta(\X \times \Y)$.
\bo{edited above slightly. I believe a technical definition of consistency would have a ``for all $\D$'', but we can probably skip that.}\jessie{Tried to make sure it had this technical correctness.}
As Definition~\ref{def:calibrated} does not involve the feature space $\X$, we will drop it for the remainder of the paper.
Note that in the finite-outcome setting, calibration is necessary and sufficient for consistency from a generalization of~\citet{tewari2007consistency} given by~\citet{ramaswamy2016convex}.  


\subsection{Embedding}

%Several consistent convex surrogates in the literature can be thought of as \emph{embeddings} of the target loss $\ell$, wherein one maps the discrete reports of $\ell$ to a vector space, and finds a convex loss which agrees with the original loss.
%A key notion to define an embedding is that of a \emph{representative set}: a set of reports $\Sc$ such that, for all label distributions, at least one report $r\in\Sc$ minimizes expected loss.
We now formalize the sense in which a convex surrogate can be an \emph{embedding} of a target loss $\ell$.
In these cases, which capture several consistent surrogates in the literature (e.g.,~\citep{ramaswamy2015hierarchical,ramaswamy2016convex,lapin2015top,wang2020weston}; see \S~\ref{sec:applications}), one maps each report (prediction) of $\ell$ to a point in $\reals^d$, then constructs a convex loss on $\reals^d$ that agrees with $\ell$ at these points.

In fact, it is not always necessary to map \emph{all} reports to such points.
It turns out to often be convenient to allow $\ell$ to have reports that are ``redundant'' in some sense (explored in \S~\ref{sec:min-rep-sets}), initially explored by~\citet{wang2020weston} under the name \emph{embedding cardinality}.
%\bo{perhaps mention and cite \citet{wang2020weston}}
Because of this redundancy, we will only require an embedding map to be defined on a \emph{representative set}: a set of reports $\Sc$ such that, for all label distributions, at least one report $r\in\Sc$ minimizes expected loss.
\begin{definition}[Representative set]
  Let $\Gamma:\simplex\toto\R$.
  We say $\Sc \subseteq \R$ is \emph{representative for $\Gamma$} if we have $\Gamma(p) \cap \Sc \neq \emptyset$ for all $p\in \simplex$.
  We further say $\Sc$ is a \emph{minimum representative set} if it has the smallest cardinality among all representative sets.
  Given a loss $L:\R\to\reals^\Y_+$, we say $\Sc$ is a (minimum) representative set for $L$ if it is a (minimum) representative set for $\prop L$.
  % We say $\Sc \subseteq \R$ is \emph{representative for $L$} if we have $\prop{L}(p) \cap \Sc \neq \emptyset$ for all $p\in \simplex$.
  % We further say $\Sc$ is a \emph{minimum representative set} if it has the smallest cardinality among all representative sets.
\end{definition}

%It turns out that it will be convenient, especially in applications, to allow the target $\ell:\R\to\reals^\Y_+$ to have reports which are ``redundant'' in some sense.
%One natural sense is that $\ell$ has a representative set that is a strict subset of $\R$, i.e., $\R$ is not a minimum representative set.
%\raft{Some discussion here about minimum representative sets and \emph{maximally informative sets} from~\citet{wang2020weston} here, though maybe best to leave it until later.  BTW, I think their latest version might use different terminology.}
%Another is that some reports $r$ are ``dominated'', meaning $r$ is never optimal or there is another report $r'$ which is always optimal when $r$ is optimal.
%A priori these alternate definitions may not seem closely related.
%In fact, we show in \S~\ref{sec:min-rep-sets} that they are equivalent in our setting.

We now formally introduce the concept of an embedding.
In addition to matching loss values, a key condition for consistency is that the original reports should be optimal exactly when the corresponding embedded points are optimal.
\begin{definition}[Embedding]\label{def:loss-embed}
  A loss $L:\reals^d\to\reals^\Y$ \emph{embeds} a loss $\ell:\R\to\reals^\Y$ if there exists a representative set $\Sc$ for $\ell$ and an injective embedding $\varphi:\Sc\to\reals^d$ such that
  (i) for all $r\in\Sc$ we have $L(\varphi(r)) = \ell(r)$, and (ii) for all $p\in\simplex,r\in\Sc$ we have
  % \begin{equation}\label{eq:embed-loss}
  %   r \in \argmin_{r'\in\R} \inprod{p}{\ell(r')} \iff \varphi(r) \in \argmin_{u\in\reals^d} \inprod{p}{L(u)}~.
  % \end{equation}
  \begin{equation}\label{eq:embed-loss}
    r \in \prop{\ell}(p) \iff \varphi(r) \in \prop{L}(p)~.
  \end{equation}
  If $\Sc$ is a minimal representative set, we say $L$ \emph{tightly embeds} $\ell$.
  %if there is some representative set $\Sc$ for $\ell$ such that $L$ embeds $\ell$ over $\Sc$.
\end{definition}

To illustrate the idea of embedding, let us examine hinge loss in detail as a surrogate for 0-1 loss for binary classification.
Recall that we have $\R = \Y = \{-1, +1\}$, with $\hinge(u)_y = (1 - uy)_+$ and $\ellzo(r)_y := \Ind{r\neq y}$, typically with link function $\psi(u) = \sgn(u)$.
We will see that hinge loss embeds (2 times) 0-1 loss, via the embedding $\varphi(r) = r$.
For condition (i), it is straightforward to check that $\hinge(r)_y = 2\ellzo(r)_y$ for all $r,y\in\{-1,1\}$.
% First, consider that
% \begin{align*}
% 	\ellzo(-1) = (0,1) && \ellzo(1) = (1,0)\\
% 	L_{hinge}(-1) = (0,2) && L_{hinge}(1) = (2,0)\\
% \end{align*}
% Therefore, $L_{hinge}$ is twice $\ellzo$ for each $r \in \R$.
For condition (ii), let us compute the property each loss elicits, i.e., the set of optimal reports for each $p$:
\[
\prop{\ellzo}(p) = \begin{cases}
1 & p_1 > 1/2 \\
\{-1,1\} & p_1 = 1/2\\
-1 & p_1 < 1/2
\end{cases}
\qquad
\prop{L_{hinge}}(p) = \begin{cases}
[1,\infty) & p_1 = 1\\
1 & p_1 \in (1/2,1) \\
[-1,1] & p_1 = 1/2\\
-1& p_1 \in (0, 1/2)\\
(-\infty, -1]& p_1 = 0
\end{cases}~.
\]
% Now, take the embedding $\varphi$ to be the identity.  % := r$ if $r \in [-1,1]$, and $\sgn(r)$ otherwise.
In particular, we see that $-1 \in \prop{\ellzo}(p) \iff p_1 \in [0, 1/2] \iff -1 \in \prop{\hinge}(p)$, and $1 \in \prop{\ellzo}(p) \iff p_1 \in [1/2,1] \iff 1 \in \prop{\hinge}(p)$.
With both conditions of Definition~\ref{def:loss-embed} satisfied, we can conclude that $\hinge$ embeds $2\ellzo$ by results in \S~\ref{subsec:match-BR}.
\raf{Forward ref \S~\ref{sec:min-rep-sets} <-- add ref to 6.2}

\begin{figure}
	\begin{minipage}{0.3\linewidth}
	\centering
	\includegraphics[width=0.95\linewidth]{figs/0-1-br.pdf}
%		\caption{Bayes risk of the 0-1 loss.}
%		\label{fig:0-1-br}
	\end{minipage}
	\hfill
	\begin{minipage}{0.3\linewidth}
	\centering		\includegraphics[width=0.95\linewidth]{figs/hinge-br.pdf}
%		\caption{Bayes risk of hinge loss.}
%		\label{fig:hinge-br}
	\end{minipage}
	\hfill
	\begin{minipage}{0.3\linewidth}
	\centering
	\includegraphics[width=0.95\linewidth]{figs/logistic-br.pdf}
%	\caption{Bayes risk of the logistic loss.}
%	\label{fig:logistic-br}
\end{minipage}
\caption{Bayes risks of 0-1, hinge, and logistic losses, respectively, in binary classification setting. \raf{Add detail}}
\label{fig:bayes-risks-01}
\end{figure}

In this particular example, it is known $(\hinge,\psi)$ is calibrated for $\psi(u) = \sgn(u)$.
More generally, however, it is not clear whether an arbitrary embedding yields a calibrated link.
Indeed, apart from mapping the embedded points back to their original reports via $\psi(\varphi(r)) = r$, how to map the remaining values is far from obvious.
When the surrogate is polyhedral, we give a construction to map the remaining values in \S~\ref{sec:calibration}, showing that embeddings always yield calibration.
We first explore the connection between embeddings and polyhedral surrogates.

%\raf{Suggest moving / incorporating below} \jessie{Moved, though I think think should be merged with above paragraph now}
While our notion of embedding is sufficient for calibration (and therefore consistency), it is worth noting that it is not \emph{necessary} for these conditions.  
For example, while logistic loss does not embed 0-1 loss, the surrogate and link for logistic loss are consistent with respect to 0-1 loss.


% Taking $\R = \{-1,1\}$ and the above embedding, we can write the expected loss over each report
% \begin{center}
% 	\begin{tabular}{ll}
% 		$\inprod{p}{\ellzo(1)} = (1-p_1)$ & $\inprod{p}{\ellzo(-1)} = (p_1)$\\
% 		$\inprod{p}{L_{hinge}(1)} = 2(1-p_1)$ & $\inprod{p}{L_{hinge}(-1)} = 2(p_1)$
% 	\end{tabular}
% \end{center}
% Thus, for every $r \in \R$ and $p \in \simplex$, we have $2 \inprod{p}{\ellzo(r)} = \inprod{p}{L_{hinge}(\varphi(r))}$.
% This leads us into the next section, where we show the Bayes Risks of the losses will be the same, accounting for the multiplicative constant, if one loss embeds the other.

% Additionally, one can see logistic loss does not embed $\ellzo$ because it elicits a continuous property, so there is an infinite set of unique optima over $\simplex$, and thus no injection exists mapping this set to $\{-1,1\}$.

% \raf{$L_{logistic}(u,y) := \ln\left(1 + \exp(-yu)\right)$}




% In another example, we consider the abstain loss and an embedding given by Ramaswamy et al.~\cite{ramaswamy2018consistent} in Section~\ref{sec:applications}, but note that the link function they provide is not unique.
% Essentially, the discrete abstain$(1/2)$ loss is minimized in expectation by predicting the most likely outcome $\argmax_y p_y$, conditioned on $p_y \geq 1/2$.
% If the most likely outcome is not likely enough, the loss is minimized by ``abstaining'' from predicting; a situation one can imagine would be useful if a false positive is costly.





\section{Embeddings and Polyhedral Losses}
\label{sec:poly-loss-embed}

In this section, we establish a tight relationship between the technique of embedding and the use of polyhedral (piecewise-linear convex) surrogate losses, showing Theorem~\ref{thm:embed-poly-informal}.
We defer the question of when such surrogates are consistent to \S~\ref{sec:calibration}. 
\raf{Reference Theorem~\ref{thm:embed-poly-informal}}

% \raf{Do we want this anymore?  We violate it right off the bat with Lemma~\ref{lem:loss-restrict}...}
% Throughout this section, unless specified otherwise, we use $\ell:\R\to\reals^\Y$ for discrete target losses and $L:\reals^d\to\reals^\Y$ for surrogates.

A first observation is that a loss $L$ restricted to some representative set $\Sc$, denoted $L|_\Sc$, elicits the same property restricted to $\Sc$.
As a consequence, restricting to representative sets preserves the Bayes risk.
We will use these observations throughout.
\begin{lemma}\label{lem:loss-restrict}
  Let $L:\R\to\reals^\Y_+$ elicit $\Gamma$, and let $\Sc\subseteq\R$ be representative for $L$.
  % such that $\Gamma(p) \cap \Sc \neq \emptyset$ for all $p\in\simplex$.
  Then $L|_\Sc$ elicits $\gamma:\simplex\toto\Sc$ defined by $\gamma(p) = \Gamma(p)\cap \Sc$.
  Moreover, $\risk{L}=\risk{L|_\Sc}$.
\end{lemma}
\begin{proof}
  Let $p\in\simplex$ be fixed throughout.
  First let $r \in \gamma(p) = \Gamma(p) \cap \Sc$.
  Then $r \in \Gamma(p) = \argmin_{u\in\R} \inprod{p}{L(u)}$, so as $r\in\Sc$ we have in particular $r \in \argmin_{u\in\Sc} \inprod{p}{L(u)}$.
  For the other direction, suppose $r \in \argmin_{u\in\Sc} \inprod{p}{L(u)}$.
  As $\Sc$ is representative for $L$, we must have some $s \in \Gamma(p) \cap \Sc$.
  On the one hand, $s\in\Gamma(p) = \argmin_{u\in\R} \inprod{p}{L(u)}$.
  On the other, as $s \in \Sc$, we certainly have $s \in \argmin_{u\in\Sc} \inprod{p}{L(u)}$.
  But now we must have $\inprod{p}{L(r)} = \inprod{p}{L(s)}$, and thus $r \in \argmin_{u\in\R} \inprod{p}{L(u)} = \Gamma(p)$ as well.
  We now see $r \in \Gamma(p) \cap \Sc$.
  Finally, the equality of the Bayes risks $\min_{u\in\R} \inprod{p}{L(u)} = \min_{u\in\Sc} \inprod{p}{L(u)}$ follows immediately by the above, as $\emptyset \neq \Gamma(p)\cap\Sc \subseteq \Gamma(p)$ for all $p\in\simplex$.
\end{proof}


Lemma~\ref{lem:loss-restrict} leads to the following useful tool for finding embeddings.
\begin{proposition}\label{prop:representative-embeds-restriction}
  Let surrogate loss $L:\reals^d \to \reals^\Y_+$ be given.
  If $L$ has a finite representative set $\U \subseteq \reals^d$, then $L$ embeds the discrete loss $L|_\U$.
\end{proposition}
\begin{proof}
  Let $\Gamma = \prop{L}$ and $\gamma = \prop{L|_\U}$.
  Define $\varphi : \U \to \U$ to be the identity embedding.
  Condition (i) of an embedding is trivially satisfied, as $L|_\U(u) = L(u)$ for all $u\in\U$.
  Now let $u\in\U$.
  From Lemma~\ref{lem:loss-restrict}, for all $p\in\simplex$ we have $u \in \gamma(p) \iff u \in \Gamma(p) \cap \U \iff u \in \Gamma(p)$.
  We conclude condition (ii) of an embedding.
\end{proof}


We now shift our focus to \emph{polyhedral} (piecewise-linear and convex) surrogates.
Our first observation is that while polyhedral surrogates cannot be finite properties, in the sense that they have infinitely many possible reports, they do elicit properties with a finite range, meaning a finite set of possible optimal sets.
This observation lets us apply results about finite representative sets to understand the structure of polyhedral surrogates and the losses they embed.
See \S~\ref{app:power-diagrams} for the full proof.

\begin{restatable}{lemma}{polyhedralrangegamma}
	\label{lem:polyhedral-range-gamma}
	Let $L:\reals^d\to\reals_+^\Y$ be a polyhedral loss, and let $\Gamma = \prop{L}$.
	Then the range of $\Gamma$, given by $\Gamma(\simplex) = \{\Gamma(p) \subseteq \reals^d : p\in\simplex\}$, is a finite set of closed polyhedra.
\end{restatable}
\begin{proof}[Sketch]
	With $\Y$ finite, there are only finitely many supporting sets over $\simplex$.
	For $p \in \simplex$, the power diagram induced by projecting the epigraph of expected loss onto $\reals^d$ is the same for any $p$ of the same support (Lemma~\ref{lem:polyhedral-pd-same}).
	Moreover, we have $\Gamma(p)$ being exactly one of the faces of the projected epigraph since the hyperplane $u \mapsto (u, \inprod{p}{L(u)})$ supports the epigraph of the expected loss at exactly the property value; moreover, since the loss is polyhedral the supporting hyperplane must support on a face of the epigraph.
	Since this epigraph has finitely many faces (as it is polyhedral), the range of $\Gamma$ is then (a subset) of elements of a finitely generated (finite supports) set of finite elements (finite faces).
	Moreover, each element of $\Gamma(\simplex)$ is a closed polyhedron since it corresponds exactly to a closed face of a polyhedral set.
\end{proof}


\begin{theorem}\label{thm:poly-embeds-discrete}
  Every polyhedral loss $L$ embeds a discrete loss.
\end{theorem}
\begin{proof}
  Let $L:\reals^d\to\reals_+^\Y$ be a polyhedral loss, and $\Gamma = \prop{L}$.
  By Lemma~\ref{lem:polyhedral-range-gamma}, $\Gamma(\simplex)$ is finite set. 
  For each $U\in \Gamma(\simplex)$, select $u_U \in U$, and let $\Sc = \{u_U : U \in\Gamma(\simplex)\}$, which is again finite.
  For any $p\in\simplex$ then, let $U = \Gamma(p)$.
  We have $U \in \Gamma(\simplex)$ by definition, and thus some $u_U \in \Sc$; in particular, $u_U \in U = \Gamma(p)$.
  We conclude that $\Sc$ is representative for $L$.
  Proposition~\ref{prop:representative-embeds-restriction} now states that $L$ embeds $L|_\Sc$.
\end{proof}

We now turn to the reverse direction: which discrete losses are embedded by some polyhedral loss?
Perhaps surprisingly, we show that \emph{every} discrete loss is embeddable,
using a construction via convex conjugate duality which has appeared several times in the literature (e.g.\ \cite{duchi2018multiclass,abernethy2013efficient,frongillo2014general}).
The proof relies on a result we will prove in \S~\ref{sec:min-rep-sets}: a minimizable surrogate embeds a discrete loss if and only if their Bayes risks match (Proposition~\ref{prop:embed-bayes-risks}).


\begin{theorem}\label{thm:discrete-loss-poly-embeddable}
  Every discrete loss $\ell:\R \to \reals^\Y_+$ is embedded by a polyhedral loss.
\end{theorem}
\begin{proof}
  Let $n = |\Y|$, and let $C:\reals^n \to \reals$ be given by $(-\risk{\ell})^*$, the convex conjugate of $-\risk{\ell}$.
  From standard results in convex analysis, $C$ is polyhedral as $-\risk{\ell}$ is, and $C$ is finite on all of $\reals^\Y$ as the domain of $-\risk{\ell}$ is bounded~\cite[Corollary 13.3.1]{rockafellar1997convex}.
  Note that $-\risk{\ell}$ is a closed convex function, as the infimum of affine functions, and thus $(-\risk{\ell})^{**} = -\risk{\ell}$.
  Define $L:\reals^n\to\reals^\Y$ by $L(u) = C(u)\ones - u$, where $\ones\in\reals^\Y$ is the all-ones vector.
  As $C$ is polyhedral, so is $L$.
  We first show that $L$ embeds $\ell$, and then establish that the range of $L$ is in fact $\reals^\Y_+$, as desired.

  We compute Bayes risks and apply Proposition~\ref{prop:embed-bayes-risks} to see that $L$ embeds $\ell$.
  Observe that $\risk{\ell}$ is polyhedral as $\ell$ is discrete.
  For any $p\in\simplex$, we have
  \begin{align*}
    \risk{L}(p)
    &= \inf_{u\in\reals^n} \inprod{p}{C(u)\ones - u}\\
    &= \inf_{u\in\reals^n} C(u) - \inprod{p}{u}\\
    &= -\sup_{u\in\reals^n} \inprod{p}{u} - C(u)\\
    &= -C^*(p) = - (-\risk{\ell}(p))^{**} = \risk{\ell}(p)~.
  \end{align*}
  It remains to show $L(u)_y \geq 0$ for all $u\in\reals^n$, $y\in\Y$.
  Letting $\delta_y\in\simplex$ be the point distribution on outcome $y\in\Y$, we have for all $u\in\reals^n$, $L(u)_y \geq \inf_{u'\in\reals^n} L(u')_y = \risk{L}(\delta_y) = \risk{\ell}(\delta_y) \geq 0$, where the final inequality follows from the nonnegativity of $\ell$.
  \btw{FUTURE: to get $n-1$, just use the same trick as before and note that the Bayes risk doesn't change.  Should be a couple lines.}
\end{proof}

\raf{TODO: smooth over these paragraphs.  I think we should combine the first two and give only a 1-sentence description of how to achieve $|\Y|-1$ dimensions.}
\jessie{Took a pass 09 Sept 21}

Note however that the number of dimensions $d$ required could be as large as $n = |\Y|$, which is particularly undesirable in structured prediction and information retrieval problems where the number of outcomes is exponential.
Recent work~\citep{ramaswamy2016convex,finocchiaro2020embedding,finocchiaro2021unifying} yield characterizations for bounding the prediction dimension $d$ for consistent convex surrogates and embeddings.
Moreover, to construct an embedding of dimension $n-1$ instead of $n$, consider an invertible linear transformation from $\reals^n \to \reals^{n-1}$ and re-define the loss appropriately so Bayes risks match.
%\raft{CITE: DKR, frongillo-kash (WINE 2014), probably one of Bob's papers?, others?}

%Moreover, to construct an embedding of dimension $n-1$ instead of $n$ (see~\citet{finocchiaro2020embedding} for significance of dimension),  consider some invertible linear transformation $A : \reals^n \to \reals^{n-1}$ that maps to an affine subspace.
%Take $L' : \reals^{n-1} \to \reals^\Y$ as $x \mapsto L(A^{-1}(x))$ and $L$ defined as above, so that $\inprod{\ell(r)}{p} = \inprod{L(\varphi(r))}{p} = \inprod{L'(A^{-1}(\varphi(r)))}{p} $ for all $p \in \simplex$.
%The Bayes risks $\risk{L}(p)$ and $\risk{L'}(p)$ are equal for all $p \in \simplex$; thus, the result holds in $n-1$ dimensions as well.
%\jessiet{Leaving this change in the macro for proofreading.}

%\btw{Cut summary for now (Why is this cool?)}
%\jessiet{Jessie: added a bit to say why this is cool.}
\jessiet{Is this following paragraph necessary here?  It's more of an intro for \S~\ref{sec:calibration}.}
We will see that embedding implies the existence of a consistent surrogate and link, so this result tells us that, for discrete losses, the embedding framework is sufficient to understand the existence of consistent surrogates for the given loss.
Even if there is a smooth surrogate for a given discrete loss, there is also a polyhedral loss that is calibrated for the given loss.
What's more is that there is a polyhedral \emph{embedding} that is calibrated for the discrete loss.
Observe that the concatenation of theorems \ref{thm:poly-embeds-discrete} and \ref{thm:discrete-loss-poly-embeddable} yield Theorem~\ref{thm:embed-poly-informal}, and highlight the connection between embeddings and polyhedral losses.


%\begin{corollary}\label{cor:finite-elicit-embed}
%  Let $\gamma$ be a finite property.
%  The following are equivalent.
%  \begin{enumerate}\setlength{\itemsep}{0pt}
%  \item $\gamma$ is elicitable.
%  \item $\gamma$ is embeddable.
%  \item $\gamma$ is embeddable via a polyhedral loss.
%  \end{enumerate}
%\end{corollary}
%\begin{proof}
%  We trivially have $3\Rightarrow 2$.
%  The direction $1\Rightarrow 3$ follows from Theorem~\ref{thm:discrete-loss-poly-embeddable}, by taking any discrete loss which elicits $\gamma$.
%  Finally, to see $2\Rightarrow 1$,
%  \raf{This follows from Proposition~\ref{prop:embed-trim}, but we're stating this as a corollary... let's see if it follows more directly from the above.  If it doesn't (one complication: there is no discrete loss starting from $2$ or $3$!) we can rephrase it as a theorem / proposition.}
%\end{proof}


\section{Consistency via Calibrated Links}
\label{sec:calibration}

We have now seen the tight relationship between polyhedral losses and embeddings; in particular, every polyhedral loss embeds some discrete loss.
The embedding itself tells us how to link the embedded points back to the discrete reports (map $\varphi(r)$ to $r$).
But it is not clear how to extend this to yield a full link function $\psi: \reals^d \to \R$, and whether such a $\psi$ can lead to consistency.
In this section, we give a construction to generate calibrated links for \emph{any} polyhedral loss, which we summarize in the following theorem.
\S~\ref{app:calibration} contains the full proof; this section provides a sketch of the main construction and result.

\linkinformal* 
%\begin{theorem}\label{thm:eps-thick-calibrated}
%  Let $L$ be polyhedral and $\ell$ the discrete loss it embeds from Theorem~\ref{thm:poly-embeds-discrete}.
%  Then there exists a calibrated link function $\psi$ from $L$ to $\ell$.
%%Then for small enough $\epsilon > 0$, the $\epsilon$-thickened link $\psi$ is well-defined and, furthermore, is a calibrated link from $L$ to $\ell$.
%\end{theorem}

\raf{I liked the new version in general.  The only thing I was suggest is that for a journal version we could add a bit more detail right here, in the proof sketch.  For example, maybe a high-level overview right at this spot would help, since when you launch into step 1, it's not clear how many steps will follow.  And after step 2, there is a definition, and more discussion after.  All the pieces are great; maybe just a splash of overview and organization can help the reader orient as they go through it.}

At a high level, the proof has three main steps.
The first step is to show indirect elicitation: we give a link $\psi: \reals^d \to \R$ such that exactly minimizing expected surrogate loss $L$, followed by applying $\psi$, always exactly minimizes expected original loss $\ell$.
%Symbolically, if $u^* \in \argmin_u \inprod{p}{L(u)} = \Gamma(p)$, then $\psi(u^*) \in \argmin_r \inprod{p}{\ell(r)} = \gamma(p)$.
For this step, the challenge is that for a generic $u \in \reals^d$, not an embedding point, we may have $u \in \Gamma(p) \cap \Gamma(p')$.
Because $u$ minimizes expected loss for both $p$ and $p'$, this \raf{suggest adding a noun here} imposes the requirement on the link that $\psi(u) \in \gamma(p) \cap \gamma(p')$.
It is not \emph{a priori} clear that these sets even have a nonempty intersection.
We use elicitation results, discussed in \S~\ref{sec:min-rep-sets}, to show that for each $u \in \reals^d$ there exists $r \in \R$ such that $\Gamma_u \subseteq \gamma_r$, i.e. if $p$ satisfies $u \in \Gamma(p)$ then it satisfies $r \in \gamma(p)$.
This implies that if $u \in \Gamma(p) \cap \Gamma(p')$, then $r \in \gamma(p) \cap \gamma(p')$, so we may safely choose $\psi(u) = r$.

%Associating each $U$ with $R_U \subseteq \R$, the set of reports whose embedding points are in $U$, we enforce that all points in $U$ link to some report in $R_U$.
%(As a special case, embedding points must link to their corresponding reports.)
%Proving that these choices are well-defined uses a chain of arguments involving the Bayes risk, ultimately showing that if $u$ lies in multiple such sets $U$, the corresponding report sets $R_U$ all intersect at some $r =: \psi(u)$.

The second step is to ensure calibration: we should modify this link so that even approximately-optimal points $u$ link to optimal reports $r$.
We do so by ``thickening'' the link by $\epsilon$ as described next.
Informally, we take each optimal report set, i.e. some $U = \Gamma(p) = \argmin_u \inprod{p}{L(u)}$, and thicken it by adding all points within distance $\epsilon$.

Finally, we impose a requirement that all points in this thickened set are linked to some element of $\gamma(p) = \argmin_r \inprod{p}{\ell(r)}$.
We use $\Psi(u)$ to denote the remaining legal choices for $\psi(u)$.
After repeating for each optimal report set, we will claim that there exists a small enough $\epsilon > 0$ so that the construction is successful, i.e. $\Psi(u)$ is nonempty for all $u$.
%A key step in the following definition will be to narrow down a ``link envelope'' $\Psi$ where $\Psi(u)$ denotes the legal or valid choices for $\psi(u)$.

\begin{definition}[$\epsilon$-thickened link] \label{def:eps-thick-link}
  First, define $\U = \{\Gamma(p) : p \in \simplex\}$ and, for each $U \in \U$, let $R_U = \{r \in \R : \phi(r) \in U\}$, the reports whose embedding points are in $U$.
  Given a polyhedral $L$ that embeds some $\ell$, an $\epsilon > 0$, and a norm $\|\cdot\|$, the \emph{$\epsilon$-thickened link} $\psi$ is constructed as follows.
  First, initialize $\Psi: \reals^d \toto \R$ by setting $\Psi(u) = \R$ for all $u$.
  Then for each $U \in \U$, for all points $u$ such that $\inf_{u^* \in U} \|u^*-u\| < \epsilon$, update $\Psi(u) = \Psi(u) \cap R_U$.
  Finally, define $\psi(u) \in \Psi(u)$, breaking ties arbitrarily.
  If $\Psi(u)$ became empty, then leave $\psi(u)$ undefined.
\end{definition}
A key point is that, by Lemma~\ref{lem:polyhedral-range-gamma}, there are only finitely many sets of the form $U = \Gamma(p)$.
We then show, along the lines discussed above, that for a small enough choice of $\epsilon > 0$, if the thickenings of $U$ and $U'$ overlap at some $u$, then the corresponding report sets $R_U, R_{U'}$ also overlap; this implies $\Psi(u)$ is nonempty.
Given a well-defined thickened link $\psi$, calibration is almost immediate because, given $p$, any report $u$ linking to a suboptimal $r = \psi(u)$ satisfies $\inf_{u^* \in \Gamma(p)} \|u^* - u\| \geq \epsilon$.
  We then show that the minimal gradient of the expected loss along any direction away from $U$ is lower-bounded, giving a constant excess expected loss at $u$.

\btw{FUTURE: we should comment in the discussion section that we probably can show that *any* loss embedding $\ell$ must be polyhedral-ish, meaning polyhedral except for stuff that is never optimal.  This theorem would then not need the ``polyhedral'' part. This is related to the ``convex envelope conjucture'', that if $L$ embeds $\ell$ via $\varphi$, you can just take the loss $L'$ such that $L_y$ is the convex envelope of points $\{(\varphi(r),L(r)_y)\}_{r\in\R}$.}
\btw{FUTURE: This prop and theorem give an excellent reason to focus on embeddings, since other techniques do not necessarily give you separated links for free.  Since we know we get them for free, we can just focus on the property, and study elicitation complexity; we know if we have a link at all it can be taken to be separated.  [[Is this true?]]}



Note that the construction given above in Definition~\ref{def:eps-thick-link} is not necessarily computationally efficient as the number of labels $n$ grows.
In practice this potential inefficiency is not typically a concern, as the family of losses typically has some closed form expression in terms of $n$, and thus the construction can proceed at the symbolic level.
We illustrate this formulaic approach in \S~\ref{sec:abstain}.
% \proposedadd{In general, we are not overly concerned with the computational complexity of computing the surrogate and link, since these calculations only need to be computed once.  They are generally derived mathematically, rather than computationally.  These symbolic calculations typically tend to be straightforward, but can sometimes be tricky.}



%%%%%%%%%%%%%%%%%%%%%%%%%%%%%%%%%%%%%%%%%%%%%%%%%%%%%%%%%%%%%%%
\section{Application to Specific Surrogates}\label{sec:applications}

Our results give a framework to construct consistent surrogates and link functions for any discrete loss, but they also provide a way to verify the consistency or inconsistency of given surrogates.
Below, we illustrate the power of this framework with specific examples from the literature, as well as new examples.
In some cases we simplify existing proofs, while in others we give new results, such as a new calibrated link for abstain loss, and the inconsistency of the recently proposed Lov\'asz hinge.
\btw{RF: The top-k and hypercube/set examples (except abstain) have the link built in, e.g.\ as the sign function), and they just work around it.  Our results may suggest that it's worth thinking ``outside the box'' (HAH, genious...) and looking for embeddings which are not the hypercube. \jessie{Added a bit discussing this, starting at ``the examples in ...''.}}
The examples below, with the exception of the abstain surrogate given by~\citet{ramaswamy2018consistent}, all present a surrogate that use the link $\psi: u \mapsto \sgn(u)$. \jessie{Is this true?  Not for top-$k$, unless I'm misreading}
While this may sometimes yield a calibrated link, this is not always the case.
In fact, we will see that most of these examples do not yield calibrated surrogates, although proving that there is \emph{no} calibrated link for a given surrogate is quite difficult.
Our results suggest that it is possible that some of the given surrogates are calibrated, but perhaps one must use a nontraditional link in order to calibrate the loss, such as the thickening construction given here.
%\citet{wang2020weston} apply the embedding framework to show that the Weston-Watkins hinge loss embeds a discrete loss for ordered partitions.

In general the approach to apply embeddings is as follows: given a surrogate $L:\reals^d \to \reals^\Y_+$, find the regions for which $L$ is affine.
Vertices of $L$ are formed by the intersection of $d$ facets of $\epi(L)$, or equivalently affine regions of $L$~\citep[Statement 3.1.7]{grunbaum2013convex}.
The enumeration of these vertices yields a finite representative set, and we can reason about embedding these vertices.

Applying the $\epsilon$-thickened link construction additionally enables one to verify the consistency of a proposed link and embedding.
For a given $\epsilon$ and norm $\|\cdot\|$, suppose one follows the routine of Definition~\ref{def:eps-thick-link} until the last step in which values for the link are selected.
Once possible choices for this $\epsilon$-thickened set-valued function are available, we can see if the proposed link aligns with these possible choices: if so, then the proposed link is calibrated.


\subsection{Consistency of abstain surrogate and link construction}
\label{sec:abstain}

%\raf{TODO: talk about generic approach: restrict to affine regions, take vertices} \jessie{Attempted above}

In classification settings with a large number of labels, several authors consider a variant of classification, with the addition of a ``reject'' or \emph{abstain} option.
For example, \citet{ramaswamy2018consistent} study the loss $\ellabs{\alpha} : [n] \cup \{\bot\} \to \reals^\Y_+$ defined by $\ellabs{\alpha}(r)_y = 0$ if $r=y$, $\alpha$ if $r = \bot$, and 1 otherwise.
%\begin{align}\label{eq:abstain-discrete}
%$\ellabs{\alpha}(r)_y = \begin{cases}
%0 & r = y\\
%\alpha & r = \bot\\
%1 & \text{otherwise}
%\end{cases}~.$
%\end{align}
Here, the report $\bot$ corresponds to ``abstaining'' if no label is sufficiently likely, specifically, if no $y\in\Y$ has $p_y \geq 1-\alpha$.
\citet{ramaswamy2018consistent} provide a polyhedral surrogate for $\ellabs{\alpha}$, which we present here for $\alpha=1/2$.
Letting $d = \ceil{\log_2(n)}$ their surrogate is $L_{1/2} : \reals^d \to \reals^\Y_+$ given by
\begin{equation}\label{eq:abstain-surrogate}
L_{1/2}(u)_y = \left(\max\nolimits_{j \in [d]}B(y)_j u_j + 1\right)_+~,
\end{equation}
where $B:[n]\to\{-1,1\}^d$ is a arbitrary injection; let us assume $n = 2^d$ so that we have a bijection.
Consistency is proven for the following link function,
\begin{equation}\label{eq:abstain-link}
  \psi(u) = \begin{cases}
	\bot & \min_{i \in [d]} |u_i| \leq 1/2\\
	B^{-1}(\sgn(-u)) &\text{otherwise}
  \end{cases}~.
\end{equation}

In light of our framework, we can see that $L_{1/2}$ is an excellent example of an embedding, where $\varphi(y) = B(y)$ and $\varphi(\bot) = 0 \in \reals^d$.
Moreover, the link function $\psi$ can be recovered from Theorem~\ref{thm:eps-thick-calibrated} with norm $\|\cdot\|_\infty$ and $\epsilon=1/2$; see Figure~\ref{fig:abstain-links}(L).
Hence, our framework would have simplified the process of finding such a link, and the corresponding proof of consistency.
To illustrate this point further, we give an alternate link $\psi_1$ corresponding to $\|\cdot\|_1$ and $\epsilon=1$, shown in Figure~\ref{fig:abstain-links}(R):
\begin{equation}\label{eq:abstain-link-1}
  \psi_1(u) = \begin{cases}
	\bot & \|u\|_1 \leq 1\\
	B^{-1}(\sgn(-u)) &\text{otherwise}
  \end{cases}~.
\end{equation}
Theorem~\ref{thm:eps-thick-calibrated} immediately gives calibration of $(L_{1/2},\psi_1)$ with respect to $\ellabs{1/2}$.
Aside from its simplicity, one possible advantage of $\psi_1$ is that it appears to yield the same constant in generalization bounds as $\psi$, yet assigns $\bot$ to much less of the surrogate space $\reals^d$.
It would be interesting to compare the two links in practice.

The embedding framework can also be applied to understand the hierarchical extension of the BEP surrogate given by~\citet{ramaswamy2015hierarchical}, who present a convex, calibrated surrogate for the hierarchical classification problem.
Given a tree $H = ([n], E,W)$ over finite class labels, they consider the discrete loss 
\begin{equation}\label{eq:tree-loss-discrete}
	\ell^H(r,y) = \text{shortest path length in $H$ between $r$ and $y$}
\end{equation}
They generalize the BEP embedding to this problem and show their extension is consistent with respect to eq.~\eqref{eq:tree-loss-discrete}.
We can also verify that their surrogate is an embedding and recover their result through Theorem~\ref{thm:poly-ie-implies-consistent}.


\begin{figure}
\begin{center}
\begin{minipage}{0.32\linewidth}
\includegraphics[width=\linewidth]{tikz/abstain-link-linf.pdf}
\end{minipage}\hfill
\begin{minipage}{0.32\linewidth}
\includegraphics[width=\linewidth]{tikz/abstain-link-U-regions-l1.pdf}
\end{minipage}\hfill
\begin{minipage}{0.32\linewidth}
\includegraphics[width=\linewidth]{tikz/abstain-link-l1.pdf}
\end{minipage}\hfill
\caption{Constructing links for the abstain surrogate $L_{1/2}$ with $d=2$. The embedding is shown in bold labeled by the corresponding reports. (L) The link envelope $\Psi$ resulting from Theorem~\ref{thm:eps-thick-calibrated} using $\|\cdot\|_\infty$ and $\epsilon = 1/2$, and a possible link $\psi$ which matches eq.~\eqref{eq:abstain-link} from~\cite{ramaswamy2018consistent}.  (M) An illustration of the thickened sets from Definition~\ref{def:eps-thick-link} for two sets $U \in \U$, using $\|\cdot\|_1$ and $\epsilon = 1$. (R) The $\Psi$ and $\psi$ from Theorem~\ref{thm:eps-thick-calibrated} using $\|\cdot\|_1$ and $\epsilon = 1$.}
\label{fig:abstain-links}
\end{center}
\end{figure}


\subsection{Inconsistency of Lov\'asz hinge}
\label{sec:lovasz-hinge}

Many structured prediction settings can be thought of as making multiple predictions at once, with a loss function that jointly measures error based on the relationship between these predictions~\cite{hazan2010direct, gao2011consistency, osokin2017structured}.
In the case of $k$ binary predictions, these settings are typically formalized by taking the predictions and outcomes to be $\pm 1$ vectors, so $\R=\Y=\{-1,1\}^k$.
One then defines a joint loss function, which is often merely a function of the set of mispredictions, meaning we may write $\ell^f(r)_y = f(\{i \in [k] : r_i \neq y_i\})$ for some set function $f:2^{[k]}\to\reals$.
For example, Hamming loss is given by $f(S) = |S|$.
In an effort to provide a general convex surrogate for these settings when $f$ is a submodular function, Yu and Blaschko~\cite{yu2018lovasz} introduce a surrogate $L^f$ which they dub the \emph{Lov\'asz hinge}, as it leverages the well-known convex Lov\'asz extension of submodular functions.
While the authors provide theoretical justification and experiments, consistency of the Lov\'asz hinge is left open, which we resolve.
As even the formal definition of the Lov\'asz hinge is quite involved, we defer the complete analysis to \S~\ref{app:lovasz} \jessie{REF} and here state the main results.

As a warm-up, let us introduce the form of the Lov\'asz hinge $L^f$ for the case $k=2$, as one can build intuition from this case.
For brevity, we write $f_\emptyset := f(\emptyset)$, $f_{1,2} := f(\{1,2\})$, etc.
%\raft{Jessie: I meant \emph{our} paper :-].  Leaving this here:\cite{yu2015lovaszarxiv}}
Assuming $f$ is normalized and increasing (meaning $f_{1,2} \geq \{f_1,f_2\} \geq f_\emptyset = 0$), the Lov\'asz hinge $L:\reals^k\to\reals^\Y_+$ is given by
\begin{multline}
  \label{eq:lovasz-hinge}
  L^f(u)_y = \max\Bigl\{(1-u_1y_1)_+ f_1 + (1-u_2y_2)_+ (f_{1,2}-f_1),\\[-4pt] (1-u_2y_2)_+ f_2 + (1-u_1y_1)_+ (f_{1,2}-f_2)\Bigr\}~,
\end{multline}
where $(x)_+ = \max\{x,0\}$.
The link function $\psi:\reals^2\to\{-1,1\}^2$ is fixed as $\psi(u)_i = \sgn(u_i)$, with ties broken arbitrarily.

We are interested in characterizing the functions $f$ for which $L^f$ is consistent.
As an example, consider the coefficients $f_\emptyset = 0$, $f_1 = f_2 = f_{1,2} = 1$, for which $\ell^f$ is merely 0-1 loss on $\Y$.
For consistency, for any distribution $p\in\simplex$, we must have that whenever $u \in \argmin_{u'\in\reals^2} \inprod{p}{L^f(u')}$, the outcome $\psi(u)$ must be the most likely, i.e., in $\argmax_{y\in\Y} p(y)$.
Simplifying eq.~\eqref{eq:lovasz-hinge}, however, we have
\begin{equation}
  \label{eq:lovasz-hinge-abstain}
  L^f(u)_y = \max\bigl\{(1-u_1y_1)_+,(1-u_2y_2)_+\bigr\} = \max\bigl\{1-u_1y_1,1-u_2y_2,0\bigr\}~,
\end{equation}
which is exactly the abstain surrogate~\eqref{eq:abstain-surrogate} for $d=2$.
We immediately conclude that $L^f$ cannot be consistent with $\ell^f$, as the origin will be the unique optimal report for $L^f$ under distributions with $p_y < 0.5$ for all $y$, and one can simply take a distribution which disagrees with the way ties are broken in $\psi$.
For example, if we take $\sgn(0) = 1$, then under $p((1,1)) = p((1,-1)) = p((-1,1)) = 0.2$ and $p((-1,-1)) = 0.4$, we have $\{0\} = \argmin_{u\in\reals^2} \inprod{p}{L^f(u)}$, yet we also have $\psi(0) = (1,1) \notin \{(-1,-1)\} = \argmin_{r\in\R} \inprod{p}{\ell^f(r)}$.

In fact, this example is typical: % using our embedding framework, and characterizing when $0\in\reals^2$ is an embedded point,
one can show that $L^f$ is consistent if and only if $f_{1,2} = f_1 + f_2$.
Moreover, in this linear case, which corresponds to $f$ being \emph{modular}, the Lov\'asz hinge reduces to weighted Hamming loss, which is trivially consistent from the consistency of hinge loss for 0-1 loss.
In \S~\ref{app:lovasz}, we generalize this observation for all $k$.
\begin{theorem}\label{thm:lovasz-consistent-main}
	Let $f$ be submodular, normalized, and increasing.
	Then $(L^f,\psi)$ is consistent if and only if $f$ is modular.
\end{theorem}
In other words, even for $k>2$, the only consistent Lov\'asz hinge is weighted Hamming loss.
This result casts doubt on the practical effectiveness of the Lov\'asz hinge as a surrogate for $\ell^f$.

In light of this negative result, together with the fact that every polyhedral surrogate embeds some target loss (Theorem~\ref{thm:poly-embeds-discrete}), we now turn to the question: for what target problem \emph{is} the Lov\'asz hinge consistent?
We show in \S~\ref{app:lovasz} that the answer is a variant of abstain loss where one can abstain on a set of labels $B\subseteq[k]$.
Recall that the symmetric difference is defined by $S\triangle T = (S\setminus T) \cup (T\setminus S)$.
\begin{theorem}\label{thm:lovasz-embed-main}
	Let $\hat\ell^f:\hat\R\times\Y\to\reals_+$, where $\hat\R = \{(A,B) \in 2^{[k]} \times 2^{[k]}: A\cap B=\emptyset\}$, be given by
	\begin{equation}
	\hat\ell^f((A,B))_S = f(A\triangle S\setminus B) + f(A\triangle S\cup B)~.
	\end{equation}
	Then the Lov\'asz hinge $L^f$ embeds $\hat\ell^f$.
  Moreover, if $f$ is strictly submodular and strictly increasing, the embedding is tight.
\end{theorem}

While Theorem~\ref{thm:lovasz-consistent-main} shows that the Lov\'asz hinge is inconsistent, Theorem~\ref{thm:lovasz-embed-main} illuminates why: when there is sufficient uncertainty around some subset of labels, the loss encourages the learning algorithm to ``hedge'' and set prediction coordinates for that set equal to 0.
This observation reveals a potential peril of using the Lov\'asz hinge in noisy label settings.
On the other hand, of course, this revelation suggests a positive result as well: under an appropriate link from our construction in Theorem~\ref{thm:eps-thick-calibrated}\raft{Which theorem number to ref?}, the Lov\'asz hinge could be fruitfully applied to set-valued prediction problems when one allows the learning algorithm to abstain on a subset of labels.

% \subsection{New surrogates for cost-sensitive classification}

% \raf{I think we should cut this subsection, and maybe just remind the reader at the prelude to this section that Theorem~\ref{thm:discrete-loss-poly-embeddable} gives a construction for any discrete loss.}
% The construction given in Theorem~\ref{thm:discrete-loss-poly-embeddable}, together with the calibrated link construction in Section~\ref{sec:calibration}, can generate consistent convex surrogates for arbitrary discrete losses.
% For example, consider cost-sensitive classification; given an arbitrary loss matrix $C \in \reals^{n\times n}_+$ describing the loss $\ell_j(i) = C_{ij}$ of predicting label $i$ when the true label is $j$, we construct a convex surrogate based on the convex conjugate of $-\risk{\ell}$.
% While of course several surrogates are known for this problem~\cite{pires2013cost}, they typically work in $n$ dimensions.
% %\raf{Jessie, could you cite \url{http://proceedings.mlr.press/v28/avilapires13.pdf} here}
% one advantage of our embedding framework is the ability to construct lower-dimensional surrogates when costs have certain geometric interpretations.
% For example, if $v_1,\ldots,v_n \in \reals^k$ are on the unit sphere and $C_{ij} = \inprod{v_i}{v_j}$, we can simply take $L_j(u) = $\raf{aaaand Noah woke up}\raf{Just realized that this loss matrix has rank $k$ and thus is covered by AA15: just elicit the linear property $\E v_Y \in \reals^k$.}

\subsection{Embedding Ordered Partitions}
\citet*{wang2020weston} use embeddings to construct and prove the calibration of the Weston-Watkins surrogate for the \emph{Ordered Partition} loss.

An ordered partition on $[n] := \{1, \ldots, n\}$ is an ordered list $S = (S_1, \ldots, S_l)$ of nonempty, pairwise disjoint sets such that $\cup_i S_i = [n]$, and assume $l \geq 2$.
The ordered partition target loss $\ell^{\OP}$ is defined, for $i \in [n]$ and $S = (S_1, \ldots, S_l) \in \OP_n$ by
\begin{align*}
\ell^\OP(S,y) &= |S_1| - 1 + \sum_{i=1}^{l-1} |S_1 \cup \ldots \cup S_{i+1}| \cdot \Ind{y \not \in S_1 \cup \ldots \cup S_i}~.~
\end{align*}
The summation in this ordered partition loss can be interpreted as a variation of 0-1 loss where one can report sets of sets, and increases the punishment for the outcome $y$ not being in the earlier sets in the ordering; outside the summation is a punishment for having a large prediction set on the first $S_1$.
This work presents an embedding for $\ell^\OP$ and uses this fact to show that the Weston-Watkins hinge loss~\citep{weston1999support} is in fact calibrated (and hence consistent, since the two are equivalent in Quadrant 1) with respect to $\ell^\OP$.

The Weston-Watkins hinge loss takes predictions $u \in \reals^n$ and assigns loss
\begin{equation}\label{eq:ww-hinge}
L(u,y) = \sum_{i \in \Y : i \neq y} (1 - (u_y - u_i))_+~.~
\end{equation}

\citet[Definition 2.3]{wang2020weston} takes the embedding $\varphi : \OP_n \to \reals^n$ entrywise so that for all $j \in S_i$, we have $\varphi(S)_j = -(i-1)$.

\begin{figure}[H]
	\centering
	\includegraphics[width=0.5\linewidth]{tikz/ordered-partition}
	\caption{Level sets of $\psi \circ \prop{WW}$, the property elicited by Weston Watkins hinge with embedding given by~\citet[Definition 2.3]{wang2020weston}.  
		$\prop{WW}$ is shown with faint dotted lines.}
	\label{fig:ordered-partition}
\end{figure}


\subsection{Inconsistency of top-$k$ losses}
In settings of search results and object recognition, one might plausibly be interested in top-$k$ classification: a generalization of $0-1$ loss where one can predict a set $S$ of size $k$, receiving loss $0$ if the true label $y \in S$, and $1$ otherwise.
While this problem is well-studied, \citep{lapin2015top, lapin2016loss, lapin2018analysis,yang2018consistency,berrada2018smooth,rastegari2011scalable} a class of surrogates for top-$k$, most of which are not simultaneously convex and consistent.
The only convex surrogate in the literature that is consistent is the softmax surrogate of~\citet{lapin2016loss} which essentially learns the entire distribution.
However, this seems to encapsulate more information than necessary and therefore presumably requires more data to learn the Bayes optimal hypothesis.
\jessie{Add more here}

Other hinge-like surrogates are shown to be non-consistent by~\citet{lapin2016loss} and~\citet{yang2018consistency}, including $\psi_2$ from~\citet[Table 1]{yang2018consistency}, which elicits essentially the same property $\prop{WW}$ shown in Figure~\ref{fig:ordered-partition} before the link is applied for certain values of $n$ and $k$.

\begin{align}\label{eq:topk-embedding}
L(u,y) &= \max \left(u_{[1]}, \max_{m \in k+1, \ldots, m} 1 - \frac k m + \frac 1 m \sum_{i=1}^m u_{[i]} \right)- u_y
\end{align}

\citet{TOPKPAPER} presents an embedding (Eq.~\eqref{eq:topk-embedding}) for top-$k$ loss that looks unlike the commonly studied surrogates in the literature.
Moreover, the most common link $\psi^{\tt argmax}(u) := \argmax_{S \subseteq \Y : |S| = k} \sum_{i \in S} u_i$ is an $\epsilon$-thickening of the embedding with $\epsilon = $ and the norm $\|\cdot\|_x$.\jessie{TODO... if this is true.}

%Commented out 07 Sept 21
%In certain classification problems when ground truth may be ambiguous, such as object identification, it is common to predict a set of possible labels.
%As one instance, the top-$k$ classification problem is to predict the set of $k$ most likely labels; formally, we have $\R := \{r \in \{0,1\}^n : \|r\|_0 = k\}$, $1<k<n$, $\Y = [n]$, and discrete loss $\elltopk(r)_y = 1-r_y$.
%Surrogates for this problem commonly take reports $u\in\reals^n$, with the link $\psi(u) = \{u_{[1]},\ldots,u_{[k]}\}$, where $u_{[i]}$ is the $i^{th}$ largest entry of $u$.
%
%\citet{lapin2015top, lapin2016loss, lapin2018analysis} provide the following convex surrogate loss for this problem, which \citet{yang2018consistency} show to be inconsistent:%
%\footnote{\citet{yang2018consistency} also introduce a consistent surrogate, but it is non-convex.}
%\begin{align}\label{eq:L-2-surrogate}
%L^k(u)_y := \left( 1-u_y + \tfrac{1}{k} \textstyle\sum_{i=1}^k (u - e_y)_{[i]} \right)_+~,
%\end{align}
%where $e_y$ is $1$ in component $y$ and 0 elsewhere.
%With our framework, we can say more.
%Specifically, while $(L^k,\psi)$ is not consistent for $\elltopk$, since $L^k$ is polyhedral (Lemma~\ref{lem:top-k-polyhedral}), we know from Theorem~\ref{thm:poly-embeds-discrete} that it embeds \emph{some} discrete loss $\ell^k$, and from Theorem~\ref{thm:eps-thick-calibrated} there is a link $\psi'$ such that $(L^k,\psi')$ is calibrated (and consistent) for $\ell^k$.
%We therefore turn to deriving this discrete loss $\ell^k$.
%
%For concreteness, consider the case with $k = 2$ over $n=3$ outcomes.
%We can re-write $L^2(u)_y = \left(1 - u_y + \tfrac 1 2 (u_{[1]}+ u_{[2]} - \min(1,u_y))\right)_+$.
%By inspection, we can derive the properties elicited by $\elltop{2}$ and $L^2$, respectively, which reveals that the set $\R'$ consisting of all permutations of $(1,0,0)$, $(1,1,0)$, and $(2,1,0)$, are always represented among the minimizers of $L^2$.
%Thus, $L^2$ embeds the loss $\ell^2(r)_y = 0$ if $r_y = 2$ or $\ell^2(r)_y = 1 - r_y + \tfrac 1 2 \inprod{r}{\ones-e_y}$ otherwise.
%Observe that $\ell^2$ is just $\elltop{2}$ with an extra term punishing weight on elements other than $y$, and a reward for a weight of 2 on $y$.
%
%Moreover, we can visually inspect the corresponding properties (Fig.~\ref{fig:top-k-simplices}) to immediately see why $L^2$ is inconsistent: for distributions where the two least likely labels are roughly equally (un)likely, the minimizer will put all weight on the most likely label, and thus fail to distinguish the other two.
%More generally, $L^2$ cannot be consistent because the property it embeds does not ``refine'' (subdivide) the top-$k$ property, so not just $\psi$, but \emph{no} link function, could make $L^2$ consistent.
%
%\begin{figure}[H]
%	\begin{minipage}{0.45\linewidth}
%		\centering
%		\includegraphics[width=\linewidth]{tikz/original-top-k}
%		\label{fig:original-top-k}
%	\end{minipage}
%	\hfill
%	\begin{minipage}{0.45\linewidth}
%		\centering
%		\includegraphics[width=\linewidth]{tikz/finite-surrogate-top-k}
%		\label{fig:finite-surrogate-top-k}
%	\end{minipage}
%	\caption{Minimizers of $\inprod{p}{\elltop{2}}$ and $\inprod{p}{\ell^2}$, respectively, varying $p$ over $\Delta_3$.}
%	\label{fig:top-k-simplices}
%\end{figure}



\btw{Hierarchical surrogate commented out; just mention in 7.1 since it builds off of that example.}
%\subsection{Embedding hierarchical classifications}
%Our final example comes from \citet{ramaswamy2015hierarchical}, who present a convex, calibrated surrogate for the hierarchical classification problem.
%Given a tree $H = ([n], E,W)$ over finite class labels, they consider the discrete loss 
%\begin{equation}\label{eq:tree-loss-discrete}
%	\ell^H(r,y) = \text{shortest path length in $H$ between $r$ and $y$}
%\end{equation}
%In this setting, the outcomes $\Y$ are composed of all nodes in the tree and not just the leaves.
%
%\begin{figure}
%	\centering
%		\includegraphics[width=0.7\linewidth]{tikz/example-tree-deeper.pdf}
%		\caption{Tree $H$ with distribution $\vec p = (0, .2, .1, .4, .3)$. $\gamma^H(\vec p) = 3$.}\label{fig:example-tree-deeper}
%\end{figure}
%
%
%\begin{figure}
%\begin{minipage}{0.45\linewidth}
%	\centering
%	\includegraphics[width=0.9\linewidth]{tikz/3-node-tree.pdf}
%	\caption{3 node tree whose property we evaluate in next image.}
%	\label{fig:3-node-tree}
%\end{minipage}
%\hfill
%\begin{minipage}{0.45\linewidth}
%	\centering
%	\includegraphics[width=0.9\linewidth]{tikz/3-node-tree-prop.pdf}
%	\caption{Property elicited by unweighted 3-node tree.}
%	\label{fig:3-node-tree-prop}
%\end{minipage}
%\end{figure}
%
%In~\cite[Theorem 1]{ramaswamy2015hierarchical}, Ramaswamy et al. show the property $\gamma^H := \prop{\ell^H}$ is the deepest node $i$ in the tree $H$ such that the probability that node $i$ or one of its descendants is the outcome, denoted $S_i(p)$, is greater than or equal to $1/2$.
%They additionally note that this property is agnostic to the weight $W$ of the tree.
%As before, consider $p_i$ to be the probability that node $i$ is the ground truth label, and $S_i(p)$ to be the probability that node $i$ or one of its descendants is the ground truth.
%For an example, see Figure~\ref{fig:example-tree-deeper}, where predicting node $3$ minimizes the expected tree loss over the distribution $\vec p = (0, 0.2, 0.1, 0.4, 0.3)$.
%
%Letting $h$ be the height of the tree $H$, Ramaswamy et al. present an embedding of $\ell^H$ that consists of learning $h$ $\abstain{1/2}$ embeddings, where the outcomes at the $j^{th}$ embedding are the nodes on level $j$ of the tree.
%At each highest level, one proceeds down the path of the tree given by the prediction of the current abstain embedding.
%If one predicts $\bot$ at level $j$, then we predict the node predicted at the $(j-1)^{st}$ level.
%
%Note that the given \emph{Cascade Surrogate} of~\citet[Equation 2]{ramaswamy2015hierarchical} simply requires the use of a surrogate that is calibrated with respect to $\abstain{1/2}$, so we can use the BEP surrogate of~\citet{ramaswamy2018consistent} to concatenate $h$ $\ceil{\log_2(n)}$ dimensional embeddings for the tree loss, yielding an embedding for $\ell^H$.
%
%It is worth noting that the cascade surrogate is not the always most efficient in terms of dimension.
%For example, the cascading surrogate on the 3 node tree $H$ in Figure~\ref{fig:3-node-tree} takes two $1$-dimensional optimization problems for distributions $p$ such that $p_1 < 1/2$.
%However, the property (Figure~\ref{fig:3-node-tree-prop}) elicited by tree distance for $H$ in Figure~\ref{fig:3-node-tree} is embeddable by a real-valued surrogate thanks to the characterization of \cite[\S~3]{finocchiaro2020embedding}, so one can see that the dimension cascade surrogate does not yield a tight bound.
%This leaves the question of embedding dimension open for the hierarchical surrogate embedding the discrete tree loss $\ell^H$.
%
%% \jessie{Don't think we need the below addition for the camera ready, but will eventually want something along these lines for the arXiv version.}
%% \proposedadd{To generalize this result, first consider that Lapin's surrogate is polyhedral, and we propose that each coordinate of an optimal report $u \in \reals_+^n$ must be at one of three points: $0$, $1$, or another ``high'' point that is a function of the number of high points and $M = |\{i : u_i = 1\}|$.  Since there are only a finite number of such optimal reports, $\R$, we know that this embeds a finite loss, namely $L^2_\R$.
%% \begin{align*}
%% \R := \left\{ \frac{|M| + k -1}{k - |H|} \mathbf{1}_H + \mathbf{1}_M : H, M \subset [n], H\cap M = \emptyset, |H| + |M| \leq k \right\}
%% \end{align*}
%% The discrete loss $\ell^{top-2} = L^2|_\R$ is then defined as follows:
%% \begin{equation}
%% \ell^{top-k}(r,y) = \begin{cases}
%% 0 & r_y = \bar r_{-y} + 1\\
%% \bar r - \frac 1 k & r_y = 1\\
%% 1 + \bar r & r_y = 0
%% \end{cases}~,~
%% \end{equation}
%% where $\bar r = \frac 1 k \sum_{i=1}^k r_{[i]}$.
%% }

%%%%%%%%%%%%%%%%%%%%%%%%%%%%%%%%%%%%%%%%%%%%%%%%%%%%%%%%%%%%%%%%%%%%%%
%% Jessie's version here 7/16/2019

% Figure~\ref{fig:top-k-simplices} (left) gives a map of the property values of $\gamma^{\text{top-2}} := \prop{\elltop{2}}$ over $\Delta_3$.
% Similarly, for every distribution $p \in \Delta_3$, $\inprod{p}{L^2}$ has a minimizer in the set $\R'$ composed of the tuples shown in Figure~\ref{fig:top-k-simplices} (right), and thus $\risk{L^2} = \risk{\ell^2}$, where $\ell^2 := L^2|_{\R'}$, so Figure~\ref{fig:top-k-simplices} (right) shows $\gamma^2 := \prop{\ell^2}$.
% %Therefore, we know $L^2$ embeds $\ell^2$ by Proposition~\ref{prop:embed-bayes-risks}.
% %(Each cell in this simplex is a level set $\gamma^{\text{top-2}}_r$ and corresponds to a flat piece of the Bayes Risk $\risk{\elltop{2}}$.)

% By visual inspection, we can observe there is no ``refining'' mapping from $\gamma^{\text{top-2}}$ to $\gamma^2$.
% In refining, we loosely mean that, for some $j \in \R$, there is no $I \subseteq \reals^n$ such that $\cup_{i \in I}\gamma^2_i = \gamma^{\text{top-2}}_{j}$.
% %If such a mapping existed, it would allude to one possible consistent link $\psi$ by which $L^2$ embeds $\elltop{2}$.
% Conversely, we can ``refine'' $\prop{L^2}$ down to $\prop{\ell^2}$ by using the axiom of choice to construct $\R'$.

% \jessie{Not sure if this is too much detail...}
% Since $\prop{\elltop{2}}$ does not refine $\prop{L^2}$, then $L^2$ does not embed $\elltop{2}$.
% For example, we can take distributions $p_1 \in \gamma^{top-2}_{\{1,2\}}$ and $p_2 \in \gamma^{top-2}_{\{1,3\}}$ such that $p_1, p_2 \in \Gamma_{(1,0,0)}$.
% Regardless of $\psi((1,0,0))$, for either $p_1$ or $p_2$, we will observe $\psi((1,0,0)) \not \in \argmin_u \inprod{p_i}{\elltop{2}(u)}$.
% Thus, we see that $L^2$ is not a consistent surrogate for $\elltop{2}$.


%%%%%%%%%%%%%%%%%%%%%%%%%%%%%% old version submitted to NuerIPS%%%%%%%%%%%%%%%%%%%%%%%%%%%%%%%%%%
%\btw{The story we discussed: YK show $L'$ is inconsistent.  Our framework lets us ask: with what discrete loss \emph{is} $L'$ consistent?  (It also assures us that $L'$ is consistent for something...) Oh look, $L'$ embeds something which is $\elltopk$ plus some other term, so (a) we can see very clearly what $L'$ ``is doing'', and (b) we can see where to look for distributions yielding inconsistency, namely ones for which we actually prefer to report a set with $|r|<k$.}
%
%In certain classification problems, for example in information retrieval, it is common to predict a set of possible labels.
%As one instance, for $k<n$ the top-$k$ classification problem has reports $\R := \{r \subseteq [n] : |r| = k\}$, with label $y \in [n]$.
%The natural discrete loss $\elltopk:\R \to \reals^\Y_+$ is given by
%\begin{align}\label{eq:top-k}
%  \elltopk(r,y) = \Ind{y \not \in r}~,
%\end{align}
%which simply gives a penalty if the label was not in the reported set.
%
%Surrogates for this problem commonly take reports $u\in\reals^n$, with the link $\psi(u) = \{u_{[1]},\ldots,u_{[k]}\}$, where $x_{[u]}$ is the $i^{th}$ largest component of $u$, with ties broken arbitrarily.
%\citet{lapin2015top, lapin2016loss, lapin2018analysis} provide the following convex surrogate loss for this problem, which \citet{yang2018consistency} show to be inconsistent:%
%\footnote{Yang and Koyejo also introduce a consistent surrogate, but it is non-convex.}
%\begin{align}\label{eq:L-2-surrogate}
%  L'(u)_y := \left( \tfrac{1}{k} \textstyle\sum_{i=1}^k (u + \ones - e_y)_{[i]} - u_y \right)_+~,
%\end{align}
%where $e_y$ is $1$ in component $y$ and 0 elsewhere.
%
%With our framework, we can say more.
%Specifically, while $(L',\psi)$ is not consistent for $\elltopk$, since $L'$ is polyhedral (Lemma~\ref{lem:top-k-polyhedral}), we know from Theorem~\ref{thm:poly-embeds-discrete} that it embeds \emph{some} discrete loss $\ell'$, and from Theorem~\ref{thm:eps-thick-calibrated} there is a link $\psi'$ such that $(L',\psi')$ is calibrated (and consistent) for $\ell'$.
%We therefore turn to deriving this discrete loss $\ell'$.
%
%In Lemma~\ref{lem:top-k-optimal-corners}\jessiet{in the Appendix... remove ref for camera ready version}, we show that the set $\U = \{u \in \{0,1\}^n : \|u\|_0 \leq k\}$ is always represented among the optimizers of $L'$, meaning for all $p \in \simplex$ we have $\prop{L'}(p) \cap \U \neq \emptyset$.
%From Lemma~\ref{lem:loss-restrict}\raft{was Lemma~\ref{lem:top-k-surrogate-embeds}}, then, $L'$ must embed $L'|_\U$, which gives us $\ell':\R'\to\reals^\Y_+$ via the natural embedding from sets $\R' = \{r \subseteq [n] : |r| \leq k\}$ to their indicator vectors:
%\begin{align}\label{eq:ell-2}
%\ell'(r)_y &= \Ind{y \not\in r} + \tfrac 1 k |r \setminus \{y\}| = (1+\tfrac 1 k)\Ind{y \not\in r} + \tfrac 1 k (|r|-1)~.
%\end{align}
%We now see that $\ell'$ is essentially $\elltopk$ (extended to sets smaller than $k$) plus an additional cardinality term which rewards smaller sets.
%
%Knowledge of what loss $L'$ actually embeds greatly simplifies the task of proving inconsistency with $\elltopk$.
%Specifically, we see that $\ell'$ allows sets of cardinality strictly less than $k$, so we can look for a distribution making one of these smaller sets optimal.
%Writing the expected loss, we have
%%$\inprod{p}{\ell'(r)} = 1 - p(r) + \tfrac 1 k (|r|-p(r)) = 1 + \tfrac 1 k |r| - (1 + \tfrac 1 k) p(r)$,\\
%$\inprod{p}{\ell'(r)} = (1+\tfrac 1 k)(1 - p(r)) + \tfrac 1 k (|r|-1) = 1 + \tfrac 1 k |r| - (1 + \tfrac 1 k) p(r)$, where $p(r) = \sum_{i\in r} p_i$.
%% It is now easy to see that the optimal $r$ for any fixed cardinality $|r|=c$ is the $c$ most likely labels.
%So let us ask when it is better to drop an element $i$ from $r$:
%% If $i$ is the index of the $c$th most likely label (so $p_i = p_{[c]}$)
%we have $\inprod{p}{\ell'(r)-\ell'(r\setminus\{i\})} = \tfrac 1 k - (1+\tfrac 1 k)p_i$, meaning we will drop elements from $r$ until they all have weight at least $\tfrac 1 {k+1}$.
%In particular, for distributions close to uniform, $r=\emptyset$ is optimal, already giving us inconsistency\raft{If space (nope) we should spell this out like Jessie did: however ties are broken at $u=0$, pick a distribution with bumps going against the ties, and $u=0$ is still optimal.}.
%More generally, as $k < n$, the set $P = \{p\in\simplex : \max_{r\in\R'} p(r) < k/(k+1)\}$ is full-dimensional and guarantees that at least one of the top $k$ labels has probability strictly less than $\tfrac 1 {k+1}$.
%
%% We now focus on the case where $k=1$; the top-$k$ loss reduces to 0-1 loss in this case.
%% To show a set of distributions where $\prop{\ell'}(p) \neq \prop{\elltopk}(p)$, we claim that $\vec{0}~\in~\prop{\ell'} \implies \risk{\ell'}(p) = 1$ for any $p$ with $\max_y p_y \leq 1/2$.
%% Additionally $\risk{\ell_{top-1}}(p) = 1 - \max_y p_y < 1 \implies \vec{0} \not\in \prop{\elltopk}(p)$.
%
%% First, $\ell(\vec{0}, \bar{p}) = 1$ for all $p \in \simplex$.
%% Taking the set $\{y\}$ for any $y \in \Y$, the expected loss $\inprod{p}{\ell'(\{y\})} = 2 (1-p_y) \geq 1$ by assumption on $p$.
%% %For any $r$ such that $|r| \geq 2$ \jessiet{In general, $|r| \geq k+1$}, we know that  $\ell'(r)_y \geq 1$ for all $y \in \Y$, so the expected loss much be greater than or equal to $1$.
%% Therefore, we have that $\vec{0} \in \prop{\ell'}(p)$ but $\vec{0} \not\in \prop{\elltopk}(p)$ for any $p$ such that $\max_y p_y \leq 1/2$.
%
%
%
%%Since we can see that $\ell' \neq \elltopk$, we can then see that $L'$ is not consistent with respect to $\elltopk$.
%%Similarly, given $L_4(u,y) := \max\left(\frac{1}{k} \sum_{i=1}^k (1 + u_{\setminus y})_{[i]} - s_y, 0\right)$, we can actually see that $L_4$ also embeds $\ell'$.
%%We will double up on notation and call this discrete loss $\ell_4$ as well, for the sake of matching subscripts of discrete losses to the surrogates that embed them.

%%%%%%%%%%%%%%%%%%%%%%%%%%%%%%%%%%%%%%%%%%%%%%%%%%%%%%%%%%%%%%%



%%%%%%%%%%%%%%%%%%%%%%%%%%%%%%%%%%%%%%%%%%%%%%%%%%%%%%%%%%%%%%%

\section{Additional Structure of Embeddings}
\label{sec:min-rep-sets}

\raf{Brief overview here}


\subsection{Structure of polyhedral Bayes risks}

While we have focused on polyhedral losses thus far, it turns out that many of our results about embeddings extend to losses with polyhedral Bayes risks.
(We say a concave function is polyhedral if its negation is a polyhedral convex function.)
To see that this is a weaker condition, recall that Theorem~\ref{thm:poly-embeds-discrete} constructs a finite representative set $\Sc$ for any polyhedral loss $L$, and thus $\risk{L} = \risk{L|_\Sc}$ by Lemma~\ref{lem:loss-restrict}, which is polyhedral.
The condition is strictly weaker, in that the Bayes risk may be polyhedral even if the loss itself is not.
For example, a modified hinge loss $L(r)_y = \max(r^2-1,1-ry)$ as shown in Figure~\ref{fig:modified-hinge}, which matches hinge loss on the interval $[-1,1]$ but is smooth outside the interval $[-2,2]$, still embeds twice 0-1 loss.

\begin{figure}
	\begin{minipage}{0.47\linewidth}
		\centering
		\includegraphics[width=0.95\linewidth]{figs/modhinge.pdf}
		%		\caption{Bayes risk of the 0-1 loss.}
		%		\label{fig:0-1-br}
	\end{minipage}
	\hfill
	\begin{minipage}{0.47\linewidth}
		\centering		\includegraphics[width=0.95\linewidth]{figs/modhinge-br.pdf}
		%		\caption{Bayes risk of hinge loss.}
		%		\label{fig:hinge-br}
	\end{minipage}
	\caption{(L) Modified (expected) hinge loss for fixed distribution, and (R) Bayes risk of modified hinge still matches the Bayes risk of hinge}
	\label{fig:modified-hinge}
\end{figure}



% \begin{lemma}
% 	\label{lem:poly-loss-poly-risk}
% 	If $L$ is polyhedral, $\risk{L}$ is polyhedral.
% \end{lemma}
% \begin{proof}
% 	Let $L:\reals^d\to\reals_+^\Y$ be a polyhedral loss, and $\Gamma = \prop{L}$.
% 	By Lemma~\ref{lem:polyhedral-range-gamma}, $\U = \Gamma(\simplex)$ is finite. 
% 	For each $U\in \U$, select $u_U \in U$, and let $\Sc = \{u_U : U \in\U\}$.
% 	$\Sc$ is representative for $L$, so Lemma~\ref{lem:loss-restrict} gives us $\risk{L} = \risk{L|_{\Sc}}$, which is polyhedral as $\Sc$ is finite.
% \end{proof}

We now present our main structural result in Lemma~\ref{lem:X}, which will lay the foundation for the rest of this section.
Lemma~\ref{lem:X} observes that a (minimizable) loss $L$ with polyhedral Bayes risk have finite representative sets, and derives equivalent conditions on the level sets of the property elicited by $L$ and tight embeddings.
\raf{TODO: Brief description of the result}\jessie{Attempted}

\begin{lemma}[\jessie{Lemma X}]\label{lem:X}
  Let $L: \R \to \reals^\Y_+$ be a minimizable loss with a polyhedral Bayes risk $\risk L$.
  Then $L$ has a finite representative set.
  Furthermore, letting $\Gamma = \prop{L}$, there exist finite sets
  $\V \subseteq \reals^\Y_+$ and
  $\Theta = \{\theta_v \subseteq \simplex \mid v\in\V\}$,
  both uniquely determined by $\risk{L}$ alone,
  such that
  \begin{enumerate}
  % \item There exists $\V = \V(\risk{L}) \subseteq \reals^\Y_+$, uniquely determined by $\risk{L}$ alone, such that $\V = L(\R^*)$.\label{item:X-V}
  \item A set $\R'\subseteq\R$ is representative if and only if $\V \subseteq L(\R')$.\label{item:X-rep-V}
  \item A set $\R'\subseteq\R$ is minimum representative if and only if $L(\R') = \V$.\label{item:X-min-V}
  \item A set $\R'\subseteq\R$ is representative if and only if $\Theta \subseteq \{\Gamma_r \mid r \in \R'\}$.\label{item:X-rep-Theta}
  \item A set $\R'\subseteq\R$ is minimum representative if and only if $\{\Gamma_r \mid r \in \R'\} = \Theta$.\label{item:X-min-Theta}
  \item Every representative set for $L$ contains a minimum representative set for $L$.\label{item:X-rep-contain-min}
  \item The set of full-dimensional level sets of $\Gamma$ is exactly $\Theta$.\label{item:X-full-dim}
  \item For any $r \in \R$, there exists $\theta \in \Theta$ such that $\Gamma_r \subseteq \theta$.\label{item:X-redundant}
  \item $L$ tightly embeds $\ell:\R'\to\reals^\Y_+$ if and only if $\ell$ is injective and $\ell(\R') = \V$.\label{item:X-tight-embed}
  \end{enumerate}
\end{lemma}

As a finite representative set implies a polyhedral Bayes risk by Lemma~\ref{lem:loss-restrict}, Lemma~\ref{lem:X} shows that polyhedral Bayes risks are equivalent to having finite representative sets, which in turn gives an embedding by
Proposition~\ref{prop:representative-embeds-restriction}.
\begin{corollary}\label{cor:poly-risk-fin-rep}
  The following are equivalent for any minimizable loss $L:\R\to\reals^\Y_+$.
  \begin{enumerate}
  \item $\risk{L}$ is polyhedral.
  \item $L$ has a finite representative set.
  \item $L$ embeds a discrete loss.
  \end{enumerate}
\end{corollary}
From Corollary~\ref{cor:poly-risk-fin-rep}, $L$ having a finite representative set is an equivalent condition to $L$ being minimizable and $\risk{L}$ being polyhedral.
(Recall that having a finite representative set already implies minimizability.)
As it is also a more succinct condition, we will use the former in the sequel.
In particular, the implications of Lemma~\ref{lem:X} follow whenever $L$ has a finite representative set.


\subsection{Equivalent condition: matching Bayes risks}\label{subsec:match-BR}


Lemma~\ref{lem:X} leads to another appealing equivalent condition to our embedding condition in Definition~\ref{def:loss-embed}: a surrogate embeds a discrete loss if and only if their Bayes risks match.


\begin{proposition}\label{prop:embed-bayes-risks}
  Let discrete loss $\ell$ and minimizable loss $L$ be given.
  Then $L$ embeds $\ell$ if and only if $\risk{L}=\risk{\ell}$.
  % Let discrete loss $\ell:\R\to\reals^Y$ be given.
  % Then $L:\reals^d\to\reals^\Y$ embeds $\ell$ if and only if $\risk{L}=\risk{\ell}$.
  % A loss $L$ embeds a discrete loss $\ell$ if and only if $\risk{L}=\risk{\ell}$.
\end{proposition}
\begin{proof}
  Define $\Gamma = \prop{L}$ and $\gamma = \prop{\ell}$.
  Suppose $L$ embeds $\ell$, so we have some $\Sc\subseteq \R$ which is representative for $\ell$ and an embedding $\varphi:\Sc\to\reals^d$; take $\U := \varphi(\Sc)$.
  Since $\Sc$ is representative for $\ell$, by embedding condition (ii) we have $\{\gamma_s \mid s\in\Sc\} = \{\Gamma_u \mid u\in\U\}$, so $\U$ is representative for $L$.
  % for all $p\in\simplex$ we have some $r \in \Sc \cap \gamma(p)$ and thus we have $\varphi(r) \in \U \cap \Gamma(p)$.
  % In particular, $\U \cap \Gamma(p)\neq \emptyset$ for all $p\in\simplex$, 
  By Lemma~\ref{lem:loss-restrict}, we have $\risk{\ell} = \risk{\ell|_{\Sc}}$ and $\risk{L} = \risk{L|_{\U}}$.
  As $L(\varphi(\cdot)) = \ell(\cdot)$ by embedding condition (i), for all $p\in\simplex$ we have
  \begin{equation*}
    \risk{\ell}(p) = \risk{\ell|_\Sc}(p) = \min_{r \in \Sc}\inprod{p}{\ell(r)} = \min_{r \in \Sc}\inprod{p}{L(\varphi(r))} = \min_{u \in \U}\inprod{p}{L(u)} = \risk{L|_\U}(p) = \risk{L}(p)~.
  \end{equation*}
  
	For the reverse implication, assume $\risk{L} = \risk{\ell}$, which are polyhedral functions as $\ell$ is discrete.
  From Lemma~\ref{lem:X}(\ref{item:X-min-V}), we have some set $\V\subseteq\reals^\Y_+$ and minimum representative sets $\R^* \subseteq \R$ and $\U^* \subseteq \U$, for $\ell$ and $L$ respectively, such that $\ell(\R^*) = \V = L(\U^*)$.\jessiet{This is a different $\Sc$ than the first paragraph, right? The first $\Sc$ is representative for $\ell$.}\raft{They are both quantified separately, so not linked; are you saying it's confusing to use $\Sc$ here though because its role is slightly different?}\jessiet{Yes... attempting to clarify.}
  As $\R^*$ and $\U^*$ are miniumum, they cannot repeat loss vectors, and thus $|\R^*|=|\ell(\R^*)|$ and $|L(\U^*)|=|\U^*|$.
  We conclude that $\R^*$ and $\U^*$ are both in bijection with $\V$.
  The map $\varphi :\R^* \to \reals^d$, given by $\varphi(r) = u \in \U^*$ where $\ell(r) = L(u)$, is therefore well-defined.
  Condition (i) of an embedding is immediate.
  From Proposition~\ref{prop:representative-embeds-restriction}, $\ell$ embeds $\ell|_{\R^*}$ and $L$ embeds $L|_{\U^*}$, both via the identity embedding.
  Using condition (ii) from both embeddings, for all $p\in\simplex$ and $r\in\R^*$, we have
  \begin{equation*}
    r \in \gamma(p) \iff r \in \prop{\ell|_{\R^*}}(p) \iff \varphi(r) \in \prop{L|_{\U^*}}(p)
    \iff \varphi(r) \in \prop{L}(p)~,
  \end{equation*}
  giving condition (ii).
\end{proof}

%\btw{RF: matching risks is not necessary for \emph{consistency}, as evidenced by logistic loss for 0-1 loss.  Maybe we should just make the observation, earlier on, that \emph{embedding} is not necessary for consistency. \jessie{Added a few sentences above, when talking about consistency vs calibration.}}

%\btw{RF: DKR (section 3.1) realized the importance of matching Bayes risks, but they could only give general results for strictly convex (concave I should say) risks, in part because they fixed the link function to be a generalization of $\sgn$.  In contrast, we focus exclusively on non-strictly-convex risks.}

Previous work from~\citet[Proposition 4]{duchi2018multiclass} realized the significance of matching Bayes risks for calibration with respect to the 0-1 loss.
Proposition~\ref{prop:embed-bayes-risks} broadens this general insight to any discrete loss.
Moreover, their result relies the Bayes risk of the surrogate being strictly concave, whereas polyhedral Bayes risks are never strictly concave.

\subsection{Trimming a loss}

Central to the structural results in Lemma~\ref{lem:X} is the existence of a canonical set of loss vectors $\V$ which match the loss vectors of any minimum representative set.
This fact may seem surprising when one considers that losses may have many mimimum representative sets.
For example, consider hinge loss with a spurious extra dimension, i.e., $L:\reals^2\to\reals^\Y$, $L(r)_y = \max(0,1-r_1y)$ for $\Y = \{-1,+1\}$.
Here the minimum representative sets are $\{(-1,a),(1,b)\}$ for any $a,b\in\reals$. 
Lemma~\ref{lem:X}(\ref{item:X-min-V}) states that, while the minimum representative set is not unique, its loss vectors are.

Motivated by this observation, let us define the ``trim'' of a loss to be this unique set $\V$ of loss vectors induced by any minimum representative set, which again is well-defined by Lemma~\ref{lem:X}(\ref{item:X-min-V}).
\begin{definition}[Trim]\label{def:trim-loss}
  Given a loss $L:\R \to \reals_+^\Y$ with a finite representative set, we define $\trimcover(L) = \{L(r) \mid r \in \R^*\}$ given any minimum representative set $\R^*$ for $L$.
\end{definition}


Using this notion of trimming a loss, we can again recast our embedding condition: a loss embeds another if and only if they have the same $\trimcover$.

\begin{proposition}\label{prop:embed-iff-trims-equal}
  Let $L:\reals^d\to\reals^\Y_+$ have a finite representative set, and let $\ell:\R\to\reals^\Y_+$ be a discrete loss.
  Then $L$ embeds $\ell$ if and only if $\trimcover(L) = \trimcover(\ell)$.
  Furthermore, $L$ tightly embeds $\ell$ if and only if $\ell$ is injective and $\trimcover(L) = \ell(\R)$.
\end{proposition}
\begin{proof}
  \raft{This is sort of a ``clever'' proof, probably not in a good way.  Something that used Lemma X more directly would probably be better than using Bayes risk matching.  But if correct I think it's not bad. \jessie{It's a little subtle in parts, but checks out to me.}}
  As $L$ has a finite representative set, it is minimizable.
  Proposition~\ref{prop:embed-bayes-risks} gives $L$ embeds $\ell$ if and only if $\risk L = \risk \ell$.
  If $\risk L = \risk \ell$, Lemma~\ref{lem:X}(\ref{item:X-min-V}) gives $\trim(L) = \trim(\ell)$.
  For the converse, suppose $\trim(L) = \trim(\ell) =: \V$.
  Define the discrete loss $\ell_\trim : \V \to \V, v\mapsto v$.
  Then $\ell_\trim$ is injective and $\ell_\trim(\V) = \V$, so from Lemma~\ref{lem:X}(\ref{item:X-tight-embed}), both $L$ and $\ell$ tightly embed $\ell_\trim$.
  We conclude $\risk L = \risk{\ell_\trim} = \risk \ell$ from Proposition~\ref{prop:embed-bayes-risks}.
  The second statement also follows directly from Lemma~\ref{lem:X}(\ref{item:X-tight-embed}).
\end{proof}


\subsection{Minimum representative sets and non-redundancy}

The condition that a representative set be minimum implies that one has identified exactly the ``active'' reports of a loss, in some sense.
We now relate this condition to another natural notion from the property elicitation literature, that of non-redundancy~\cite{frongillo2014general,lambert2018elicitation}.
Intuitively, a loss is non-redundant if no report is weakly dominated by another report.

\begin{definition}[Non-redundancy]\label{def:nonredundant}
  A loss $L : \R \to \reals^\Y_+$ eliciting $\Gamma:\simplex \toto \R$ is \emph{redundant} if there are reports $r, r' \in \R$ such that $\Gamma_r \subseteq \Gamma_{r'}$, and \emph{non-redundant} otherwise.
\end{definition}

From the structural result of Lemma~\ref{lem:X}, we can see that in fact these two notions are equivalent when $L$ has a polyhedral Bayes risk.
\begin{proposition}\label{prop:tfae-min-rep-nonredundant}
  Let $L:\R\to\reals^\Y_+$ have a finite representative set $\R'$.
  Then $\R'$ is a minimum representative set for $L$ if and only if $L|_{\R'}$ is non-redundant.
\end{proposition}
\begin{proof}
  Let $\Gamma = \prop{L}$.
  Suppose first that $L|_{\R'}$ is redundant.
  Then there exist $r,r' \in \R'$ such that $\Gamma_r \subseteq \Gamma_{r'}$.
  Thus, for all $p \in \Gamma_r$, we have $\{r, r'\} \subseteq \Gamma(p)$.
  Therefore $\R' \setminus \{r\}$ still a representative set, so $\R'$ is not minimum.

  Now suppose $L|_{\R'}$ is non-redundant.
  As $\R'$ is a representative set, Lemma~\ref{lem:X}(\ref{item:X-rep-contain-min}) gives some minimum representative set $\Sc \subseteq \R'$.
  Suppose we had some $r \in \R' \setminus \Sc$.
  Now Lemma~\ref{lem:X}(\ref{item:X-min-Theta},\ref{item:X-redundant}) gives some $s\in\Sc$ such that $\Gamma_r \subseteq \Gamma_s$, which contradicts $L|_{\R'}$ being non-redundant.
  We conclude $\Sc=\R'$, meaning $\R'$ is a minimum representative set.
\end{proof}

\begin{corollary}\label{cor:tight-embed-min-rep}
  Let loss $L:\R\to\reals^\Y_+$ with finite representative set $\R'$ be given.
  Then $L$ tightly embeds $L|_{\R'}$ if and only if $L|_{\R'}$ is non-redundant.
\end{corollary}

In fact, we can show something stronger: the reports in minimum representative sets are precisely those which are not strictly redundant.
To formalize this statement, given $\Gamma : \simplex \toto \R$, let $\red(\Gamma) := \{r\in\R \mid \exists r'\in\R,\; \Gamma_r \subsetneq \Gamma_{r'}\}$ be the set of strictly redundant reports.
Similarly, for minimizable $L$, let $\red(L) := \red(\prop L)$.

\begin{proposition}
  Let $L : \R \to \reals^\Y_+$ have a finite representative set.
  Let $\R'$ be the union of all minimum representative sets for $L$.
  Then $\R' = \R \setminus \red(L)$.
  \raft{I was a bit surprised how involved this proof ended up being.  Quite possibly there is a faster way, though at the same time, I do like how the proof shows something stronger.  Old version is in before-reorg-sec-3.tex.\jessie{It doesn't feel that involved as a reader, though maybe I'm not clear what you mean by ``involved''}}
\end{proposition}

\begin{proof}
  Let $\Gamma = \prop L$.
  Let $\Sc$ be a minimum representative set for $L$, and let $s\in\Sc$.
  Suppose for a contradiction that $s\in\red(\Gamma)$.
  Then we have some $r\in\R$ with $\Gamma_s \subsetneq \Gamma_r$.
  From Lemma~\ref{lem:X}(\ref{item:X-min-Theta},\ref{item:X-redundant}) we have some $s'\in\Sc$ such that $\Gamma_r \subseteq \Gamma_{s'}$.
  But now $\Gamma_s \subsetneq \Gamma_r \subseteq \Gamma_{s'}$, contradicting $\Sc$ being minimum representative.
  Thus $\Sc \subseteq \R \setminus \red(\Gamma)$.

  For the reverse inclusion, let $r\in\R\setminus\red(\Gamma)$.
  Let $\Sc$ again be a minimum representative set for $L$.
  From Lemma~\ref{lem:X}(\ref{item:X-min-Theta},\ref{item:X-redundant}), we have some $s\in\Sc$ such that $\Gamma_r \subseteq \Gamma_s$.
  By definition of $\red(L)$, we conclude $\Gamma_r = \Gamma_s$.
  Now take $\Sc' = (\Sc \setminus \{s\}) \cup \{r\}$, that is, the same set of reports with $r$ replacing $s$.
  We have $\{\Gamma_s \mid s\in\Sc\} = \{\Gamma_{s'} \mid s'\in\Sc'\}$, and thus $\Sc'$ is minimum representative for $L$ by Lemma~\ref{lem:X}(\ref{item:X-min-Theta}).
  As $r\in\Sc'$, we have $r \in \R'$ and we are done.
\end{proof}


As a corollary, we can state another characterization of $\trim$ in terms of redundant reports.
The result follows immediately from the definition of $\trim$.

\begin{corollary}\label{cor:trim-loss-red}
  Let $L : \R \to \reals^\Y_+$ have a finite representative set.
  Then $\trimcover(L) = L(\R \setminus \red(L))$
  % $\{L(r) \mid r \notin \red(\Gamma)\}$.
  % \in \R,\; \forall r'\in\R,\; \Gamma_r = \Gamma_{r'} \textnormal{ or } \Gamma_r \not\subseteq \Gamma_{r'}\}$
  % $\mid u \in \R \textrm{ s.t. } \neg\exists u'\in\R,u'\neq u,\, \Gamma_u \subsetneq \Gamma_{u'}\}$
\end{corollary}



\subsection{A property elicitation perspective on trimmed losses}
\jessie{Should this be a subsubsection?}

We conclude this section with a similar structural result about the properties elicited or embedded by another property.
We say a property $\Gamma:\simplex\toto\reals^d$ embeds a finite property $\gamma:\simplex\toto\R$ if condition (ii) of Definition~\ref{def:loss-embed} holds.
In other words, $\Gamma$ embeds $\gamma$ if we have some representative set $\Sc\subseteq\R$ for $\gamma$ and embedding $\varphi:\Sc\to\reals^d$ such that for all $s\in\Sc$ we have $\gamma_s = \Gamma_{\varphi(s)}$.
% The definitions of representative sets and embeddings naturally extend to properties.
% We say a set 
% As we define embeddings from one loss to another, one can define embeddings from one property to another, where essentially condition (i) is relaxed; see \S~\ref{app:embed-props} for a precise definition.
Furthermore, when working with properties, it is more natural to define ``trim'' to remove any level sets which are strictly redundant.
Precisely, we define $\trimred(\Gamma) := \{\Gamma_r \mid r \in \R\setminus\red(\Gamma)\}$.
%\R \textrm{ s.t. } \neg\exists r'\in\R,r'\neq r,\, \Gamma_r \subsetneq \Gamma_{r'}\}~.\]

With these definitions, we now state a structural result for embeddable properties, and the properties that embed them.
First, if $\Gamma$ embeds $\gamma$ and $\gamma$ is non-redundant, the level sets of $\Gamma$ must all be redundant relative to $\gamma$.
In other words, $\Gamma$ is exactly the property $\gamma$ \jessie{up to relabelling reports}, just with other reports filling in the gaps between the embedded reports of $\gamma$.
When working with convex losses, these extra reports often arise as the convex hull of the embedded reports.
In this sense, we can regard embedding as only a slight departure from direct elicitation: if a loss $L$ elicits $\Gamma$ which embeds $\gamma$, we can think of $L$ as essentially eliciting $\gamma$ itself.
Finally, we have an important converse: if $\Gamma$ has finitely many full-dimensional level sets, or if $\trimred(\Gamma)$ is finite, then $\Gamma$ must embed some finite elicitable property with those same level sets.

We will make use of another corollary of Proposition~\ref{prop:embed-iff-trims-equal}, stated for properties.
\begin{corollary}\label{cor:trim-prop-red}
  Let $\Gamma : \simplex \toto \R$ be an elicitable property with a finite representative set.
  Then $\trimred(\Gamma)$ is the set of full-dimensional level sets of $\Gamma$.
\end{corollary}
\begin{proof}
  Let $L$ elicit $\Gamma$.
  From Lemma~\ref{lem:X}(\ref{item:X-min-Theta},\ref{item:X-full-dim}), for any finite minumum representative set $\Sc\subseteq\R$, the set $\{\Gamma_s\mid s\in\Sc\}$ is exactly the set of full-dimensional level sets $\Theta$ of $\Gamma$.
  From Proposition~\ref{prop:tfae-min-rep-nonredundant}, we have $r \in \R\setminus \red(\Gamma)$ if and only if $r$ is an element of some minimum representative set.
  As $\Gamma$ has at least one minimum representative set, we conclude $\trimred(\Gamma) = \{\Gamma_r \mid r\in \R\setminus\red(\Gamma)\} = \Theta$.  
\end{proof}

\jessie{Adding in small motivation of Prop 6}
This allows us to think about the structure of properties elicited by losses with finite representative sets in the following equivalent ways through $\trimred(\Gamma)$.

\begin{proposition}\label{prop:embed-trim}
  Let $\Gamma:\simplex\toto\reals^d$ be an elicitable property.\jessiet{$\reals^d$ vs $\R$?}\raft{Can't recall why we needed $\reals^d$.  Probably don't... in any case, can't use $\R$ since that clashes with $\gamma$.}
  The following are equivalent:
  \begin{enumerate}\setlength{\itemsep}{0pt}
  \item $\Gamma$ embeds a elicitable finite property $\gamma:\simplex \toto \R$.
  \item $\trimred(\Gamma)$ is a finite set.%, and $\cup\,\trimred(\Gamma) = \simplex$.
  % \\ $\hat\Gamma := \trimred(\Gamma)$ is a finite property, and $\cup\,\hat\Gamma) = \simplex$.
  \item There is a finite minimum representative set $\U$ for $\Gamma$.
  \item There is a finite set of full-dimensional level sets $\hat\Theta$ of $\Gamma$, and $\cup\,\hat\Theta = \simplex$.
  \end{enumerate}
  % \raft{I see why you wanted this addition here, but (a) it's clear from the def of embedding, and (b) we don't have varphi defined.  So let's leave it out.}
  Moreover, when any of the above hold, $\trimred(\gamma) = \trimred(\Gamma) = \{\Gamma_u \mid u\in\U\} = \hat\Theta$.
\end{proposition}

\begin{proof}
  Let $L$ be a fixed loss eliciting $\Gamma$, so that in particular $\risk L$ is fixed.
  By definition of elicits, $L$ is minimizable.
  In each case, we will show that $\risk L$ is polyhedral, and thus Lemma~\ref{lem:X} will give us the set $\Theta$ of full-dimensional level sets of $\Gamma$, uniquely determined by $\risk L$.
  We will prove $1 \Rightarrow 2 \Rightarrow 3 \Rightarrow 4 \Rightarrow 1$, and in each case show that the relevant set of level sets is equal to $\Theta$, giving the result.

  Case 1:
  Let $\Sc$ be the representative set for $\gamma$ and $\varphi:\Sc\to\reals^d$ the embedding.
  Since $\Sc$ is finite, $\varphi(\Sc)$ is a finite representative set for $\Gamma$ (and $L$).
  Corollary~\ref{cor:trim-prop-red} now gives $\trimred(\Gamma) = \Theta$, which is finite, showing Case 2.

  Case 2:
  \raf{Somehow I had like 5 typos in here which made it impossible to follow.  Hopefully better now!}
  If $\trimred(\Gamma)$ is finite, then in particular we have a finite set of reports $\Sc \subseteq \reals^d\setminus\red(\Gamma)$ such that $\trimred(\Gamma) = \{\Gamma_s \mid s\in\Sc\}$.
  As $\Gamma$ is elicitable, $\reals^d$ is representative for $\Gamma$.
  By definition of $\trimred$, we have $\simplex = \cup_{r\in\reals^d} \Gamma_r = \cup \trimred(\Gamma) = \cup_{s\in\Sc} \Gamma_s$, and therefore $\Sc$ is representative for $\Gamma$ and for $L$.\jessiet{I'm a bit confused by this- how do we get $\cup_{\reals^d}\Gamma_r \subseteq \cup \trimred(\Gamma)$?}\raft{Each level set on the left is contained in one on the right}
  From Lemma~\ref{lem:X}(\ref{item:X-rep-contain-min}), we have some minimum representative set $\U\subseteq\Sc$ for $L$ and $\Gamma$, showing Case 3.
  Moreover, Lemma~\ref{lem:X}(\ref{item:X-min-Theta},\ref{item:X-full-dim}) gives $\{\Gamma_u \mid u\in\U\} = \Theta$.
  \jessie{My understanding: we have a representative set $\R$, but don't know if it's finite.  However, we have $\Sc$ that [nonredundantly] generates trim, which is finite by assumption, and need to show that's representative.  Once we have $\Sc$ a finite representative set, we can apply Lemma X to say it also contains a finite \emph{minimum} rep set.}

  Case 3:
  Let $\U$ be a finite minimum representative set.
  Lemma~\ref{lem:X}(\ref{item:X-min-Theta},\ref{item:X-full-dim}) once again gives $\{\Gamma_u \mid u\in\U\} = \Theta$.
  We simply let $\hat\Theta = \Theta$, giving Case 4 as $\U$ is representative.

  Case 4:
  Let $\Sc\subseteq\R$ such that $\{\Gamma_s \mid s\in\Sc\} = \hat \Theta$.
  Then $\Sc$ is representative for $\Gamma$ and $L$, as $\cup\hat\Theta = \simplex$.
  Lemma~\ref{lem:loss-restrict} now states that $L$ embeds $L|_\Sc$, so $\Gamma$ embeds $\gamma := \Gamma|_\Sc$, giving Case 1.
  Finally, Corollary~\ref{cor:trim-prop-red} gives $\trimred(\gamma) = \Theta$.
  %
  % Letting $\Theta_\ell$ be the full-dimensional level sets of $\gamma$, Corollary~\ref{cor:trim-prop-red} also gives $\trimred(\gamma) = \Theta'$.
  % From Lemma~\ref{lem:X}(\ref{item:X-rep-contain-min}), we have some minimum representative set $\U\subseteq\U'$ for $L$ and $\Gamma$.
  % We have $\{\gamma_r\mid \varphi(r)\in\U\} = \{\Gamma_u \mid u\in\U\} = \Theta$ by Lemma~\ref{lem:X}(\ref{item:X-min-Theta}) and definition of embedding.
  % As $\Theta$ is a set of full-dimensional level sets of $\gamma$, we must have $\Theta \subseteq \Theta_\ell$.
  % But as $\cup \Theta = \simplex$, we must have $\Theta = \Theta_\ell$, as otherwise there exists $s\in\Sc$ with $\gamma_s \in \Theta_\ell \setminus \Theta$ and thus $\Sc \setminus \{s\}$ would still be representative.
  %
  % 
  % It remains to show $\trimred(\gamma) = \Theta$.
  % Moreover, $\Theta$ is exactly the full-dimensional level sets of $\Gamma$ by Lemma~\ref{lem:X}(\ref{item:X-full-dim}).
  % 
  % From Lemma~\ref{lem:X}(\ref{item:X-min-Theta},\ref{item:X-full-dim}) there exists some set of level sets $\Theta'$ which are the full-dimensional level sets of $\gamma$ and there exists a minimum representative set $\Sc\subseteq\R$ with $\{\gamma_s\mid s\in\Sc\} = \Theta'$.
  % Finally, Lemma~\ref{lem:X}(\ref{item:X-redundant}) gives $\trimred(\gamma) = \Theta$.
\end{proof}

\btw{JOURNAL: cool to point out that when $L$ is polyhedral, $\prop{L}$ has a finite range (this result) and so does its (multivalued map) inverse (trim result)!}


\section{Polyhedral Indirect Elicitation Implies Consistency}
\label{sec:poly-ie-consistency}

Our last result concerns indirect elicitation as a necessary condition for consistency.
Intuitively, a loss $L$ indirectly elicits a property $\gamma$ if we can compute $\gamma$ from $\prop L$.
To formalize the condition, we use the notion of a property refining another from~\citet{frongillo2014general} \raf{also (and maybe only) Biometrika paper with Ian}.

\begin{definition}[Refines]
	Let $\Gamma:\simplex \toto \R$ and $\Gamma':\simplex\toto \R'$.
	Then $\Gamma$ \emph{refines} $\Gamma'$ if for all $r \in \R$, there exists $r' \in \R'$ such that $\Gamma_{r} \subseteq \Gamma'_{r'}$.
\end{definition}
Equivalently, $\Gamma$ refines $\Gamma'$ if there is some ``link'' function $\psi:\R\to\R'$ such that $r\in\Gamma(p) \implies f(r) \in \Gamma'(p)$ for all $p\in\simplex$.
We will use the fact that refinement is transitive: if $\Gamma_1$ refines $\Gamma_2$ and $\Gamma_2$ refines $\Gamma_3$, then $\Gamma_1$ refines $\Gamma_3$.

\begin{definition}[Indirectly elicits]
  A loss $L$ \emph{indirectly elicits} a property $\gamma$ if $\prop L$ refines $\gamma$.
\end{definition}

It is straightforward to verify that consistency, and therefore calibration, implies indirect elicitation\raft{Add cites}.
Indirect elicitation may appear much weaker than calibration, since in particular it does not depend on the loss except through the property it elicits.
Surprisingly, for polyhedral surrogates, we show the converse: indirect elicitation implies calibration, and therefore consistency.

A useful lemma is that for polyhedral losses, indirect elicitation must always pass through an embedding.
This result holds more generally whenever $L$ has a finite representative set, as in \S~\ref{sec:min-rep-sets}.
\begin{lemma}\label{lem:ie-iff-embeds-refinement}
  Let $L$ be a polyhedral loss.
  Then $L$ indirectly elicits a property $\gamma$ if and only if $L$ tightly embeds a discrete loss $\ell$ such that $\prop \ell$ refines $\gamma$.
\end{lemma}
\begin{proof}
  Let $L:\reals^d\to\reals^\Y_+$ be polyhedral, and $\Gamma = \prop L$.
  Then $L$ tightly embeds a discrete loss from Lemma~\ref{lem:X}(\ref{item:X-tight-embed}).
  Furthermore, Lemma~\ref{lem:X}(\ref{item:X-min-Theta},\ref{item:X-redundant},\ref{item:X-tight-embed}) implies that $\prop L$ refines $\prop \ell$ for any discrete loss $\ell$ that $L$ tightly embeds.

  We claim that, for any property $\gamma$, and any loss $\ell$ that $L$ tightly embeds, $\prop L$ refines $\gamma$ if and only if $\prop \ell$ refines $\gamma$.
  If $\prop \ell$ refines $\gamma$, then $\prop L$ refines $\gamma$ by transitivity.
  For the other direction, Lemma~\ref{lem:X}(\ref{item:X-min-Theta},\ref{item:X-tight-embed}) shows that the level sets of $\prop \ell$ are contained in the set $\{\Gamma_u \mid u\in\reals^d\}$.
  Thus, if $\prop L$ refines $\gamma$, then in particular $\prop \ell$ refines $\gamma$.
  The result now follows immediately from the claim.
\end{proof}



\begin{theorem}\label{thm:poly-ie-implies-consistent}
	Let $L$ be a polyhedral loss which indirectly elicits a finite property $\gamma$.
  For any loss $\ell$ eliciting $\gamma$, there exists a link $\psi$ such that $(L, \psi)$ is calibrated (and consistent) with respect to $\ell$.
\end{theorem}
\begin{proof}
  \raft{Jessie: well done with this proof!  There were a bunch of type check problems where you got the links confused but the overall thread was there.}
	Let $L:\reals^d \to \reals^\Y_+$ be a polyhedral loss indirectly eliciting $\gamma: \simplex \toto \R$, and let $\ell$ be a discrete loss eliciting $\gamma$.
  By Lemma~\ref{lem:ie-iff-embeds-refinement}, $L$ tightly embeds a discrete loss $\ell^\emb:\R^\emb\to\reals^\Y_+$ such that $\gamma^\emb := \prop{\ell^\emb}$ refines $\gamma$.
  From refinement, we can define a function $\psi^\R: \R^\emb \to \R$ such that for all $r\in\R^\emb$ and $p\in\simplex$ we have $r\in\gamma^\emb(p) \implies \psi^\R(r) \in \gamma(p)$. 
  Finally, Theorem~\ref{thm:eps-thick-calibrated} gives a link function $\psi^\emb : \reals^d \to \R^\emb$ such that $(L,\psi^\emb)$ is calibrated with respect to $\ell^\emb$.

	
  Consider $\psi := \psi^\R \circ \psi^\emb$ and fix $p\in\simplex$.
	For any $u\in\reals^d$, if $\psi^\emb(u) \in \gamma^\emb(p)$, then $\psi(u) = \psi^\R(\psi^\emb(u)) \in \gamma(p)$ by definition of $\psi$ and $\psi^\R$.
  Contrapositively\raft{You heard me},
  $\psi(u) \notin \gamma(p) \implies \psi^\emb(u) \notin \gamma^\emb(p)$.
  Thus, we have
  \begin{equation}
    \label{eq:link-set-inclusion}
    \{u\in\reals^d \mid \psi(u) \not \in \gamma(p) \} \subseteq \{u\in\reals^d \mid \psi^\emb(u) \not \in \gamma^\emb(p) \}~.
  \end{equation}
  Combining eq.~\eqref{eq:link-set-inclusion} with the fact that $(L,\psi^\emb)$ is calibrated with respect to $\ell^\emb$, we have
	\begin{align*}
\inf_{u\in\reals^d : \psi(u) \not \in \gamma(p)} \inprod{p}{L(u)} \geq	\inf_{u\in\reals^d : \psi^\emb(u) \not \in \gamma^\emb(p)} \inprod{p}{L(u)} > \inf_{u\in\reals^d}\inprod{p}{L(u)}~,
	\end{align*}
  showing calibration of $\psi$.
	Consistency follows as calibration and consistency are equivalent in this setting~\citep{ramaswamy2016convex}\raft{I think we need a better reference, since I believe that's only for binary classification...}\jessiet{Changed to Hari's proof (which tbh, isn't much of a proof but rather a claim IIRC)}.
\end{proof}

Theorem~\ref{thm:poly-ie-implies-consistent} gives a somewhat surprising result: despite the fact that indirect elicitation appears to be a somewhat weak necessary condition for consistency in general, the two conditions are equivalent for polyhedral surrogates.




\section{Conclusions} \label{sec:conclusion}

\raf{Old recap; perhaps cut}
This paper formalizes an intuitive way to design convex surrogate losses for finite prediction problems---by embedding the reports into $\reals^d$.
We establish a close relationship between embeddings and polyhedral surrogates, showing both that every polyhedral loss embeds a discrete loss (Theorem~\ref{thm:poly-embeds-discrete}) and that every discrete loss is embedded by some polyhedral loss (Theorem~\ref{thm:discrete-loss-poly-embeddable}).
We then construct a calibrated link function from any polyhedral loss to the discrete loss it embeds, giving consistency for all such losses (Theorem~\ref{thm:eps-thick-calibrated}).

\jessie{New version...?}
In this paper, we introduce the embeddings tool for studying finite prediction problems, and connect this tool to the design of piecewise-linear and convex (polyhedral) surrogates (Theorem~\ref{thm:embed-poly-informal}), and show that embeddings by polyhedral surrgates are consistent with respect to the given target problem (Theorem~\ref{thm:link-informal}).
We additionally show that embedding discrete prediction tasks is tightly connected to surrogates with polyhedral Bayes risks (Corollary~\ref{cor:poly-risk-fin-rep}).
In \S~\ref{sec:applications}, this tool is applied to a variety examples, including high-confidence classification, ordered partitions, top-$k$ prediction, and structured prediction.
Finally, when restricting to polyhedral surrogate losses, we show that indirect property elicitation is equivalent to consistency in \S~\ref{sec:poly-ie-consistency}.

Some open questions remain: 

\jessie{I don't think we want all of these... I feel like it sends a signal about the work being incomplete.}
\raf{Future work and conjectures}
\begin{itemize}
\setlength \itemsep{0.1em}
\item %\raf{This is false I believe, from BEP surrogate $n=4$ and large (small?) $\alpha$ -- you can add a little curvature between the spokes and it still embeds.  Maybe we can rephrase / reformulate.}
%  In fact, we conjecture that \emph{any} loss embedding a discrete $\ell$ must be polyhedral on the convex hull of the embedded reports. 
%(The convex hull of the embedded reports follows since any point not in the convex hull will never minimize the expected loss.)
\btw{Modified from a conjecture Raf thought was false. Old version in comments - J, 15 Sept 21}
While the modified hinge example in Figure~\ref{fig:modified-hinge} shows that not every embedding must necessarily be polyhedral, this curvature is added on a set that is never uniquely optimal.
A natural question is to posit if an embedding my be polyhedral on the convex hull of the embedded reports; we conjecture negatively, though some conditions on linearity seem necessary.


\item 
  Since discrete losses have a finite set of reports, and in turn, minimizers, any surrogate embedding the discrete loss must also have a finite set of unique minimizers.
  This is in turn related to another conjecture about the ``convex envelope'' of embeddings: if $L$ embeds $\ell$ by the embedding $\varphi$, the (polyhedral) surrogate $L'$ such that $L'_y$ is the convex envelope of $\{(\varphi(r),L(r)_y)\}_{r\in\R}$ also embeds $\ell$.
  \jessie{What is the significance of this conjecture?}

\item
  Since polyhedral surrogates have a finite trimmed loss, it is an open line of work to relate embeddings to the to geometry of losses~\citep{williamson2014geometry} through superprediction sets, as these sets encapsulate losses that are dominated by elements of $\trim(L)$.
  
\item
  In \S~\ref{sec:poly-ie-consistency}, we show that indirect elicitation is equivalent to consistency when restricting to the class of polyhedral surrogates.
  It remains a line of future work to characterize other settings in which indirect elicitation and consistency are equivalent.

\item
  \citet{finocchiaro2020embedding} introduce and present lower bounds for the notion of \emph{embedding dimension} of a target loss $\ell$: the minimum prediction dimension $d$ such that a polyhedral surrogate $L:\reals^d\to \reals^\Y_+$ embeds $\ell$.
  However, it is unclear if convex consistency dimension restricting to polyhedral losses, embedding dimension, or unrestricted convex consistency dimension are equal for all target losses.
  %Characterizing embedding dimension \cite{finocchiaro2020embedding}; does poly dim == embedding dim?  ccdim == embedding dim?

\item
  ...
\end{itemize}

%%%% Pre-neurips
%This work is part of a broader research program to understand convex surrogates through the lens of property elicitation.  % the relationship between finite losses and convex surrogates, the link functions connecting them, and the properties they elicit.
%We seek a general theory that, given a property, can prescribe when and how to construct convex surrogate losses that elicit it, and specifically, determine the minimum dimension required.
%Even more broadly, one could replace ``convex'' by any notion of ``nice'' surrogate.
%
%This work formalized the \emph{embedding} approach where labels are identified with points in $\reals^d$.
%We saw in Theorems~\ref{thm:poly-embeds-discrete} and~\ref{thm:discrete-loss-poly-embeddable} that this approach is tightly connected to the use of polyhedral (i.e. piecewise linear convex) loss functions.
%We also saw that it is a general technique that can be used for any finite elicitable property.
%
%Moreover, we established the relationship between polyhedral surrogates and \emph{calibrated links} to the discrete losses they embed.
%\jessie{Add more?}

%We then investigated the \emph{dimensionality} of $\reals^d$ required for such embeddings, giving a characterization in terms of the structure of the property in the simplex.
%This gave a complete understanding of the one-dimensional case, and a complete characterization albeit weaker understanding in higher dimensions.
%RF got here
%The two key conditions are an optimality condition that relates the structure of each level set to existence of polytopes in $\reals^d$ satisfying certain conditions; and a monotonicity condition relating such polytopes for different level sets.
%This yields new lower bounds in particular for the abstain loss.

% \paragraph{Directions.}
%\jessie{Getting rid of this given the COLT-20 paper}
%\iffalse
%One open question of particular interest involves the dimension of the surrogate prediction space; given a discrete loss, can we construct a surrogate that embeds it \emph{of minimum dimension}?
%If we na\"ively embed the reports into an $n$-dimensional space, the dimensionality of the problem scales linearly in the number of possible labels $n$.
%As the dimension of the optimization problem is a function of this \emph{embedding dimension} $d$, a promising direction is to leverage tools from elicitation complexity~\cite{lambert2008eliciting,frongillo2015elicitation} and convex calibration dimension~\cite{ramaswamy2016convex} to understand when we can take $d <\!\!< n$.
%\fi

%There are several direct open questions involving the dimension of the surrogate loss.
%It would be interesting and perhaps practically useful to develop further techniques for upper bounds: automatically constructing embeddings in $\reals^d$ and associated polyhedral losses from a given property, with $d$ as small as possible.
%Another direction is additional lower bound techniques, or further development of our necessary conditions.

%This paper also suggests an agenda of defining more nuanced classes of surrogate losses and studying their properties.
%For example, a tangential topic in this work was the characteristics of a ``good'' link function; formalizing and exploring this question is an exciting direction.
%We would also like to move toward a full understanding of the differences between these classes.
%For example, how does embedding dimension compare in general to convex elicitation dimension (the dimensionality $d$ required of \emph{any} convex surrogate loss)?
%These questions have both theoretical interest and potential practical significance.

%\begin{conjecture}
%  $\mathrm{elic}_{embed}(\Gamma) = \mathrm{elic}_{Pcvx}(\Gamma) = \mathrm{elic}_{cvx}(\Gamma)$ for all finite elicitable $\Gamma$.
%\end{conjecture}


\subsection*{Acknowledgements}
We thank Arpit Agarwal and Peter Bartlett for many early discussions, which led to several important insights.
% to a proof of %\raf{1-d reduction} among other insights.
% Additionally, we thank Arpit Agarwal for early insights and bringing out attention to the abstain property.
We thank Eric Balkanski for help with a lemma about submodular functions, Stephen Becker for his insights about finding a constant rate of separation in proving $\epsilon$-thickenings are calibrated.
 %Lemma~\ref{lem:bar-f}\jessie{We don't have the Lov\'asz hinge appendix here?}.
This material is based upon work supported by the National Science Foundation under Grants No.\ 1657598 and No.\ DGE 1650115.
\raf{Nishant, anyone else now?}
\newpage
\bibliographystyle{plainnat}
\bibliography{diss,extra}

%%%%%%%%%%%%%%%%%%%%%%%%%%%%%%%%%%%%%%%%%%%%%%%%%%%%%%%%%%%%%%%
\appendix

\newpage
\section{Power diagrams}\label{app:power-diagrams}
First, we present several definitions from Aurenhammer~\cite{aurenhammer1987power}.
\begin{definition}\label{def:cell-complex}
  A \emph{cell complex} in $\reals^d$ is a set $C$ of faces (of dimension $0,\ldots,d$) which (i) union to $\reals^d$, (ii) have pairwise disjoint relative interiors, and (iii) any nonempty intersection of faces $F,F'$ in $C$ is a face of $F$ and $F'$ and an element of $C$.
\end{definition}

\begin{definition}\label{def:power-diagram}
  Given sites $s_1,\ldots,s_k\in\reals^d$ and weights $w_1,\ldots,w_k \geq 0$, the corresponding \emph{power diagram} is the cell complex given by
  \begin{equation}
    \label{eq:pd}
    \cell(s_i) = \{ x \in\reals^d : \forall j \in \{1,\ldots,k\} \, \|x - s_i\|^2 - w_i \leq \|x - s_j\|^2 - w_j\}~.
  \end{equation}
\end{definition}

\begin{definition}\label{def:affine-equiv}
  A cell complex $C$ in $\reals^d$ is \emph{affinely equivalent} to a (convex) polyhedron $P \subseteq \reals^{d+1}$ if $C$ is a (linear) projection of the faces of $P$.
\end{definition}

Proposition~\ref{prop:embed-bayes-risks}, focuses on matching the values of Bayes Risks, while the following result from~\citet{aurenhammer1987power} allows us to move towards understanding the projection of the Bayes Risk onto the simplex $\simplex$.
In particular, one can consider the epigraph of a polyhedral convex function on $\reals^d$ and the projection down to $\reals^d$; in this case we call the resulting power diagram \emph{induced} by the convex function.

\begin{theorem}[Aurenhammer~\cite{aurenhammer1987power}]\label{thm:aurenhammer}
	A cell complex is affinely equivalent to a convex polyhedron if and only if it is a power diagram.
\end{theorem}

\raf{Here's what I'm thinking for the text from here to Lemma~\ref{lem:poly-loss-poly-risk} (let's not implement just yet though): Let's move Lemma~\ref{lem:polyhedral-pd-same} to the appendix, and state Lemma~\ref{lem:polyhedral-range-gamma} but move the proof to the appendix.  Then we can keep the proof of Lemma~\ref{lem:poly-loss-poly-risk}.}
We extend Theorem~\ref{thm:aurenhammer} to a weighted sum of convex functions, showing that the induced power diagram is the same for any choice of strictly positive weights.

\begin{lemma}\label{lem:polyhedral-pd-same}
	Let $f_1,\ldots,f_m:\reals^d\to\reals$ be polyhedral convex functions.
	The power diagram induced by $\sum_{i=1}^m p_i f_i$ is the same for all $p \in \inter(\simplex)$.
\end{lemma}
\begin{proof}
	For any convex function $g$ with epigraph $P$, the proof of~\citet[Theorem 4]{aurenhammer1987power} shows that the power diagram induced by $g$ is determined by the facets of $P$.
	Let $F$ be a facet of $P$, and $F'$ its projection down to $\reals^d$.
	It follows that $g|_{F'}$ is affine, and thus $g$ is differentiable on $\inter(F')$ with constant derivative $d\in\reals^d$.
	Conversely, for any subgradient $d'$ of $g$, the set of points $\{x\in\reals^d : d'\in\partial g(x)\}$ is the projection of a face of $P$; we conclude that $F = \{(x,g(x))\in\reals^{d+1} : d\in\partial g(x)\}$ and $F' = \{x\in\reals^d : d\in\partial g(x)\}$.
	
	Now let $f := \sum_{i=1}^k f_i$ with epigraph $P$, and $f' := \sum_{i=1}^k p_i f_i$ with epigraph $P'$.
	By Rockafellar~\cite{rockafellar1997convex}, $f,f'$ are polyhedral.
	We now show that $f$ is differentiable whenever $f'$ is differentiable:
	\begin{align*}
	\partial f(x) = \{d\}
	&\iff \sum_{i=1}^k \partial f_i(x) = \{d\} \\
	&\iff \forall i\in\{1,\ldots,k\}, \; \partial f_i(x) = \{d_i\} \\
	&\iff \forall i\in\{1,\ldots,k\}, \; \partial p_i f_i(x) = \{p_id_i\} \\
	&\iff \sum_{i=1}^k \partial p_if_i(x) = \left\{\sum_{i=1}^k p_id_i\right\} \\
	&\iff \partial f'(x) = \left\{\sum_{i=1}^k p_id_i\right\}~.
	\end{align*}
	From the above observations, every facet of $P$ is determined by the derivative of $f$ at any point in the interior of its projection, and vice versa.
	Letting $x$ be such a point in the interior, we now see that the facet of $P'$ containing $(x,f'(x))$ has the same projection, namely $\{x'\in\reals^d : \nabla f(x) \in \partial f(x')\} = \{x'\in\reals^d : \nabla f'(x) \in \partial f'(x')\}$.
	Thus, the power diagrams induced by $f$ and $f'$ are the same.
	The conclusion follows from the observation that the above held for any strictly positive weights $p$, and $f$ was fixed.
\end{proof}

We now include the full proof of Lemma~\ref{lem:polyhedral-range-gamma}.

\polyhedralrangegamma*
\begin{proof}
	For all $p$, let $P(p)$ be the epigraph of the convex function $u\mapsto \inprod{p}{L(u)}$.
	From Lemma~\ref{lem:polyhedral-pd-same}, we have that the power diagram $D_\Y$ induced by the projection of $P(p)$ onto $\reals^d$ is the same for any $p\in\inter(\simplex)$.
	Let $\F_\Y$ be the set of faces of $D_\Y$, which by the above are the set of faces of $P(p)$ projected onto $\reals^d$ for any $p\in\inter(\simplex)$.
	
	We claim for all $p\in\inter(\simplex)$, that $\Gamma(p) \in \F_\Y$.
	To see this, let $u \in \Gamma(p)$, and $u' = (u,\inprod{p}{L(u)}) \in P(p)$.
	The optimality of $u$ is equivalent to $u'$ being contained in the face $F$ of $P(p)$ exposed by the normal $(0,\ldots,0,-1)\in\reals^{d+1}$.
	Thus, $\Gamma(p) = \argmin_{u\in\reals^d} \inprod{p}{L(u)}$ is a projection of $F$ onto $\reals^d$, which is an element of $\F_\Y$.
	
	Now for $p \not \in \inter(\simplex)$, consider $\Y'\subsetneq \Y$, $\Y'\neq\emptyset$.
	Applying the above argument, we have a similar guarantee: a finite set $\F_{\Y'}$ such that $\Gamma(p) \in \F_{\Y'}$ for all $p$ with support exactly $\Y'$.
	Taking $\F = \bigcup\{\F_{\Y'} | \Y'\subseteq\Y, \Y'\neq\emptyset\}$, we have for all $p\in\simplex$ that $\Gamma(p) \in \F$, giving $\U \subseteq \F$.
	As $\F$ is finite, so is $\U$, and the elements of $\U$ are closed polyhedra as faces of $D_{\Y'}$ for some $\Y'\subseteq\Y$.
\end{proof}


\section{Embedding properties}\label{app:embed-props}
\jessie{Do we use this anymore?}

While Definition~\ref{def:loss-embed} gives the notion of one \emph{loss} embedding another, we now define the notion of one \emph{property} embedding another.
\begin{definition}\label{def:prop-embed}
	A property $\Gamma : \simplex \toto \reals^d$ embeds a property $\gamma:\simplex \toto \R$ if there exists some injective embedding $\varphi:\R \to \reals^d$ such that for all $p \in \simplex$ and $r \in \R$, we have $r \in \gamma(p) \iff \varphi(r) \in \Gamma(p)$.
\end{definition}

By condition (ii.) of Definition~\ref{def:loss-embed}, we then have $L$ embedding $\ell$ implies $\prop{L}$ embeds $\prop{\ell}$.

This also prompts us to think about redundancy in properties.
\begin{definition}[Non-redundant property]\label{def:nonredundant-prop}
	A property $\Gamma:\simplex \toto \R$ is \emph{redundant} if there are reports $r, r' \in \R$ such that $\Gamma_r \subseteq \Gamma_{r'}$, and \emph{non-redundant} otherwise.
\end{definition}


%It is often convenient to work directly with properties and set aside the losses which elicit them.
%To this end, we say a property to embeds another if eq.~\eqref{eq:embed-loss} holds.
%We begin with the notion of redundancy.
%\begin{definition}[Finite property, non-redundant]
%  A property $\gamma:\simplex\toto\R$ is \emph{redundant} if for some $r,r'\in\R$ with $r \neq r'$, we have $\gamma_r \subseteq \gamma_{r'}$, and \emph{non-redundant} otherwise.
%  $\gamma$ is \emph{finite} if it is non-redundant and $\R$ is a finite set.
%\end{definition}

%With the terminology of properties in hand, we can restate our definition of embedding.
%First, we formalize the notion of embedding properties.
%\begin{definition}
%  A property $\Gamma : \simplex \toto \reals^d$ \emph{embeds} a property $\gamma : \simplex \toto \R$ if there exists some injective embdedding $\varphi:\R\to\reals^d$ such that for all $p\in\simplex,r\in\R$ we have $r \in \gamma(p) \iff \varphi(r) \in \Gamma(p)$.
%  Similarly, we say a loss $L:\reals^d\to\reals^\Y$ embeds $\gamma$ if $\prop{L}$ embeds $\gamma$.
%\end{definition}
%We can now see that a surrogate $L:\reals^d\to\reals^\Y$ embeds $\ell:\R\to\reals^\Y$ if and only if $\prop{L}$ embeds $\prop{\ell}$ via $\varphi$ and for all $r\in\R$ we have $L(\varphi(r)) = \ell(r)$.

When working with convex losses which are not strictly convex, one quickly encounters redundant properties: if $\inprod{p}{L(\cdot)}$ is minimized by a point where $p\cdot L$ is flat, then there will be an uncountable set of reports which also minimize the loss.
As results in property elicitation typically assume non-redundant properties (e.g.~\cite{frongillo2014general,frongillo2015elicitation}), it is useful to consider a transformation which removes redundant level sets.
We capture this transformation as the trim operator presented below.

\begin{definition}\label{def:trim-prop-nonred}
  Given an elicitable property $\Gamma:\simplex \toto\R$, we define $\trimred(\Gamma) = \{\Gamma_u \mid \neg \exists u' \neq u$ s.t. $\Gamma_u \subsetneq \Gamma_{'u}\}$.
  %$\trimcover(\Gamma) = \{\Gamma_u : u \in \U \}$ as the set of maximal level sets of $\Gamma$ for any minimum representative set $\U \subseteq \R$.
\end{definition}

\begin{definition}\label{def:trim-prop-cover}
	Given an elicitable property $\Gamma:\simplex \toto\R$, we define $\trimcover(\Gamma) = \{\Gamma_u : u \in \U \}$ as the set of maximal level sets of $\Gamma$ for any minimum representative set $\U \subseteq \R$.
\end{definition}

The following corollary follows from Corollary~\ref{cor:trim-loss-condition}.
\begin{corollary}
	Let $\Gamma$ be an elicitable property with a finite minimum representative set.
	$\trimcover(\Gamma) = \trimred(\Gamma)$.
\end{corollary}

\btw{RF: Note for later: should be able to show that the union of trim is the simplex.\jessie{This is part of the proposition statement 2 now.}}
Take note that the unlabeled property $\trim(\Gamma)$ is non-redundant, meaning that for any $\theta \in \trim(\Gamma)$, there is no level set $\theta' \in \trim(\Gamma)$ such that $\theta \subset \theta'$.

\iffalse
\hrule
\bigskip
\jessie{Add results (subsection?) relating trim and positive normal sets.  From here until the hrule}
\subsection{Relation to Positive Normal Sets}

\proposedadd{The concept of $\trim$ is closely related to the \emph{positive normal set} of Ramaswamy et al.~\cite{ramaswamy2016convex}.
However, they define positive normal sets in terms of the loss vector, agnostic to the report yielding such a loss vector.
In constructing the $\trim$ of a property, we yield the positive normal sets of the loss eliciting such a property, without the attachment to loss vectors.
This allows us to frame necessary and sufficient conditions for constructing a calibrated surrogate in terms of finite properties.}


\begin{definition}
	Let $L:\reals^d \to \reals^n_+$, and define $\Sc_L := \conv(L(\reals^d))$ as in~\cite[Definition 8]{ramaswamy2016convex}.
	For $z \in \Sc_L$, we define the \emph{positive normal set} of $L$ at $z$ as
	\begin{equation}
	\N^L(z) = \left\{ p \in \simplex: \inprod{p}{z} = \inf_{z' \in \Sc_L} \inprod{p}{z} \right\}~.~
	\end{equation}
\end{definition}

\begin{conjecture}
	For $L$ convex and $p \in \simplex$, we have $\inf_{z' \in \Sc_L} \inprod{p}{z'} = \inf_{z' \in L(\reals^d)}\inprod{p}{z'}$.
\end{conjecture}
\begin{proof}
	First, we have $\inf_{z' \in \Sc_L} \inprod{p}{z'} \leq \inf_{z' \in L(\reals^d)}\inprod{p}{z'}$ since $L(\reals^d) \subseteq \Sc_L$.
	
	Now, we want to show $\inf_{z' \in \Sc_L} \inprod{p}{z'} \geq \inf_{z' \in L(\reals^d)}\inprod{p}{z'}$.
	Consider that for all $z \in \Sc_L$, we have $z \in \inter(\Sc_L) \implies z \not \in \arginf_{z' \in \Sc_L}\inprod{p}{z'}$, where $\partial \Sc_L$ is the boundary of $\Sc_L$.
	Therefore, we have $\inf_{z' \in \Sc_L}\inprod{p}{z'} = \inf_{z' \in \partial\Sc_L}\inprod{p}{z'}$.
	
	If we can then show $L(\reals^d) \supseteq \partial \Sc_L$, then we have $\inf_{z' \in \Sc_L} \inprod{p}{z'} \geq \inf_{z' \in L(\reals^d)}\inprod{p}{z'}$.

	\jessie{??? Not sure why this should be true, so it probably isn't, but I can't think of a counterexample.}
\end{proof}

\begin{proposition}
	Consider the loss $L:\reals^d \to \reals^n_+$ and $\Gamma := \prop{L}$ nondegenerate.
	Fix a finite set of $\{z_i\}_{i=1}^k$ so that each $z_i \in \Sc_L$ and $\cup_{z_i} \N^L(z_i) = \simplex$.
	Then $\N^L(\mathcal{Z}) = \{\N^L(z_i)\}_{i=1}^k = \trim(\Gamma)$.
\end{proposition}
\begin{proof}
  \jessie{still a conjecture.}
  Take $\theta \in \N^L(\mathcal{Z})$.
  For all $p \in \theta$, we then have 
  \begin{align*}
  p \in \theta \iff \inprod{p}{z} &= \inf_{z' \in \Sc_L} \inprod{p}{z'}\\
  &= \inf_{z' \in L(\reals^d)} \inprod{p}{z'}\\
  &= \inf_{u \in \reals^d} \inprod{p}{L(u)}\\
  &= \inf_{u\in \reals^d} \E_p L(u, Y)\\
  &\iff p \in \Gamma_u~.~
  \end{align*}
  \jessie{Not quite... we have $\inf_{z' \in \Sc_L} \inprod{p}{z'} \leq \inf_{z' \in L(\reals^d)} \inprod{p}{z'}$, by $L(\reals^d) \subseteq \Sc_L$, but we need the fact that $\N^L(z) = \emptyset$ for all $z \in \inter(\Sc_L)$ or something similar for the first to second lines of equality.  Otherwise we just have $\Gamma_u \subseteq \N^L(z)$.  i.e. we need Conjecture 1.
  }
  Since this is true for all $p \in \theta$, we have $\theta = \Gamma_u$.
  As this is true for a finite set of $\mathcal{Z}$ whose positive normal sets union to the simplex, we have $\N^L(\mathcal{Z}) = \trim(\prop{L})$.
\end{proof}

\begin{corollary}
	\jessie{The necessary and sufficient conditions for calibrated surrogates from Ramaswamy \cite{ramaswamy2016convex}, but in terms of the unlabeled property.}
\end{corollary}
\hrule

Before we state the Proposition needed to prove many of the statements in Section~\ref{sec:poly-loss-embed}, we will need to general lemmas about properties and their losses.
The first follows from standard results relating finite properties to power diagrams (see Theorem~\ref{thm:aurenhammer}), and its proof is omitted.
The second is closely related to the trim operator: it states that if some subset of the reports are always represented among the minimizers of a loss, then one may remove all other reports and elicit the same property (with those other reports removed).

\begin{lemma}\label{lem:finite-full-dim}
  Let $\gamma$ be a finite (non-redundant) property elicited by a loss $L$.
  Then the negative Bayes risk $G$ of $L$ is polyhedral, and the level sets of $\gamma$ are the projections of the facets of the epigraph of $G$ onto $\simplex$, and thus form a power diagram.
  In particular, the level sets of $\gamma$ are full-dimensional in $\simplex$ (i.e.,\ of dimension $n-1$).
\end{lemma}

\begin{lemma}\label{lem:loss-restrict}
  Let $L$ elicit $\Gamma:\simplex\toto\R_1$, and let $\R_2\subseteq\R_1$ such that $\Gamma(p) \cap \R_2 \neq \emptyset$ for all $p\in\simplex$.
  Then $L|_{\R_2}$ ($L$ restricted to $\R_2$) elicits $\gamma:\simplex\toto\R_2$ defined by $\gamma(p) = \Gamma(p)\cap \R_2$.
  Moreover, the Bayes risks of $L$ and $L|_{\R_2}$ are the same.
\end{lemma}
\begin{proof}
  Let $p\in\simplex$ be fixed throughout.
  First let $r \in \gamma(p) = \Gamma(p) \cap \R_2$.
  Then $r \in \Gamma(p) = \argmin_{u\in\R_1} \inprod{p}{L(u)}$, so as $r\in\R_2$ we have in particular $r \in \argmin_{u\in\R_2} \inprod{p}{L(u)}$.
  For the other direction, suppose $r \in \argmin_{u\in\R_2} \inprod{p}{L(u)}$.
  By our assumption, we must have some $r^* \in \Gamma(p) \cap \R_2$.
  On the one hand, $r^*\in\Gamma(p) = \argmin_{u\in\R_1} \inprod{p}{L(u)}$.
  On the other, as $r^* \in \R_2$, we certainly have $r^* \in \argmin_{u\in\R_2} \inprod{p}{L(u)}$.
  But now we must have $\inprod{p}{L(r)} = \inprod{p}{L(r^*)}$, and thus $r \in \argmin_{u\in\R_1} \inprod{p}{L(u)} = \Gamma(p)$ as well.
  We now see $r \in \Gamma(p) \cap \R_2$.
  Finally, the equality of the Bayes risks $\min_{u\in\R_1} \inprod{p}{L(u)} = \min_{u\in\R_2} \inprod{p}{L(u)}$ follows immediately by the above, as $\emptyset \neq \Gamma(p)\cap\R_2 \subseteq \Gamma(p)$ for all $p\in\simplex$.
\end{proof}

We now state a useful result for proving the existence of an embedding loss, which shows remarkable structure of embeddable properties, and the properties that embed them.
First, we conclude that any embeddable property must be elicitable.
We also conclude that if $\Gamma$ embeds $\gamma$, the level sets of $\Gamma$ must all be redundant relative to $\gamma$.
In other words, $\Gamma$ is exactly the property $\gamma$, just with other reports filling in the gaps between the embedded reports of $\gamma$.
(When working with convex losses, these extra reports are typically the convex hull of the embedded reports.)
In this sense, we can regard embedding as a minor departure from direct elicitation: if a loss $L$ elicits $\Gamma$ which embeds $\gamma$, we can think of $L$ as essentially eliciting $\gamma$ itself.
Finally, we have an important converse: if $\Gamma$ has finitely many full-dimensional level sets, or if $\trim(\Gamma)$ is finite, then $\Gamma$ must embed some finite elicitable property with those same level sets.





\jessie{Added from COLT-19 but not discussed in COLT-20}

\begin{definition}
	We say a link $\psi:\reals^d \to \R$ is \emph{calibrated from $\Gamma$ to $\gamma$} if there is a calibrated link from $L$ to $\ell$, where $\Gamma := \prop{L}$ and $\gamma:= \prop{\ell}$.
\end{definition}

\begin{proposition}
	Let $L$ elicit $\Gamma:\simplex \toto \reals^d$ which embeds a finite property $\gamma$.
	Then there is a calibrated link from $\Gamma$ to $\gamma$.
\end{proposition}
\begin{proof}
	Let $\gamma: \simplex \toto \R$.
	Proposition~\ref{prop:embed-trim} gives us that $\trim(\Gamma) = \{\gamma_r : r \in \R\}$.
	We conclude that for any $u \in \reals^d$, there is a calibrated link from $\Gamma$ to $\gamma$.\jessie{Added definition for calibrated link for properties above.}
\end{proof}
\jessie{Not sure what we want to say about calibrated links, if anything.}

\subsection{Refining properties}

\begin{definition}
	Let $\Gamma:\simplex \toto \R$ and $\Gamma':\simplex\toto \R'$.
	Then $\Gamma'$ \emph{refines} $\Gamma$ if for all $r' \in \R'$, we have $\Gamma'_{r'} \subseteq \Gamma_r$ for some $r \in \R$.
	That is, the cells of $\Gamma'$ are all contained in the cells of $\Gamma$.
\end{definition}

\begin{theorem}
	Every polyhedral loss embeds a finite elicitable property.
	Moreover, a polyhedral loss $L$ indirectly elicits a finite elicitable property $\gamma$ if and only if $\gamma$ is finite and $L$ embeds a property which refines $\gamma$.
\end{theorem}
\begin{proof}
	Let $L:\reals^d\to\reals^\Y_+$ be a polyhedral loss.
	For all $p$, let $P(p)$ be the epigraph of the convex function $u\mapsto \inprod{p}{L(u)}$.
	From Lemma~\ref{lem:polyhedral-pd-same}, we have that the power diagram induced by the projection of $P(p)$ onto $\reals^d$ is constant whenever $p\in\inter(\simplex)$.
	Let $q\in\inter\simplex$ be the uniform distribution on $\Y$, and $V_\Y$ be the set of vertices of $P(q)$ projected onto $\reals^d$.
	By the above, this set is the same had we replaced $q$ by any $p\in\inter\simplex$.
	
	Now let $\Gamma := \Gamma[L]$.
	We claim for all $p\in\inter(\simplex)$, that $\Gamma(p) \cap V_\Y \neq \emptyset$.
	To see this, let $u \in \Gamma(p)$, and $u' = (u,\inprod{p}{L(u)}) \in P(p)$.
	The optimality of $u$ is equivalent to $u$ being contained in the face exposed by the normal $(0,\ldots,0,-1)\in\reals^{d+1}$, which is a face of $P(p)$.
	Let $v'\in\reals^{d+1}$ be a vertex on such a face, and $v\in V_\Y$ its projection onto $\reals^d$.
	Then $v$ is also optimal, and therefore $v\in\Gamma(p)$.
	
	Now consider $\Y'\subset \Y$.
	Applying the above argument on distributions $p$ with support exactly $\Y'$, we have a similar guarantee: a finite set $V_{\Y'}$ such that $\Gamma(p) \cap V_{\Y'} \neq \emptyset$ for all $p$ with support exactly $\Y'$.
	(When $\Y' = \{y\}$ is a singleton, we simply take the projected vertices of $L(\cdot)_y$.)
	
	Thus, taking $V = \bigcup_{\Y'\subseteq\Y} V_{\Y'}$, we have for all $p\in\simplex$ that $\Gamma(p) \cap V \neq \emptyset$.
	This implies that $\trim(\Gamma) \subseteq \{\Gamma_v : v\in V\}$, which is finite, so Proposition~\ref{prop:embed-trim} now gives the conclusion.
	\jessiet{Do we need to show the preimage of $\Gamma$ is $V$?}
	
	\raft{I might be delusional, but this second part ended up being much slicker than I'd thought, by essentially chaining definitions and maps.  Please check!}
	For the second part, let $\gamma':\simplex\toto\R'$ be the finite elicitable property embedded by $L$, with embedding $\varphi:\R'\to\reals^d$, and let $\psi$ be a calibrated link to a non-redundant elicitable property $\gamma:\simplex\toto\R$.
	Then letting $\psi' = (\psi \circ \varphi):\R'\to\R$, we see that $\psi'$ is a calibrated link from $\gamma'$ to $\gamma$:
	for all $r'\in\R'$, we have $\gamma'_{r'} = \prop{L}_{\varphi(r')} \subseteq \gamma_{\psi(\varphi(r'))}$.
	In particular, $\gamma'$ refines $\gamma$, and as $\gamma'$ is finite, $\gamma$ must be finite.
\end{proof}

\jessie{Other results on refined properties? Assuming we don't want the embedding dimension conjecture brought up on 02.03.2020 in here.}
\begin{conjecture}
	Let $\gamma'$ be a refinement of $\gamma$.
	Then a loss $\ell$ embedding for $\gamma'$ also embeds $\gamma$. \jessie{Not quite.  I think something similar is true, but we need more assumptions. See below for a related statement though.}
\end{conjecture}

\begin{proposition}
	Take $\prop{\ell} =: \gamma : \simplex \toto \R$ and $\prop{\ell'} =:\gamma' : \simplex \toto \R'$.
	If $\gamma'$ refines $\gamma$ and $L'$ is calibrated with respect to  the discrete loss $\ell'$, then there exists a link $\psi$ so that $(L', \psi)$ is calibrated with respect to $\ell$.
\end{proposition}
\begin{proof}
	Let us construct the link $\psi$ such that, for all $r' \in \R'$ consider $r \in \R$ so that $\gamma'_{r'} \subseteq \gamma_r$.
	Define $\psi$ so that $\psi'(u) = r' \implies \psi(u) = r$.
	
	To see this link and surrogate are calibrated with respect to $\ell$, consider that for any fixed $p \in \simplex$, $\{u \in \reals^d : \psi(u) \not \in \gamma(p)\} \subseteq \{u \in \reals^d : \psi'(u) \not \in \gamma'(p)\}$, which in turn implies that the infimum over the first term of the expected loss is at least the infimum over the second term, which is strictly greater than the Bayes Risk of $L'$ at $p$ by calibration of $(L', \psi')$.
	
	\jessiet{Probably need to be more thorough on the subset argument.}
	That is, 
	\begin{align*}
		\{u \in \reals^d : \psi(u) \not \in \gamma(p)\} &\subseteq \{u \in \reals^d : \psi'(u) \not \in \gamma'(p)\}\\
		\implies
		\inf_{u \in \reals^d : \psi(u) \not \in \gamma(p)} \inprod{p}{L'(u)} &\geq \inf_{u \in \reals^d : \psi'(u) \not \in \gamma'(p)} \inprod{p}{L'(u)} > \inf_u \inprod{p}{L'(u)}\\
		\implies 		\inf_{u \in \reals^d : \psi(u) \not \in \gamma(p)} \inprod{p}{L'(u)} &> \inf_u \inprod{p}{L'(u)}~.~
	\end{align*}
	Thus, as $p$ is arbitrary, we observe $(L', \psi)$ is calibrated with respect to $\ell$.
\end{proof}

\begin{conjecture}
	Let $\gamma'$ refine $\gamma = \prop{\ell}$ and $L'$ embeds $\gamma'$ by the injection $\varphi'$.
	Define $\phi : \R \to \R'$ such that $\phi(r) = r' \implies \gamma'_{r'} \subseteq \gamma_r$.
	Let $\varphi:\R \to \reals^d = \phi \circ \varphi'$
	Then $(-\risk{L'|_{\varphi(R)}})^*$ embeds $\ell$.
\end{conjecture}
\fi

%\section{Polyhedral losses}\label{app:polyhedral-losses}


%\raft{The following statement is true I believe, but low priority: ``An elicitable property $\Gamma:\simplex\toto\reals$ is convex elicitable (elicited by a convex $L : \reals \to \reals^\Y$) if and only if it is monotone.''  Start of the proof commented out.  Just need to show that $b$ is the upper limit of $a$ and $a$ the lower of $b$; should follow from elicitability of $\Gamma$.}
%\begin{lemma}\label{lem:prop-L-monotone}
%  For any convex $L : \reals \to \reals^\Y_+$, the property $\prop{L}$ is monotone.
%\end{lemma}
%\begin{proof}
%  If $L$ is convex and elicits $\Gamma$, let $a,b$ be defined by $a(r)_y = \partial_- L(r)_y$ and $b(r) = \partial_+ L(r)_y$, that is, the left and right derivatives of $L(\cdot)_y$ at $r$, respectively.
%  Then $\partial L(r)_y = [a(r)_y,b(r)_y]$.
%  We now have $r \in \prop{L}(p) \iff 0 \in \partial \inprod{p}{L(r)} \iff \inprod{a(r)}{p} \leq 0 \leq \inprod{b(r)}{p}$, showing the first condition.
%  The second condition follows as the subgradients of $L$ are monotone functions (see e.g.~\citet[Theorem 24.1]{rockafellar1997convex}).
%  % Conversely, given such an $a,b$, we appeal to~\citet[Theorem 24.2]{rockafellar1997convex}, which gives us that $L(u)_y := \int_0^u a(u)_y$ is convex, and
%\end{proof}

%\newcommand{\Pbar}{\overline P}
%\begin{lemma}\label{lem:pbar}
%  Let $\gamma:\simplex\toto\R$ be a finite elicitable property, and suppose there is a calibrated link $\psi$ from an elicitable $\Gamma$ to $\gamma$.
%  For each $r\in\R$, define $P_r = \bigcup_{u\in\psi^{-1}(r)} \Gamma_u \subseteq \simplex$, and let $\Pbar_r$ denote the closure of the convex hull of $P_r$.
%  Then $\gamma_r = \Pbar_r$ for all $r\in\R$.
%\end{lemma}
%\begin{proof}
%  As $P_r \subseteq \gamma_r$ by the definition of calibration, and $\gamma_r$ is closed and convex, we must have $\Pbar_r \subseteq \gamma_r$.
%  Furthermore, again by calibration of $\psi$, we must have $\bigcup_{r\in\R} P_r = \bigcup_{u\in\reals} \Gamma_u = \simplex$, and thus $\bigcup_{r\in\R} \Pbar_r = \simplex$ as well.
%  Suppose for a contradiction that $\gamma_r \neq \Pbar_r$ for some $r\in\R$.
%  From Lemma~\ref{lem:finite-full-dim}, $\gamma_r$ has nonempty interior, so we must have some $p\in\inter\gamma_r \setminus \Pbar_r$.
%  But as $\bigcup_{r'\in\R} \Pbar_{r'} = \simplex$, we then have some $r'\neq r$ with $p\in\Pbar_{r'} \subseteq \gamma_{r'}$.
%  By Theorem~\ref{thm:aurenhammer}, the level sets of $\gamma$ form a power diagram, and in particular a cell complex, so we have contradicted point (ii) of Definition~\ref{def:cell-complex}: the relative interiors of the faces must not be disjoint.
%  Hence, for all $r\in\R$ we have $\gamma_r = \Pbar_r$.
%\end{proof}



%\begin{proof}[Proof of Theorem~\ref{thm:polyhedral-embed-prop}]
%  Let $L:\reals^d\to\reals_+^\Y$ be a polyhedral loss, and $\Gamma = \prop{L}$.
%  By Lemma~\ref{lem:polyhedral-range-gamma}, $\U = \Gamma(\simplex)$ is finite.
%  For any $U \in \U$, let $\Gamma_U = \{p\in\simplex | \Gamma(p) = U\}$, which is nonempty by definition.
%  Observe that for any $p\in\simplex$ and $u\in\reals^d$, we have $p \in \Gamma_u \iff u \in \Gamma(p) \iff U = \Gamma(p) \land u \in U \iff p\in\Gamma_U \land u \in U$.
%  Thus, we have for all $u\in\reals^d$ that $\Gamma_u = \cup\{\Gamma_U | U\in\U,u\in U\}$.
%  Now $\trim(\Gamma)$ is finite because the powerset of $\U$ is finite, and we apply Proposition~\ref{prop:embed-trim}.
%
%%
%%  \raf{I might be delusional, but this second part ended up being much slicker than I'd thought, by essentially chaining definitions and maps.  Please check!}
%%  For the second part, let $\gamma':\simplex\toto\R'$ be the finite elicitable property embedded by $L$, with embedding $\varphi:\R'\to\reals^d$, and let $\psi$ be a calibrated link to a non-redundant elicitable property $\gamma:\simplex\toto\R$.
%%  Then letting $\psi' = (\psi \circ \varphi):\R'\to\R$, we see that $\psi'$ is a calibrated link from $\gamma'$ to $\gamma$:
%%  for all $r'\in\R'$, we have $\gamma'_{r'} = \prop{L}_{\varphi(r')} \subseteq \gamma_{\psi(\varphi(r'))}$.
%%  In particular, $\gamma'$ refines $\gamma$, and as $\gamma'$ is finite, $\gamma$ must be finite.
%\end{proof}

\section{Thickened link and calibration} \label{app:calibration}

We define some notation and assumptions to be used throughout this section.
Let some norm $\|\cdot\|$ on finite-dimensional Euclidean space be given.
Given a set $T$ and a point $u$, let $d(T,u) = \inf_{t \in T} \|t-u\|$.
Given two sets $T,T'$, let $d(T,T') = \inf_{t\in T, t' \in T'} \|t-t'\|$.
Finally, let the ``thickening'' $B(T,\epsilon)$ be defined as
  \[ B(T,\epsilon) = \{u \in \R' : d(T,u) < \epsilon \} . \]

\begin{assumption} \label{assume:cal}
  $\ell: \R \times \Y \to \reals^{\Y}_+$ is a loss on a finite report set $\R$, eliciting the property $\gamma: \simplex \toto \R$.
  It is embedded by $L: \reals^d \times \Y \to \reals^{\Y}_+$, which elicits the property $\Gamma: \simplex \toto \reals^d$.
  The embedding points are $\{\varphi(r) : r \in \R\}$.
\end{assumption}

Given Assumption \ref{assume:cal}, let $\mathcal{S} \subseteq 2^{\R}$ be defined as $\mathcal{S} = \{\gamma(p) : p \in \Delta_{\Y}\}$.
In other words, for each $p$, we take the set of optimal reports $R = \gamma(p) \subseteq \R$, and we add $R$ to $\mathcal{S}$.
Let $\U \subseteq 2^{\reals^d}$ be defined as $\U = \{\Gamma(p) : p \in \Delta_{\Y}\}$.
For each $U \in \U$, let $R_U = \{r: \varphi(r) \in U\}$.

The next lemma shows that if a subset of $\U$ intersect, then their corresponding report sets intersect as well.
\begin{lemma} \label{lemma:calibrated-pos}
  Let $\U' \subseteq \U$.
  If $\cap_{U\in\U'} U \neq \emptyset$ then $\cap_{U\in\U'} R_U \neq \emptyset$.
\end{lemma}
\begin{proof}
  Let $u \in \cap_{U\in\U'} U$.
  Then we claim there is some $r$ such that $\Gamma_u \subseteq \gamma_r$.
  This follows from Proposition \ref{prop:embed-trim}, which shows that $\trim(\Gamma) = \{ \gamma_r : r \in \R\}$.
  Each $\Gamma_u$ is either in $\trim(\Gamma)$ or is contained in some set in $\trim(\Gamma)$, by definition, proving the claim.

  For each $U \in \U'$, for any $p$ such that $U = \Gamma(p)$, we have in particular that $u$ is optimal for $p$, so $p \in \Gamma_u$, so $p \in \gamma_r$, so $r$ is optimal for $p$.
  This implies that $\phi(r)$, the embedding point, is optimal for $p$, so $\phi(r) \in U$.
  This holds for all $U \in \U'$, so $r \in \cap_{U\in\U'} R_U$, so it is nonempty.
\end{proof}

\begin{lemma} \label{lemma:enclose-halfspaces}
  Let $D$ be a closed, convex polyhedron in $\reals^d$.
  For any $\epsilon > 0$, there exists an \emph{open}, convex set $D'$, the intersection of a finite number of open halfspaces, such that
    \[ D \subseteq D' \subseteq B(D,\epsilon) . \]
\end{lemma}
\begin{proof}
  Let $S$ be the standard open $\epsilon$-ball $B(\{\vec{0}\},\epsilon)$.
  Note that $B(D,\epsilon) = D + S$ where $+$ is the Minkowski sum.
%  Now let $S' = \{u : \|u\|_1 < \delta\}$ be the open $\delta$ ball in $L_1$ norm.
%  By equivalence of norms in Euclidean space, \bo{cite} we can take $\delta$ small enough yet positive such that $S' \subseteq S$.
%  By Lemma \ref{lemma:open-plus-closed-poly}, the Minkowski sum $D' = D + S'$ is an open polyhedron, i.e. the intersection of a finite number of open halfspaces.
  Now let $S' = \{u : \|u\|_1 \leq \delta\}$ be the closed $\delta$ ball in $L_1$ norm.
  By equivalence of norms in Euclidean space~\cite[Appendix A.1.4]{boyd2004convex}, we can take $\delta$ small enough yet positive such that $S' \subseteq S$.
  By standard results, the Minkowski sum of two closed, convex polyhedra, $D'' = D + S'$ is a closed polyhedron, i.e. the intersection of a finite number of closed halfspaces. (A proof: we can form the higher-dimensional polyhedron $\{(x,y,z) : x \in D, y \in S', z = x+y\}$, then project onto the $z$ coordinates.)

  Now, if $T' \subseteq T$, then the Minkowksi sum satisfies $D + T' \subseteq D + T$.
  In particular, because $\emptyset \subseteq S' \subseteq S$, we have
    \[ D \subseteq D'' \subseteq B(D,\epsilon) . \]
  Now let $D'$ be the interior of $D''$, i.e. if $D'' = \{x : Ax \leq b\}$, then we let $D' = \{x: Ax < b\}$.
  We retain $D' \subseteq B(D,\epsilon)$.
  Further, we retain $D \subseteq D'$, because $D$ is contained in the interior of $D'' = D + S'$.
  (Proof: if $x \in D$, then for some $\gamma$, $x + B(\{\vec{0}\},\gamma) = B(x,\gamma)$ is contained in $D + S'$.)
  This proves the lemma.
\end{proof}

\begin{lemma} \label{lemma:thick-nonempty}
  Let $\{U_j : j \in \mathcal{J}\}$ be a finite collection of closed, convex sets with $\cap_{j\in\mathcal{J}} U_j \neq \emptyset$.
  Then there exists  $\epsilon > 0$ such that $\cap_j B(U_j,\epsilon) \subseteq B(\cap_j U_j, \delta)$.
\end{lemma}
\begin{proof}
  We induct on $|\mathcal{J}|$.
  If $|\mathcal{J}|=1$, set $\epsilon = \delta$.
  If $|\mathcal{J}|>1$, let $j\in\mathcal{J}$ be arbitrary, let $U' = \cap_{j'\neq j} U_{j'}$, and let $C(\epsilon) = \cap_{j' \neq j} B(U_{j'},\epsilon)$.
  Let $D = U_j \cap U'$.
  We must show that $B(U_j,\epsilon) \cap C(\epsilon) \subseteq B(D,\delta)$.
  By Lemma \ref{lemma:enclose-halfspaces}, we can enclose $D$ strictly within a polyhedron $D'$, the intersection of a finite number of open halfspaces, which is itself strictly enclosed in $B(D,\delta)$.
  (For example, if $D$ is a point, then enclose it in a hypercube, which is enclosed in the ball $B(D,\delta)$.)
  We will prove that, for small enough $\epsilon$, $B(U_j,\epsilon) \cap C(\epsilon)$ is contained in $D'$.
  This implies that it is contained in $B(D,\delta)$.

  For each halfspace defining $D'$, consider its complement $F$, a closed halfspace.
  We prove that $F \cap B(U_j,\epsilon) \cap C(\epsilon) = \emptyset$.
  Consider the intersections of $F$ with $U$ and $U'$, call them $G$ and $G'$.
  These are closed, convex sets that do not intersect (because $D$ in contained in the complement of $F$).
  So $G$ and $G'$ are separated by a nonzero distance, so $B(G,\gamma) \cap B(G',\gamma) = \emptyset$ for small enough $\gamma$.
  And $B(G,\gamma) = F \cap B(U_j,\gamma)$ while $B(G',\gamma) = F \cap B(U',\gamma)$.
  This proves that $F \cap B(U_j,\gamma) \cap B(U',\gamma) = \emptyset$.
  By inductive assumption, $C(\epsilon) \subseteq B(U',\gamma)$ for small enough $\epsilon = \epsilon_F$.
  So $F \cap B(U_j,\gamma) \cap C(\epsilon) = \emptyset$.
  We now let $\epsilon$ be the minimum over these finitely many $\epsilon_F$ (one per halfspace).
\end{proof}

\begin{figure}
\caption{Illustration of a special case of the proof of Lemma \ref{lemma:thick-nonempty} where there are two sets $U_1,U_2$ and their intersection $D$ is a point. We build the polyhedron $D'$ inside $B(D,\delta)$. By considering each halfspace that defines $D'$, we then show that for small enough $\epsilon$, $B(U_1,\epsilon)$ and $B(U_2,\epsilon)$ do not intersect outside $D'$. So the intersection is contained in $D'$, so it is contained in $B(D,\delta)$.}
\includegraphics[width=0.24\textwidth]{figs/separated-proof-2} \hfill
\includegraphics[width=0.24\textwidth]{figs/separated-proof-3} \hfill
\includegraphics[width=0.24\textwidth]{figs/separated-proof-4} \hfill
\includegraphics[width=0.24\textwidth]{figs/separated-proof-5}
\end{figure}

\begin{lemma} \label{lemma:thick-empty}
  Let $\{U_j : j \in \mathcal{J}\}$ be a finite collection of nonempty closed, convex sets with $\cap_{j\in\mathcal{J}} U_j = \emptyset$.
  Then for all $\delta > 0$, there exists  $\epsilon > 0$ such that $\cap_{j\in\mathcal{J}} B(U_j,\epsilon) = \emptyset$.
\end{lemma}
\begin{proof}
  By induction on the size of the family.
  Note that the family must have size at least two.
  Let $U_j$ be any set in the family and let $U' = \cap_{j' \neq j} U_{j'}$.
  There are two possibilities.

  The first possibility, which includes the base case where the size of the family is two, is the case $U'$ is nonempty.
  Because $U_j$ and $U'$ are non-intersecting closed convex sets, they are separated by some distance $\epsilon$.
  By Lemma \ref{lemma:thick-nonempty}, for any $\epsilon > 0$, there exists $\delta > 0$ such that $\cap_{j'\neq j} B(U_{j'},\delta) \subseteq B(U', \epsilon/3)$.
  Then we have $B(U_j, \epsilon/3) \cap B(U', \epsilon/3) = \emptyset$.

  The second possibility is that $U'$ is empty.
  This implies we are not in the base case, as the family must have three or more sets.
  By inductive assumption, for small enough $\delta$ we have $\cap_{j' \neq j} B(U_{j'},\delta) = \emptyset$, which proves this case.
\end{proof}


\begin{corollary} \label{cor:thick-intersect}
  There exists a small enough $\epsilon > 0$ such that, for any subset $\{U_j : j \in \mathcal{J}\}$ of $\U$, if $\cap_j U_j = \emptyset$, then $\cap_j B(U_j,\epsilon) = \emptyset$.
\end{corollary}
\begin{proof}
  For each subset, Lemma \ref{lemma:thick-empty} gives an $\epsilon$.
  We take the minimum over these finitely many choices.
\end{proof}

\begin{theorem} \label{thm:small-eps-thick}
  For all small enough $\epsilon$, the epsilon-thickened link $\psi$ (Definition \ref{def:eps-thick-link}) is a well-defined link function from $\R'$ to $\R$, i.e. $\psi(u) \neq \bot$ for all $u$.
\end{theorem}
\begin{proof}
  Fix a small enough $\epsilon$ as promised by Corollary \ref{cor:thick-intersect}.
  Consider any $u \in \R'$.
  If $u$ is not in $B(U,\epsilon)$ for any $U \in \U$, then we have $\Psi(u) = \R$, so it is nonempty.
  Otherwise, let $\{U_j : j \in \mathcal{J}\}$ be the family whose thickenings intersect at $u$.
  By Corollary \ref{cor:thick-intersect}, because of our choice of $\epsilon$, the family themselves has nonempty intersection.
  By Lemma \ref{lemma:calibrated-pos}, their corresponding report sets $\{R_j : j \in \mathcal{J}\}$ also intersect at some $r$, so $\Psi(u)$ is nonempty.
\end{proof}

In the rest of the section, for shorthand, we write $L(u;p) := \langle p, L(u) \rangle$ and similarly $\ell(r;p)$.

\begin{lemma} \label{lemma:exposed-shortest}
  Let $U$ be a convex, closed set and $u \not\in U$.
  Then $\inf_{u^* \in U} \|u-u^*\|$ is achieved by some unique $u^* \in U$.
  Furthermore, $u^*$ is the unique member of $U$ such that $u = u^* + \alpha v$ for some $\alpha > 0$ and unit vector $v$ that exposes $u^*$.
\end{lemma}
\begin{proof}
  Unique achievement of the infimum is well-known.
  (Achievement follows e.g. because the set $U \cap \{u' : \|u - u'\| \leq d(U,u) + 1\}$ is closed and compact, so the continuous function $u' \mapsto \|u-u'\|$ achieves its infimum.
  Uniqueness follows because for two different points $u',u''$ at the same distance from $u$, the point $0.5u' + 0.5u''$ is strictly closer and also lies in the convex set $U$.)
  Now suppose $u = u' + \alpha' v'$ where $v'$ is a unit vector exposing $u'$.
  Then $U$ is contained in the halfspace $\{u'': \langle u'', v'' \rangle \leq \langle u',v' \rangle \}$.
  But every point in this halfspace is distance at least $\alpha'$ from $u$, as $\|u-u''\| \geq \langle v, u - u'' \rangle \geq \langle v, u-u'\rangle = \alpha'$.
  So $u'$ uniquely achieves this minimum distance.
\end{proof}

\begin{lemma} \label{lemma:distance-loss}
  \raft{What I changed: linear ftn $\to$ affine; name the cell $U_f$ for $f$; also name set of functions $\F$, normals $V_f$, etc}
  If $L$ is a polyhedral loss, then for each $p$, there exists a constant $c$ such that, for all $u$,
    \[ L(u;p) - \inf_{u^* \in \R'} L(u^*;p) \geq c \cdot d(\Gamma(p),u) . \]
\end{lemma}
\begin{proof}
  Fix $p$ and let $U = \Gamma(p)$.
  If $u \in U$, then both sides are zero.
  So it remains to find a $c$ such that the inequality holds for all $u \not\in U$.

  $L(\cdot;p)$ is a convex polyhedral function, so it is the pointwise maximum over finitely many affine functions.
  Recall that $\risk{L}(p) = \min_{u} L(u;p)$, the Bayes risk.
  Construct the convex polyhedral function $\hat{L}(\cdot;p)$ by dropping from the maximum those affine functions that are never equal to $L$ for any $u^* \in U$.
  We have $\hat{L}(u^*;p) = \risk{L}(p)$ for all $u^* \in U$ and $\hat{L}(u;p) \leq L(u;p)$ for all $u \not\in U$.
  Now $\hat{L}$ is also a maximum over finitely many affine functions $\F$.
  Each such function $f\in\F$ is equal to $\hat{L}$ above a closed, convex cell $U_f\subseteq\reals^d$ in the power diagram formed by projecting $\hat{L}(\cdot;p)$.
  If $f$ has nonzero gradient, then $U_f \cap U$ is a face of $U$.
  We will prove that there exists $c_f > 0$ such that, for all $u\in U_f$,
    \[ \hat{L}(u;p) \geq \risk{L}(p) + c_f \cdot d(U,u) . \]
  Taking $c$ to be the minimum of $c_f$ over the finitely many $f\in\F$ with nonzero gradient (which covers all points $u \not\in U$) will complete the proof.

  Consider the set of unit vectors $V = \{v \in \reals^d : \|v\|=1\}$ and the boundary of $U$, denoted $\partial U$.
  For any $u^*\in\partial U$, $v\in V$ such that $v$ exposes $u^*$, let
  \raft{NOTE: we need to define ``exposes'' or just phrase in terms of normals: $v$ is normal to $U$ at $u^*$}
  $G_{u^*,v} = \left\{ u^* + \beta v : \beta \geq 0 \right\}$
  be the ray leaving $U$ from $u^*$ in direction $v$.
  % Note that $\{ u : u \not\in U\} \subseteq \cup_{u^*,v} G_{u^*,v}$.
  For each $f\in\F$, we define the set $R_f \subseteq \partial U \times V$ to be the points $(u^*,v)$ such that there exists $\epsilon>0$ with $G_{u^*,v} \cap U_f = G_{u^*,v} \cap B(u^*,\epsilon)$; that is, such that the ray $G_{u^*,v}$ starts its journey in $U_f$.
  Futhermore, define $U^*_{f,v} = \{u^* \in \partial U : (u^*,v)\in R_f\}$ and $V_f = \{v\in V: \exists u^*\in U_f\cap U,\: (u^*,v) \in R_f\}$.
  (That is, $U^*_{f,v}$ is the set of points from which the ray in direction $v$ begins in $U_f$, and $V_f$ is the set of all normal directions in which some ray begins in $U_f$.)
  Finally, define $G_f = \cup_{(u^*,v)\in R_f} G_{u^*,v}$ as the union of all such rays beginning in $U_f$.
  Note that $\cup_{f\in\F} G_f \supseteq \reals^d \setminus U$; this follows as every point not in $U$ is on a normal ray out of $U$, which must begin in some cell $U_f$.

  We will prove the following steps:
  \begin{enumerate}
  % \item For all $u^*\in\partial U$, $v\in V$ such that $v$ exposes $u^*$, there is some $f\in\F$ such that $u^* \in U_f$ and $G_{u^*,v} \cap U_f \neq \{u^*\}$.
  %   (That is, the ray $G_{u^*,v}$ begins its journey away from $U$ within $U_f$.)
    \item For all $f\in\F$, $v\in V_f$, there exists a constant $c_{f,v} > 0$ such that $L(u;p) \geq \risk{L}(p) + c_{f,v} \cdot d(U,u)$ for all $u \in G_{u^*,v}$ and all $u^*\in U^*_{f,v}$.
    \item For all $f\in \F$, the set $V_f$ is compact, and the map $v \mapsto c_{f,v}$ is continuous on $V_f$.
    \item Hence, there is an infimum $c_f > 0$ such that $f(u) \geq \risk{L}(p) + c_f \cdot d(U,u)$ for all $u\in G_f$.
    \item Let $c = \min \{c_f : f\in\F, \nabla f \neq 0\}$; then $L(u;p) \geq \risk{L}(p) + c \cdot d(U,u)$ for all $u \not\in U$.
  \end{enumerate}

  % (1) Follows because $G_{u^*,v}$ is contained in a cell of the power diagram, which is a set of points where $\hat{L}(\cdot;p) = f(\cdot)$. \bo{Should have more justification.}

  (1) Let $\nabla f$ denote the gradient of the affine function $f$.
  Note that because $u^*$ is on the boundary of $U$, we have $f(u^*) = \hat{L}(u^*;p) = \risk{L}(p)$.
  So we can write, using Lemma \ref{lemma:exposed-shortest},
  \begin{align*}
    f(u) &= f(u^*) + (\nabla f)\cdot (u - u^*)  \\
    f(u) &= f(u^*) + (\nabla f)\cdot (d(u^*,u) v)  \\
         &= \risk{L}(p) + c_{f,v} \cdot d(U,u)  \\
  \end{align*}
  where $c_{f,v} = (\nabla f)\cdot v$.
  We must have $c_{f,v} > 0$ because the set $U$ minimizes $L(\cdot;p)$, so $f(u) > f(u^*) = \risk{L}(p)$.
  The result now follows as $L(u;p) \geq \hat L(u;p) \geq f(u)$.

  (2) The intersection $U_f \cap U$ is a face of $U$, and thus decomposes as the union of relative interiors of subfaces, $U_f \cap U = \cup_i \mathrm{ri}(F_i)$.
  For each $i$, let $V_i = \{v\in V: \exists u^*\in \mathrm{ri}(F_i),\: (u^*,v)\in R_f\}$.
  % set $N_U(u^*)$ is the same for all $u^*\in\mathrm{ri}(F_i)$~\cite{lu2008normal}.
  % \raft{That is a total punt reference---not sure the result is in there, but it easily could be!}
  For any $v\in V_i$, we may consider the power diagram restricted to $A$, the affine hull of $\{u + \alpha v: u\in U, \alpha\in\reals\}$.
  As there is some $u^*\in U$ such that $(u^*,v)\in R_f$, in particular, $U_f\cap A$ intersects $A\cap\mathrm{ri}(F_i)$ and thus must contain $A\cap\mathrm{ri}(F_i)$.
  We conclude that $(u',v)\in R_f$ for all other $u'\in \mathrm{ri}(F_i)$.
  \raft{The idea here is that if you can start in $\mathrm{ri}(F_i)$ and move in direction $v$ and land immediately in $U_f$, then you can do that anywhere from $\mathrm{ri}(F_i)$; otherwise $U_f$ intersects only part of $\mathrm{ri}(F_i)$ (at least when restricting to $A$), a contradiction.}
  Thus, we have $\{(u^*,v) \in R_f : u^*\in F_i\} = F_i \times V_i$.
  For closure, pick any $u^*\in \mathrm{ri}(F_i)$, and consider a sequence $\{v_j\}_j$ with $(u^*,v_j)\in R_f$, and corresponding witnesses $\{\epsilon_j\}_j$.
  Then we have $u^* + \epsilon_j v_j \in U_f$ for all $j$, and as $u^*\in U_f$ and $U_f$ is closed and convex, the limiting point must be contained in $U_f$ as well.
  \raft{Thus is totally bogus actually, since the limiting point could be $u^*$ itself.  Need a different approach I think.}
  We have now shown $V_f$ to be the union of finitely many closed convex sets, and thus closed.
  Boundedness follows as $V$ is bounded.
  Finally, $c_{f,v}$ is linear in $v$, and thus continuous.

  Steps (3) and (4) are immediate and complete the proof.
\end{proof}

\begin{theorem} \label{thm:app-eps-thick-sep}
  For small enough $\epsilon$, the $\epsilon$-thickened link $\psi$ (Definition \ref{def:eps-thick-link}) satisfies that, for all $p$, there exists $\delta > 0$ such that, for all $u \in \R'$,
    \[ L(u;p) - \inf_{u^* \in \R'} L(u^*;p) \geq \delta \left[ \ell(\psi(u);p) - \min_{r^* \in \R} \ell(r^*;p) \right] . \]
\end{theorem}
\begin{proof}
  We take the $\epsilon$ thickened link, which is well-defined by Theorem \ref{thm:small-eps-thick}.
  Fix $p$ and let $U = \Gamma(p)$.
  The left-hand side is nonnegative, so it suffices to prove the result for all $u$ such that the right side is strictly positive, i.e. for all $u$ such that $\psi(u) \not\in \gamma(p)$.
  By definition of the $\epsilon$-thickened link, we must have $d(U,u) \geq \epsilon$.
  By Lemma \ref{lemma:distance-loss}, we have $L(u;p) - \inf_{u^*} L(u^*;p) \geq C$ where $C = c\epsilon$ for some $c > 0$.
  This holds for all $u$.
  Meanwhile,
    \[ \ell(\psi(u);p) - \min_{r^*} \ell(r^*;p) \leq \max_{r \in \R} \ell(r;p) - \min_{r^* \in \R} \ell(r^*;p) =: D, \]
  for some constant $D$.
  This also holds for all $u$.
  Set $\delta = \frac{C}{D}$ to complete the proof.
\end{proof}

\begin{proof}[Proof of Theorem \ref{thm:eps-thick-calibrated}]
  The two claims are Theorems \ref{thm:small-eps-thick} and \ref{thm:app-eps-thick-sep}.
\end{proof}

\section{Omitted Proofs}\label{app:omitted-proofs}
% While Definition~\ref{def:loss-embed} gives the notion of one \emph{loss} embedding another, we now generalize the notion of one \emph{property} embedding another. \jessie{Merge with earlier definitions if we keep this section}
% \begin{definition}\label{def:prop-embed}
%   A property $\Gamma : \simplex \toto \reals^d$ embeds a property $\gamma:\simplex \toto \R$ on a set $\Sc$ if there exists some injective embedding $\varphi:\R \to \reals^d$ such that for all $p \in \simplex$ and $r \in \R$, we have $r \in \gamma(p) \iff \varphi(r) \in \Gamma(p)$.
%   If there is a representative set $\Sc$ such that $\Gamma$ embeds $\gamma$ on $\Sc$, then we simply say $\Gamma$ embeds $\gamma$.
%   Moreover, we say a loss embeds a property when it embeds a loss eliciting the property.
% \end{definition}
% 
% 
% By condition (ii.) of Definition~\ref{def:loss-embed}, we then have $L$ embedding $\ell$ implies $\prop{L}$ embeds $\prop{\ell}$.
% However, Definitions~\ref{def:loss-embed} and~\ref{def:prop-embed} are not immediately equivalent because property embedding does not capture the requirement of matching losses on embedded points; i.e., that $L(\varphi(r)) = \ell(r)$ for all $r \in \R$.  
% However, Lemma~\ref{lem:embed-defs-equiv} shows an equivalence follows without loss of generality.
% 
% \begin{lemma}\label{lem:embed-defs-equiv}
%	% \raft{Introduce $L$}
%	Let $L : \reals^d \to \reals^\Y_+$ be a loss function whose infimum is attained in expectation for all $p \in \simplex$.
%	If the property $\Gamma := \prop{L}$ embeds the finite property $\gamma : \simplex \toto \R$, then there is a discrete loss $\ell:\R \to \reals^\Y_+$ such that $\ell$ elicits $\gamma$ and $L$ embeds $\ell$.
% \end{lemma}
% \begin{proof}
%   \raft{All the important pieces are here, but take it slower.  State $\ell$ as a definition (let $\ell$ be given by...) and show why $L$ embeds it.\jessie{Took another pass but changed technique, adding in this detail.  Now very similar to the proof of Prop 1 forward direction.}}
%   Consider the embedding $\varphi$ given by the property embedding and take the discrete loss $\ell:r \mapsto L(\varphi(r))$.
%   Now, we claim that $L$ embeds $\ell$, and by Proposition~\ref{prop:embed-bayes-risks}, we can simply show $\risk{L} = \risk{\ell}$, and the risks being polyhedral follows from $\ell$ being discrete.
%   
%   First, we show that for all $p \in \simplex$, we have $(1) \inf_{u \in \reals^d}\inprod{p}{L(u)} = (2) \inf_{r \in \R}\inprod{p}{L(\varphi(r))}$, and the equality of risks follows.
%   $(1) \leq (2)$ follows from the fact that $\varphi(\R) \subset \reals^d$ and definition of infimum. 
%   Consider that if we had $(1) < (2)$ for some $p \in \simplex$, then there would be no $r$ such that $\varphi(r) \in \Gamma(p)$, and therefore, we would have $\gamma(p) = \emptyset$, yielding a contradiction. 
%   This gives us $(1) = (2)$, so we have $\risk{L} = \risk{L|_{\varphi(\R)}} = \risk{\ell}$.
%	% 
%	% Now consider the property $\Gamma' := p \mapsto \Gamma(p) \cap \varphi(\R)$.
%	% We have $\Gamma'$ nondegenerate since $\Gamma$ embedding $\gamma$ implies that, for all $p \in \simplex$, there is some $r \in \R$ such that $\varphi(r) \in \Gamma(p)$.
%	% Therefore, for all $p \in \simplex$, we have $\inf_{r \in \R}\inprod{p}{L(\varphi(r))} = \inf_{r \in \R} \inprod{p}{\ell(r)}$, which yields $\risk{L|_{\varphi(\R)}} = \risk{\ell}$.
%	% Chaining the two equalities, we now have $\risk{L} = \risk{\ell}$.
%	
%	% To verify $\ell$ elicits $\gamma$, consider $r \in \gamma(p) \iff \varphi(r) \in \Gamma(p)$ by embedding, $ \iff \varphi(r) \in \argmin_{u \in \reals^d} \inprod{p}{L(u)}$ by $L$ eliciting $\Gamma$, and $ \iff r \in \argmin_{r \in \R} \inprod{p}{\ell(r)}$ by applying the definition of properties to the embedding definition.
%	% \hrule
%	%	If $\Gamma$ embeds a finite $\gamma$, $L$ must embed a discrete loss $\ell$ such that $\ell(r) = L(\varphi(r))$ for all $r \in \R$.  
%	%	Moreover, we can see $\ell$ elicits $\gamma$ since we have $r \in \gamma(p) \iff \varphi(r) \in \Gamma(p) \iff \varphi(r) \in \argmin_{u \in \reals^d} \inprod{p}{L(u)} \iff r \in \argmin_{r \in \R} \inprod{p}{\ell(r)}$.
%	% 
%	% \raft{Spell this out; I don't think these distributions are enough, for the same reason as above.  \jessie{Took a different approach of using Prop 1 to show the risks are equal.}}
%	%	The final condition of matching loss values is satisfied by considering the equality of the expected loss on each $\delta_y$ distribution.
% \end{proof}

\jessiet{GO THROUGH THIS}
When working with convex surrogates which are not strictly convex, one quickly encounters redundant properties: if $\inprod{p}{L(\cdot)}$ is minimized by a point where $\inprod{p}{L}$ is flat, then there will be an uncountable set of reports which also minimize the expected loss.
As results in property elicitation typically assume properties are non-redundant (e.g.~\cite{frongillo2014general,frongillo2015elicitation}), 
% it is useful to consider a transformation which removes redundant level sets, captured by the \emph{trim} operation.

\begin{definition}\label{def:trim}
  Given an elicitable property $\Gamma:\simplex \toto\R$, we define $\trim(\Gamma) = \{\Gamma_u : u \in \R \text{ s.t. } \neg\exists u'\in\R,u'\neq u,\, \Gamma_u \subsetneq \Gamma_{u'}\}$ as the set of maximal level sets of $\Gamma$.
\end{definition}
\jessie{Do we want to re-define $\trim(\Gamma) = \{\Gamma_r \mid r \in \Sc\}$ for any minimum representative set $\Sc$?}

% \btw{RF: Note for later: should be able to show that the union of trim is the simplex.\jessie{This is part of the proposition statement 2 now.}}
% Take note that the unlabeled property $\trim(\Gamma)$ is non-redundant, meaning that for any $\theta \in \trim(\Gamma)$, there is no level set $\theta' \in \trim(\Gamma)$ such that $\theta \subset \theta'$; this is a corollary of Proposition~\ref{prop:embed-trim}.


Before we state Proposition~\ref{prop:embed-trim}, we will need to general lemmas about properties and their losses.
The first follows from standard results relating finite properties to power diagrams (see Theorem~\ref{thm:aurenhammer}), and its proof is omitted.
The second is closely related to the definition of the trim operator: it states that if some subset of the reports are always represented among the minimizers of a loss, then one may remove all other reports and elicit the same property (with those other reports removed).

\begin{lemma}\label{lem:finite-full-dim}
  Let $\gamma$ be a finite (non-redundant) property elicited by a loss $L$.
  Then the negative Bayes risk $G$ of $L$ is polyhedral, and the level sets of $\gamma$ are the projections of the facets of the epigraph of $G$ onto $\simplex$, and thus form a power diagram.
  In particular, the level sets of $\gamma$ are full-dimensional in $\simplex$ (i.e.,\ of dimension $n-1$).
\end{lemma}

\btw{Commented out result about old trim def = new trim def - Jessie 2 June 21}
%\btw{JESSIE: Figures/an example might help clarify this}
%When a property $\Gamma$ embeds a finite property $\gamma$, we can show that the level sets of $\gamma$ correspond exactly to their embedded level sets of $\Gamma$, and that these embedded level sets are exactly $\trim(\Gamma)$.
%\jessiet{This lemma new to journal version}
%\begin{lemma}\label{lem:embedded-level-sets-trim}
%  Let $\Gamma$ be an elicitable property.
%  If $\Gamma$ embeds a (non-redundant) finite property $\gamma : \simplex \toto \R$ by the injection $\varphi$, then $\{\gamma_r : r \in \R\} = \{\Gamma_u : u \in \varphi(\R)\} = \trim(\Gamma)\proposedadd{=\trim(\gamma)}$.
%  \proposedadd{RETRY: Let $\Gamma$ be an elicitable property.
%    If $\Gamma$ embeds a finite property $\gamma : \simplex \toto \R$ by the injection $\varphi$, then for any minimum representative set $\Sc$ for $\gamma$, we have $\{\gamma_r : r \in \Sc\} = \{\Gamma_u : u \in \varphi(\Sc)\} = \trim(\Gamma)=\trim(\gamma)$.}
%\end{lemma}
%\begin{proof}
%  Let $L$ elicit $\Gamma$.
%  Take $\ell$ to be the loss embedded by $L$ from Lemma~\ref{lem:embed-defs-equiv}.
%  Moreover, by Proposition~\ref{prop:embed-bayes-risks}, we have $\risk{L} = \risk{\ell}$.
%  % \raft{Implicitly? Justify the claim.  \jessie{In definition of finite properties, we say that we assume we are talking about non-redundant.}}
%  As $\gamma$ is finite (and non-redundant by assumption), we know that each level set of $\gamma$ must be full-dimensional in $\affhull(\simplex)$. 
%  For all $r \in \R$, we know $\gamma_r = \Gamma_{\varphi(r)}$, so we must also have $\Gamma_{\varphi(r)}$ full-dimensional.
%  Moreover, we have $\{\gamma_r : r \in \R\} = \{\Gamma_{\varphi(r)} : r \in \R\}$ as a corollary since this is true for each $r \in \R$.
%  \raft{The main action is here.  (1) Epigraph of $-\risk{L}$. (2) There is really only one case: every level set is a face of the power diagram, which must be contained (weakly, so $\subseteq$) in a facet. (3) To show that level sets are faces, you'll want to appeal to some piece of a proof in the prev section.}
%  Since the cell $\Gamma_{\varphi(r)}$ is the projection of a facet of the epigraph of $-\risk{L}$, which is convex, any other level set intersecting $\Gamma_{\varphi(r)}$ can be considered as faces of the epigraph of $-\risk{L}$, and therefore a face of the induced power diagram.
%  Therefore, we remove lower-dimensional faces, as they are proper subsets of the level sets of embedded reports, and conclude such $\Gamma_u \not \in \trim(\Gamma)$.
%  % in one of two cases:
%  % First, if the level set $\Gamma_u$ (for $u \not \in \varphi(\R)$) is a projection of a lower-dimensional face of $\risk{L}$, which is contained in a facet of the epigraph.
%  % Therefore, we must have $\Gamma_u \subset \Gamma_{\varphi(r)}$ for some $r \in \R$, and therefore $\Gamma_u \not \in \trim(\Gamma)$.
%  
%  Second, $\Gamma_u = \Gamma_{\varphi(r)}$ for some $r \in \R$, then the level set is in both $\{\Gamma_{\varphi(r)} : r \in \R\}$ and $\trim(\Gamma)$.
%
%  \proposedadd{First, construct $\Sc := \Sc(L)$ as Definition~\ref{cons:rep-set}\jessiet{Argue there is a bijection from this set to any other minimum representative set}, which is a minimum representative set by Corollary~\ref{cor:poly-risk-fin-rep}. 
%    Moreover, we have $r \in \gamma(p) \iff \varphi(r) \in \Gamma(p)$ by embedding, and therefore $\gamma_r = \Gamma_{\varphi(r)}$, yielding $\{\gamma_r \mid r \in \Sc\} = \{\Gamma_{\varphi(r)} \mid r \in \Sc\}$ as this is true for all $r \in \Sc$.}
%
%  \proposedadd{Now let $T := \{\gamma_r \mid r \in \Sc\} = \trim(\gamma)$. 
%    First, if $\gamma_r \in T$, then it is not a subset of any level set as $\gamma_r$ is full-dimensional in the simplex as it corresponds to a facet of the epigraph of the negative risk $E_\ell$, and the supporting hyperplane is therefore unique.
%    Now, if $\theta \in \trim(\gamma)$, then there is some report $r'$ such that $\theta = \gamma_{r'}$.
%    If $r' \not \in \Sc$, we want to claim there is some $r$ such that $\gamma_{r'} = \gamma_r$.
%    As $\theta$ is a maximal level set, it has a nonempty interior.
%    Therefore, for any $p \in \inter{\theta}$ (and one such $p$ must exist), the hyperplane $p \mapsto (p, \ell(r'))$ uniquely supports the epigraph of negative risk.
%    If $r' \not \in \Sc$, then there must be some $r \in \Sc$ such that $p \mapsto (p, \ell(r))$ also supports the same facet of $E_\ell$; if no such $r \in \Sc$ existed, we would contradict our claim that $\Sc$ is representative for $\ell$.
%    Similarly, we have $\{\Gamma_u \mid u \in \varphi(\Sc)\} = \trim(\Gamma)$ by the same logic.}
%\end{proof}

%\jessiet{03.01.21 - Old version of Prop~\ref{prop:embed-trim} moved to comment}
% \jessiet{Can we remove this?}
% \begin{proof}[Original]
%   Let $L$ elicit $\Gamma$.
%   
%   1 $\Rightarrow$ 2:
%   By the embedding condition, taking $\R_1 = \reals^d$ and $\R_2 = \varphi(\R)$ satisfies the conditions of Lemma~\ref{lem:loss-restrict}: for all $p\in\simplex$, as $\gamma(p) \neq \emptyset$ by definition, we have some $r\in\gamma(p)$ and thus some $\varphi(r) \in \Gamma(p)$.
%   Let $G(p) := -\min_{u\in\reals^d} \inprod{p}{L(u)}$ be the negative Bayes risk of $L$, which is convex, and $G_{\R}$ that of $L|_{\varphi(\R)}$.
%   By the Lemma, we also have $G = G_\R$.
%   As $\gamma$ is finite, $G$ is polyhedral.
%   Moreover, the projection of the epigraph of $G$ onto $\simplex$ forms a power diagram, with the facets projecting onto the level sets of $\gamma$, the cells of the power diagram.
%   (See Theorem~\ref{thm:aurenhammer}.)
%   As $L$ elicits $\Gamma$, for all $u\in\reals^d$, the hyperplane $p\mapsto \inprod{p}{L(u)}$ is a supporting hyperplane of the epigraph of $G$ at $(p,G(p))$ if and only if $u\in\Gamma(p)$.
%   This supporting hyperplane exposes some face $F$ of the epigraph of $G$, which must be contained in some facet $F'$.
%   Thus, the projection of $F$, which is $\Gamma_u$, must be contained in the projection of $F'$, which is a level set of $\gamma$.
%   We conclude that $\Gamma_u \subseteq \gamma_r$ for some $r\in\R$.
%   Hence, $\trim(\Gamma) = \{\gamma_r : r\in\R\}$, which is finite, and unions to $\simplex$.
%   
%   2 $\Rightarrow$ 3: let $\R = \{u_1,\ldots,u_k\} \subseteq\reals^d$ be a set of distinct reports such that $\trim(\Gamma) = \{\Gamma_{u_1},\ldots,\Gamma_{u_k}\}$.
%   Now as $\cup\,\trim(\Gamma) = \simplex$, for any $p\in\simplex$, we have $p\in\Gamma_{u_i}$ for some $u_i\in\R$, and thus $\Gamma(p) \cap \R \neq \emptyset$.
%   We now satisfy the conditions of Lemma~\ref{lem:loss-restrict} with $\R_1 = \reals^d$ and $\R_2 = \R$.
%   The property $\gamma:p\mapsto\Gamma(p)\cap\R$ is non-redundant by the definition of $\trim$, finite, and elicitable.
%   Now from Lemma~\ref{lem:finite-full-dim}, the level sets $\Theta = \{\gamma_r:r\in\R\}$ are full-dimensional, and union to $\simplex$.
%   Statement 3 then follows from the fact that $\gamma_r = \Gamma_r$ for all $r\in\R$.
%   
%   3 $\Rightarrow$ 1: let $\Theta = \{\theta_1,\ldots,\theta_k\}$.
%   For all $i\in\{1,\ldots,k\}$ let $u_i\in\reals^d$ such that $\Gamma_{u_i} = \theta_i$.
%   Now define $\gamma:\simplex\toto\{1,\ldots,k\}$ by $\gamma(p) = \{i : p\in\theta_i\}$, which is non-degenerate as $\cup\,\Theta = \simplex$.
%   By construction, we have $\gamma_i = \theta_i = \Gamma_{u_i}$ for all $i$, so letting $\varphi(i) = u_i$ we satisfy the definition of embedding, namely statement 1.
% \end{proof}






% \begin{lemma}\label{lem:fdls-in-trim}
%   If $\theta$ is a full-dimensional level set (in the simplex) of the elicitable property $\Gamma$, then $\theta \in \trim(\Gamma)$.
% \end{lemma}
% \begin{proof}
%   Let $u$ be the report such that $\theta = \Gamma_u$.
%   If there is a $u'$ so that $\Gamma_u = \Gamma_{u'}$, one of the level sets is in $\trim(\Gamma)$, and then we have $\theta$ equal to that level set.
%   If $\Gamma_u \subsetneq \Gamma_{u'}$, then either $\Gamma_u$ is not a full-dimensional level set or $\Gamma_u \cap \Gamma_{u'}$ is full-dimensonal.
%   If $\Gamma_u$ is not full-dimensional, then we contradict our assumption.
%   If $\Gamma_u \cap \Gamma_{u'}$ is full dimensional, then we contradict our elicitability condition (Lemma~\ref{lem:elicitable-level-sets}).
%   Therefore, full-dimensional level sets of an elicitable property must be in the trim operation of the same property.
% \end{proof}



\section{General characteristics of polyhedra}
\subsection{Definitions and preliminaries}
\begin{definition}[Closed halfspace]
	For any $w\in\reals^d, b \in \reals$, let $H_{(w,b)}^+ := \{ x \in \reals^d \mid \inprod{x}{w} \geq b\}$ be the closed halfspace defined by $(w,b)$.
\end{definition}
\begin{definition}[Hyperplane]
	For any $(w,b)\in\reals^d \times \reals$, let $H_{(w,b)} := \{ x \in \reals^d \mid \inprod{x}{w} = b\}$ be the hyperplane generated by $(w,b)$, and the boundary of $H^+_{(w,b)}$.
\end{definition} 

\begin{definition}[Polyhedron - halfspace representation]
	\btw{Def from Ziegler page 28}
	A \emph{polyhedron} $P$ is an intersection of a finite set of closed halfspaces $\H = \{H^+_{(w_i,b_i)}\}_{i=1}^k$ presented in the form $P = \cap \H$.
\end{definition}

%\raf{Maybe add def supporting hyperplane; cor of face def is that nonempty face iff supporting hyperplane}
\begin{definition}[Supports]
	A hyperplane $H$ \emph{supports} the polyhedron $P$ if 
	(i) $P \subseteq H^+$ or $P \subseteq -H^+$, and
	(ii)$H \cap \mathrm{bd}(P) \neq \emptyset$.
	Moreover, $H$ supports $P$ at $x$ if $x \in H \cap \mathrm{bd}(P)$.
\end{definition}


\begin{definition}[Face, facet]
  Let $P \subseteq \reals^d$ be a convex polytope.
  A halfspace $H^+_{(w,b)}$ is \emph{valid} for $P$ if $P \subseteq H^+_{(w,b)}$.
  % A linear inequality $\inprod{w}{x} \geq b$ is \emph{valid} for $P$ if it is satisfied for all $x \in P$.
  A \emph{face} $F_{(w,b)}$ of the polytope $P$ is any set of the form
  \begin{equation*}
    F_{(w,b)} = P \cap H_{(w,b)}~,~
  \end{equation*}
  for any valid halfspace $H^+_{(w,b)}$.
  The dimension of a face $F$ is the dimension of its affine hull $\dim(F) := \dim(\affhull(F))$.
  A face $F$ with $\dim(F) = \dim(P) - 1$ is called a facet.
\end{definition}

\begin{corollary}
  A face $F_{(w,b)}$ of the polyhedron $P$ such that $F_{(w,b)} = P \cap H_{(w,b)}$ is nonempty if and only if $H_{(w,b)}$ is a supporting hyperplane of $P$.
\end{corollary}


%\raf{Let's use $\H^*$ for the minimum one, here and below; $\H$ for another one}
%\raf{Similarly $\V^*$ for the minimum one, below; $\V$ for another one}
\begin{theorem}\label{thm:polyhedron-uniquely-gen-facets}
  Given a $d$-dimensional polyhedron $P \subseteq \reals^d$, there is a unique finite set of closed halfspaces $\H^*$ such that $P = \cap \H^*$, and for all finite sets of closed halfspaces $\H$ such that $P = \cap \H$, we have $\H^* \subseteq \H$.
  Moreover, $\{H\cap P \mid H^+ \in \H^*\}$ is the set of facets of $P$.
\end{theorem}
\begin{proof}
	Since $P$ is $d$-dimensional in $\reals^d$, it therefore has nonempty interior.	
  \jessiet{\url{https://arxiv.org/abs/0805.0292} Prop 4.5? Ziegler Lectures on Polytopes Theorem 2.15 (7) also kind of close to this, but on polytopes not polyhedron.}
\end{proof}

\subsubsection{Notation}
Within this appendix, we use some self-contained notation.
First, for a fixed $g$ as above, define $G = \{(x,c) \mid c \leq g(x)\} \subseteq \reals^\Y_+ \times \reals$ to be the hypograph of $g$.

We define $H^+_v := H^+_{([v;-1], 0)}$ for a given $v$, where $[v; -1]\in\reals^\Y_+\times\reals$ define a halfspace for some $v \in \V$.
Similarly, we denote $H_y^+ := H_{([e_y; 0], 0)}^+$ for any $y \in \Y$; the latter will help us restrict our hypograph to the nonnegative orthant. 
Similarly, we let $H_v := H_{([v;-1], 0)}$ for $v \in \reals^\Y_+$ and define $H_y := H_{([e_y;-1], 0)}$. 

Finally, we define the face generated by $v \in \V$ with $F_v := H_v \cap G$ for any $v\in\V$.
Finally, we let $\H_{\V} = \{H_v^+ \mid v\in\V\}$, $\H_\Y = \{H_y^+ \mid y\in\Y\}$, $\H = \H_{\V} \cup \H_\Y$.


\subsection{Constructing a unique minimum set of facets from a finitely generated polyhedron}\label{appsubsec:phase1}
We will later consider losses over a finite set of outcomes $\Y$; to make notation consistent, we use $\reals^\Y_+$ to denote $\reals^{|\Y|}_+$ throughout, and let $d := |\Y|+1$.
Suppose we have a finite set $\V \subset \reals_+^\Y \times \reals$.

\begin{definition}
  Define the function $g : \reals_+^\Y \to \reals_+$ by
  \begin{align*}
    g(x) = \min_{v\in\V} \inprod{x}{v} - \delta(x \mid \reals_+^\Y)~,~
  \end{align*}
\end{definition}



First, we observe that the region generated by the intersection of the $\H_y$ halfspaces restricts the hypograph to the nonnegative orthant.
\begin{claim}\label{claim:x-nonneg-orthant-iff-intersection-HY}
  $\cap \H_\Y = \reals^\Y_+ \times \reals$.
\end{claim}
\begin{proof}
  The result follows if we show $x \in \reals^\Y_+ \iff (x,c) \in \cap \H_\Y$ for all $c \in \reals$.

  $\implies$
  Fix any $c \in \reals$.
  $x \in \reals^\Y_+ \iff x_y \geq 0$ for all $y \in \Y$.
  This means that for any $y \in \Y$, $(x,c) \in \{(x,c) \mid x_y \geq 0\} = H_y$.
  As $y$ and $c$ were arbitrary, this shows the forward direction.
  
  $\impliedby$
  $(x,c) \in \cap \H_\Y$ implies $x_y \geq 0$ for all $y \in \Y$, and therefore $x \in \reals^d_+$.	
\end{proof}

Now, we can define $G$ as the intersection of halfspaces generated by $\V$ on the nonnegative orthant.
\begin{claim}\label{claim:G-intersection-H}
  $G = \cap \H$.
\end{claim}
\begin{proof}
  $(x,c) \in G \iff g(x) - c \geq 0 \iff \min_{v \in \V}\inprod{v}{x} - c \geq 0$ and $x \in \reals_+^\Y$, which is true if and only if $\inprod{v}{x} - c \geq 0 \,\forall v \in \V$ and $x_y \geq 0$ for all $y$.
  In turn, this statement holds if and only if $(x,c) \in H^+_v$ for all $v \in \V$ and in $H^+_y$ for all $y \in \Y$ respectively, so $(x,c) \in \cap \H$.
\end{proof}

We proceed with some observations about facets and dimension of the hypograph of $g$.

\begin{claim}\label{claim:G-full-dimensional}
  $G$ is $d$-dimensional.  %\raf{$g$ is nonnegative on $\reals^\Y_+$, so $G$ contains $\{(x,c) \mid x\in\reals^\Y_+, c \leq 0\}$, which is full-dim.}
\end{claim}
\begin{proof}
  Since $g$ is nonnegative on $\reals_+^\Y$, $G$ therefore contains $\{(x,c) \mid x\in\reals^\Y_+, c \leq 0\}$, which is $(|\Y| + 1)$-dimensional.
\end{proof}


\begin{lemma}\label{lem:G-unique-facets-Hstar}
  %\raf{This is the important general polyhedra result to cite/state.  Note: we'll have to work half-spaces of the form $H_{(w,\beta)}^+ = \{ z \in \reals^{n+1} \mid \inprod{z}{w} - \beta \geq 0\}$.}  
  There is some unique $\H^* \subseteq \H$ such that $G = \cap \H^*$. 
  Moreover, for each $H_{(w_i, b_i)} \in\H^*$, the face $F_{(w_i, b_i)} = H_{(w_i, b_i)} \cap G$ is a facet.
\end{lemma}
\begin{proof}
  Since $G$ is full-dimensional, this follows immediately from Theorem~\ref{thm:polyhedron-uniquely-gen-facets}.
\end{proof}

\begin{lemma}\label{lem:HY-subset-Hstar}
  $\H_\Y \subseteq \H^*$.
\end{lemma}
\begin{proof}
  If there was a $y \in \Y$ such that $H_y = \{(x,c) \mid x_y \geq 0\}$ was not in $\H^*$, then we would either have some $c_1 > 0$ such that $\{(x,c) \mid x_y \geq c_1\} \in \H^*$, or we have a point $x$ such that $x_y < 0$ but $g(x) > -\infty$.
  The first cannot happen as we take $g$ to be defined (and finite) on $x = e_y \in \reals^\Y_+$ and is concave.
  Moreover, the second cannot be true by construction of $g$ including the $0-\infty$ indicator on $\reals^\Y_+$.	
\end{proof}

\begin{claim}\label{claim:unique-set-loss-vectors-defining-facets}
  There is a unique finite set $\V^* \subseteq \reals^\Y_+$ such that $\H^* = \H_\Y \cup \H_{\V^*}$.
  Moreover, $F_v$ is a facet of $G$ for each $v\in\V^*$.
\end{claim}
\begin{proof}
%  \raf{Argue that all facets of $G$ are an $H_y \cap G$ or some $H_v \cap G$.  Thus, we can define $\V^*$ by $\H_\Y \cup \H_{\V^*} = \H^*$.  (I.e., $\H^* \setminus \H_\Y = \H_{\V^*}$ for some $\V^*$.)}
%
  Since $G$ is full-dimensional, all facets of $G$ are uniquely determined by the hyperplanes $H$ whose halfspaces $H^+$ compose $\H^*$ by Lemma~\ref{lem:G-unique-facets-Hstar}.
  Any facet $F$ must then be some intersection of an $H_y \cap G$ or $H_v \cap G$.

  The moreover follows immediately since $\H_{\V^*} \subseteq \H^*$ and every $H^+ \in \H^*$ defines a facet by Theorem~\ref{thm:polyhedron-uniquely-gen-facets}.
\end{proof}



\begin{corollary}\label{cor:anything-gen-G-subset-Hstar}
  Let $\V \subseteq \reals^\Y_+$ be a finite set such that $\H = \H_\Y \cup \H_\V$ satisfies $G = \cap \H$.  Then $\V^* \subseteq \V$.
\end{corollary}
\begin{proof}
  $\H_\Y \cup \H_{\V^*} = \H^* \subseteq \H = \H_\Y \cup \H_{\V}$, ergo $\V^* \subseteq \V$.
\end{proof}

We now show that we can equivalently construct $g$ through the unique finite set $\V^*$ instead of a generic set of vectors $\V$.
\begin{claim}\label{claim:g-gen-by-Vstar}
  $g(x) = \min_{v \in \V^*}\inprod{v}{x} - \delta(x \mid \reals_+^\Y)$
\end{claim}
\begin{proof}
  $G = \cap \H = \cap \H^* = \cap (\H_\Y \cup \H_{\V^*}) = \{(x,c) \in \reals_+^\Y \times \reals \mid \inprod{v^*}{x} \geq c$ for all $v^* \in \V^* \}$ where the first equality follows as $\H^* \subseteq \H$.
  Since $G$ is the hypograph of $g$, this means $g$ can be written as 
  $g(x) = \min_{v \in \V^*}\inprod{v}{x} - \delta(x \mid \reals^\Y_+)$. 
\end{proof}

\subsection{Infinitely generated polyhedron}\label{appsubsec:phase2}

Suppose $L$ is a minimizable loss function.
For $x\in\reals^\Y_+$, consider $\risk{L}_+(x) := \inf_{r\in\R} \inprod{x}{L(r)}$ polyhedral.
Throughout, suppose $g(x) = \risk{L}_+(x)$.
As before, take $G$ to be the hypograph of $g$.
We define $\V = L(\R) \subseteq \reals^\Y_+$ and $\H_{\V} = \{H_v^+ \mid v\in\V\}$.  Unlike above, these could be infinite sets.  
Now let $\H = \H_\Y \cup \H_{\V}$; again, this may be infinitely generated.
\begin{claim}
	$G = \cap\H$.
\end{claim}
\begin{proof}
	Observe that $x \in \reals^\Y_+ \iff (x, c) \in \cap \H_\Y$.
	Let $x \in \reals^\Y_+$.
	\begin{align*}
	(x,c) \in G
	&\iff g(x) \geq c & \text{ since $G$ is the hypograph of $g$}\\
	&\iff \risk L_+(x) \geq c & \text{ as $g=\risk L_+$}\\
	&\iff \inprod{v}{x} \geq c \,\, \forall v \in \V & \text{By def of $\risk L_+$ as the infimum over $v \in \V$ of}\\
	& & \text{the inner product with $x$ and minimizable.} \\
	&\iff (x,c) \in H^+_{v} \,\, \forall v \in L(\R) & \text{By definition of each halfspace}\\
	&\iff (x,c) \in \cap \H_\V & \text{Since true for all $v \in \V$} \\ 
	\end{align*}
	Combining the two equalities (e.g., $\cap \H = (\cap \H_\Y) \cap (\cap \H_\V)$), we have $G = \cap \H$.
	
\end{proof}

%\raf{Need to use minimizable}
Take $\V^*$ to be the unique finite set such that $\H^* = \H_\Y \cup \H_{\V^*}$, where $\H^*$ is the unique minimum halfspace representation for $G$ as in \S~\ref{appsubsec:phase1}.

\begin{claim}
	$\H_{\V^*} \subseteq \H$
\end{claim}
\begin{proof}
	Observe that it is not immediately clear that $\H_{\V^*} \subseteq \H$, since $\H$ is not finite, and therefore does not immediately form a halfspace representation for $G$.
	If there was a $v^* \in \V^*$ such that $H^+_{v^*} \not \in \H$, then as $\V^*$ generates facets, the projection $\pi(F_{v^*})$ has nonempty interior.
	Fix  $x \in \inter(\pi(F_{v^*}))$; we know $(x,g(x))$ supports $H_v$ at $x$ by Lemma~\ref{lem:projected-faces-iff-support-iff-argmin}.
	Since $L$ is minimizable, we have $\risk L_+ = \min_{v \in \V}\inprod{v}{x} = g(x)$.
	However, since $F_{v^*}$ is a facet \jessie{ref}, it is unique, so $\argmin_{v \in \V^*}\inprod{v}{x} = \{v^*\}$ and there can be no $v' \in \V^*$ such that $F_{v'}$ supports $G$ at $x$.
	That is to say that $v^* \not \in \V$, we then contradict the fact that $L$ is minimizable.
	Thus, we must have $\H_{v^*} \subseteq \H$.
\end{proof}

\begin{corollary}\label{cor:finite-rep-set}
	There is a (finite) set $\R^* \subseteq \R$ such that $L(\R^*) = \V^*$ (without duplicates).
\end{corollary}



\subsection{Projecting from $G$ to $g$}
We now consider projections from $\reals^\Y_+ \times \reals$ onto the positive orthant $\reals^\Y_+$.

\begin{claim}\label{claim:vstar-supporting-G}
  For all $x\in\reals^\Y_+$, there exists $v^*\in\V^*$ such that $H_{v^*}$ supports $G$ at $(x,g(x))$.
\end{claim}
\begin{proof}
% \jessie{Notes in meeting:
%   For any $x \in \reals^d_+$, consider $v^* \in \argmin_{v \in \V} \inprod{v}{x}$.
%   Argue $G \subseteq H^+_{v^*}$ by definition.
%   Argue support: $g(x) = \inprod{v^*}{x}$.  WTS $(x,g(x)) \in G$.  Already have $(x,g(x)) \in H_{v^*}$.}
%
  By Claim~\ref{claim:g-gen-by-Vstar}, we know that $g(x) = \min_{v \in \V^*}\inprod{v}{x}$.
  In particular, consider the normal $v^* \in \argmin_{v \in \V^*}\inprod{v}{x}$; we claim that $H_{v^*}$ supports $G$ at $(x,g(x))$.
  First, $G \subseteq H^+_{v^*}$ by definition of $G$ as the intersection of halfspaces including $H^+_{v^*}$.
  Thus, it us just left to show that $(x, \inprod{v^*}{x}) \in H_{v^*} \cap G$.
  $H_{v^*} = \{(x,c) \mid \inprod{v^*}{x} = c\}$.
  By definition of $g$, we have $(x,g(x)) = \inprod{v^*}{x}$, so $(x,g(x)) \in H_{v^*}$.
  Moreover, $(x,g(x)) \in G = \{(x,c) \mid g(x) \geq c\}$ trivially since $g(x) \geq g(x)$.
\end{proof}

Define the projection $\pi:\reals^\Y\times \reals \to \reals^\Y, (x,c) \mapsto x$.

\begin{corollary}\label{cor:projected-Vstar-faces-cover-pos-orthant}
  $\cup_{v\in\V^*} \pi(F_v) = \reals^\Y_+$.
\end{corollary}

\begin{claim}\label{claim:pi-preserves-dim}
  For all $v\in\reals^\Y_+$, $\dim(\pi(F_v)) = \dim(F_v)$.
\end{claim}
\begin{proof}
  \jessiet{Can use a careful check}
  \btw{See \url{http://homepages.cae.wisc.edu/~linderot/classes/ie418/lecture11.pdf} for def of AI}
  %\jessie{just need to argue that you never needed $c$ (i.e., $c$ was always just $g(x)$).  Maybe: take an affine basis, project, and show it's an affine basis.}
  As a reminder, we define the dimension of a polytope to be the dimension of its affine hull.
  Suppose we are given $|\Y|+1$ affinely independent vectors $z_i$ in $F_v$. 
  We claim their projections $\{\pi(z_i)\}$ are affinely independent.
  Let $a_1 + \ldots + a_{\Y+1} = 0$, such that $\sum_i a_i \pi(z_i) = 0$.
  We want to conclude that we must have $a_i = 0$ for all $i$, meaning they are affinely independent.
  
  Observe $z_i = (x_i, \inprod{v}{x_i})$ for all $i$; therefore, if $z_i \in F_v$ (e.g., $F_v$ supports $G$ at $x, \inprod{v}{x})$), then we also have $z_i \in H_v$.
  So $0 = \sum_i a_i \pi(z_i) = \sum_i a_i x_i$.
  Moreover, the sum $\sum_i a_i z_i = \sum_i a_i (x_i, \inprod{v}{x_i}) = (\sum_i a_i x_i, \inprod{v}{\sum_i a_i x_i}) = (\vec 0,0) = \vec 0$.
  Thus, since $a_i = 0$ for all $i$, the set $\{z_i\}$ is affinely independent and the dimensions of the affine hulls are therefore equal.
\end{proof}


%\begin{itemize}
%\item REP $\iff$ $\V^* \subseteq \V' := L(\R')$\jessie{Claim~\ref{claim:Vprime-projected-faces-cover-iff-Vstar-subset-Vprime}}
%\item REP $\iff \jessie{\Lambda := \{\pi(F_v) \mid v \in \V^*\}; \Lambda \subseteq \{ \pi(F_{v}) \mid v \in L(\R')\}}$.
%  \jessie{Claim~\ref{claim:Vprime-projected-faces-cover-iffprojected-faces-subsets}}
%\item $\Lambda$ == Full-dim \jessie{Claim~\ref{claim:projected-Vstar-faces-full-dim}.}
%\item Lev sets subset $\theta$ \jessie{Just work with $\pi(F_{v})$ where $v \in L(\R)$}
%\end{itemize}

Since we preserve the dimension of these projected spaces, we can now study equivalence of projected faces of the hypograph and regions of support of $g$ for any $v \in \V$.

\begin{lemma}\label{lem:projected-faces-iff-support-iff-argmin}
  Fix any $x \in \reals^\Y_+$.
  For any $v \in \V$, that the following are equivalent: 
  

  (1) $(x,g(x)) \in F_v$

  (2) $\inprod{v}{x}= g(x)$

  (3) $v \in \argmin_{v' \in \V} \inprod{v'}{x}$
  
  (4) $x \in \pi(F_v)$
\end{lemma}
\begin{proof}
  \begin{align*}
    (x,g(x)) \in F_v
    &\iff (x,g(x)) \in \{(x',c) \in G \mid \inprod{v}{x'} = c\}\\
    &\iff  \inprod{v}{x} = g(x)\\
    &\iff \inprod{v}{x} = \min_{v' \in \V}\inprod{v'}{x}\\
    &\iff v \in \argmin_{v' \in \V}\inprod{v'}{x}
  \end{align*}
  This covers $1 \iff 2 \iff 3$.

  For $1 \iff 4$, the forward implication follows trivially by applying the definition of the projection $\pi$.
  For the reverse implication, consider some $x \in \pi(F_v)$.
  There must be a $c \in \reals$ so that $(x,c) \in F_v$.
  Expanding, this is actually saying $(x,c) \in \{(x',c') \in G \mid \inprod{v}{x'} = c\}$.
  In particular, this is true when $c = \inprod{v}{x}$, which defines a face of $G$ at $x$ if any only if $\inprod{v}{x} = g(x)$.
  Therefore, we have $(x, g(x)) \in F_v$.
\end{proof}


%\raf{For Lemma X: NTS $\cup_{v\in\V} \pi(F_v) = \reals^\Y_+ \implies \V^* \subseteq \V$.}
%\raf{If $G = \cap( \H_\Y \cup \H_\V)$...}
\begin{claim}\label{claim:Vprime-projected-faces-cover-iff-Vstar-subset-Vprime}
  For $\V' \subseteq \V$, we have
  $\cup_{v\in\V'} \pi(F_v) = \reals^\Y_+ \iff \V^* \subseteq \V'$.\btw{Will get us to REP $\iff$ $\V^* \subseteq L(R')$ (3).}
\end{claim}
\begin{proof}
  Would like to show
  $\cup_{v\in\V'} \pi(F_v) = \reals^\Y_+ \iff \cap(\H_\Y \cup \H_{\V'}) = G$ then apply Corollary~\ref{cor:anything-gen-G-subset-Hstar}.

  Suppose $\cup_{v\in\V'} \pi(F_v) = \reals^\Y_+$.
  Take any $x\in\reals^\Y_+$.
  %\raf{Argue that we have $\cap(\H_\Y \cup \H_{\V'}) \subseteq G$}
  %\raf{Another approach: assume $v^* \in \V^* \setminus \V'$.  Take $z = (x,g(x)) \in \relint(F_{v^*})$.  Argue that $x \not\in \cup_{v\in\V'} \pi(F_v)$.}
  
  $\implies$ 
  Fix any $x \in \reals^\Y_+$.
  Since $\V' \subseteq \V$, and $G = \cap (\H_\Y \cup \H_\V)$, we immediately have $G \subseteq (\H_\Y \cup \H_{\V'})$.
  Therefore, we just need to show the other direction of inclusion.
%  \raf{Need to add logic: know $G \subseteq \cap(\H_\Y \cup \cap \H_{\V'})$, so only need to show $\cap(\H_\Y \cup \cap \H_{\V'}) \subseteq G$}
  $(x,c) \in \cap \H_\Y$ for all $c \in \reals$ by Claim~\ref{claim:x-nonneg-orthant-iff-intersection-HY}, so it is left to show that $(x,c) \in \cap \H_{\V'}$, yielding the desired result.
  First, $(x,c) \in \cap \H_{\V'} \implies (x,c) \in H^+_{v'}$ for all $v' \in \V'$.
  This implies $c \leq \inprod{v'}{x}$ for all $v' \in \V'$.
  By Lemma~\ref{lem:projected-faces-iff-support-iff-argmin}, there exists a $v' \in \V'$ such that $(x,g(x)) \in F_{v'}$, and $\inprod{v'}{x} = g(x)$.
  Therefore, we have $c \leq \inprod{v'}{x} = g(x)$, and $(x,c) \in G$ follows since $G$ is the hypograph of $G$.
  
  %\raf{Want to show $c \leq g(x)$, then done by def $G$}
  %\raf{Now $(x,c) \in \cap \H_{\V'} \implies (x,c) \in H^+_{v'} \implies c \leq \inprod{v'}{x} = g(x) \implies (x,c) \in G$}

  % ...$\leq \inprod{v}{x}$ for all $v \in \V$, so $(x,g(x)) \in \cap \H_\V \implies (x,g(x)) \in G = \cap (\H_\Y \cup \H_\V)$.
  
  $\impliedby$
  $G = \cap (\H_\Y \cup \H_{\V'})$ if and only if $G = \cap \H_{\V'}$ for allwhen restricting to $x \in \reals_+^\Y$.
  This implies that for all $x \in \reals_+^\Y$, there exists a $v' \in \V'$ such that $(x,g(x)) \in H_{v'} \cap G = F_{v'} \implies x \in \pi(F_{v'})$.
  As this is true for all $x \in \reals_+^\Y$, we have $\cup_{v \in \V'} \pi(F_{v}) = \reals_+^\Y$.

\end{proof}


\begin{claim}\label{claim:projected-Vstar-faces-full-dim}
  For all $v \in \V^*$, $\pi(F_v)$ is full dimensional in $\reals_+^\Y$.
  \btw{$\Lambda$ is exactly the set of FDLS; Will lead to $\Theta$ are exactly the full dimensional level sets over simplex (6).}
\end{claim}
\begin{proof}
  By Claim~\ref{claim:unique-set-loss-vectors-defining-facets}, $F_v$ is a facet of $G$ in $\reals^d_+$, meaning it is $(d - 1)$-dimensional.
  Moreover, Claim~\ref{claim:pi-preserves-dim} states that the dimension of $F_v$ is preserved for each $v \in \V^*$.
  Thus, $\pi(F_v)$ is $|\Y|$-dimensional.
\end{proof}




Denote $\Lambda := \{\pi(F_v) \mid v \in \V^*\}$
\begin{claim}\label{claim:Vprime-projected-faces-cover-iffprojected-faces-subsets}
  For $\V' \subseteq \V$, we have $\cup_{v \in \V'} \pi(F_v) = \reals_+^\Y \iff \Lambda \subseteq \{ \pi(F_v) \mid v \in L(\R') \}$.
  \btw{$1$-homogeneous version of rep iff $\Theta$ subset of level sets generated by $\R'$.}
\end{claim}
\begin{proof}
  $\implies$
  The result follows if $\V^* \subseteq \V'$, which is exactly the forward implication of Claim~\ref{claim:Vprime-projected-faces-cover-iff-Vstar-subset-Vprime}.
  Explicitly, for all $v \in \V^*$ we also have $v \in \V'$, so, $\pi(F_v) \in \Lambda \cap \{\pi(F_v) \mid v \in \V'\} = \Lambda$.
  
  $\impliedby$
  If $\Lambda \subseteq \{ \pi(F_v) \mid v \in L(\R') \}$, then $\cup \Lambda \subseteq \cup_{v \in \V'} \pi(F_v)$.
  By Corollary~\ref{cor:projected-Vstar-faces-cover-pos-orthant}, we have $\cup \Lambda = \reals_+^\Y$, so $\reals_+^\Y \subseteq \cup_{v \in \V'} \pi(F_v)$.
  The other direction of subset inequality following from $G$ being finite only on $\reals_+^\Y$.	
\end{proof}

Let $r$ be a report, and let $v = L(r)$.
As $H_{v}^+ \in \H$, it supports $G$ or it is redundant in $\H$ (or both).

\begin{claim}\label{claim:faces-contained-in-facets}
  If a $H_v \in \H$ does support $G$, then $F_{v} = H_{v} \cap G$ is a nonempty face of $G$, and thus a subset of a facet. 
\end{claim}
\begin{proof}
  Since $H_v$ supports $G$, we know that $G \subseteq H_v^+$ and $F_v$ is not empty.
  Since $H^+_v$ is valid, we have $F_v = F_{([v;-1],0)}$ is a face by definition.
  
  % subset of facet
  Moreover, the faces of $G$ are all convex polyhedra.
  Any face of $G$ is must then be a lower-dimensional face of a facet, and therefore a subset.
  \btw{Gr\"unbaum Page 35, Section 3.1 statement 7}
\end{proof}

\begin{claim}\label{claim:vface-subset-some-vstar-face}
  For any $v \in \V$, the face $F_{v}$ must be a subset of one of the $\V^*$ facets $F_{v^*}$ (as opposed to the vertical ones)
\end{claim}
\begin{proof}
	\jessiet{Can use a careful read}
  $\{F_{v^*}\}_{v^* \in \V^*}$ is exactly the set of facets of $G$ on $\reals_+^d$ by Theorem~\ref{thm:polyhedron-uniquely-gen-facets}.
  Any $k$-face is contained in a facet by Claim~\ref{claim:faces-contained-in-facets}.
  Since $\H = \cap (\H_{\V^*} \cup \H_\Y)$, we are done if $F_v$ is contained in a face of $\H_{\V^*}$ and need to consider the case where $F_v \in \H_\Y$.
  
  If $F_v$ is a subset of a vertical facet, we would have $F_v = H_v \cap G \subseteq H_y \cap G$ for some $y \in \Y$.
  Expanding, this is $\{(x,g(x)) \mid \inprod{v}{x} = g(x)\} \subseteq \{(x,g(x)) \mid x_y = 0 \}$.
  While this can happen, it must be on the boundary of $\reals_+^d$, and must be at the intersection of $H_y$ and some other $H_{v^*}$ for $v^* \in \V^*$, since $g$ is continuous on $\reals^\Y_{+}$, and determined by $\V^*$ on $\reals^\Y_{++}$.
  
  
  
  \jessie{Justification: $g$ is continuous on $\reals_+^\Y$, and there has to be a unique $v^*$ on $\reals^\Y_{++}$.}
  
\end{proof}

\begin{corollary}\label{cor:projected-faces-subset-projected-facet}
  For $v,v^*$ such that $F_v \subseteq F_{v^*}$ and $v^*\in \V^*$, $\pi(F_{v}) \subseteq \pi(F_{v^*})$. 
\end{corollary}

\begin{claim}\label{claim:exists-vstar-projected-face-subset}
  For any $v \in \V$, there is a $v^* \in \V^*$ such that $\pi(F_v) \subseteq \pi(F_{v^*})$.
  \btw{shows (7) on $\reals^\Y_+$}
\end{claim}
\begin{proof}
  This is exactly Claim~\ref{claim:vface-subset-some-vstar-face} and Corollary~\ref{cor:projected-faces-subset-projected-facet} chained together.
\end{proof}


% \begin{claim}
%   \jessie{This should probably move down to translating for properties since representative sets are at the core, about properties}
%   $\R'$ is a representative set for $L$ if and only if $\V^* \subseteq L(\R')$.
% \end{claim}
% \begin{proof}
%   Fix $v^* \in \V^*$.
%   If $\R'$ is representative, then for any $p \in \simplex$ such that , 
%   As $g$ is $1$-homogeneous and minimizable, for every $v^* \in \V^*$, there must be some $p \in \simplex$ such that $\inprod{v^*}{p} = \inf_v \inprod{v}{x}$.
%   Thus, it suffices to show subset inclusion for any $p \in \simplex$.
%   Moreover, since $\R'$ is representative, there is some $r' \in \R'$ such that $\inprod{L(r')}{p} = \inf_{v} \inprod{v}{x}$.
%   
%   For the reverse implication, $\V^*$ is representative (since $\cup_{v^* \in \V^*} F_{v^*} = \reals_+^\Y$ in Corollary 8\jessie{make reference}), so any set containing $\V^*$ is also representative.
% \end{proof}


% \begin{corollary}
%   $\R^*$ is a min rep set for $L$ if and only if $\V^* = L(\R^*)$. \raf{(2)}
% \end{corollary}


\subsection{Translating to properties: projecting the domain of $g$ to $\simplex$}

Let $f:\reals^\Y\to\reals_+\cup\{-\infty\}$ be a polyhedral concave function with $\dom(f) = \simplex$.
\begin{claim}
	We may write $f(p) = \min_{v\in\V} \inprod{p}{v} + \delta(p \mid \simplex)$ for some finite set $\V \subset \reals^\Y_+$. %\raf{show you can do this using the $h$ and $\delta$ representation from Rockafellar -- argue that because $p\in\simplex$ we can write $\inprod{p}{b_i}-\beta_i = \inprod{p}{b_i-\beta_i\ones}$.}
\end{claim}
\begin{proof}
  We will think of $f$ as defined $f:\reals^\Y\to\reals_+\cup\{-\infty\}$  with $\dom(f) = \simplex$.
  For $p \in \simplex$, we know $\sum_i p_i = 1$, and can write any inner product $\inprod{p}{b} - \beta = \inprod{p}{b} - \inprod{p}{\beta \ones} = \inprod{p}{b - \beta \ones}$.
  If $p \not \in \simplex$, then $f(p) = -\infty$ and inner products are not used to compute $F$.
  Moreover, since $f$ is polyhedral, it is finitely generated \citep[Proposition 19.1.2]{rockafellar1997convex} and can be written 
	\begin{align*}
	f(p) &= h(p) - \delta(p \mid C)\\
		 &= \min(\inprod{p}{b_1} - \beta_1, \ldots, \inprod{p}{b_k} - \beta_k) - \delta(p \mid \simplex)\\
		 &= \min(\inprod{p}{b_1 - \beta_1 \ones}, \ldots, \inprod{p}{b_k - \beta_k \ones}) - \delta(p \mid \simplex)~.~
	\end{align*}
\end{proof}

\begin{corollary}\label{cor:f-matches-g-on-simplex}
  For all concave $f : \reals^\Y_+ \to \reals_+ \cup \{-\infty\}$ with $\dom(f) = \simplex$, there is a polyhedral concave function $g : \reals^\Y_+ \to \reals_+$ with $\dom(g) = \reals^\Y_+$ so that $f(p) = g(p)$ for all $p \in \simplex$.
\end{corollary}
Given the function $f$, we consider $g$ to be its extension and $L$ such that $\risk L_+ = g$ so that we may use the tools in \S~\ref{appsubsec:phase1} and~\ref{appsubsec:phase2} to draw conclusions about the property $\Gamma := \prop{L}$.

Define the function $\theta(v) = \{p \in \simplex \mid \inprod{v}{p} = f(p)\}$ as the level sets of the loss vector $v \in \V$, and the set $\Theta := \{\theta(v) \mid v \in \V^*\}$ to be the set of (minimal) level sets uniquely defined by the loss vectors.

\begin{claim}
  For all $v \in \V$, $\theta(v) = \pi(F_v) \cap \simplex$.
\end{claim}
\begin{proof}
  % Argue $\theta(v)$ is full-dim in $\simplex$ if and only if $\pi(F_v)$ is $|\Y|$-dim, if and only if $F_v$ is a facet. \raf{(2)}
  \btw{Moved previous proof outline to comment}

  Fix $p \in \simplex$.
  \begin{align*}
    p \in \theta(v)
    &\iff \inprod{v}{p} = f(p) & \text{Definition of $\theta$}\\
    &\iff \inprod{v}{p} = \min_{v' \in \V}\inprod{v'}{p} & \text{$f=g$ on $\simplex$ (Cor.~\ref{cor:f-matches-g-on-simplex})}\\
    &\iff v \in \argmin_{v' \in \V}\inprod{v'}{p} &\\
    &\iff p \in \pi(F_v) & \text{Lemma~\ref{lem:projected-faces-iff-support-iff-argmin}}
  \end{align*}
\end{proof}


\begin{claim}\label{claim:level-set-is-projected-face}
  For all $r \in \R$ with $v = L(r)$, $\Gamma_r = \theta(v) = \pi(F_{v}) \cap \simplex$.
\end{claim}
\begin{proof}
  % Follows immediately from Lemma~\ref{lem:projected-faces-iff-support-iff-argmin}.
  Let us rewrite
  \begin{align*}
    \Gamma_r
    &= \{p \in \simplex \mid r \in \argmin_{r' \in \R} \inprod{L(r')}{p}\}\\
    &= \{p \in \simplex \mid v \in \argmin_{v' \in \V} \inprod{v'}{p}\}\\
    &= \{p \in \simplex \mid \inprod{v}{p} = \min_{v' \in \V}\inprod{v'}{p}\}\\
    &= \{p \in \simplex \mid \inprod{v}{p} = f(p) \}\\
    &= \theta(v)
  \end{align*}
  Now, we have $p \in \theta(v)$ if and only if,
  \begin{align*}
    p \in \theta(v)
    &\iff p \in \{p' \in \simplex \mid \inprod{v}{p'} = f(p')\} & \text{Definition of $\theta$}\\
    &\iff p \in \{p' \in \simplex \mid \inprod{v}{p'} = g(p')\} &\text {$f = g$ on $\simplex$ (Cor.~\ref{cor:f-matches-g-on-simplex})} \\
    &\iff p \in \{p' \in \reals_+^\Y \mid \inprod{v}{p'} = g(p')\} \cap \simplex & \\
    &\iff p \in \pi(F_v) \cap \simplex & \text{Definition of $\pi$, $F_v$}
  \end{align*}
  The result follows.
   
\end{proof}


\subsection{Proving Lemma~\ref{lem:X}}
\begin{lemma}\label{lem:g-1-homog}
	For the finite set $\V^* \subseteq \reals^\Y_+$, the function defined by $g(x) = \min_{v \in \V^*}\inprod{v}{x}$ is $1$-homogeneous.
\end{lemma}
\begin{proof}
	If $x \not \in \reals^\Y_+$, then $c g(x) = -\infty = g(cx)$ for any $c > 0$.
	If $x \in \reals^\Y_+$, then we have $g(cx) = \min_{v \in \V^*}\inprod{v}{cx} = c \min_{v \in \V^*}\inprod{v}{x} = c g(x)$ for any $c > 0$ by linearity of the inner product.
\end{proof}

We now define the \emph{extended level set} $\bar \Gamma_r := \{x \in \reals^\Y_+ \mid \inprod{L(r)}{x} = \risk L_+(x)\}$.

\begin{lemma}\label{lem:levelset-to-extended-levelset}
	For any $r \in \R$ and $c > 0$, if $p \in \Gamma_r$, then $cp \in \bar \Gamma_r$. 
\end{lemma}
\begin{proof}
	Fix $r \in \R$ and $c > 0$.
	We have
	\begin{align*}
	p \in \Gamma_r
	&= \{p' \in \simplex \mid r \in \argmin_{r' \in \R} \inprod{L(r')}{p'} \} & \text {Definition of level set} \\
	&= \{p' \in \simplex \mid v \in \argmin_{v' \in \V} \inprod{v'}{p'} \} & \text {$\V = L(\R)$} \\
	&= \{p' \in \simplex \mid \inprod{v}{p'} = \min_{v' \in \V} \inprod{v'}{p'} \} & \text {$L$ minnable} \\
	&= \{p' \in \simplex \mid \inprod{v}{p'} = g(p') \} & \text {Definition of $g$} \\
	&= \{p' \in \simplex \mid c \inprod{v}{p'} = c g(p') \} &  \\
	&= \{p' \in \simplex \mid  \inprod{v}{cp'} = g(cp') \} & \text {Lemma~\ref{lem:g-1-homog}} \\
	\implies cp
	&\in \{x \in \reals^\Y_+ \mid \inprod{v}{x} = g(x)\}\\
	&= \bar \Gamma_r
	\end{align*}
\end{proof}

\begin{lemma}\label{lem:extended-levelset-equals-projected-face}
	For any $r\in \R$ with $v = L(r)$, $\bar \Gamma_r = \pi(F_v)$.  
\end{lemma}
\begin{proof}
	\begin{align*}
	\bar \Gamma_r
	&= \{x \in \reals^\Y_+ \mid \inprod{L(r)}{x} = \risk L_+(x)\} & \text{Definition of $\bar \Gamma_r$}\\
	&= \{x \in \reals^\Y_+ \mid \inprod{L(r)}{x} = g(x)\} & \text{Assumption that $\risk L_+(x) = g(x)$}\\
	&= \{x \in \reals^\Y_+ \mid \inprod{v}{x} = g(x)\} & \text{$v = L(r)$}\\
	&= \pi(F_v) & \text{Since $F_v = \{(x,g(x)) \mid \inprod{v}{x} = g(x)\}$}\\
	\end{align*}
\end{proof}


\begin{claim}\label{claim:projected-faces-cover-RY-iff-representative}
	A set $\R' \subseteq \R$ with $\V' := L(\R')$ is representative for $L$ if and only if $\cup_{v \in \V'} \pi(F_v) = \reals^\Y_+$.  
\end{claim}
\begin{proof}
	%\raf{Need 1-homog here -- suggest separating out statement like cone(simplex cap pi(Fv)) = pi(Fv)}
	$\implies$
	This proof follows from three lemmas: first, we observe that $g$ is $1$-homogeneous (Lemma~\ref{lem:g-1-homog}).
	Then we extend the notion of a level set $\Gamma_r$ to the nonnegative orthant $\bar \Gamma_r$, and show that any scalar transformation of a distribution in the level set is contained in the same (extended) level set via Lemma~\ref{lem:levelset-to-extended-levelset}.
	Finally, we show the extended level set is exactly the projection $\pi(F_v)$ (Lemma~\ref{lem:extended-levelset-equals-projected-face}).
	As a corollary, we chain the results to observe $\cup_{r \in \R'} \Gamma_r = \simplex \implies \cup_{r \in \R'} \bar \Gamma_r = \reals_+^\Y = \cup_{v \in L(\R')}\pi(F_v) = \reals^\Y_+$, yielding the forward implication.
	
	
	$\impliedby$
	Fix $p \in \simplex \subseteq \reals^\Y_+$.
	By the assumption, there is a $v \in \V'$ such that $p \in \pi(F_v)$.
	By Claim~\ref{claim:level-set-is-projected-face}, we have $p \in \pi(F_v) \cap \simplex = \Gamma_r$ for the $r \in \R'$ such that $v = L(r)$.
	As this is true for all $p \in \simplex$, we have $\R'$ representative.
	
	
\end{proof}

\begin{lemma}\label{lem:lemX1-rep-iff-subset-vectors}
	\btw{(1)}
	A finite set $\R' \subseteq \R$ with $\V' = L(\R')$ is representative if and only if $\V^* \subseteq \V'$.
\end{lemma}
\begin{proof}
	Chain Claim~\ref{claim:projected-faces-cover-RY-iff-representative} and Claim~\ref{claim:Vprime-projected-faces-cover-iff-Vstar-subset-Vprime} to yield the result.
\end{proof}

\begin{lemma}\label{lem:lemX-3-rep-iff-FDLS-subsets}
	\btw{(3)}
	A finite set $\R' \subseteq \R$ with $\V' = L(\R')$ is representative if and only if $\Theta \subseteq \{\theta(v) \mid v \in \V'\}$.
\end{lemma}
\begin{proof}
	Chain Claim~\ref{claim:projected-faces-cover-RY-iff-representative} and Claim~\ref{claim:Vprime-projected-faces-cover-iffprojected-faces-subsets} to yield the result.
\end{proof}

Recall $\Theta := \{\theta(v) \mid v \in \V^*\}$; it follows that this set is exactly the set of level sets of the propertly elicited by $L$
\begin{corollary}
	$\Theta = \{\Gamma_r \mid r \in \R^*\}$ 
\end{corollary}

\begin{lemma}\label{lem:fdls-exactly-theta}
	\btw{(6)}
	$\Theta = \{\Gamma_r \mid r\in\R, \dim(\Gamma_r) = |\Y|-1\}$.
	\raft{The set of full-dimensional level sets of $\Gamma$ is exactly $\Theta$.}
\end{lemma}
\begin{proof}
	From Claim~\ref{claim:projected-Vstar-faces-full-dim}, we know $\Lambda$ is exactly the set of full-dimensional level sets in $\reals^\Y_+$.
	Each element of $\Lambda$ is $\pi(F_v)$ for some $v \in \V^*$.
	Take $r \in \R^*$ so that $v = L(r)$.
	By Claim~\ref{claim:level-set-is-projected-face}, we have $\theta(v) = \Gamma_r = \pi(F_v) \cap \simplex$ is full-dimensional relative to the simplex.
	The result follows.
\end{proof}


\begin{lemma}\label{lem:any-levelset-contained-in-minlevelset}
  For any $r \in \R$, there exists a $v^* \in \V^*$ such that $\Gamma_r \subseteq \theta(v^*)$.\btw{(7)}
\end{lemma}
\begin{proof}
  Take $v = L(r)$.
  It suffices to show there is a $v^* \in \V$ such that $\theta(v) \subseteq \theta(v^*)$ by their equality in Claim~\ref{claim:level-set-is-projected-face}.

  By Claim~\ref{claim:exists-vstar-projected-face-subset}, there is a $v^* \in \V^* \subseteq \V$ such that $\pi(F_v) \subseteq \pi(F_{v^*})$.
  Therefore, $\pi(F_v) \cap \simplex \subseteq \pi(F_{v^*})\cap \simplex$.  
  We know $\theta(v) = \pi(F_v) \cap \simplex$ and similarly for $\theta(v^*)$ by Claim~\ref{claim:level-set-is-projected-face}.
  The result follows.
  % Let $v = L(r) \in \V$.
  % By Claim \ref{claim:vface-subset-some-vstar-face}, the face $F_v$ must be a subset of some $F_{v^*}$ for $v^* \in \V^*$.
  % We must then have $\pi(F_v) \cap \simplex = \Gamma_r \subseteq \pi(F_{v^*}) \cap \simplex = \theta(v^*)$.
\end{proof}

Now this brings us to Lemma~\ref{lem:X}.
The framework in this appendix cues up this proof: any loss $L$ satisfying the assumptions of Lemma~\ref{lem:X} has some $g = \risk{L}_+$ as in this section that we can work with. 
For such a $g$, Lemma~\ref{lem:lemX1-rep-iff-subset-vectors} is exactly statement \eqref{item:X-rep-V}.
This immediately implies statement~\eqref{item:X-min-V}.
Moreover, Lemma~\ref{lem:lemX-3-rep-iff-FDLS-subsets} is exactly statement~\eqref{item:X-rep-Theta}, and again statement~\eqref{item:X-min-Theta} immediately follows.
Statement~\eqref{item:X-rep-contain-min} is a corollary of the existence of a finite representative set, as shown in Corollary~\ref{cor:finite-rep-set}.
Again, Statement~\eqref{item:X-full-dim} is exactly Lemma~\ref{lem:fdls-exactly-theta}.
Statement~\eqref{item:X-redundant} is exactly Lemma~\ref{lem:any-levelset-contained-in-minlevelset}.
Finally, Statement~\eqref{item:X-tight-embed} follows as a corollary of statement~\eqref{item:X-min-V} and Corollary~\ref{cor:tight-embed-min-rep}.



\end{document}
%%% Local Variables:
%%% mode: latex
%%% TeX-master: t
%%% End:
