\documentclass{article}
\usepackage[utf8]{inputenc}
\usepackage{mathtools, amsthm, amsmath, amssymb, graphicx, verbatim}
\usepackage{bbm}

\newcommand{\Comments}{1}
\newcommand{\mynote}[2]{\ifnum\Comments=1\textcolor{#1}{#2}\fi}
\newcommand{\mytodo}[2]{\ifnum\Comments=1%
  \todo[linecolor=#1!80!black,backgroundcolor=#1,bordercolor=#1!80!black]{#2}\fi}
\newcommand{\raf}[1]{\mynote{green}{[RF: #1]}}
\newcommand{\raft}[1]{\mytodo{green!20!white}{RF: #1}}
\newcommand{\jessie}[1]{\mynote{purple}{[JF: #1]}}
\newcommand{\jessiet}[1]{\mytodo{purple!20!white}{JF: #1}}
\newcommand{\bo}[1]{\mynote{blue}{[Bo: #1]}}
\newcommand{\botodo}[1]{\mytodo{blue!20!white}{[Bo: #1]}}
\ifnum\Comments=1               % fix margins for todonotes
  \setlength{\marginparwidth}{1in}
\fi


\newcommand{\reals}{\mathbb{R}}
\newcommand{\posreals}{\reals_{>0}}%{\reals_{++}}
\newcommand{\nonnegreals}{\reals_{\geq 0}}%{\reals_{++}}
\newcommand{\dom}{\mathrm{dom}}

\newcommand{\prop}[1]{\Gamma[#1]}
\newcommand{\eliccts}{\mathrm{elic}_\mathrm{cts}}
\newcommand{\eliccvx}{\mathrm{elic}_\mathrm{cvx}}
\newcommand{\elicpoly}{\mathrm{elic}_\mathrm{pcvx}}
\newcommand{\elicembed}{\mathrm{elic}_\mathrm{embed}}

\newcommand{\cell}{\mathrm{cell}}

\newcommand{\abstain}[1]{\mathrm{abstain}_{#1}}
\newcommand{\mode}{\mathrm{mode}}

\newcommand{\simplex}{\Delta_\Y}

% alphabetical order, by convention
\newcommand{\C}{\mathcal{C}}
\newcommand{\D}{\mathcal{D}}
\newcommand{\E}{\mathbb{E}}
\newcommand{\F}{\mathcal{F}}
\newcommand{\I}{\mathcal{I}}
\newcommand{\R}{\mathcal{R}}
\newcommand{\X}{\mathcal{X}}
\newcommand{\Y}{\mathcal{Y}}

\newcommand{\inprod}[2]{\langle #1, #2 \rangle}%\mathrm{int}(#1)}
\newcommand{\inter}[1]{\mathring{#1}}%\mathrm{int}(#1)}
%\newcommand{\expectedv}[3]{\overline{#1}(#2,#3)}
\newcommand{\expectedv}[3]{\E_{Y\sim{#3}} {#1}(#2,Y)}
\newcommand{\toto}{\rightrightarrows}
\newcommand{\strip}{\mathrm{strip}}
\newcommand{\trim}{\mathrm{trim}}
\newcommand{\fplc}{finite-piecewise-linear and convex\xspace} %xspace for use in text
\newcommand{\conv}{\mathrm{conv}}
\newcommand{\ones}{\mathbbm{1}}
\DeclarePairedDelimiter\ceil{\lceil}{\rceil}

\newcommand{\Ind}{\mathbf{1}}

\DeclareMathOperator*{\argmax}{arg\,max}
\DeclareMathOperator*{\argmin}{arg\,min}
\DeclareMathOperator*{\arginf}{arg\,inf}
\DeclareMathOperator*{\sgn}{sgn}


\newtheorem{theorem}{Theorem}
\newtheorem{lemma}{Lemma}
\newtheorem{claim}{Claim}
\newtheorem{corollary}{Corollary}

\theoremstyle{definition}
\newtheorem{definition}{Definition}

\begin{document}

Given a convex set $Q \subseteq \reals^n$ and a point $p \in Q$, the \emph{cone of feasible directions} of $Q$ at $p$ is $F_Q(p) = \{v \in \reals^n: \exists \epsilon > 0 \text{ s.t. } p + \epsilon v \in Q\}$. The \emph{feasible subspace dimension} of $Q$ at $p$ is the dimension of the subspace $F_Q(p) \cap -F_Q(p)$.

Write it $FSD(Q,p)$.


\vskip1em
%Given a discrete loss $\ell: \R \times \Y \to \nonnegreals$, we say a loss $L: \reals^d \times \Y \to \nonnegreals$ \emph{indirectly elicits $\ell$} if there exists a ``link'' $\psi: \reals^d \to \R$ such that $\psi\left(\argmin_u L(u;p)\right) \subseteq \argmin_r \ell(r;p)$, for all $p$.
Given a finite property $\gamma: \Delta_{\Y} \to 2^{\R}$, we say a loss $L: \reals^d \times \Y \to \nonnegreals$ \emph{indirectly elicits $\gamma$} if there exists a ``link'' $\psi: \reals^d \to \R$ such that $\psi\left(\argmin_u L(u;p)\right) \subseteq \gamma(p)$, for all $p$.

\vskip1em
\begin{theorem}
  Suppose $L: \reals^d \times \Y \to \nonnegreals$ is convex for each $y$ and indirectly elicits $\gamma$. Then for all $p \in \Delta_{\Y}$ and all $r \in \gamma(p)$,
    \[ d \geq \|p\|_0 - FSD(\gamma_r, p) - 1 . \]
\end{theorem}
\begin{proof}[IN PROGRESS]
  Let $u \in \argmin_{u'} L(u';p)$ and let $r = \psi(u)$.
  This implies $p \in \gamma_r$.

  Because $L$ is convex, we have $\vec{0} \in \partial L(u;p)$.
  In particular, there exist $\{w_y \in \partial L(u,y) : y \in \Y\}$ such that $\sum_y p(y) w_y = \vec{0}$.

  Now consider any $p'$ such that $\sum_y p'(y) w_y = \vec{0}$.
  This implies that $u$ is also an optimal report for $p'$, so $p' \in \gamma_r$ as well.
  Hence, if we write $W \in \reals^{d \times y$ as a matrix whose columns are $w_y$, we have
    \[ \gamma_r \supseteq \{p' \in \Delta_{\Y} : p' W = \vec{0} \}.  \]
  or maybe?
    \[ \gamma_r \supseteq \{p' \in \Delta_{\Y} : (p'-p) W = \vec{0} \}.  \]
  This is an equation in $|\Y|-1$ nonnegative variables with $d$ constraints, so the dimensionality of its solution is \bo{CHECK} at least $|\Y|-d-1$.
  
  Note: If $\|p\|_0 < \|\Y|$, then everything drops down to that many dimensions because the zero coefficients from $p$ can be dropped from columns of the matrix, etc.

  To look into: do the ``flats'' always extend all the way to the edge of the simplex??

\end{proof}

\end{document}

