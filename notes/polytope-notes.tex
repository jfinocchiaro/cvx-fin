\documentclass[12pt]{article}
\usepackage[utf8]{inputenc} % allow utf-8 input
\usepackage[T1]{fontenc}    % use 8-bit T1 fonts
\usepackage{lmodern}
\usepackage{hyperref}       % hyperlinks  %[implicit=false, bookmarks=false]
\usepackage{url}            % simple URL typesetting
\usepackage{booktabs}       % professional-quality tables
\usepackage{amsfonts}       % blackboard math symbols
\usepackage{nicefrac}       % compact symbols for 1/2, etc.
\usepackage{microtype}      % microtypography
\usepackage[margin=1.0in]{geometry}

\usepackage{mathtools, amsmath, amssymb, amsthm, graphicx, verbatim}
%\usepackage[thmmarks, thref, amsthm]{ntheorem}
\usepackage{color}
\definecolor{darkblue}{rgb}{0.0,0.0,0.2}
\hypersetup{colorlinks,breaklinks,
            linkcolor=darkblue,urlcolor=darkblue,
            anchorcolor=darkblue,citecolor=darkblue}
\usepackage{wrapfig}
\usepackage{subcaption}
\usepackage[colorinlistoftodos,textsize=tiny]{todonotes} % need xargs for below
%\usepackage{accents}
\usepackage{bbm}
\usepackage{xspace}

\usetikzlibrary{calc}
\newcommand{\Comments}{1}
\newcommand{\mynote}[2]{\ifnum\Comments=1\textcolor{#1}{#2}\fi}
\newcommand{\mytodo}[2]{\ifnum\Comments=1%
  \todo[linecolor=#1!80!black,backgroundcolor=#1,bordercolor=#1!80!black]{#2}\fi}
\newcommand{\raf}[1]{\mynote{green}{[RF: #1]}}
\newcommand{\raft}[1]{\mytodo{green!20!white}{RF: #1}}
\newcommand{\jessie}[1]{\mynote{purple}{[JF: #1]}}
\newcommand{\jessiet}[1]{\mytodo{purple!20!white}{JF: #1}}
\newcommand{\bo}[1]{\mynote{blue}{[Bo: #1]}}
\newcommand{\botodo}[1]{\mytodo{blue!20!white}{[Bo: #1]}}
\newcommand{\btw}[1]{\mytodo{orange!20!white}{BTW: #1}}
\ifnum\Comments=1               % fix margins for todonotes
  \setlength{\marginparwidth}{1in}
\fi


\newcommand{\reals}{\mathbb{R}}
\newcommand{\posreals}{\reals_{>0}}%{\reals_{++}}
\newcommand{\dom}{\mathrm{dom}}
\newcommand{\epi}{\text{epi}}
\newcommand{\relint}{\mathrm{relint}}
\newcommand{\prop}[1]{\Gamma[#1]}
\newcommand{\eliccts}{\mathrm{elic}_\mathrm{cts}}
\newcommand{\eliccvx}{\mathrm{elic}_\mathrm{cvx}}
\newcommand{\elicpoly}{\mathrm{elic}_\mathrm{pcvx}}
\newcommand{\elicembed}{\mathrm{elic}_\mathrm{embed}}

\newcommand{\cell}{\mathrm{cell}}

\newcommand{\abstain}[1]{\mathrm{abstain}_{#1}}
\newcommand{\mode}{\mathrm{mode}}

\newcommand{\simplex}{\Delta_\Y}

% alphabetical order, by convention
\newcommand{\C}{\mathcal{C}}
\newcommand{\D}{\mathcal{D}}
\newcommand{\E}{\mathbb{E}}
\newcommand{\F}{\mathcal{F}}
\renewcommand{\H}{\mathcal{H}}
\newcommand{\N}{\mathcal{N}}
\newcommand{\I}{\mathcal{I}}
\newcommand{\R}{\mathcal{R}}
\newcommand{\T}{\mathcal{T}}
\newcommand{\U}{\mathcal{U}}
\newcommand{\V}{\mathcal{V}}
\newcommand{\X}{\mathcal{X}}
\newcommand{\Y}{\mathcal{Y}}
\renewcommand{\P}{\mathcal{P}}

\newcommand{\risk}[1]{\underline{#1}}
\newcommand{\inprod}[2]{\langle #1, #2 \rangle}%\mathrm{int}(#1)}
\newcommand{\inter}[1]{\mathrm{int}(#1)}%\mathrm{int}(#1)}
%\newcommand{\expectedv}[3]{\overline{#1}(#2,#3)}
\newcommand{\expectedv}[3]{\E_{Y\sim{#3}} {#1}(#2,Y)}
\newcommand{\toto}{\rightrightarrows}
\newcommand{\strip}{\mathrm{strip}}
\newcommand{\trim}{\mathrm{trim}}
\newcommand{\fplc}{finite-piecewise-linear and convex\xspace} %xspace for use in text
\newcommand{\conv}{\mathrm{conv}}
\newcommand{\indopp}{\bar{\mathbbm{1}}}
\newcommand{\ones}{\mathbbm{1}}
\DeclarePairedDelimiter\ceil{\lceil}{\rceil}

\newcommand{\Ind}[1]{\mathbf{1}\{#1\}}

\DeclareMathOperator*{\argmax}{arg\,max}
\DeclareMathOperator*{\argmin}{arg\,min}
\DeclareMathOperator*{\arginf}{arg\,inf}
\DeclareMathOperator*{\sgn}{sgn}

\newtheorem{theorem}{Theorem}
\newtheorem{lemma}{Lemma}
\newtheorem{proposition}{Proposition}
\newtheorem{definition}{Definition}
\newtheorem{corollary}{Corollary}
\newtheorem{conjecture}{Conjecture}
\newtheorem{claim}{Claim}


\title{Polytope notes}
\date{}

\begin{document}
\maketitle

\section{Preliminaries}
\begin{definition}
	A \emph{polyhedra} in $\reals^d$ is defined by the intersection of a finite number of half-spaces.
	A \emph{polytope} is a bounded polyhedra.
\end{definition}

\begin{definition}[Valid inequality]
	Let $S$ be a set in $\reals^d$.
	A \emph{valid inequality} for $S$ is an inequality that holds for all vectors in $S$.
	That is, the pair $(a,\beta)$ is a valid inequality for $S$ if and only if 
	\begin{align*}
	\inprod{a}{x} &\leq \beta \; \; \forall x \in S~.~
	\end{align*}
\end{definition}

\begin{definition}[Face]
	For any valid inequality of a polytope, the subset of the polytope of vectors which are tight for the inequality is called a \emph{face} of the polytope.
	That is, the set $F$ is a face of the polytope $P$ if and only if 
	\begin{align*}
	F &= \{x \in P : \inprod{a}{x} = \beta \}
	\end{align*}
	for some valid inequality $(a, \beta)$ of $P$.
\end{definition}

\begin{definition}[Supporting function]
	Let $S$ be a nonempty bounded set in $\reals^d$.
	We call the \emph{supporting function} of $S$ the function $H_S:\reals^d \to \reals$ by
	\begin{align*}
	H_S(a) := \sup_{x \in S}\inprod{a}{x}~.~
	\end{align*} 
\end{definition}

\begin{definition}[Maximizers]
  Let $S \subseteq \reals^d$, and $a \in \reals^d$.
  The \emph{set of maximizers} of $a$ over $S$ is defined as
  \begin{align*}
    \mathcal{S}(S;a) &= \{x \in S : \inprod a x = H_S(a)\}
  \end{align*}
\end{definition}

\begin{definition}[Normal cones]
  Let $P$ be a polytope in $\reals^d$.
  For any face $F$ of $P$, we define its \emph{normal cone} $\N(F;P)$ as the set of vectors for which $F$ is the maximizer set over $P$.
  That is,
  \begin{align*}
    \N(F;P) = \left\{a : F = \mathcal{S}(P; a) \right\}~.~
  \end{align*}
\end{definition}

  It is worth noting that normal cones are generally not closed by this definition, but sometimes we may want to think about the closure of the normal cone.

\begin{definition}[Minkowski sum]
	Let $S_1, S_2, \ldots, S_n$ be sets of vectors.
	We can define their \emph{Minkowski sum} as the set of vectors which can be written as the sum of a vector in each set.
	Namely,
	\begin{align*}
	S_1 \oplus \ldots \oplus S_n &= \{x_1 + \ldots + x_n : x_i \in S_i \; \forall i \}
	\end{align*}
\end{definition}

\begin{theorem}[EPFL Thesis Theorem 3.1.2]
	Let $P_1, \ldots, P_n$ be polytopes in $\reals^d$ and let $F$ be a face of the Minkowski sum $P := P_1 \oplus \ldots \oplus P_n$.
	Then there are faces $F_1, \ldots, F_n$ of $P_1, \ldots, P_n$ respectively such that $F = F_1 \oplus \ldots \oplus F_n$.
	Moreover, this decomposition is unique.
\end{theorem}

\begin{corollary}[EPFL Cor 3.1.3]\label{cor:face-decomp-normal-cones}
  Let $P = P_1 \oplus \ldots \oplus P_n$ be a Minkowski sum of polytopes in $\reals^d$, let $F$ be a nonempty face of $P$ , and let $F_1, \ldots, F_n$ be its decomposition.
  Then $\N(F;P) = \N(F_1;P_1) \cap \ldots \cap \N(F_n; P_n)$.
\end{corollary}

\begin{corollary}[EPFL Cor 3.1.4]
  Let $F_1, \ldots, F_n$ be nonempty faces of the polytopes $P_1, \ldots, P_n$ respectively, then $F_1 \oplus \ldots \oplus F_n$ is a face of $P_1 \oplus \ldots \oplus P_n$ if and only if the intersection of their normal cones is nonempty.
\end{corollary}

\begin{theorem}[EPFL Th 3.1.6]\label{thm:support-minksum}
  The supporting function of a Minkowski sum is the sum of the supporting functions of its summands.
\end{theorem}

\begin{definition}[Weighted Minkowski sum]
  If $P_1, \ldots, P_n$ are polytopes in $\reals^d$, we can call $P(\vec \lambda)$ their \emph{weighted} Minkoski sum for $\vec \lambda \in \reals^n_+$
  \begin{align*}
    P(\lambda_1, \ldots, \lambda_n) = \lambda_1 P_1 \oplus \ldots \oplus \lambda_n P_n
  \end{align*}
\end{definition}

  From the thesis:  \emph{``It is easy to see that the normal fan (undefinedhere, but consequently normal cones) of $\lambda_i P_i$ does not change as long as $\lambda_i$ is positive.  Since the normal fan of a Minkowski sum can be deduced from that of its summands, we can deduce from this that the conbimatorial properties of $P(\lambda_1, \ldots, \lambda_n)$ stay the same as long as all $\lambda_i$ are positive.''}

  \section{Starting with polytopes}\label{sec:start-polytope}
  In this section, we will suppose we start with a finite set of $n$ polytopes $\T := \{T_1, \ldots, T_n\}$, and we will call $T := T_1 \oplus \ldots \oplus T_n$ their Minkowski sum.
  We know that every polytope has both a halfspace and vertex representation ($\H$-representation and $\V$ representation, respectively.)
  In order to construct the $\H$-representation, we know there must be a matrix $V$ and vector $e$ such that $T = \{x : Vx \leq e\}$, but it is unclear how to find such a set, which we discuss later.

  \subsection{Notation and new definitions}
  We will use the following notation throughout the rest of this section.
  Suppose we are given a polytope $T_y \in \reals^d$ and set of vectors $V \in \reals^{k \times d}$.
  Call $e^y \in \reals^k$ the vector such that $e^y_i = \max_{x \in T_y}\inprod{v_i}{x}$.  
  For a finite set $\T = \{T_1, , \ldots, T_n\}$, let us denote the matrix $E = (e^y)_{y=1}^n$.
  \begin{definition}
    We say a set of normals $V$ is \emph{complete} with respect to a polytope $T_y$ if $T_y = \{x \in \reals^d: Vx \leq e^y\}$.

    Moreover, $V$ is complete with respect to the set of polytopes $\T$ if and only if $V$ is complete with respect to each $T_y \in \T$.
  \end{definition}

  \subsection{Finding $V$}
  \btw{Not sure if we want to go this far, or just argue that a complete set of normals exists.  This would be where Corollaries 1 and 2 come in though.}
  In order to find a $V$ that is complete with respect to $T := \oplus_y T_y$, we can look at the normal cones for each $T_i$ and find a vector in the normal cone $\N(F_y, T_y)$ in the decomposition of $F$ in $T$ for each face $F$.
  Since we know $T_1, \ldots, T_n$ are polytopes with $\H$-representations, one of the first natural steps is to find a (possibly minimal) set of valid inequalities describing each $T_i$.
  For each face $F_j$ of a given polytope $T$, we can choose a normal $v_j \in \N(F_j; T)$ and describe the face $F$ by $\{x \in T:  \inprod{v_j}{x} = \sup_{x \in T} \inprod{v_j}{x} =: e_j\}$.
  For each face $F_{j}^{y}$ that is in the decomposition of the face $F_j$, we have $\inprod {v_j}{x} \leq e^y_j$ forming a tight bound on $F_j$. 
  If we do this over each $T_y$, we can call $e^y$ the vector of such $e^y_j$ vectors for each face of $T_y$.
  We then concatenate these $e^y$ vectors to form the matrix $E$.
  
%  Therefore, if we take a matrix $V$ that is a finite set of the elements of the normal cones for each face in $T$, such a matrix can also be used to define the $\H$-representation for not only $T$, but each $T_y$.
%  For each face $F_i$ of $T$ with $\N(F_i, T) \neq \emptyset$, we can choose one normal $v_i$ so that the face $F_i = \{x : \inprod{v_i}{x} = \sup_{x\in T}\inprod{v_i}{x} =: e_i\}$.
%  As this happens for each face with a nonempty normal cone, we can repeat this for each $T_y \in \T$.
%  This yields the matrix $E := (e_{yj})$ where $e_{yj} = \sup_{x \in T_y}\inprod{v_j}{x}$.
%  
  
  \begin{conjecture}
  	The set of normals $V$ and matrix $E$ constructed by taking an element of the normal cone $\N(F, T)$ for each face $F$ in $T$ is complete for both $T$ and each $T_y \in \T$.
  \end{conjecture}
  \begin{proof}
  	First, to see this constructed $V$ is complete for $T$, consider that the set of normals to the facets (in the affine hull) of $T$ is complete for $T$.
  	Moreover, each additional normal is a redundant constraint, so
%   since the inner product of elements of the normal cone is less than or equal to the support for the polytope for every element of $T$.
%  	Thus, 
  	this set of normals is complete for $T$.
  	
  	To see why $V$ is complete with respect to each $T_y$, consider each case of subset inclusion.
  	We will let $\bar T_y$ be the polytope described by $V$ and $E$, and $T_y$ be the given polytope.
  	
  	First,  to see $T_y \subseteq \bar T_y  := \{ x : Vx \leq E_{;y}\}$, consider some $x \in T_y$.
  	For all $i \in [k]$, we then have $\inprod{v_i}{x} \leq H_{T_y}(v_i) = e^y_i$.
  	By definition, this holds for all $v_i$ such that $v_i \in \N(F, T_y)$ for some $F$ a face of $T_y$.
  	Moreover, we want to claim this is true for all $v_i$, i.e. for all $i$, we have $v_i \in \N(F_y, T_y)$ for some face $F_y$ of $T_y$.
  	
  	This follows from Corollary~\ref{cor:face-decomp-normal-cones}, since we know that we must have $v_i \in \N(F, T)$ for some face $F$ of $T$, and we necessarily have $v_i \in \N(F_y, T_y)$, where $F_y$ is the face of $T_y$ in the decomposition of $F$.
  	
  	Now to see $\bar T_y \subseteq T_y$, we use a contradiction.
  	Suppose there was some $x \in \bar T_y$ such that $x \not \in T_y$.
  	Then there must be some $v_i \in V$ so that $\inprod {v_i}{x} > \sup_{y \in T_y} \inprod{v_i}{y}$.
  	This implies that for all faces $F_{j}^y$ of $T_y$, we have $v_i \not \in \N(F_{j}^y, T_y)$.
  	However, by construction of $V$, we can consider the face $F$ of $T$ such that $v_i \in \N(F,T) = \cap_y \N(F_y, T_y)$ by Corollary~\ref{cor:face-decomp-normal-cones}.
  	We necessarily have $v_i \in \N(F_y, T_y)$, yielding a contradiction.
  	Thus, we have $v_k \in \N(F_y, T_y)$, so we conclude $\bar T_y \subseteq T_y$, yielding the equality of each $\bar T_y = T_y$.
  \end{proof}
  

  
  \subsection{Once we have $V$}
  Now, for a given polytope $T(p) := \oplus p_y T_y$, we will later want to ask when a given $z \in \reals^d$ is in the polytope $T(p)$.
  We will then use this to understand the set of $p \in \simplex$ for which the given $z \in T(p)$.
  Throughout, assume we have $V$ which is complete for $\T$ and $E$ defined by the support of each normal in $V$.
  
  Observe that since we define $T_y = \{x : Vx \leq E_{;y}\}$, we can multiply the right side of the inequality by the constant $p_y \geq  0$ to yield $p_y T_y = \{x : Vx \leq p_y E_{;y}\}$.
  Now in taking the Minkowski sum of polytopes described by the same set of normals, we can take 
  \begin{align*}
  \oplus_y p_y T_y &= \{x : Vx \leq p_1 E_{;1}\} \oplus \ldots \oplus \{x : Vx \leq p_n E_{;n}\} \\
  &= \{x : Vx \leq p_1 E_{;1} + \ldots + p_n E_{;n}\}\\
  &= \{x : Vx \leq E p\}~.~
  \end{align*}
  Observe the first to second line follows from Theorem~\ref{thm:support-minksum} and preservation of inequalities under addition.
  
  Now, we have $z \in T(p) \iff \inprod{v_i}{z} \leq (Ep)_i$ for all $v_i \in V$.
  As we are particularly interested in when $\vec 0 \in T(p)$, we have $\vec 0 \in T(p)$ if and only if $E p \geq 0$ by substitution.  \btw{For later, common result to say $T(p) = \partial \inprod p {L(\cdot)}$, so if $0 \in T(p)$, we have a minimizer of the loss.}


  As we assume $p \in \simplex$, we now describe the cell $D := \{p \in \simplex : Ep \geq \vec 0\}$ as the set of distributions such that $\vec 0 \in T(p)$.


  \section{Starting at the cell $B$}\label{sec:start-cell}
  
  Now let us consider some cell $C \subseteq \simplex$.
  Let $A$ be the affine span of the simplex, $A := \{p \in \reals^n : \sum_y p_y = 1\}$.
  By existence of its $\H$-representation, we can represent the cell $C := \{p \in A : Bp \geq c\}$.
  In fact, we can set $c := \vec 0$ in order to enforce a unique representation of $B$, up to row reordering and scaling.
  This also allows us to use the same matrix for the $\H$-representation even as we scale the simplex by a positive constant.
 
  \jessie{Notes post NeurIPS} 
  Still not sure if we can construct $B$ with the assumption that $p \in \simplex$ rather than $p \in A$, but gathering from our conversations at NeurIPS, that question seems worth investigating.
  I mean that in the sense that this $B$ matrix captures our necessary condition.
  The vertex representation seems to capture our sufficient condition, possibly rendering the simplex halfspace constraints unnecessary when we think of them as necessary conditions.

  \section{Equivalences; moving from polytopes to cells and back}
  
  In the big picture, we want to know if we are given a cell $C$, how we can come up with a finite set of polytopes $\T = \{T_1, \ldots, T_n\}$ where $T(p) = \oplus_y p_y T_y$ such that $\vec 0 \in T(p) \iff p \in C$, or vice versa: starting from $\T$, can we derive the cell $C \subseteq \simplex$ where $\vec 0 \in T(p)$ for all $p \in C$.
  
  We know that if we are given $\T$ and a complete set of normals $V$, we can describe $D = \{p \in \simplex : Ep \geq \vec 0\}$ as in Section~\ref{sec:start-polytope}.
  However, we do not necessarily have $D' := \{p \in A : Ep \geq \vec 0\} \subseteq \simplex$, and this poses some issues given the construction of the cell $C$ only assuming that we are in the affine span of the simplex, but not necessarily in the simplex.
  
  \begin{lemma}\label{lem:describe-D}
    Suppose we have polytopes $\T = \{T_1, \ldots, T_n\}$ and a set of normals $V$ that is complete for $\T$.
 	Take $E = (e_{iy})$ where $e_{iy} = \max_{x \in T_y} \inprod{v_i}{x}$, and $D = \{p \in \simplex : Ep \geq \vec 0\}$.
 	
 	Then $\{p \in \simplex : \vec 0 \in \oplus_y p_y T_y\} = \{p \in \simplex: Ep \geq \vec 0\}$.\btw{``stuff about $E$ and $D$.''}
  \end{lemma}
  \begin{proof}
    First, let us fix a distribution $p \in \simplex$.
    We define $T(p) := \oplus_y p_y T_y$ to be the $p$-weighted Minkowski sum over the polytopes in $\T$.
    By Theorem~\ref{thm:support-minksum}, we have the support of the (weighted) Minkowski sum is the (weighted) sum of the support of each polytope, which we can re-write as the product $Ep$.
    
    As each halfspace is bounded by the support by construction of $E$, then the support of the weighted polytope defined by an inequality on $v_i$ can be described as $\inprod{v_i}{z} \leq \inprod{E_i}{p}$.
    Since we this is true for all $v_i$, we then have $T(p) = \oplus_y p_y T_y = \{x \in \reals^d : Vx \leq Ep\}$.
    	
    Therefore, for fixed $p$, we have $\vec 0 \in \oplus_y p_y T_y \iff Ep \geq \vec 0$.
    Since this is true for all $p \in \simplex$, we observe the stated set equality.
  \end{proof}
  
  \begin{proposition}\label{prop:relate-E-B}
    Suppose we have polytopes $\T = \{T_1, \ldots, T_n\}$ and a set of normals $V$ that is complete for $\T$.
    Take $E = (e_{iy})$ where $e_{iy} = \max_{x \in T_y} \inprod{v_i}{x}$, and take $D = \{p \in \simplex : Ep \geq \vec 0\}$ and $C = \{p \in A : Bp \geq \vec 0\}$.
    
    Then $\{p \in \simplex : \vec 0 \in \oplus_y p_y T_y\} = C$	if and only if $C = D$.
    \btw{``stuff about $E$ and $B$.''}
  \end{proposition}
  \begin{proof}
  	By Lemma~\ref{lem:describe-D}, we have $D = \{p \in \simplex : \vec 0 \in \oplus_y p_y T_y\}$ by construction, so the double implication follows from substitution.
  \end{proof}
  
  \begin{definition}
  	We say a vector $v$ is \emph{redundant} with respect to matrix $Y$ if we have $\{z \in A : Yz \geq \vec 0\} = \{z \in A : [Y;v]z \geq \vec 0\}$.
  \end{definition}

  \begin{proposition}\label{prop:relate-rows}
  	Suppose we have polytopes $\T = \{T_1, \ldots, T_n\}$ and a set of normals $V$ that is complete for $\T$.
  	Take $E = (e_{i}^y)$ where $e_{i}^y = \max_{x \in T_y} \inprod{v_i}{x}$, and take $D = \{p \in \simplex : Ep \geq \vec 0\}$ and $C = \{p \in A : Bp \geq \vec 0\}$.
  	
  	Then $\{p \in \simplex : \vec 0 \in \oplus_y p_y T_y\} = C$	if and only if non-simplex rows of $B$ appear in $E$ (possibly scaled) and every other row of $E$ is redundant with respect to $B$.
  	\btw{``replace prop 1 statement with `non-simplex rows of $B$ appear in $E$ (possible scaled) and every other row of $E$ is redundant with respect to $B$.'''}
  \end{proposition}
  \begin{proof}
	\begin{itemize}
		\item [$\implies$] First, assume $C = \{p \in \simplex: \vec 0 \in \oplus_y p_y T_y\}$.
		By Proposition~\ref{prop:relate-E-B}, we know that $C = D := \{p \in \simplex : Ep \geq \vec 0\}$.
		We can re-write $D$ as $\{p \in A : [E;I_n] p \geq \vec 0\}$, building simplex constraints into the inequality rather than the domain assumption.
		By the assumed equality, every row of $B$ must then appear in $[E;I_n]$, where $I_n$ is the $n$-element identity matrix.
		As $\{p \in A : I_n p \geq \vec 0\}$ strictly defines the simplex, any non-simplex row of $B$ must then be described by a row of $E$, up to rescaling.
		
		What's more is that since the polytopes are the same, they have a unique minimal $\H$-representation (up to rescaling) since they are defined by an inequality on $\vec 0$.
		\btw{Elaborated this justification in the comment below ``As $D$ is defined... yielding a contradiction.''}
%		As $D$ is defined by distributions on $\simplex$ and $C$ on $A$, and $\simplex \subset A$, we have the rest of the rows in $E$ necessarily being redundant with respect to $B$.
%		To see this, suppose there was a row of $E$ that was not redundant with respect to $B$.
%		If such a row did not partition the simplex, it would necessarily be redundant by the restriction that $C \subseteq \simplex$.
%		Thus, such a row much partition the simplex, but not appear in $B$ up to rescaling.
%		Again, if the constraint does not partition the cell, the constraint is redundant, and if it does, then $C \neq D$, yielding a contradiction.
		
		\item [$\impliedby$] Suppose that any non-simplex rows of $B$ appear in $E$, and every other row of $E$ is redundant with respect to $B$.
		Then we have $D = \{p\in \simplex : Ep \geq \vec 0\} = \{p \in \simplex : Bp \geq \vec 0\}$.
		This polytope is equal to $\{p \in A : Bp \geq \vec 0\} = C$ since we require $C\subseteq \simplex$, so $B$ encapsulates any non-redundant simplex constraints.
				
		Then $D = C$, and therefore, by Proposition~\ref{prop:relate-E-B}, we have $C = \{p \in \simplex : \vec 0 \in \oplus_y p_y T_y\}$.
	\end{itemize}
  \end{proof}

  \bigskip
  \hrule
  \bigskip

  \section{Necessary and sufficient conditions}
  \btw{This probably shouldn't be in this document... let's move it later.}
  \subsection{Necessary conditions}
    Halfspace representation
  \subsection{Sufficient conditions}
    Vertex representation
  
\end{document}
%%% Local Variables:
%%% mode: latex
%%% TeX-master: t
%%% End:
