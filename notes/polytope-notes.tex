\documentclass[12pt]{article}
\usepackage[utf8]{inputenc} % allow utf-8 input
\usepackage[T1]{fontenc}    % use 8-bit T1 fonts
\usepackage{lmodern}
\usepackage{hyperref}       % hyperlinks  %[implicit=false, bookmarks=false]
\usepackage{url}            % simple URL typesetting
\usepackage{booktabs}       % professional-quality tables
\usepackage{amsfonts}       % blackboard math symbols
\usepackage{nicefrac}       % compact symbols for 1/2, etc.
\usepackage{microtype}      % microtypography
\usepackage[margin=1.0in]{geometry}

\usepackage{mathtools, amsmath, amssymb, amsthm, graphicx, verbatim}
%\usepackage[thmmarks, thref, amsthm]{ntheorem}
\usepackage{color}
\definecolor{darkblue}{rgb}{0.0,0.0,0.2}
\hypersetup{colorlinks,breaklinks,
            linkcolor=darkblue,urlcolor=darkblue,
            anchorcolor=darkblue,citecolor=darkblue}
\usepackage{wrapfig}
\usepackage{subcaption}
\usepackage[colorinlistoftodos,textsize=tiny]{todonotes} % need xargs for below
%\usepackage{accents}
\usepackage{bbm}
\usepackage{xspace}

\usetikzlibrary{calc}
\newcommand{\Comments}{1}
\newcommand{\mynote}[2]{\ifnum\Comments=1\textcolor{#1}{#2}\fi}
\newcommand{\mytodo}[2]{\ifnum\Comments=1%
  \todo[linecolor=#1!80!black,backgroundcolor=#1,bordercolor=#1!80!black]{#2}\fi}
\newcommand{\raf}[1]{\mynote{green}{[RF: #1]}}
\newcommand{\raft}[1]{\mytodo{green!20!white}{RF: #1}}
\newcommand{\jessie}[1]{\mynote{purple}{[JF: #1]}}
\newcommand{\jessiet}[1]{\mytodo{purple!20!white}{JF: #1}}
\newcommand{\bo}[1]{\mynote{blue}{[Bo: #1]}}
\newcommand{\botodo}[1]{\mytodo{blue!20!white}{[Bo: #1]}}
\newcommand{\btw}[1]{\mytodo{orange!20!white}{BTW: #1}}
\ifnum\Comments=1               % fix margins for todonotes
  \setlength{\marginparwidth}{1in}
\fi


\newcommand{\reals}{\mathbb{R}}
\newcommand{\posreals}{\reals_{>0}}%{\reals_{++}}
\newcommand{\dom}{\mathrm{dom}}
\newcommand{\epi}{\text{epi}}
\newcommand{\relint}{\mathrm{relint}}
\newcommand{\prop}[1]{\Gamma[#1]}
\newcommand{\eliccts}{\mathrm{elic}_\mathrm{cts}}
\newcommand{\eliccvx}{\mathrm{elic}_\mathrm{cvx}}
\newcommand{\elicpoly}{\mathrm{elic}_\mathrm{pcvx}}
\newcommand{\elicembed}{\mathrm{elic}_\mathrm{embed}}

\newcommand{\cell}{\mathrm{cell}}

\newcommand{\abstain}[1]{\mathrm{abstain}_{#1}}
\newcommand{\mode}{\mathrm{mode}}

\newcommand{\simplex}{\Delta_\Y}

% alphabetical order, by convention
\newcommand{\C}{\mathcal{C}}
\newcommand{\D}{\mathcal{D}}
\newcommand{\E}{\mathbb{E}}
\newcommand{\F}{\mathcal{F}}
\renewcommand{\H}{\mathcal{H}}
\newcommand{\N}{\mathcal{N}}
\newcommand{\I}{\mathcal{I}}
\newcommand{\R}{\mathcal{R}}
\newcommand{\T}{\mathcal{T}}
\newcommand{\U}{\mathcal{U}}
\newcommand{\V}{\mathcal{V}}
\newcommand{\X}{\mathcal{X}}
\newcommand{\Y}{\mathcal{Y}}
\renewcommand{\P}{\mathcal{P}}

\newcommand{\risk}[1]{\underline{#1}}
\newcommand{\inprod}[2]{\langle #1, #2 \rangle}%\mathrm{int}(#1)}
\newcommand{\inter}[1]{\mathrm{int}(#1)}%\mathrm{int}(#1)}
%\newcommand{\expectedv}[3]{\overline{#1}(#2,#3)}
\newcommand{\expectedv}[3]{\E_{Y\sim{#3}} {#1}(#2,Y)}
\newcommand{\toto}{\rightrightarrows}
\newcommand{\strip}{\mathrm{strip}}
\newcommand{\trim}{\mathrm{trim}}
\newcommand{\fplc}{finite-piecewise-linear and convex\xspace} %xspace for use in text
\newcommand{\conv}{\mathrm{conv}}
\newcommand{\indopp}{\bar{\mathbbm{1}}}
\newcommand{\ones}{\mathbbm{1}}
\DeclarePairedDelimiter\ceil{\lceil}{\rceil}

\newcommand{\Ind}[1]{\mathbf{1}\{#1\}}

\DeclareMathOperator*{\argmax}{arg\,max}
\DeclareMathOperator*{\argmin}{arg\,min}
\DeclareMathOperator*{\arginf}{arg\,inf}
\DeclareMathOperator*{\sgn}{sgn}

\newtheorem{theorem}{Theorem}
\newtheorem{lemma}{Lemma}
\newtheorem{proposition}{Proposition}
\newtheorem{definition}{Definition}
\newtheorem{corollary}{Corollary}
\newtheorem{conjecture}{Conjecture}
\newtheorem{condition}{Condition}
\newtheorem{claim}{Claim}


\title{Polytope notes}
\date{}

\begin{document}
\maketitle

\section{Preliminaries}
\begin{definition}
	A \emph{polyhedra} in $\reals^d$ is defined by the intersection of a finite number of half-spaces.
	A \emph{polytope} is a bounded polyhedra.
\end{definition}

\begin{definition}[Valid inequality]
	Let $S$ be a set in $\reals^d$.
	A \emph{valid inequality} for $S$ is an inequality that holds for all vectors in $S$.
	That is, the pair $(a,\beta)$ is a valid inequality for $S$ if and only if 
	\begin{align*}
	\inprod{a}{x} &\leq \beta \; \; \forall x \in S~.~
	\end{align*}
\end{definition}

\begin{definition}[Face]\label{def:face}
	For any valid inequality of a polytope, the subset of the polytope of vectors which are tight for the inequality is called a \emph{face} of the polytope.
	That is, the set $F$ is a face of the polytope $T$ if and only if 
	\begin{align*}
	F &= \{x \in T : \inprod{a}{x} = \beta \}
	\end{align*}
	for some valid inequality $(a, \beta)$ of $T$.
\end{definition}

\begin{definition}[Supporting function]
	Let $S$ be a nonempty bounded set in $\reals^d$.
	We call the \emph{supporting function} of $S$ the function $H_S:\reals^d \to \reals$ by
	\begin{align*}
	H_S(a) := \sup_{x \in S}\inprod{a}{x}~.~
	\end{align*} 
\end{definition}

\iffalse
\begin{definition}[Maximizers]
  Let $S \subseteq \reals^d$, and $a \in \reals^d$.
  The \emph{set of maximizers} of $a$ over $S$ is defined as
  \begin{align*}
    \mathcal{S}(S;a) &= \{x \in S : \inprod a x = H_S(a)\}
  \end{align*}
\end{definition}

\begin{definition}[Normal cones]
  Let $T$ be a polytope in $\reals^d$.
  For any face $F$ of $T$, we define its \emph{normal cone} $\N(F;T)$ as the set of vectors for which $F$ is the maximizer set over $T$.
  That is,
  \begin{align*}
    \N(F;T) = \left\{a : F = \mathcal{S}(T; a) \right\}~.~
  \end{align*}
\end{definition}

  It is worth noting that normal cones are generally not closed by this definition, but sometimes we may want to think about the closure of the normal cone.
\fi

\begin{definition}[Minkowski sum]
	Let $S_1, S_2, \ldots, S_n$ be sets of vectors.
	We can define their \emph{Minkowski sum} as the set of vectors which can be written as the sum of a vector in each set.
	Namely,
	\begin{align*}
	S_1 \oplus \ldots \oplus S_n &= \{x_1 + \ldots + x_n : x_i \in S_i \; \forall i \}
	\end{align*}
\end{definition}

\begin{theorem}[EPFL Thesis Theorem 3.1.2]\label{thm:unique-face-decomp}
	Let $T_1, \ldots, T_n$ be polytopes in $\reals^d$ and let $F$ be a face of the Minkowski sum $T := T_1 \oplus \ldots \oplus T_n$.
	Then there are faces $F_1, \ldots, F_n$ of $T_1, \ldots, T_n$ respectively such that $F = F_1 \oplus \ldots \oplus F_n$.
	Moreover, this decomposition is unique.
\end{theorem}

\iffalse 
%used in the construction of V
\begin{corollary}[EPFL Cor 3.1.3]\label{cor:face-decomp-normal-cones}
  Let $T = T_1 \oplus \ldots \oplus T_n$ be a Minkowski sum of polytopes in $\reals^d$, let $F$ be a nonempty face of $T$ , and let $F_1, \ldots, F_n$ be its decomposition.
  Then $\N(F;T) = \N(F_1;T_1) \cap \ldots \cap \N(F_n; T_n)$.
\end{corollary}

%used in proving completeness of constructed V
\begin{corollary}[EPFL Cor 3.1.4]
  Let $F_1, \ldots, F_n$ be nonempty faces of the polytopes $T_1, \ldots, T_n$ respectively, then $F_1 \oplus \ldots \oplus F_n$ is a face of $T_1 \oplus \ldots \oplus T_n$ if and only if the intersection of their normal cones is nonempty.
\end{corollary}
\fi

\begin{theorem}[EPFL Th 3.1.6]\label{thm:support-minksum}
  The supporting function of a Minkowski sum is the sum of the supporting functions of its summands.
\end{theorem}

\begin{definition}[Weighted Minkowski sum]
  If $T_1, \ldots, T_n$ are polytopes in $\reals^d$, we can call $T(\vec p)$ their \emph{weighted} Minkowski sum for $\vec p \in \reals^n_+$
  \begin{align*}
    T(\vec p) &:= \oplus_y p_y T_y = p_1 T_1 \oplus \ldots \oplus p_n T_n
  \end{align*}
\end{definition}

  From the thesis:  \emph{``It is easy to see that the normal fan (undefined here, but consequently normal cones) of $p_i T_i$ does not change as long as $p_i$ is positive.  Since the normal fan of a Minkowski sum can be deduced from that of its summands, we can deduce from this that the conbimatorial properties of $\oplus_y p_y T_y$ stay the same as long as all $p_i$ are positive.''}


  \subsection{Notation and new definitions}
  We will use the following notation throughout the rest of this section.
  Suppose we are given a polytope $T_y \in \reals^d$ and set of vectors $V \in \reals^{k \times d}$.
  Call $e^y \in \reals^k$ the vector such that $e^y_i = \max_{x \in T_y}\inprod{v_i}{x}$.  
  For a finite set $\T = \{T_1, , \ldots, T_n\}$, let us denote the \emph{support matrix} $E = (e^y)_{y=1}^n$.
  \begin{definition}
    We say a set of normals $V$ is \emph{complete} with respect to a polytope $T_y$ if $T_y = \{x \in \reals^d: Vx \leq e^y\}$.
  \end{definition}
  Moreover, we say $V$ is complete with respect to the set of polytopes $\T$ if and only if $V$ is complete with respect to each $T_y \in \T$.

  \subsubsection*{Optimality Condition}
  In general, we know that $\vec 0$ in the subgradient of $f(r)$ if and only if $r$ is an optimum of convex $f$ in general dimensions.
  
  For nice\footnote{for reports on the relative interior of the domain, and something else} conditions, we have $\vec 0 \in \oplus_i f_i$ if and only if $\vec 0 \in \oplus_i \partial f_i(r)$.
  This allows us to consider the following optimality condition:
  \begin{definition}\label{def:optimality}[Optimality]
  	For all $r \in \R$ and $y \in \Y$, we have $r \in \gamma_r \iff \vec 0 \in \oplus_y p_y T_y(r)$.
  \end{definition}
  The polytope $T_y(r)$ can be thought of as $\partial L_y(\varphi(r))$.

  \section{From polytopes to cells}\label{sec:start-polytope}
  In this section, we will suppose we start with a finite set of $n$ polytopes $\T := \{T_1, \ldots, T_n\}$, and we will call $T := T_1 \oplus \ldots \oplus T_n \in \reals^d$ their Minkowski sum.
  We know that every polytope has both a halfspace and vertex representation ($\H$-representation and $\V$-representation, respectively.)
  By existence of the $\H$-representation, we know there must be a matrix $V \in \reals^{k \times d}$ and vector $e \in \reals^k$ such that $T = \{x \in \reals^d : Vx \leq e\}$.
  In fact, with a complete set of normals $V$, we know that $e$ can be the support vector of each of the normals.
  However, finding $V$ is not always easy, so we assume that we are given $V$ for now.
  
  \subsection{Given normals $V$}
  Now, for a given polytope $T(p)$, we want to ask when a given $z \in \reals^d$ is in the polytope $T(p)$.
  We will later generalize to finding the set of $p \in \simplex$ for which $\vec 0 \in T(p)$ by substituting $z= \vec 0$.
  Throughout, assume we have $V$ which is complete for $\T$ and $E$ defined by the support of each normal in $V$ for all $T_y \in \T$.
  We denote $e^y = E_{;y}$ as the $y^{th}$ column of $E$, or equivalently, the support vector for $T_y$ given $V$.
  
  Since we define $T_y = \{x : Vx \leq e^y\}$, we can multiply the right side of the inequality by the constant $p_y \geq  0$ to yield $p_y T_y = \{x : Vx \leq p_y e^y\}$.
  Taking the Minkowski sum of polytopes described by the same set of normals, we can take 
  \begin{align*}
  \oplus_y p_y T_y &= \{x : Vx \leq p_1 E_{;1}\} \oplus \ldots \oplus \{x : Vx \leq p_n E_{;n}\} \\
  &= \{x : Vx \leq p_1 E_{;1} + \ldots + p_n E_{;n}\}\\
  &= \{x : Vx \leq E p\}~.~
  \end{align*}
  The first to second line follows from Theorem~\ref{thm:support-minksum} and preservation of inequalities under addition.
  Now, we have $z \in T(p) \iff \inprod{v_i}{z} \leq (Ep)_i$ for all $v_i \in V$.
  
  Observe that we have $\vec 0 \in T(p)$ if and only if $E p \geq 0$ by substitution, which lines up with our optimality condition in Definition~\ref{def:optimality}.  
  
  We assume $p \in \simplex$, so we now describe the cell $D^\T := \{p \in \simplex : Ep \geq \vec 0\}$ as the set of distributions such that $\vec 0 \in T(p)$.
  In cases where $\T$ is obvious from context, we omit the subscript and just write $D$.
  
  Given the complete set of normals $V$ and constructing the support matrix for $V$ and $\T$, $E$, we observe that $E$ is unique up to rescaling.
  However, as discussed earlier, there are always multiple complete sets of normals for $\T$, and so in that sense, $E$ is not unique.
  
  \subsection{Equivalences}
  In the big picture, we want to know two things.
  First, if we are given a cell $C$, how we can construct a finite set of polytopes $\T = \{T_1, \ldots, T_n\}$ such that $\vec 0 \in T(p) \iff p \in C$.
  Second, we want to know the opposite case: starting from $\T$, can we derive the cell $C \subseteq \simplex$ where $\vec 0 \in T(p)$ for all $p \in C$?
  We start with the latter, leaving the former for future work.
  
  We know that if we are given $\T$ and a complete set of normals $V$, we can describe $D = \{p \in \simplex : Ep \geq \vec 0\}$ as in Section~\ref{sec:start-polytope}.
  However, we do not necessarily have $D' := \{p \in A : Ep \geq \vec 0\} \subseteq \simplex$, and this poses some issues given the construction of the cell $C$ only assuming that we are in the affine span of the simplex, but not necessarily in the simplex.
  
  \begin{lemma}\label{lem:describe-D}
    Suppose we are given polytopes $\T = \{T_1, \ldots, T_n\}$ and a set of normals $V$ that is complete for $\T$.
 	Take $E = (e_{i}^y)$ where $e_{i}^y = \max_{x \in T_y} \inprod{v_i}{x}$, and $D^\T = \{p \in \simplex : Ep \geq \vec 0\}$.
 	
 	Then $\{p \in \simplex : \vec 0 \in \oplus_y p_y T_y\} = \{p \in \simplex: Ep \geq \vec 0\}$.\btw{on the whiteboard: ``stuff about $E$ and $D$.''}
  \end{lemma}
  \begin{proof}
    First, let us fix a distribution $p \in \simplex$.
    %We define $T(p) := \oplus_y p_y T_y$ to be the $p$-weighted Minkowski sum over the polytopes in $\T$.
    By Theorem~\ref{thm:support-minksum}, we have the support of the (weighted) Minkowski sum is the (weighted) sum of the support of each polytope, which we can re-write the weighted support as the product $Ep$.
    
    Each halfspace is bounded by the support function of the weighted polytope by construction of $E$, so the support of the weighted polytope defined by an inequality on $v_i$ can be described as $\inprod{v_i}{z} \leq \inprod{E_i}{p}$.
    Taking this for all $v_i$, we then have $\oplus_y p_y T_y = \{x \in \reals^d : Vx \leq Ep\}$.
    	
    Therefore, for fixed $p$, we have $\vec 0 \in \oplus_y p_y T_y \iff Ep \geq \vec 0$.
    As $p \in \simplex$ was arbitrary, we observe the stated set equality.
  \end{proof}
  
  \begin{proposition}\label{prop:relate-E-B}
    Suppose we are given polytopes $\T = \{T_1, \ldots, T_n\}$ and a set of normals $V$ that is complete for $\T$.
    Take $E = (e_{iy})$ where $e_{iy} = \max_{x \in T_y} \inprod{v_i}{x}$, and take $D = \{p \in \simplex : Ep \geq \vec 0\}$ and take the minimal rank $B \in \reals^{k \times n}$ such that we have the given cell $C = \{p \in \simplex : Bp \geq \vec 0\}$.
    
    Then $\{p \in \simplex : \vec 0 \in \oplus_y p_y T_y\} = C$	if and only if $C = D$.
    \btw{whiteboard: ``stuff about $E$ and $B$.''}
  \end{proposition}
  \begin{proof}
  	By Lemma~\ref{lem:describe-D}, we have $D = \{p \in \simplex : \vec 0 \in \oplus_y p_y T_y\}$, and the result follows.
  \end{proof}

  \begin{definition}
  	We say a vector $v$ is \emph{redundant} with respect to matrix $Y$ if we have $\{z: Yz \geq \vec b\} = \{z : [Y;v]z \geq \vec b^*\}$, where $b^* = [b;c]$ for some constant $c \in \reals$.
  \end{definition}

  \begin{proposition}\label{prop:relate-rows}
  	Suppose we have polytopes $\T = \{T_1, \ldots, T_n\}$ and a set of normals $V$ that is complete for $\T$.
  	Take $E = (e_{i}^y)$ where $e_{i}^y = \max_{x \in T_y} \inprod{v_i}{x}$, and take $D = \{p \in \simplex : Ep \geq \vec 0\}$ and take the minimal matrix $B$ such that a given cell $C = \{p \in \simplex : Bp \geq \vec 0\}$.
  	
  	Then $\{p \in \simplex : \vec 0 \in \oplus_y p_y T_y\} = C$	if and only the rows of $B$ appear in $E$ (possibly scaled) and every other row of $E$ is redundant with respect to $B$.
  	\btw{``replace prop 1 statement with `non-simplex rows of $B$ appear in $E$ (possible scaled) and every other row of $E$ is redundant with respect to $B$.'''}
  \end{proposition}
  \begin{proof}
	\begin{itemize}
		\item [$\implies$] First, assume $C = \{p \in \simplex: \vec 0 \in \oplus_y p_y T_y\}$.
		By Proposition~\ref{prop:relate-E-B}, we know that $C = D^\T := \{p \in \simplex : Ep \geq \vec 0\}$.
		Then we have $\{p \in \simplex : Bp \geq \vec 0\} = \{p \in \simplex : Ep \geq \vec 0\}$.
		As $B$ is minimal, we must have that every row of $B$ appears (possibly scaled) in $E$.
		Otherwise, we would contradict equality of the polytopes $C$ and $D$.
		Moreover, all rows in $E$ not in $B$ are redundant with respect to $B$ by equality of the polytopes.
		
		
		\item [$\impliedby$] Suppose that all rows of $B$ appear in $E$, and every other row of $E$ is redundant with respect to $B$.
		Then we have $D = \{p\in \simplex : Ep \geq \vec 0\} = \{p \in \simplex : Bp \geq \vec 0\} = C$.
				
		Then $D = C$, and by Proposition~\ref{prop:relate-E-B}, we have $C = \{p \in \simplex : \vec 0 \in \oplus_y p_y T_y\}$.
	\end{itemize}
  \end{proof}

  \section{Necessary and sufficient conditions for optimality}
  \subsection{Necessary conditions}
    With the above results, we actually realize that the matrix $B$ captures a necessary condition for optimality.
    As we showed earlier, we derive the support matrix $E$ such that $\vec 0 \in \oplus_y p_y T_y \iff Ep \geq \vec 0$ by substitution, and in Proposition~\ref{prop:relate-rows}, derive necessary and sufficient conditions on the matrix $B$ such that the optimality condition ($\vec 0$ in the weighted Minkowski sum) holds, and these conditions are in relation to the support matrix $E$.
    
    \begin{condition}[$\H$-condition]\label{cond:H-condition}
    	Given polytopes $\T$ and the cell $C = \{p \in \simplex : Bp \geq \vec 0\}$, there exist normals $v_1, \ldots, v_k$ such that $\max_{z \in T_y} \inprod{v_i}{z} = B_{iy}$ for all $i \in [k]$ and $y \in [n]$.
    \end{condition}
  

  \subsection{Sufficient conditions}
    Now in order to think about sufficient conditions for optimality, consider that convex polytopes all have $\V$-representations.
    Therefore, we can equivalently define the cell $C = \{p \in \simplex : Bp \geq \vec 0\} := \conv(\{p^1, \ldots, p^\ell\})$ for a finite set of distributions $p^j \in \simplex$, indexing with $j \in [\ell]$.
    Any distribution $p \in C$ can then be written as a convex combination of $p^j$ making up the convex hull.
%    , so $\vec 0$ being in those Minkowski sums implies that $\vec 0$ is also in the weighted Minkowski sum of any convex combination. \jessiet{Too hand wavy}
    \begin{condition}[$\V$-condition]\label{cond:V-condition}
    	Given $\T$ and $C = \conv(\{p^1, \ldots, p^\ell\})$, there exist $x_{jy} \in T_y$ such that $\oplus_y p^j_y x_{jy} = \vec 0$ for all $j \in [\ell]$.
    \end{condition}

  These two conditions together give us necessary and sufficient conditions for optimality.
  
  \begin{theorem}\label{thm:nasc-optimality-conditions}
  	Consider $B \in \reals^{k \times n}$ minimally describing $C = \{p \in \simplex : Bp \geq \vec 0\} = \conv(\{p^1, \ldots, p^\ell\})$ for a given convex cell $C$.
  	There exists $\T = \{T_1, \ldots, T_n\}$ such that $D^\T = C$ if and only if there exist normals $v_1, \ldots, v_k$, and points $x_{jy}$ such that the Conditions~\ref{cond:H-condition} and~\ref{cond:V-condition} hold.
  \end{theorem}
  \begin{proof}
  	$\implies$
    First, let there exist polytopes $\T$ so that $D^\T = C$.
    
    Suppose $V^*$ is complete for $\T$.
    Take $V = [v_1, \ldots, v_k]^T$ to be the normals corresponding to the nonredundant rows of $E$ with respect to $B$.
    Each entry of the support matrix $E$ is equal to $B$ for these nonredundant normals by Proposition~\ref{prop:relate-rows}.
    Therefore, Condition~\ref{cond:H-condition} is satisfied with the constructed set of normals $V$.    
%    Condition~\ref{cond:H-condition} is satisfied by the existence of a halfspace representation of the polytopes $T_y$, which requires a superset $V^*$ of $V = [v_1, \ldots, v_k]^T$ such that $V^*$ is complete for $\T$ and reduction of redundant rows from the support matrix $E$, which make up $B$.
    
    Moreover, as $\{ p : Ep \geq \vec 0\} \subseteq \{p : Bp \geq \vec 0\}$ (since rows of $E$ are either in $B$ or redundant), we apply Lemma~\ref{lem:describe-D} to observe that there is a witness $x_y \in T_y$ for all $p \in D^\T$ such that $\oplus_y p_y x_y = \vec 0$, as this is a corollary of our optimality condition.
    In particular, this is true for every distribution $p^j \in \{p^1, \ldots, p^\ell\}$, thus Condition~\ref{cond:V-condition} is satisfied.
    
    \bigskip
    $\impliedby$
    
    Consider the set of polytopes $\T$ given so that there are normals $v_i$ and witnesses $x_{jy}$ satisfying Conditions~\ref{cond:H-condition} and~\ref{cond:V-condition}.
    With these polytopes, we want to show that $D^\T = C$.    

    \bigskip
    $D^\T \subseteq C$.
    \bigskip
    
    We have $D^\T = \{p \in \simplex : \vec 0 \in \oplus_y p_y T_y\}$, and $C = \{p \in \simplex : Bp \geq \vec 0\}$.
    Take $p \not \in C$, and therefore $Bp < \vec 0$, and we will show $p \not \in D^\T$.
    For some row $\bar i$ of $B$, we must have $\inprod{B_{\bar i}}{p} = b < 0$.
    
    By Condition~\ref{cond:H-condition}, we can write polytopes as $T_y = \{x : \inprod{v_i}{x} \leq B_{iy} \; \forall i \in [k]\}$, since the $max$ being equal $B_{iy}$ implies an inequality on all $x \in T_y$.
    Moreover, by Theorem~\ref{thm:support-minksum}, we know the support of the weighted Minkowski sum is $Bp$, so the Minkowski sum can be written $\oplus_y p_y T_y = \{x : \inprod{v_i}{x} \leq (Bp)_i \; \forall i \in [k] \}$.
    As $(Bp)_{\bar i} < 0$, we have that $\vec 0$ is not in $T(p)$, so we have $p \not \in D^\T$.
    Thus, we have shown the contrapositive and have $D^\T \subseteq C$.
    
    \bigskip
    $C \subseteq D^\T$.
    \bigskip
    
    Take $p \in C$ so that $Bp \geq \vec 0$.
    Observe that $D^\T$ is a convex polytope, as there is a complete set of normals $V^*$ for $\T$,\jessiet{haven't shown this in this doc, but doesld be easy enough to argue.} and we know that the support matrix $E$ can be constructed so that $D = \{p \in \simplex : Ep \geq \vec 0\}$, which is convex as it is the intersection of a finite number of halfspaces, bounded by simplex constraints, and closed by the weak inequality.
%    it is bounded by simplex constraints, closed by definition, and convex by \jessie{ARGUE}.
    By Condition~\ref{cond:V-condition}, we have that $p^1, \ldots, p^\ell$ are all in $D^\T$, since such witnesses imply $\vec 0$ is in the weighted Minkowski sum.
    Moreover, as $C = \conv(\{p^1, \ldots, p^\ell\})$, we have $C \subseteq D^\T$.
    
  \end{proof}

  \section{Quadratic program}
  Let us generalize our Conditions~\ref{cond:H-condition} and~\ref{cond:V-condition} into a quadratic feasibility program without given polytopes $\T$.
  
  \begin{theorem}[Quadratic Program]
  	Given a cell $C = \{p \in \simplex : Bp \geq \vec 0\} = \conv(\{p^1, \ldots, p^\ell\}) \subseteq \simplex$, we aim to find normals $v_i$ and witnesses $x_{jy}$ such that
  	\begin{align*}
  	\inprod{v_i}{x_{jy}} \leq B_{iy} && \forall i \in [k], j \in [\ell], y \in [n]\\
  	\sum_y p_y^j x_{jy} = 0 && \forall j \in [\ell]\\
  	v_i, x_{jy} \in \reals^d && \forall i \in [k], j \in [\ell], y \in [n]
 	\end{align*}
  \end{theorem}

  \begin{theorem}
  	Suppose we are given a convex polytope $C \subseteq \simplex$.
  	There exist polytopes $\T$ such that $\{p \in \simplex : \vec 0 \in \oplus_y p_y T_y\} = C$ iff there is a feasible solution to the above quadratic program.
  \end{theorem}
  \begin{proof}
  	$\implies$
    First, suppose there were polytopes $\T$ such that $\{p \in \simplex : \vec 0 \in \oplus_y p_y T_y\} = C$.
        
    We know that since each $\{p^1, \ldots, p^\ell\} \in C$, and therefore $\vec 0 \in \oplus_y p_y T_y$ for each $p \in \{p^1, \ldots, p^\ell\}$, there is a set of witnesses $x_{jy}$ yielding that sum of zero, satisfying the second constraint.
    
    To see these witnesses satisfy the first constraint when their inner product is taken with some set of normals, consider each $v_i$ to be the normal of $\oplus_y T_y$ corresponding to the $i^{th}$ row of $B$.
    We know that every row of $B$ appears in the support matrix of the polytopes $E$ by Proposition~\ref{prop:relate-rows}, and therefore, we have $\inprod{v_i}{x_j} \leq B_{iy}$ for all $x_j \in T_y$, thus the first constraint holds.

    \bigskip
    $\impliedby$
    Suppose there was a feasible solution to the quadratic program above.
    We claim that $\T$ such that $T_y = \conv(\{x_{1y}, \ldots, x_{\ell y} \})$ \jessiet{Tried with these polytopes since they guarantee polytopes, but I can't see where $\{x : Vx \leq B_{;y}\}$ guarantees boundedness...} forms a valid set of polytopes so that $D^\T = C$, and therefore we satisfy our optimality condition by Proposition~\ref{prop:relate-E-B}.
    
    To see $C \subseteq D^\T$, take some $p \not \in D^\T$.
    We then have $\vec 0 \not \in \oplus_y p_y T_y$, which in turn implies that, for al\jessiet{this quantifier seems really important, but I'm not confident it's correct.} $j \in [\ell]$, there are no witnesses $\bar x_{jy}$ so that $\sum_y p^j_y \bar x_{jy} = 0$.
    If $p \in C$, then we contradict our feasibility constraints, so we conclude that we must have $p \not \in C$.
    Thus, we have $C \subseteq D^\T$.
        
    To see $D^\T \subseteq C$, take some $p \in D^\T$.
    We know that $p \in D^\T \iff \vec 0 \in \oplus_y p_y T_y$.  
    In particular, $\vec 0$ is in the Minkowski sum for all $p^j$, demonstrated by the witnesses $x_{jy}$ for all $y$.
    By construction, each polytope $T_y$ contains $x_{jy}$, and any $p\in C$ is some convex combination of the $p^j$s, so there is a witness that is a convex combination of the $x_{j}$ witnesses for each $y$ so that $\vec 0$ is in the Minkowski sum.
    Such a witness is in the convex hull of the other witnesses, as the new witness can simply be written as a reweighted sum.
    Therefore, we have $D^\T \subseteq C$
    
  \end{proof}

  \begin{corollary}
  	The minimum dimension $d$ such that the QP has a feasible solution is a lower bound on the embedding dimension of loss $\ell$ eliciting $\gamma$. 
  \end{corollary}
  
  
\end{document}
%%% Local Variables:
%%% mode: latex
%%% TeX-master: t
%%% End:
