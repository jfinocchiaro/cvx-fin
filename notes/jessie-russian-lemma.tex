\documentclass{article}
\usepackage[T1]{fontenc}
\usepackage{lmodern}

\PassOptionsToPackage{numbers, compress, sort}{natbib}

\usepackage{float}
\usepackage[utf8]{inputenc} % allow utf-8 input
\usepackage{hyperref}       % hyperlinks  %[implicit=false, bookmarks=false]
\usepackage{url}            % simple URL typesetting
\usepackage{booktabs}       % professional-quality tables
\usepackage{amsfonts}       % blackboard math symbols
\usepackage{nicefrac}       % compact symbols for 1/2, etc.
\usepackage{microtype}      % microtypography

\usepackage{mathtools, amsmath, amssymb, amsthm, graphicx, verbatim}
%\usepackage[thmmarks, thref, amsthm]{ntheorem}
\usepackage{color}
\definecolor{darkblue}{rgb}{0.0,0.0,0.2}
\hypersetup{colorlinks,breaklinks,
	linkcolor=darkblue,urlcolor=darkblue,
	anchorcolor=darkblue,citecolor=darkblue}
\usepackage{wrapfig}
\usepackage[font=small]{caption}
\usepackage{subcaption}
\usepackage[colorinlistoftodos,textsize=tiny]{todonotes} % need xargs for below
%\usepackage{accents}
\usepackage{bbm}
\usepackage{xspace}

\usetikzlibrary{calc}
\newcommand{\Comments}{1}
\newcommand{\mynote}[2]{\ifnum\Comments=1\textcolor{#1}{#2}\fi}
\newcommand{\mytodo}[2]{\ifnum\Comments=1%
	\todo[linecolor=#1!80!black,backgroundcolor=#1,bordercolor=#1!80!black]{#2}\fi}
\newcommand{\raf}[1]{\mynote{green}{[RF: #1]}}
\newcommand{\raft}[1]{\mytodo{green!20!white}{RF: #1}}
\newcommand{\jessie}[1]{\mynote{purple}{[JF: #1]}}
\newcommand{\jessiet}[1]{\mytodo{purple!20!white}{JF: #1}}
\newcommand{\proposedadd}[1]{\mynote{orange}{#1}}
\newcommand{\bo}[1]{\mynote{blue}{[Bo: #1]}}
\newcommand{\botodo}[1]{\mytodo{blue!20!white}{[Bo: #1]}}
\newcommand{\btw}[1]{\mytodo{gray!20!white}{BTW: #1}}%TURN OFF FOR NOW \mytodo{gray}{#1}}
\ifnum\Comments=1               % fix margins for todonotes
\setlength{\marginparwidth}{1in}
\fi


\newcommand{\reals}{\mathbb{R}}
\newcommand{\posreals}{\reals_{>0}}%{\reals_{++}}
\newcommand{\dom}{\mathrm{dom}}
\newcommand{\effdom}{\mathrm{effdom}}

\newcommand{\prop}[1]{\mathrm{prop}[#1]}
\newcommand{\eliccts}{\mathrm{elic}_\mathrm{cts}}
\newcommand{\eliccvx}{\mathrm{elic}_\mathrm{cvx}}
\newcommand{\elicpoly}{\mathrm{elic}_\mathrm{pcvx}}
\newcommand{\elicembed}{\mathrm{elic}_\mathrm{embed}}

\newcommand{\cell}{\mathrm{cell}}

\newcommand{\abstain}[1]{\mathrm{abstain}_{#1}}
\newcommand{\mode}{\mathrm{mode}}

\newcommand{\simplex}{\Delta_\Y}

% alphabetical order, by convention
\newcommand{\B}{\mathcal{B}}
\newcommand{\C}{\mathcal{C}}
\newcommand{\D}{\mathcal{D}}
\newcommand{\E}{\mathbb{E}}
\newcommand{\F}{\mathcal{F}}
\newcommand{\I}{\mathcal{I}}
\newcommand{\N}{\mathcal{N}}
\renewcommand{\P}{\mathcal{P}}
\newcommand{\R}{\mathcal{R}}
\newcommand{\Sc}{\mathcal{S}}
\newcommand{\U}{\mathcal{U}}
\newcommand{\V}{\mathcal{V}}
\newcommand{\X}{\mathcal{X}}
\newcommand{\Y}{\mathcal{Y}}


\newcommand{\risk}[1]{\underline{#1}}
\newcommand{\inprod}[2]{\langle #1, #2 \rangle}%\mathrm{int}(#1)}
\newcommand{\inter}[1]{\mathring{#1}}%\mathrm{int}(#1)}
%\newcommand{\expectedv}[3]{\overline{#1}(#2,#3)}
\newcommand{\expectedv}[3]{\E_{Y\sim{#3}} {#1}(#2,Y)}
\newcommand{\toto}{\rightrightarrows}
\newcommand{\strip}{\mathrm{strip}}
\newcommand{\trim}{\mathrm{trim}}
\newcommand{\fplc}{finite-piecewise-linear and convex\xspace} %xspace for use in text
\newcommand{\conv}{\mathrm{conv}}
\newcommand{\indopp}{\bar{\mathbbm{1}}}
\newcommand{\ones}{\mathbbm{1}}
\DeclarePairedDelimiter\ceil{\lceil}{\rceil}

\newcommand{\Ind}[1]{\ones\{#1\}}

\DeclareMathOperator*{\argmax}{arg\,max}
\DeclareMathOperator*{\argmin}{arg\,min}
\DeclareMathOperator*{\arginf}{arg\,inf}
\DeclareMathOperator*{\sgn}{sgn}

\newtheorem{theorem}{Theorem}
\newtheorem{lemma}{Lemma}
\newtheorem{proposition}{Proposition}
\newtheorem{corollary}{Corollary}
\newtheorem{conjecture}{Conjecture}

\newtheorem{definition}{Definition}
\newtheorem{assumption}{Assumption}

\title{Lemma 1}
\author{Jessie}

\begin{document}
\maketitle
\paragraph{Setting} 
$\effdom(L) = \reals^d$, and $E_q L(u,Y) < \infty$ for all $u \in \reals^d$ and $p \in \P$ \jessie{$y \in \Y$?}

$\effdom(L) = \reals^d$ and $L$ convex means that $\partial \E_q L(u,Y) \neq \emptyset$ for all $q \in \P$.

We define $\E_p \partial L(u,Y) = \{x \in \reals^d : x = \E_p V(Y), V(y) \in \partial L(u,y), V \text{ measurable}\}$.

Moreover, let $\V_{u} = \{V : \Y \to \reals^d : V \text{ measurable}, V(\omega) \in \partial L(u,\omega)\}$.

\begin{lemma}
	For any $z \in \reals^d$, there exists a $v \in \E_p \partial L(u,Y)$ such that $\inprod{z}{v} = \E_p L'(u,Y;z)$.
\end{lemma}
\begin{proof}
	\jessie{Meeting notes: - Life will be easier if you forgot/redefined expected value of the subgradient -Want $Q_z$ to be subgradients, and get the everywhere selection - Careful about pulling out expectation in bullet 2; when is inner product of RV = expectation of inner product? - Point out contradicting the second line would yield contradiction of subgradient inequality - Argue $Q_z$ is measurable - Go slowly through the definition of measurable}

Fix $z$.
Consider the set-valued function $Q_z : \Omega \to 2^{\reals^d}$ defined by $Q_z(\omega) := \{V \in \V_u: \inprod{z}{V(\omega)} = L'(u,\omega;z) \}$.
Now, we want to take a measurable function $V \in \V_u$ that is a selection of $Q_z$ satisfying the following conditions:
\begin{enumerate}
	\item $V \in \V_u$.
	\item $\E_p V \in \E_p \partial L(u,Y)$.
	\item $\inprod{z}{\E_p V} = \E_p L'(u,Y,z)$
\end{enumerate} 
If these are true, then we can take $v := \E_p V$ and observe the result.

\bigskip

	
1.  
If $Q_z$ is measurable and closed-valued, then by \jessie{Rockafellar VA 14.6}, we have a measurable selection $V \in \V_u$ such that $\inprod{z}{V(\omega)} = L'(u,\omega; z)$ for all $\omega \in \Omega$.
Thus, it just remains to show $Q_z$ is measurable and closed-valued.
First, we have $Q_z$ closed-valued by definition \jessie{Being lazy; will fill in later.}
%Now, we have $Q_z$ measurable if, for every open set $O \subset \reals^d$, the set $Q_z^{-1}(O) \in \B(\reals^d)$.
%\jessie{Rockafellar Wets VA Def 14.1}.
As measurable functions are closed under (countable) intersections, we have that $Q_z$ is the intersection of subgradients of $L$ and functions where the subgradient equals the directional derivative.
$V \in \V_u$ is measurable by definition.
Moreover, the function $\omega \mapsto \partial L(u,\omega)$ is measurable by \jessie{Rockafellar: Measurable Dependence, Corollary 4.6}, which also ensures $Q_z(\omega) \neq \emptyset$ for any $\omega \in \Omega$.
Now, we aim to show the set $\{c \in \reals^d: \inprod{z}{c} = L'(u,\omega;z)\}$ is measurable; as $L'$ is measurable and $\inprod{z}{\cdot}$ is continuous in its second argument, so the set is measurable. \jessie{Copying Russian argument}

As $Q_z$ is the intersection of these three sets, we conclude $Q_z$ is measurable and closed-valued, and therefore has a measurable selection $V \in \V_u$.


\bigskip 

2.  
This statement follows trivially as $V(\omega) \in \partial L(u,\omega)$ for all $\omega \in \Omega$.

\iffalse
First, we want to show $\E_p V \in \E_p \partial L(u,Y)$.
That is, $\int V(\omega) dp\omega \in \int \partial L(u,\omega) dp\omega = \{\E_p V(Y) : V \textrm{ measurable, }\, V(\omega) \in \partial L(u,\omega)\}$.
Since $V$ is measurable, it is just left to show $V(\omega) \in \partial L(u,\omega)$ $p$-almost surely.

By construction, for all $\omega \in \Omega$, we have $\inprod{V(\omega)}{z} = L'(u,\omega;z)$.
This implies 
\begin{align*}
\inprod{V(\omega)}{z} &= L'(u,\omega;z)\\
\inprod{V(\omega)}{z} &= \sup_{x \in \partial L(u,\omega)} \inprod{x}{z} \jessie{\textrm{Boyd and Vandenberghe slides}}\\
\implies \inprod{V(\omega)}{z} &\geq \inprod{x}{z} \; \forall x \in \partial L(u,\omega)\\
z := \omega - t \text{\footnotemark}\implies \inprod{V(\omega)}{\omega - t} &\geq \inprod{x}{\omega - t} \geq L(u,\omega) - L(u,t) \; \forall x \in \partial L(u,\omega)\\
\implies \inprod{V(\omega)}{\omega-t} &\geq L(u,\omega) - L(u,t)\,\forall t \in \reals^d\\
V(\omega) &\in \partial L(u,\omega) \\ 
\end{align*}
\footnotetext{can we do this?  since $z$ is fixed but $t$ is a variable}
\jessie{This is actually a bit stronger than we need; is that part of my issue?}
\fi

\bigskip

3.
We want to show $\inprod{z}{v} = \E_p L'(u,Y,z)$.
This follows from linearity of expectation and construction of $V$; to see this, consider
\begin{align*}
\inprod{z}{v} &= \inprod{z}{\E_p V} \\
 &= \E_p \inprod{z}{V} \jessie{Re-evaluate}\\
 &= \int \inprod{z}{V(\omega)} dp\omega \\
 &= \int L'(u,\omega; z) dp\omega \\
 &= \E_p L'(u,Y;z)
\end{align*}



\iffalse
	Suppose we are given $z\in \reals^d$.
	We have $\E_p \partial L(u,Y)$ compact (by convexity of $L$).
	Additionally, we know for all $\omega \in \Omega$, $\sup_{x \in \partial L'(u,\omega,z)} \inprod{x}{z} := \delta_{\partial L(u,\omega)}(z)$ is attained, and for any $v \in \delta_{\partial L(u,\omega)}(z)$, we particularly have $v \in \partial L'(u,\omega, z)$.
	
	
	
	\begin{itemize}
		\item $v \in \E_p \partial L(u,Y)$
		\item $\inprod{z}{v'} = \E_p L'(u,Y,z)$.
	\end{itemize}
\fi	
	 
\end{proof}


\end{document}