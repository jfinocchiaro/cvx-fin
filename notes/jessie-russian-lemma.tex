\documentclass{article}
\usepackage[T1]{fontenc}
\usepackage{lmodern}

\PassOptionsToPackage{numbers, compress, sort}{natbib}

\usepackage{float}
\usepackage[utf8]{inputenc} % allow utf-8 input
\usepackage{hyperref}       % hyperlinks  %[implicit=false, bookmarks=false]
\usepackage{url}            % simple URL typesetting
\usepackage{booktabs}       % professional-quality tables
\usepackage{amsfonts}       % blackboard math symbols
\usepackage{nicefrac}       % compact symbols for 1/2, etc.
\usepackage{microtype}      % microtypography

\usepackage{mathtools, amsmath, amssymb, amsthm, graphicx, verbatim}
%\usepackage[thmmarks, thref, amsthm]{ntheorem}
\usepackage{color}
\definecolor{darkblue}{rgb}{0.0,0.0,0.2}
\hypersetup{colorlinks,breaklinks,
	linkcolor=darkblue,urlcolor=darkblue,
	anchorcolor=darkblue,citecolor=darkblue}
\usepackage{wrapfig}
\usepackage[font=small]{caption}
\usepackage{subcaption}
\usepackage[colorinlistoftodos,textsize=tiny]{todonotes} % need xargs for below
%\usepackage{accents}
\usepackage{bbm}
\usepackage{xspace}

\usetikzlibrary{calc}
\newcommand{\Comments}{1}
\newcommand{\mynote}[2]{\ifnum\Comments=1\textcolor{#1}{#2}\fi}
\newcommand{\mytodo}[2]{\ifnum\Comments=1%
	\todo[linecolor=#1!80!black,backgroundcolor=#1,bordercolor=#1!80!black]{#2}\fi}
\newcommand{\raf}[1]{\mynote{green}{[RF: #1]}}
\newcommand{\raft}[1]{\mytodo{green!20!white}{RF: #1}}
\newcommand{\jessie}[1]{\mynote{purple}{[JF: #1]}}
\newcommand{\jessiet}[1]{\mytodo{purple!20!white}{JF: #1}}
\newcommand{\proposedadd}[1]{\mynote{orange}{#1}}
\newcommand{\bo}[1]{\mynote{blue}{[Bo: #1]}}
\newcommand{\botodo}[1]{\mytodo{blue!20!white}{[Bo: #1]}}
\newcommand{\btw}[1]{\mytodo{gray!20!white}{BTW: #1}}%TURN OFF FOR NOW \mytodo{gray}{#1}}
\ifnum\Comments=1               % fix margins for todonotes
\setlength{\marginparwidth}{1in}
\fi


\newcommand{\reals}{\mathbb{R}}
\newcommand{\posreals}{\reals_{>0}}%{\reals_{++}}
\newcommand{\dom}{\mathrm{dom}}
\newcommand{\effdom}{\mathrm{effdom}}

\newcommand{\prop}[1]{\mathrm{prop}[#1]}
\newcommand{\eliccts}{\mathrm{elic}_\mathrm{cts}}
\newcommand{\eliccvx}{\mathrm{elic}_\mathrm{cvx}}
\newcommand{\elicpoly}{\mathrm{elic}_\mathrm{pcvx}}
\newcommand{\elicembed}{\mathrm{elic}_\mathrm{embed}}

\newcommand{\cell}{\mathrm{cell}}

\newcommand{\abstain}[1]{\mathrm{abstain}_{#1}}
\newcommand{\mode}{\mathrm{mode}}

\newcommand{\simplex}{\Delta_\Y}

% alphabetical order, by convention
\newcommand{\B}{\mathcal{B}}
\newcommand{\C}{\mathcal{C}}
\newcommand{\D}{\mathcal{D}}
\newcommand{\E}{\mathbb{E}}
\newcommand{\F}{\mathcal{F}}
\newcommand{\I}{\mathcal{I}}
\newcommand{\N}{\mathcal{N}}
\renewcommand{\P}{\mathcal{P}}
\newcommand{\R}{\mathcal{R}}
\newcommand{\Sc}{\mathcal{S}}
\newcommand{\U}{\mathcal{U}}
\newcommand{\V}{\mathcal{V}}
\newcommand{\X}{\mathcal{X}}
\newcommand{\Y}{\mathcal{Y}}
\newcommand{\Z}{\mathcal{Z}}

\newcommand{\risk}[1]{\underline{#1}}
\newcommand{\inprod}[2]{\langle #1, #2 \rangle}%\mathrm{int}(#1)}
\newcommand{\inter}[1]{\mathring{#1}}%\mathrm{int}(#1)}
%\newcommand{\expectedv}[3]{\overline{#1}(#2,#3)}
\newcommand{\expectedv}[3]{\E_{Y\sim{#3}} {#1}(#2,Y)}
\newcommand{\toto}{\rightrightarrows}
\newcommand{\strip}{\mathrm{strip}}
\newcommand{\trim}{\mathrm{trim}}
\newcommand{\fplc}{finite-piecewise-linear and convex\xspace} %xspace for use in text
\newcommand{\conv}{\mathrm{conv}}
\newcommand{\indopp}{\bar{\mathbbm{1}}}
\newcommand{\ones}{\mathbbm{1}}
\DeclarePairedDelimiter\ceil{\lceil}{\rceil}

\newcommand{\Ind}[1]{\ones\{#1\}}

\DeclareMathOperator*{\argmax}{arg\,max}
\DeclareMathOperator*{\argmin}{arg\,min}
\DeclareMathOperator*{\arginf}{arg\,inf}
\DeclareMathOperator*{\sgn}{sgn}

\newtheorem{theorem}{Theorem}
\newtheorem{lemma}{Lemma}
\newtheorem{proposition}{Proposition}
\newtheorem{corollary}{Corollary}
\newtheorem{conjecture}{Conjecture}

\newtheorem{definition}{Definition}
\newtheorem{assumption}{Assumption}

\title{Lemma 1}
\author{Jessie}

\begin{document}
\maketitle
\paragraph{Setting} 
$\effdom(L) = \reals^d$, and $\E_p L(u,Y) < \infty$ for all $u \in \reals^d$ and $p \in \P$ \jessie{$y \in \Y$?}

$\effdom(L) = \reals^d$ and $L$ convex means that $\partial \E_q L(u,Y) \neq \emptyset$ for all $q \in \P$.

We define $\E_p \partial L(u,Y) = \{\E_p V: V(y) \in \partial L(u,y) p\text{-almost surely}, V \text{ measurable}\}$.
Moreover, let $\V_{u} = \{V : \Y \to \reals^d : V \text{ measurable}, V(\omega) \in \partial L(u,\omega) \forall \omega \in \Omega\}$.
Observe the set $\Z_u(p) := \{\E_p V(Y) \mid V \in \V_u\} \subseteq \E_p \partial L(u,Y)$.
Consider the measurable space $(\Omega, B)$. 

\begin{definition}[Measurable (multifunction)]
	%Let $\B(\reals^d)$ be the Borel $\sigma$-algebra of $\reals^d$.
	Suppose $(\Omega, B)$ is a measurable space and $f:\Omega \toto \reals^d$ a multifunction taking values in the set of nonempty closed subsets of $\reals^d$.
	We say $f$ is weakly measurable if, for every open $U \subseteq \reals^d$, we have
	\begin{equation*}
	\{\omega : f(\omega) \cap U \neq \emptyset\} \in B~.~
	\end{equation*}
\end{definition}

\begin{lemma}
	For any $z \in \reals^d$, there exists a $v \in \E_p \partial L(u,Y)$ such that $\inprod{z}{v} = \E_p L'(u,Y;z)$.
\end{lemma}
\begin{proof}

Fix $z$.
Consider the set-valued function $Q_z : \Omega \to 2^{\reals^d}$ defined by $Q_z(\omega) := \{x \in \partial L(u,\omega) : \inprod{z}{x} = L'(u,\omega;z)\}$.
Now, we want to take a measurable selection $V$ of $Q_z$ satisfying the following conditions:
\begin{enumerate}
	\item $V \in \V_u$.
	\item $\E_p V \in \E_p \partial L(u,Y)$.
	\item $\inprod{z}{\E_p V} = \E_p L'(u,Y,z)$
\end{enumerate} 
If these are true, then we can take $v := \E_p V$ and observe the result.

\bigskip

	
1.  

\bigskip
\hrulefill
\bigskip

\jessie{Direct; Aliprentis and Border Theorem 18.19}\jessiet{Infinite Dimensional Analysis Hitch-hiker's guide}
\begin{theorem}[\cite{aliprantis2006infinite} Theorem 18.19]
Let $X$\jessie{for us, $\reals^d$} be a separable, metrizable space and $(\Omega, B)$ a measurable space.
Let $\varphi : \Omega \toto X$ be a weakly measurable correspondence (multi function) with nonempty compact values\jessiet{Need $L$ finite for all $u, \omega$ for this.  Not just in $p$.}, and suppose $f:\Omega \times X \to \reals$ is a Caratheodory function.
Define the value function $m : \Omega \to \reals$ by $m(\omega) = \max_{x \in \varphi(\omega)} f(\omega,x)$, and the multifunction $\mu : \Omega \toto X$ of maximizers by $\mu(\omega) = \{x \in \varphi(\omega) : f(\omega, x) = m(\omega)\}$.

Then $m$ is measurable, $\mu$ has nonempty and compact values, and is measurable and admits a measurable selection.
\end{theorem}

Consider the following functions:
\begin{align*}
f(\omega, x) &= \inprod{z}{x}\\
\varphi(\omega) &= \partial L(u,\omega)\\
m(\omega) &= \max_{x \in \varphi(\omega)} f(\omega,x)\\
 &= L'(u,\omega;z)\jessie{\textrm{Double check for edge conditions}}\\
\mu(\omega) &= \{x \in \varphi(\omega) \mid f(\omega,x) = m(s)\}\\
&= \{x \in \partial L(u,\omega) \mid \inprod{z}{x} = L'(u,\omega;z)\}\\
&= Q_z
\end{align*}

If we can apply\cite[Theorem 18.19]{aliprantis2006infinite}, we have $\mu = Q_z$ nonempty with compact values, measurable, and admits a measurable selection.

To see this, first observe $\reals^d$ is a separable, metrizable space.
We have $\varphi : \omega \mapsto \partial L(u,\omega)$ weakly measurable by~\cite[Corollary 4.6]{ROCKAFELLAR1969measurable}, and has nonempty compact values by properties of the subdifferential \jessie{since $L(u,\omega)$ is finite for all $u,\omega$}.
It is then left to show $f$ is a Caratheodory function, meaning it is measurable in $\omega$, and continuous in $\reals^d$.
We have measurability as it is constant in $\omega$, and linear (and therefore continuous) in $\reals^d$.
Therefore, we can apply this function to admit a measurable selection of $Q_z$.

\bigskip
\hrulefill
\bigskip
\iffalse
\jessie{Complete measure}

If $Q_z$ is measurable and closed-valued, then by~\cite[Theorem III.6]{castaing2006convex} \jessiet{Also Rockafellar VA 14.6}, we have a measurable selection $V \in \V_u$ such that $\inprod{z}{V(\omega)} = L'(u,\omega; z)$ for all $\omega \in \Omega$.

Thus, we show $Q_z$ is measurable and closed-valued.
We first have $Q_z$ is closed-valued since the directional derivative is the argmax of a linear function over a compact set (the subgradient), which is also closed.


If $B$ is a complete $\sigma$-algebra with respect to $p$, then \jessiet{So we get equivalence of measurability of the function and the graph} $Q_z$ is $B$-measurable if and only if $gr(Q_z)$ is $B \otimes \B(\reals^d)$-measurable~\cite[Chapter III]{castaing2006convex}. 
Now, consider $gr(Q_z) = gr(Q) \cap gr(R)$, where $Q: \omega \mapsto \partial L(u,\omega)$ and $R : \omega \mapsto \{x \mid \inprod{z}{x} = L'(u,\omega;z)\}$.
As $Q$ is $B$-measurable~\cite[Corollary 4.6]{ROCKAFELLAR1969measurable}\jessiet{Also Borel by that same statement}, we have $gr(Q)$ is $B \otimes \B(\reals^d)$-measurable by~\cite[Chapter III]{castaing2006convex}.
Moreover, $gr(R)$ is measurable as it is the graph of a linear multifunction, hence measurable.
Now we can take $gr(Q_z) = gr(Q) \cap gr(R)$, which is measurable by closure under intersections \jessiet{Hand waving}, so $gr(Q_z)$ is $B \otimes \B(\reals^d)$-measurable, and hence $Q_z$ is $B$-measurable~\cite[Chapter III]{castaing2006convex}.
\fi

%=======================================

\iffalse
\jessie{Bounded $R'$}\jessiet{Note from Mar 4 meeting: Look at strong measurability or see if there's a result of closure under intersections}

If $Q_z$ is measurable and closed-valued, then by~\cite[Theorem III.6]{castaing2006convex} \jessiet{Also Rockafellar VA 14.6}, we have a measurable selection $V \in \V_u$ such that $\inprod{z}{V(\omega)} = L'(u,\omega; z)$ for all $\omega \in \Omega$.

Thus, we show $Q_z$ is measurable and closed-valued.
We first have $Q_z$ is closed-valued since the directional derivative is the argmax of a linear function over a compact set (the subgradient), which is also closed.


By~\cite[Prop iii.4]{castaing2006convex}, if we can simply show $Q_z$ intersection of (weakly) measurable, compact functions, then $Q_z$ itself is measurable.
Observe that we have $Q_z(\omega) = Q: \omega \mapsto \partial L(u,\omega) \cap R' : \omega \mapsto \{x \in \reals^d : \inprod{z}{x} = L'(u,\omega;z)\}$.
However, $R'$ is an affine hyperplane, and therefore not compact, so we need to be a bit more nuanced in how we reason about $Q_z$.

Instead of $R'$, consider $\hat R : \omega \mapsto \{x \mid \inprod{z}{x} = L'(u,\omega;z)$ and $\|x\| \leq c(\omega)\}$, where $c(\omega)$ is some constant function of $\omega$.
In particular, take $c(\omega) = \sup_{x' \in \partial L(u,\omega) \mid \inprod{z}{x'} = L'(u,\omega;z)} \|x'\|$.
WTS $c$ is bounded and measurable.
$c(\omega)$ is first bounded by compactness of $\partial L(u,\omega)$, and therefore the norms of the supremum will be bounded. \jessiet{I don't have a proof of this, but it seems true}



Moreover, the function $Q$ is measurable by~\cite[Corollary 4.6]{ROCKAFELLAR1969measurable}.
Moreover, $R'$ is measurable as $L'$ is measurable and $\inprod{z}{\cdot}$ is continuous in its second argument, hence Borel.

As $Q_z$ is the intersection of these two functions, we conclude $Q_z$ is measurable by~\cite[Prop iii.4]{castaing2006convex} and closed-valued, and therefore has a measurable selection $V \in \V_u$ such that $\inprod{z}{V(\omega)} = L'(u,\omega;z)$ for all $\omega \in \Omega$.
\fi

\bigskip

2.  
This statement follows trivially as $\E_p V \in \Z_p(u) \subseteq \E_p \partial L(u,Y)$. 

\bigskip

3.
We want to show $\inprod{z}{v} = \E_p L'(u,Y,z)$.
This follows from linearity of expectation and construction of $V$; to see this, consider
\begin{align*}
\inprod{z}{v} &= \inprod{z}{\E_p V} \\
 &= \E_p \inprod{z}{V}
 \end{align*}
 The expectation can be pulled out of the inner product because $V$ is finite for all $\omega \in \Omega$ as it is contained in $\partial L(u,\omega)$, which is compact as $L$ is finite and convex for all $u \in \reals^d$ and $\omega \in \Omega$.
 \begin{align*}
 \E_p \inprod{z}{V} &= \int \inprod{z}{V(\omega)} dp\omega \\
 &= \int L'(u,\omega; z) dp\omega \\
 &= \E_p L'(u,Y;z)
\end{align*}



\iffalse
	Suppose we are given $z\in \reals^d$.
	We have $\E_p \partial L(u,Y)$ compact (by convexity of $L$).
	Additionally, we know for all $\omega \in \Omega$, $\sup_{x \in \partial L'(u,\omega,z)} \inprod{x}{z} := \delta_{\partial L(u,\omega)}(z)$ is attained, and for any $v \in \delta_{\partial L(u,\omega)}(z)$, we particularly have $v \in \partial L'(u,\omega, z)$.
	
	
	
	\begin{itemize}
		\item $v \in \E_p \partial L(u,Y)$
		\item $\inprod{z}{v'} = \E_p L'(u,Y,z)$.
	\end{itemize}
\fi	
	 
\end{proof}

Recall the set $\Z_u(p) = \{\E_p V \mid V \in \V_u\}$.
\begin{lemma}
	The set $\Z_u(p) \subseteq \reals^d$ is convex and closed.
\end{lemma}
\begin{proof}
	Recall that $V \in \V_u$ implies that $V(\omega) \in \partial L(u,\omega)$ for all $\omega \in \Omega$ and $V$ measurable.
	
	\emph{Convex: }
	Consider $V, V' \in \V_u$.
	For all $\lambda \in [0,1]$, we have $V_\lambda = \lambda V + (1-\lambda) V' \in \partial L(u,\omega)$ for all $\omega \in \Omega$ by convexity of the subgradients.
	As this is true for all $\omega \in \Omega$, we also have $\E_p V_\lambda \in \Z_u(p)$ if $V_\lambda \in \V_u$.
	We can see $V_\lambda$ is measurable as it is a convex combination of two measurable functions, and therefore in $\V_u$.
	
	\emph{Closed: } A set $S \subseteq X$ (with $(X,\tau)$ a first-countable topological space) is closed if and only if $S = cl(S) := \{v \mid \exists \{v_m\} \to v, v_m \in S \forall m\}$.
	
	We have $\Z_u(p) \subseteq cl(\Z_u(p))$ trivially, so it remains to show $\Z_u(p) \supseteq cl(\Z_u(p))$.
	
	Take $v \in cl(\Z_u(p))$.
	Since $\Z_u(p)$ is first countable, there exists a sequence $\{v_m\} \to v$ with $v_m \in \Z_u(p)$ for all $m$.
	This means for each $m$, there is a function $V_m \in \V_u$ such that $v_m = \E_p V_m$.

	\jessie{Want to claim $V_m \to V$ (pointwise? though I don't think this is true.  Maybe in distribution) and $\E_p V = v$}	
	%We claim $V_m \to V$
	%Take $\{V_m\} \to V$ pointwise\jessiet{Can we do that?}, and observe that $V$ must be measurable as each $V_m$ is measurable. \jessie{CITE}.
	
\end{proof}

\section{More direct proof}
Consider the sets
\begin{align*}
A &:= \partial \E_p L(u,Y)\\
B &:= \{\E_p V(Y) \mid V\textrm{ measurable, } V(y) \in \partial L(u,y) \, p-a.s.\}\\
Z &:= \{\E_p V(Y) \mid V\textrm{ measurable, } V(y) \in \partial L(u,y) \, \forall y \in \Y \} 
\end{align*}

We want to show $A = Z$; the Russian paper shows $A =B$.
First, we have $A \supseteq B$ by definition of subdifferential, and $B \supseteq Z$ trivially, so $A \supseteq Z$.
It is left to show $A \subseteq Z$.

As $A$ is a subdifferential of a finite, convex loss defined on $\reals^d$, we have $A$ convex and compact.

In particular, if \jessie{CONDITION} in $A$ are described by elements of $Z$, then anything in $A$ can be described by an element of $Z$ by its convexity and closedness.

Hypothesis: \jessie{CONDITION} = exposed faces in the relative boundary of $A$.

\section{The story of Lemma 2: a subset of saga.}

\jessie{This isn't quite typechecking for me, with the function $P$.  In their paper, $P$ should be a function of a subset of $\Y$, but for monotonicity, we want it to be the CDF?}

If we want to show $A := \{\E_p V \mid V(y) \in \partial L(u,y) \, p-a.s.\}$ is closed, and do this by showing $A = cl(A)$.  Thankfully, $A \subseteq cl(A)$ follows trivially, so it is left to show $A \supseteq cl(A)$.

Consider a sequence $\{v_m\} \to v$ such that $v_m \in A$ for all $m$.  We have (by part 1 of the lemma) measurable selections $V_m$ such that $v_m = \E_p V_m$ and the $v_m$ has bounded norm.
If we have $v \in A$, then we have $A \supseteq cl(A)$, and therefore $A$ closed.

Now define functions $P_m : 2^\Y \to \reals^d$ so that $P_m(E) = \int_E V_m(y) dpy$.  
We can use Helly's Theorem (Theorem~\ref{thm:helly}?) to claim that $P_m$ converges pointwise to some function $P$ of bounded variation.\jessie{We need montonicity of the functions $P_m$, but is this definition $A \subseteq B \implies f(A) \leq f(B)$? We don't have that.}
We have uniform boundedness from their first past of the result, and monotonicity from the nonnegativity of probability densities. \jessie{Nope}%from each $P_m$ being a CDF.

\begin{theorem}[Helly's Theorem]\label{thm:helly}
	A uniformly bounded sequence of monotone real functions admits a (pointwise) convergent subsequence
\end{theorem}

Therefore, for any $z \in \reals^d$, by the Dominated Convergence Theorem \jessie{or Helly-Bray \url{https://en.wikipedia.org/wiki/Helly\%E2\%80\%93Bray_theorem}}, we have 
\begin{align*}
\int \inprod{z}{P(y)} dy &= \lim_{m \to \infty} \int \inprod{z}{P_m(y)} dy
\end{align*}

In particular, we can now take the selection $V(y) = P(y)$ as the density of $P$.
Define $P_c$ as the singular part of $P$; that is, the region where $P$ is not defined, concentrated on a set of measure zero.

If $P_c$ is empty (i.e., $P$ is defined for all $y \in \Y$)\jessie{I think this is what we need to claim for an everywhere selection}, then for any $z \in \reals^d$, we have
\begin{align*}
\int \inprod{z}{P(y)} dy &= \lim_{m \to \infty} \int \inprod{z}{P_m(y)} dy \leq \int L'(u,y;z) dy
\end{align*}

From this, it follows that $\inprod{z}{V(y)} \leq L'(u,y;z)$ for all $y \in \Y$, and therefore $V(y) \in \partial L(u,y)$ for all $y \in \Y$.

This yields $P(\Y) = \int_\Y V(y) dpy \in \int_\Y \partial L(u,y) dpy$.

Therefore, we have 
\begin{align*}
\lim_{m\to \infty} V_m &= V \\
\implies \lim_{m \to \infty} \E_p V_m &= \E_p V \in A\\
\lim_{m \to \infty} v_m &= v \in A
\end{align*}

\section{Zooming out: complete measures}
We are working on the measure space $(\Omega, B)$ with probability measure $p$.

Ioffe and Tikhomirov define expectation of a multifunction as 
\begin{align*}
A = \E_p \partial L(u,Y) &= \{\E_p V \mid V(y) \in \partial L(u,y) p-a.s. \}
\end{align*}
%written 17 April. commented out 18 April with new verison
%Let's think about the same set as before, but take the closure first.
%\begin{align*}
%Z &= cl\{\E_p V \mid V \in \V_u\}
%\end{align*}
%Let's show $Z$ yields a selection $V$ such that (a) $V \in \V_u$, (b) $\E_p V = \vec 0$.
%Applying the Aliprantis Border result, we have (a) with more info than we need, but know that $V$ is measurable.  
%If $V$ is in the boundary of $Z$, consider that there is a sequence of measurable functions in $Z$ that converges to $V$ pointwise almost everywhere, and therefore $V$ is measurable as well. \jessie{check claim??}
%
%Moreover, we can apply the DCT to show that $\E_p V = 0$, as it is the limit of a sequence of measurable functions in $Z$ s.t. $E_p V_m = \vec 0$.?
%\jessie{Need to show $\vec 0 \in \E_p \partial L(u,Y) = cl\{\E_p V \mid V(y) \in \partial L(u,y) p-a.s. \} \implies \vec 0 \in Z$}.
%If $0 \not \in Z$, then there are some sets $E_1, E_2 \subseteq \Y$ such that $V(y) \neq \partial L(u,y)$ for all $y \in E_1$ and $E_2$.
%However we claim $p(E_1) = 0$ and $p(y) = 0$ for all $y \in E_2$.
%Therefore, 
%
%
%
%We then have $V$ measurable and $q \in \ker_\P V \implies \E_q V = \vec 0 \implies \vec 0 \in \E_p \partial L(u,Y) = \partial \E_p L(u,Y) \implies u \in \Gamma(q) \implies q \in \Gamma_u$.
%Thus the proof is complete.

%new 18 Apr 21
Consider the set $Z = \{\E_p V \mid V \in \V_u\}$.
We claim that $Z = A$.
$Z \subseteq A$ is trivial, so it is left to show $A \subseteq Z$.

Consider $v \in A$.
There is some measurable function $V$ such that $\E_p V = v$ such that $V(y) \in \partial L(u,y)$ almost surely in $p$; we want to modify $V$ into $V' \in\Z$ so that $\E_p V = \E_p V'$, and $V'$ is still measurable.

Now consider $E = \{y \in \Y \mid V(y) \in \partial L(u,y)\}$.
We have $p(E^c) = 0$ and $E \in B$ by definition of $p$-almost surely.
Therefore $E^c \in B$.
Moreover, since $p$ is a complete measure, every $N \subseteq E^c\in B$.

Therefore, we can take $V'$ to be any function that matches $V$ on $y \in E$, and takes any $V'(y) \in \partial L(u,y)$ (which is nonempty and compact by minnability and finiteness of $L$, respectively) elsewhere. \jessiet{Not sure if we want something explicit, but wasn't sure how to define it to be unique.  Mostly was thinking about minning the norm, but still needs more specificity.}
As $V$ is measurable and $p$ is a complete measure, we have $V'$ measurable.
Consider that for all $x \in \reals^d$, we have $V'^{-1}(x) = V^{-1}(x) \cup \{y \in E^c : V'(y) = x\}$.
As $V$ is measurable, we have $V^{-1}(x) \in B$ and since $p$ is a complete measure, we know that $\{y \in E^c : V'(y) = x\} \in B$ as well.
Since sigma algebras are closed under finite unions, we have $V'^{-1}(x) \in B$ for all $x$, and therefore $V'$ is measurable.

%, for all $y \in E^c$, we can set $V'(y) = 0$ and $V'(y) = V(y)$ for $y \in E$. \jessie{Need to have $V'(y) \in \partial L(u,y)$}
%We have $V'$ measurable as $V'^{-1}(\{0\}) = E^c \cup \{x \mid V(x) = 0\}$, which is the union of elements of the sigma algebra, and therefore is also in the sigma algebra.  
Moreover, we have
\begin{align*}
v &= \int_\Y V(y) dpy \\
&= \int_{E} V(y) dpy + \int_{E^c} V(y) dpy\\
&= \int_{E} V(y) dpy\\
&= \int_{E} V'(y) dpy\\
&= \int_{E} V'(y) dpy + \int_{E^c} V'(y) dpy \\
&= \int_\Y V'(y) dpy~,~
\end{align*}
and thus $v = \E_p V = \E_p V'$, so $v \in Z$.
Therefore, $v \in Z$, so $A \subseteq Z$.

\bibliographystyle{ieeetr}
\bibliography{refs-selection}

\end{document}