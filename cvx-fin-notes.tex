\documentclass[11pt]{article}
\usepackage[utf8]{inputenc}
\usepackage{mathtools, amsmath, amsthm, amssymb, graphicx, verbatim}
%\usepackage[thmmarks, thref, amsthm]{ntheorem}
\usepackage{color}
\usepackage{wrapfig}
\usepackage{subcaption}
\usepackage[colorinlistoftodos,textsize=tiny]{todonotes} % need xargs for below
%\usepackage{accents}
\usepackage{bbm}
\usepackage{xspace}

\newcommand{\Comments}{1}
\newcommand{\mynote}[2]{\ifnum\Comments=1\textcolor{#1}{#2}\fi}
\newcommand{\mytodo}[2]{\ifnum\Comments=1%
  \todo[linecolor=#1!80!black,backgroundcolor=#1,bordercolor=#1!80!black]{#2}\fi}
\newcommand{\raf}[1]{\mynote{green}{[RF: #1]}}
\newcommand{\raft}[1]{\mytodo{green!20!white}{RF: #1}}
\newcommand{\jessie}[1]{\mynote{purple}{[JF: #1]}}
\newcommand{\jessiet}[1]{\mytodo{purple!20!white}{JF: #1}}
\ifnum\Comments=1               % fix margins for todonotes
  \setlength{\marginparwidth}{1in}
\fi


\newcommand{\reals}{\mathbb{R}}
\newcommand{\posreals}{\reals_{>0}}%{\reals_{++}}

% alphabetical order, by convention
\newcommand{\E}{\mathbb{E}}
\newcommand{\F}{\mathcal{F}}
\newcommand{\I}{\mathcal{I}}
\renewcommand{\P}{\mathcal{P}}
\newcommand{\R}{\mathcal{R}}
\newcommand{\X}{\mathcal{X}}
\newcommand{\Y}{\mathcal{Y}}
\renewcommand{\P}{\mathcal{P}}


\newcommand{\inter}[1]{\mathring{#1}}%\mathrm{int}(#1)}
%\newcommand{\expectedv}[3]{\overline{#1}(#2,#3)}
\newcommand{\expectedv}[3]{\E_{Y\sim{#3}} {#1}(#2,Y)}
\newcommand{\toto}{\rightrightarrows}
\newcommand{\trim}{\mathrm{trim}}
\newcommand{\fplc}{finite-piecewise-linear and convex\xspace} %xspace for use in text

\DeclareMathOperator*{\argmax}{arg\,max}
\DeclareMathOperator*{\argmin}{arg\,min}
\DeclareMathOperator*{\arginf}{arg\,inf}
\DeclareMathOperator*{\sgn}{sgn}

\newtheorem{theorem}{Theorem}
\newtheorem{lemma}{Lemma}
\newtheorem{proposition}{Proposition}
\newtheorem{definition}{Definition}
\newtheorem{corollary}{Corollary}


\title{Finite Property Convex Elicitation Notes}
\author{Raf + \ldots}

\begin{document}

\maketitle

\section{Notation and Definitions}

Let $\Y$ be a finite outcome space, with $n:=|\Y|$, and $\Delta(\Y)$ be the set of probability distributions over $\Y$.
A \emph{property} is a set-valued function $\Gamma: \P \toto \R$, which assigns a subset of the possible reports $\R$ to each probability distribution in a convex set $\P \subseteq \Delta(\Y)$.
(Here $\Gamma: \P \toto \R$ is shorthand for $\Gamma: \P \to 2^\R$, the power set of $\R$.)
We will often consider $\P \subset \reals^n$, meaning we identify each distribution with the corresponding vector of probabilities, for some fixed ordering of the outcomes $\{y_1,\ldots,y_n\} = \Y$.

A propertiy is \emph{elicitable} if it can be expressed as the minimizer of expected loss for some loss function.
When this loss function is convex, we say the property is \emph{convex elicitable}.
\begin{definition}
  \label{def:elicits}
  A loss function $L: \R \times \Y \to \reals$ \emph{elicits} a property $\Gamma$ if for all $p \in \P$,
  \begin{equation}
    \label{eq:elicits}
    \Gamma(p) = \argmin_{r\in\R} \E_{Y \sim p} L(r, Y)~.
  \end{equation}
  In this case, we say $\Gamma$ is \emph{elicitable}.
  We define $\Gamma_L:p\mapsto \argmin_{r\in\R} \E_{Y \sim p} L(r, Y)$ to be the unique property elicited by $L$.
  If $L(\cdot,y)$ is convex for every $y \in \Y$, we say $\Gamma$ is \emph{convex elicitable}.
\end{definition}

In what follows, we will often write $L(r) := (L(r,y_1),\ldots,L(r,y_n)) \in \reals^\Y \equiv \reals^n$, so that $L : \R \to \reals^\Y$.
The elicitation condition in eq.~\eqref{eq:elicits} then becomes, for all $p\in\P\subset \reals^n$,
\begin{equation}
  \label{eq:elicits-vectorized}
  \Gamma(p) = \argmin_{r\in\R} \; p \cdot L(r)~.
\end{equation}

We define the \emph{level set} of a property at report value $r\in\R$ to be the set $\Gamma_r := \{p\in\P : r \in \Gamma(p)\}$.

\begin{definition}
  Let $r,r'\in\R$.
  We say report $r$ is \emph{dominated by $r'$} if $\Gamma_r \subsetneq \Gamma_{r'}$, \emph{equivalent to $r'$} if $\Gamma_r = \Gamma_{r'}$, and \emph{weakly dominated by $r'$} if it is either dominated or equivalent.
  A report is \emph{redundant} if it is dominated by or equivalent to another report.
  A property $\Gamma$ is \emph{non-redundant} if there are no redundant reports.
\end{definition}

\begin{definition}
A property is \emph{finite} if $\R$ is finite.
\jessiet{If $\R$ is finite or $\Y$?  The set of minimizers will be finite, but don't we want $\R$ to be an interval/convex?}\raft{As written.}
\end{definition}

\begin{definition}
  A finite property is \emph{orderable} if there is an ordering $\R = \{r_1,\ldots,r_k\}$ such that the intersection of consecutive level sets forms a hyperplane.
  \raf{Still not sure what the right definition is for non-finite $\R$... and it would be nice to have a more natural version for finite $\R$!}
  \raf{Also, once we figure out what the 2d version is, we should call this $d$-orientable, as in manifolds, so orderable would become $1$-orientable.}
  In other words, for all $i \in \{1,\ldots,k-1\}$ there is some $a_i\in\reals^n$ such that $\Gamma_{r_i}\cap\Gamma_{r_{i+1}} = \{p\in\P : a_i\cdot p = 0\}$.
\end{definition}

\begin{definition}
  A property is \emph{degenerate} if $\Gamma(p) = \emptyset$ for some $p\in\P$, and \emph{non-degenerate} otherwise.
\end{definition}

\begin{definition}
  We will use the following notation:
  $\Gamma' = \Gamma \cap \R'$ is the property $\Gamma' : p \mapsto \Gamma(p)\cap\R'$.
  $L|_{\R'}: \R' \to \reals^\Y$ is the loss $\L|_{\R'} : r' \mapsto L(r')$.
\end{definition}

\begin{lemma}\label{lem:restrict-reports}
  Let $\R'\subseteq\R$ be report spaces, where $L:\R\to\reals^\Y$ elicits $\Gamma$, and $L|_{\R'}$ elicits $\Gamma'$.
  Then for all $r'\in\R'$, we have $\Gamma_{r'} \subseteq \Gamma'_{r'}$.  
\end{lemma}

\begin{lemma}\label{lem:trim}
  Let $L:\R\to\reals^\Y$ elicit a non-degenerate $\Gamma$.
  There is a unique non-degenerate non-redundant property $\Gamma'$, up to relabeling of the reports, such that $\Gamma' = \Gamma\cap\R'$ for some $\R'\subseteq\R$ .
  Moreover, the loss $L|_{\R'}$ elicits $\Gamma'$.
\end{lemma}
\begin{proof}
  Let $\R_1$ be the set of dominated reports of $\Gamma$, and define $\R_2 = \R\setminus\R_1$
  For all $r \in \R_2$, let $[r]$ be the set of equivalent reports to $r$, including $r$ itself.
  Then we let $\R_2' = \{[r] : r\in\R_2\}$ and $\R' = \varphi(\R_2') \subseteq \R$ where $\varphi$ selects an arbitrary representative of each equivalence class.
  \raf{I think I just used the axiom of choice... but for this paper, $\R'$ will be finite, so only a finite number of equivalence classes.}
  Let $\Gamma' = \varphi \circ \Gamma$; clearly $\Gamma'$ is unique up to the choice of relabeling $\varphi$.

  The non-redundancy of $\Gamma'$ follows immediately from our construction: $\Gamma'$ has no dominated reports, and each report is equivalent only to itself.
  For non-degeneracy, note that a property is non-degenerate if and only if its level sets union to all of $\P$ (every distribution is in some level set).
  By our construction, for all $r\in\R$ there is some $r'\in\R'$ such that $\Gamma_{r} \subseteq \Gamma_{r'}$, either because $r'$ dominates $r$ or the two are equivalent and $r' = \varphi([r])$.
  Thus, as $\Gamma$ was non-degenerate, we have $\P = \cup_{r\in\R} \Gamma_r = \cup_{r'\in\R'} \Gamma_{r'}$, giving non-degeneracy of $\Gamma'$.

\raf{TODO: finish proof that $L'$ elicits $\Gamma'$.  $\Gamma' \supseteq \Gamma \cap \R'$ is easy.}
  Finally, let $L' = L|_{\R'}$, that is, $L':\R'\to\reals^\Y$, $L:r'\mapsto L(r')$.
  \jessiet{What is this notation?}\raft{Restricted to; clarified now I hope.}
  We have $r' \in \Gamma(p) \implies r' \in \argmin_{r\in\R} p\cdot~L(r)$ $ \argmin_{r\in\R'} p\cdot~L'(r') $
\end{proof}
Note that non-degeneracy is necessary to make Lemma~\ref{lem:trim} interesting, as otherwise $\Gamma' : p \mapsto \emptyset$ would always suffice.


\begin{definition}\label{def:trim}
  We define $\trim(\Gamma)$ to be the unique $\Gamma'$ in Lemma~\ref{lem:trim}, up to relabeling of the reports.
\end{definition}

\begin{definition}
  We say $L:\R'\to\reals^\Y$ \emph{essentially elicits} a property $\Gamma : \P \toto \R$ if there exists some injective embdedding $\varphi:\R\to\R'$ such that for all $p\in\P,r\in\R$ we have $r \in \Gamma(p) \iff \varphi(r) \in \Gamma_L(p)$.
  Such a property is \emph{essentially elicitable} and if $L$ is convex, \emph{essentially convex elicitable}.
\end{definition}
Note that in particular, $L$ essentially elicits $\trim(\Gamma_L)$.

We will most often use this definition as follows.
Given a finite property $\Gamma : \P \toto \{1,\ldots,k\}$, we will seek some convex loss function $L : \reals^d \to \reals^\Y$ and a set of points $\X=\{x_1,\ldots,x_k\} \subset \reals^d$ such that the map $\varphi:i\mapsto x_i$ exhibits the essential elicitability of $\Gamma$.
In particular, using the set-valued link function $\psi:A\mapsto \varphi^{-1}(A\cap\X)$, we will have $\Gamma = \psi \circ \Gamma_L$, so that $\Gamma$ will be indirectly convex elicitable.

\begin{definition}
  Let $\Gamma:\P\toto\R$ and $\Gamma':\P\toto\R'$.
  We write $\Gamma \equiv \Gamma'$ if there is some bijection $\varphi:\R\to\R'$ such that $\Gamma_r = \Gamma'_{\varphi(r)}$ for all $r\in\R$.
\end{definition}
That is, $\Gamma\equiv\Gamma'$ if the two properties are the same up to relabeling the reports.

\begin{lemma}
  Let $L$ be a convex loss function.
  Then $\Gamma_L$ is non-degenerate if and only if $L(\cdot,y)$ is bounded from below for all $y\in\Y$.
\end{lemma}

\begin{definition}
  We say a loss $L$ is \emph{finite-piecewise-linear and convex (FPLC)} if for all $y\in\Y$, the function $L(\cdot)_y$ is a piecewise-linear function with finitely many pieces, and is bounded from below.
\end{definition}

\section{One dimension}

From Lambert 2018, Theorem 3, a finite property has a strictly order-sensitive score if and only if it is orderable...

\begin{lemma}\label{lem:fplc-directional-deriv}
  Let $L$ be FPLC, and let $r\in\reals$ be a point of differentiability of $L$.
  Let $d^-(r)_y$ and $d^+(r)_y$ denote the left and right derivative, respectively, of $L(\cdot)_y$ at $r$, so that $d^-(r),d^+(r)\in\reals^\Y$.
  Then for any open interval $(a,b)$ containing $r$ where $L$ is differentiable, we have
$d^+(a) = d^-(r) = d^+(r) = d^-(b)$.
\end{lemma}

\begin{lemma}\label{lem:simple-intervals}
  Let $a<a'$ and $b\leq b'$ such that either $a< 0 < b$ or $b' < 0 < a'$.
  Then there exists $\alpha \in (0,1)$ such that $0 \in \alpha [b,b'] + (1-\alpha) [a,a']$.
\end{lemma}
\begin{proof}
  Suppose $a < 0 < b$.
  Let $w = (a + \min(a',0))/2 < 0$ and take $\alpha = w/(w-b) \in (0,1)$.
  Then $\alpha b + (1-\alpha) w = wb/(w-b) - bw/(w-b) = 0$.
  As $a < w < a'$, the result follows:
  \begin{align*}
    & \alpha b + (1-\alpha) a < \alpha b + (1-\alpha) w = 0~,
    \\
    & \alpha b' + (1-\alpha) a' > \alpha b + (1-\alpha) w = 0~.
  \end{align*}
  Similarly, if $b' < 0 < a'$, we take $w = (\max(a,0)+a')/2 > 0$, and same construction holds.
\end{proof}

\begin{proposition}\label{prop:fplc-trim-char}
  Let $L : \reals \to \reals^\Y$ be FPLC and let $\R$ be the finite set of its nondifferentiable points.
  Then $\trim(\Gamma_L) \equiv \Gamma_L\cap\R'$ where $\underline r$ is the smallest $r\in\R$ such that $d^+(r)_y > 0$ for some $y\in\Y$, $\overline r$ is the largest such that $d^-(r)_y < 0$ for some $y\in\Y$, and $\R' = \R \cap [\underline r, \overline r]$.
\end{proposition}
\begin{proof}
  Since $L$ is FPLC, for each $y\in\Y$ we can take $\R_y$ to be the finite set of nondifferentiable points of $L(\cdot)_y$, and then $\R = \cup_{y\in\Y} \R_y$.
  As $L$ is bounded from below, we must have $d^-(\min \R)_y \leq 0$ and $d^+(\max \R)_y \geq 0$ for all $y\in\Y$.
  Thus, $\underline r$ and $\overline r$ are well-defined.

  A fact we will use throughout is that $r\in\reals$ is an element of $\Gamma_L(p)$, if and only if,
  \begin{equation}
    \label{eq:1d-optimality}
    0 \in \partial \; p \cdot L(r) = [p \cdot d^-(r), p \cdot d^+(r)]~,
  \end{equation}
  where $\partial$ denotes the subdifferential with respect to $r$.

  First, we argue that reports outside $[\underline r, \overline r]$ are weakly dominated.
  Let $r < \underline r$ and let $r'$ be the smallest value in $[r,\underline r]\cap \R$, i.e.,\ the next largest nondifferentiable point, which exists as $L$ is FPLC.
  If $r \in \Gamma_L(p)$, then we must have $d^+(r)\cdot p = 0$, as $d^+(r)\cdot p > 0$ would contradict the definition of $\underline r$.
  Then by Lemma~\ref{lem:fplc-directional-deriv}, as $L$ is differentiable on $(r,r')$, we have $0 = d^+(r)\cdot p = d^-(r')\cdot p$.
  Now, if $r' < \underline r$, we again have $d^+(r')\cdot p = 0$, and so on by induction (recall that $\R$ is finite) until we conclude $d^-(\underline r) = 0$, meaning $\underline r \in \Gamma_L(p)$ as well.
  Similarly, all reports $r > \overline r$ are weakly dominated by $\overline r$.

  We next show that all reports at differentiable points of $L$ are weakly dominated.
  We have already consider points outside of $[\underline r, \overline r]$; in that interval, nondifferentiable points satisfy $r' < r < r''$ where $r',r''\in\R$ and $L$ is differentiable on $(r',r'')$.
  If $r\in\Gamma_L(p)$, then from eq.~\eqref{eq:1d-optimality} we have $d^-(r)\cdot p = 0 = d^+(r) \cdot p$, and by Lemma~\ref{lem:fplc-directional-deriv}, we must also have $d^+(r')\cdot p = 0$ and $d^-(r'')\cdot p = 0$, giving $r',r'' \in \Gamma_L(p)$ as well.

  % Finally, we show that for every $r\in \R'$, there exists some $p\in\P$ such that $\Gamma_L(p)\cap\R' = \{r\}$.
  % For $r = \underline r$, let $\underline y$ exhibiting $d^+(\underline r)_{\underline y} > 0$.
  % Note that $d^-(\underline r)_{\underline y} \leq 0$ by definition of $\underline r$ (see the argument above).
  % Thus, letting $p = \delta_{\underline y}$ be the point distribution on $\underline y$, we then have $\underline r \in \Gamma_L(p)$.
  % As $d^-(r)_{\underline y} > 0$ for $r > \underline r$, we must have $\Gamma_L(p)\cap\R' = \{\underline r\}$.
  % \raft{Note: this is the only place in the argument where I needed to intersect with $\R'$.  The reason: losses like the hinge loss can have... }
  % Similarly for $\overline r$ and the outcome $\overline y$.
  % Now take $r \in (\underline r,\overline r) \cap \R$ and let $y$ be the outcome such that $L(\cdot)_y$ is nondifferentiable at $r$; in particular, $d^-(r)_y < d^+(r)_y$.

  Finally, we show that for every $r\in \R'$, there exists some $p\in\P$ such that $\Gamma_L(p) = \{r\}$.
  Let $\underline y$ exhibit $d^+(\underline r)_{\underline y} > 0$, and $\overline y$ exhibit $d^-(\overline r)_{\overline y} < 0$, and note that $d^-(\underline r)_{\underline y} \leq 0$ and $d^+(\overline r)_{\overline y} \geq 0$ by the argument above.
  We observe that if $r > \underline r$, then $0 < d^-(r)_{\underline y} \leq d^+(r)_{\underline y}$ by definition of $\underline r$ (and monotonicity of $d^+,d^-$).
  Similarly, if $r < \overline r$, then $d^-(r)_{\overline y} \leq d^+(r)_{\overline y} < 0$.

  Now for $r = \underline r$, we have $d^-(r)_{\underline y} \leq 0 < d^+(r)_{\underline y}$ and $d^-(r)_{\overline y} \leq d^+(r)_{\overline y} < 0$, and Lemma~\ref{lem:simple-intervals} gives $\alpha \in (0,1)$ such that $p = \alpha \delta_{\underline y} + (1-\alpha) \delta_{\overline y}$ gives $\Gamma_L(p) = \{\underline r\}$.
  Similarly, Lemma~\ref{lem:simple-intervals} gives an $\alpha$ such that $\Gamma_L(p) = \{\overline r\}$, where $p = \alpha \delta_{\overline y} + (1-\alpha) \delta_{\underline y}$.
  Finally, take $r \in \R' \cap (\underline r, \overline r)$ and let $y$ be the outcome such that $L(\cdot)_y$ is nondifferentiable at $r$; in particular, $d^-(r)_y < d^+(r)_y$.
  If $d^-(r)_y < 0 < d^+(r)_y$ we simply take $p = \delta_y$ and are done.
  Otherwise, consider the case $d^-(r)_y < d^+(r)_y \leq 0$, and as $r > \underline r$, recall that $0 < d^-(r)_{\underline y} \leq d^+(r)_{\underline y}$.
  Lemma~\ref{lem:simple-intervals} again gives $\alpha \in (0,1)$ such that $\Gamma_L(p) = \{r\}$ where $p = \alpha \delta_{y} + (1-\alpha) \delta_{\underline y}$.
  In the other case, we choose $p = \alpha \delta_{y} + (1-\alpha) \delta_{\overline y}$, with $\alpha$ again given by Lemma~\ref{lem:simple-intervals}.
  
  Putting everything together, we see that no report in $\R'$ is weakly dominated, and every report in $\reals\setminus\R'$ is weakly dominated by some report in $\R'$.
  We conclude that $\trim(\Gamma_L) \equiv \Gamma_L \cap \R'$.
\end{proof}

\begin{corollary}\label{cor:fplc-orderable}
  Every FPLC $L : \reals \to \reals^\Y$ essentially elicits a finite orderable property.
\end{corollary}
\begin{proof}
  From Proposition~\ref{prop:fplc-trim-char}, it suffices to show that $\Gamma_L\cap\R'$ is orderable.
  \raf{I need to reference some other stuff above about $\trim(\Gamma_L)$.}
  Let $\R' = \{r_1,\ldots,r_m\}$ with $r_1 < \cdots < r_m$.
  Recalling the optimality condition for $r_i \in \Gamma_L(p)$ in eq.~\eqref{eq:1d-optimality}, we see that
  \begin{equation*}\label{eq:fplc-level-set}
    \Gamma_{r_i} = \{ p \in \Delta_\Y : p\cdot d^-(r_i) \leq 0 \leq p\cdot d^+(r_{i}) \}~.
  \end{equation*}
  Thus, observing that $d^+(r_i) = d^-(r_{i+1})$ from Lemma~\ref{lem:fplc-directional-deriv}, for all $i$ the intersection $\Gamma_{r_i} \cap \Gamma_{r_{i+1}} = \{ p \in \Delta_\Y : p\cdot d^+(r_i) = 0\}$ is a hyperplane.
\end{proof}


\begin{theorem}\label{thm:fplc-orderable}
  A finite property is essentially convex elicitable in $1$ dimension if and only if it is orderable.
  Moreover, this remains true when restricting to piecewise linear loss functions.
\end{theorem}
\begin{proof}
  From Corollary~\ref{cor:fplc-orderable}, it remains to show two things: (i) \raf{general convex losses}, and (ii) a finite ordeable property is elicitable by a FPLC loss function.  
\end{proof}


\section{General dimensions}

\begin{lemma}\label{lem:sum-fplc}
  The sum of FPLC functions $f_1,\ldots,f_k$ is itself FPLC.
\end{lemma}
\begin{proof}
  Math.
\end{proof}

\begin{proposition}\label{cor:fplc-finite}
  Every FPLC $L : \reals^d \to \reals^\Y$ essentially elicits a finite property.
\end{proposition}
\begin{proof}
  Let $f(r) = \sum_{y\in\Y} L(r)_y$ which is FPLC by Lemma~\ref{lem:sum-fplc}.
  \raf{Perhaps its worth getting into power diagrams here...}
  Let $\R$ be the vertices of the cell complex given by $f$, that is, the points at which $\dim\partial f(r) = d$.
  As $f$ is FPLC, $\R$ is finite.
  We claim that $L$ essentially elicits $\Gamma_L\cap\R$.
  
  \raf{Now we just need to show that every point *not* on a vertex is weakly dominated by a vertex.}
\end{proof}

\raf{I further conjecture that there is a convex subset $C$ of $\reals^d$ such that $\R' = \R\cap C$ gives $\trim(\Gamma_L) = \Gamma_L\cap \R'$, just like in the 1-d case.  If that's true, it would be really helpful in figuring out what the constraints are in the simplex between neighbering reports!}

\raf{Note to self: (1) focus on distributions of full support to characterize the minimizers, (2) try to directly generalize 1d ``extreme'' points to the boundary of $C$.}


\section{Jessie's notes}
\raf{... have got it growing oats?}

\jessie{Notes partially for myself, but also with some questions I'm not sure of.  Feel free to correct}

We want some (invertible) representation $\phi: \reals^2 \to \reals$ so that $\forall p \in \P$, $\Gamma(p) = \phi([ x_1, x_2 ])$.

We want $\Gamma(p) = \phi(x) = \argmin_r \E_p L(r, Y)$

\subsection{Geometric median}
Consider $\Gamma(p) = \argmin E_p ||r - Y ||_2$
Let $\phi(x)$ be the representation of $\Gamma(p)$ in $\reals^n$.
Can we represent $\phi$ in some lower dimensional manner so that $\phi(x) = \Gamma(p)$ for all $p \in \P$?

The level sets of the geometric median intersect (see the git repo) so it's not elicitable.

Can we/do we want to construct a voronoi diagram of the geometric median?

\end{document}
%%% Local Variables:
%%% mode: latex
%%% TeX-master: t
%%% End:
