\documentclass[12pt]{article}
\usepackage[utf8]{inputenc}
\usepackage{mathtools, amsmath, amssymb, graphicx, verbatim}
%\usepackage[thmmarks, thref, amsthm]{ntheorem}
\usepackage{color}
\usepackage{wrapfig}
\usepackage{subcaption}
\usepackage[colorinlistoftodos,textsize=tiny]{todonotes} % need xargs for below
%\usepackage{accents}
\usepackage{bbm}

\newcommand{\Comments}{0}
\newcommand{\mynote}[2]{\ifnum\Comments=1\textcolor{#1}{#2}\fi}
\newcommand{\mytodo}[2]{\ifnum\Comments=1%
  \todo[linecolor=#1!80!black,backgroundcolor=#1,bordercolor=#1!80!black]{#2}\fi}
\newcommand{\raf}[1]{\mynote{green}{[RF: #1]}}
\newcommand{\raft}[1]{\mytodo{green!20!white}{RF: #1}}
\newcommand{\jessie}[1]{\mynote{purple}{[JF: #1]}}
\newcommand{\jessiet}[1]{\mytodo{purple!20!white}{JF: #1}}
\ifnum\Comments=1               % fix margins for todonotes
  \setlength{\marginparwidth}{1in}
\fi


\newcommand{\reals}{\mathbb{R}}
\newcommand{\posreals}{\reals_{>0}}%{\reals_{++}}
\newcommand{\dz}{\frac{d}{dz}}
\newcommand{\dx}{\frac{d}{dx}}
\newcommand{\dr}{\frac{d}{dr}}
\newcommand{\du}{\frac{d}{du}}

%m upper and lower bounds
\newcommand{\mup}{\overline{m}}
\newcommand{\mlow}{\underline{m}}


% alphabetical order, by convention
\newcommand{\E}{\mathbb{E}}
\newcommand{\F}{\mathcal{F}}
\newcommand{\I}{\mathcal{I}}
\renewcommand{\P}{\mathcal{P}}
\newcommand{\R}{\mathcal{R}}
\newcommand{\Y}{\mathcal{Y}}
\renewcommand{\P}{\mathcal{P}}

\newcommand{\inter}[1]{\mathring{#1}}%\mathrm{int}(#1)}
%\newcommand{\expectedv}[3]{\overline{#1}(#2,#3)}
\newcommand{\expectedv}[3]{\E_{Y\sim{#3}} {#1}(#2,Y)}

\DeclareMathOperator*{\argmax}{arg\,max}
\DeclareMathOperator*{\argmin}{arg\,min}
\DeclareMathOperator*{\arginf}{arg\,inf}
\DeclareMathOperator*{\sgn}{sgn}

%\newtheorem{theorem}{Theorem}
%\newtheorem{lemma}{Lemma}
%\newtheorem{proposition}{Proposition}
\newtheorem{claim}{Claim}
%\newtheorem{definition}{Definition}
%\newtheorem{corollary}{Corollary}
\newtheorem{condition}{Condition}
\newtheorem{case}{Case}
%\theorempostwork{\setcounter{case}{0}}

%%% Seriously COLT, seriously??
% \let\oldciteyear\citeyear
% \renewcommand{\citeyear}[1]{(\oldciteyear{#1})}


\title{Finite Property Convex Elicitation Notes}

\begin{document}

\maketitle

\section{1-d case}

Theorem 3 of Lambert's new version says that a finite property has a strictly order-sensitive score if and only if the reports are ordered and the border between adjacent reports in a hyperplane.

I believe we can add: ``, if and only if the finite property is indirectly convex elicitable via loss $L : \reals \times \Y \to \reals$ which is piecewise linear, i.e.,\ $L(r,y)$ is piecewise linear in $r$ for each $y$.''

Proof sketch: Since $L$ is piecewise linear, we can take $\R = \{r_1,...,r_m\}$ to be the set of all nondifferentiable points of $L(\cdot,y)$ for some $y$.
After some thought, it's clear that $\R$ contains all interesting minimizers: $\E_p L(r,Y)$ always attains a minimum at one of these points (assuming it attains one at all).
Now let $a_i(y)$ be the left derivative of $L(r_i,y)$ and $a_{m+1}(y)$ the right derivative of $L(r_m,y)$.
(So $a_i(y)$ is the ``$i$th slope'' of $L(\cdot,y)$.)
If you think about it a bit, the subgradient $\partial L(r_i,y)$ is just the interval $[a_i(y), a_{i+1}(y)]$.

Now the condition for $r_i$ to be optimal is simply
\begin{equation*}\label{eq:1}
0 \in \partial \E_p L(r_i,Y) = \sum_y p(y) [a_i(y), a_{i+1}(y)] = \left[\sum_y p(y) a_i(y), \sum_y p(y) a_{i+1}(y)\right]~.
\end{equation*}
Rewriting in vector notation, thinking of $p,a_i \in \reals^\Y$, we have
\begin{equation*}\label{eq:1}
\Gamma_{r_i} = \{ p \in \Delta_\Y : p\cdot a_i \leq 0 \leq p\cdot a_{i+1} \}~.
\end{equation*}
Thus, for all $i$ the intersection $\Gamma_{r_i} \cap \Gamma_{r_{i+1}} = \{ p \in \Delta_\Y : p\cdot a_{i+1} = 0\}$ which is a hyperplane.

For the converse... I actually didn't do it out, but I'm 99\% confident it will work, maybe as a corollary of the continuous version of this statement.
But this first direction was the most salient for now...

I also don't think lifting the piecewise linear restriction matters.

\jessie{Notes partially for myself, but also with some questions I'm not sure of.  Feel free to correct}
\section{2-d case}
We want some (invertible) representation $\phi: \reals^2 \to \reals$ so that $\forall p \in \P$, $\Gamma(p) = \phi([ x_1, x_2 ])$.

We want $\Gamma(p) = \phi(x) = \argmin_r \E_p L(r, Y)$

\subsection{Geometric median}
Consider $\Gamma(p) = \argmin E_p ||r - Y ||_2$
Let $\phi(x)$ be the representation of $\Gamma(p)$ in $\reals^n$.
Can we represent $\phi$ in some lower dimensional manner so that $\phi(x) = \Gamma(p)$ for all $p \in \P$?

The level sets of the geometric median intersect (see the git repo) so it's not elicitable.

Can we/do we want to construct a voronoi diagram of the geometric median?

\end{document}
%%% Local Variables:
%%% mode: latex
%%% TeX-master: t
%%% End:
