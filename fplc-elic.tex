\documentclass[11pt]{article}
\usepackage[utf8]{inputenc}
\usepackage{mathtools, amsmath, amsthm, amssymb, graphicx, verbatim}
%\usepackage[thmmarks, thref, amsthm]{ntheorem}
\usepackage{color}
\usepackage{wrapfig}
\usepackage{subcaption}
\usepackage[colorinlistoftodos,textsize=tiny]{todonotes} % need xargs for below
%\usepackage{accents}
\usepackage{bbm}
\usepackage{xspace}

\newcommand{\Comments}{1}
\newcommand{\mynote}[2]{\ifnum\Comments=1\textcolor{#1}{#2}\fi}
\newcommand{\mytodo}[2]{\ifnum\Comments=1%
  \todo[linecolor=#1!80!black,backgroundcolor=#1,bordercolor=#1!80!black]{#2}\fi}
\newcommand{\raf}[1]{\mynote{green}{[RF: #1]}}
\newcommand{\raft}[1]{\mytodo{green!20!white}{RF: #1}}
\newcommand{\jessie}[1]{\mynote{purple}{[JF: #1]}}
\newcommand{\jessiet}[1]{\mytodo{purple!20!white}{JF: #1}}
\ifnum\Comments=1               % fix margins for todonotes
  \setlength{\marginparwidth}{1in}
\fi


\newcommand{\reals}{\mathbb{R}}
\newcommand{\posreals}{\reals_{>0}}%{\reals_{++}}

% alphabetical order, by convention
\newcommand{\D}{\mathcal{D}}
\newcommand{\E}{\mathbb{E}}
\newcommand{\F}{\mathcal{F}}
\newcommand{\I}{\mathcal{I}}
\renewcommand{\P}{\mathcal{P}}
\newcommand{\R}{\mathcal{R}}
\newcommand{\V}{\mathcal{V}}
\newcommand{\X}{\mathcal{X}}
\newcommand{\Y}{\mathcal{Y}}


\newcommand{\inter}[1]{\mathring{#1}}%\mathrm{int}(#1)}
%\newcommand{\expectedv}[3]{\overline{#1}(#2,#3)}
\newcommand{\expectedv}[3]{\E_{Y\sim{#3}} {#1}(#2,Y)}
\newcommand{\toto}{\rightrightarrows}
\newcommand{\trim}{\mathrm{trim}}
\newcommand{\strip}{\mathrm{strip}}
\newcommand{\fplc}{finite-piecewise-linear and convex\xspace} %xspace for use in text
\newcommand{\FPLC}{\mathrm{FPLC}}
\newcommand{\conv}{\mathrm{conv}}
\newcommand{\elic}{\mathrm{elic}}
\newcommand{\ones}{\mathbbm{1}}

\DeclareMathOperator*{\argmax}{arg\,max}
\DeclareMathOperator*{\argmin}{arg\,min}
\DeclareMathOperator*{\arginf}{arg\,inf}
\DeclareMathOperator*{\sgn}{sgn}

\newtheorem{theorem}{Theorem}
\newtheorem{lemma}{Lemma}
\newtheorem{proposition}{Proposition}
\newtheorem{definition}{Definition}
\newtheorem{corollary}{Corollary}
\newtheorem{conjecture}{Conjecture}
\newtheorem{example}{Example}


\title{Elicitation Complexity via Convex Piecewise Linear Loss Functions}
\author{Jessie, Raf, Bo}

\begin{document}

\maketitle

\section{Introduction}
Big questions right now in the field:
\begin{itemize}
\item Characterize convex-elicitable properties
\item Bound convex elicitation complexity
\item Understand the role of smoothness and strict convexity: are non-smooth non-strict convex losses strictly more powerful?
\end{itemize}

This paper: progress toward all three by studying properties ``directly'' elicited by FPLC losses.

\section{Setting}

\raf{Standard defs: property, elicit, ...}

Previous papers on general indirect elicitation have largely ignored set-valued properties.
How to define?
The crux: what should the link be?

\begin{definition}
  Let properties $\gamma:\P\toto\R$ and $\Gamma:\P\toto\V$ be given.
  The function $f:\V\to\R$ is a \emph{selective link function} if (i) $v\in\Gamma(p) \implies f(v) \in \gamma(p)$, and (ii) $f(\Gamma(p)) = \gamma(p)$.
  The function $f:\V\toto\R$ is an \emph{exhaustive link function} if $v\in\Gamma(p) \iff f(v) = \gamma(p)$.
\end{definition}

The intuition: selective links map optimal reports to an optimal report, whereas exhaustive links map optimal reports to the \emph{full set} of optimal reports.

\begin{example}
  Hinge loss and logistic loss.
  Link $f = \sgn$ is selective for hinge but exhaustive for logistic.
\end{example}

\raf{If we want to state this general definition for posterity; won't be needed.}
\begin{definition}
  Elicitation complexity of $\gamma:\P\toto\R$... min dimension of $\Gamma\in\mathcal{C}\cap\mathcal{E}$ with either a selective or exhaustive link to $\gamma$.
  \raf{Maybe break into ``selective'' and ``exhaustive'' versions?}
\end{definition}

\section{Embedded Properties}

\begin{definition}
  Let $\gamma:\P\toto\R$.
  We say a property $\Gamma:\P\toto\reals^d$ \emph{embeds $\gamma$} if there is an injection $\varphi:\R\to\reals^d$ such that $r\in\gamma(p) \iff \varphi(r) \in \Gamma(p)$.
  \raft{Note: definition is general and not specific to FPLC, finite propertes, etc.}
\end{definition}

\begin{lemma}
  \label{lem:embed-lev-set-subset}
  If elicitable $\Gamma:\P\toto\reals^d$ embeds non-degenerate $\gamma:\P\toto\R$, then for all $v\in\reals^d$, we have $\Gamma_v \subseteq \gamma_r$ for some $r\in\R$.
\end{lemma}

\begin{proposition}
  If elicitable $\Gamma:\P\toto\reals^d$ embeds non-degenerate $\gamma:\P\toto\R$, there is a selective link $f:\reals^d\to\R$ from $\Gamma$ to $\gamma$.
\end{proposition}
\begin{proof}
  Let $\V = \bigcap\{ \Gamma(p) : p\in\P\} \subseteq \reals^d$ be the image of $\Gamma$.
  Define $F:\V\toto\R$ as follows: $F(v) = \bigcap\{\gamma(p) : v\in\Gamma(p)\} = \bigcap\{\gamma(p) : p \in\Gamma_v\}$.
  That is, $F$ encodes all reports of $\gamma$ which could be correct when $v$ is chosen from $\Gamma$.
  Now let $v\in\V$, and let $r\in\R$ be given by Lemma~\ref{lem:embed-lev-set-subset}, so that $\Gamma_v \subseteq \gamma_r$.
  Then $p\in\Gamma_v \implies r\in\gamma(p)$, so we have $r \in F(v)$, and in particular, $F(v)\neq\emptyset$.
  (The latter would be vacuous if there were no $p\in\Gamma_v$, but we have assumed so by definition of $\V$.)
  We simply define $f:\reals^d \to \R$ to be any single-valued function such that $f(\varphi(r)) = r$ for all $r\in\R$, $f(v) \in F(v)$ whenever $v\in\V$, and $f(v) \in \R$ arbitrary when $v\notin\V$.
  (This is a consistent choice by definition of $\varphi$.)

  It remains to show that $f$ is a selective link.
  For (i), note that $r'\in\Gamma'(p) \implies F(r') \subseteq \Gamma(p) \implies f(r') \in \Gamma(p)$.
  For (ii), let $p$ be given.
  For all $r\in\Gamma(p)$, we must have $\varphi(r) \in \gamma(p)$ by definition of $\varphi$ \jessiet{Doesn't matter too much, but notation doesn't match Definition 3}.
  Thus, $f(\gamma(p)) \supseteq \{f(\varphi(r)) : r \in \Gamma(p)\} = \Gamma(p)$.
  By (i), this must be an equality.
\end{proof}

\begin{proposition}
  Let $\Gamma:\P\toto\reals^d$ be elicitable.
  The following are equivalent:
  \begin{enumerate}
  \item $\Gamma$ is an embedding of a finite elicitable property.
  \item $\Gamma$ has finitely many full-dimensional level sets which union to $\P$.
  \item \rotatebox{180}{$\Gamma$ has finitely many full-dimensional level sets which union to $\P$.}
  \item $\strip(\Gamma)$ is finite.
  \item $\trim(\Gamma)$ is finite.
  \item $\trim(\Gamma)$ is finite and elicitable.
  \end{enumerate}
\end{proposition}
\begin{proof}
  \raf{Much of the proof in Jessie-notes}
\end{proof}


\section{Piecewise-Linear Convex Lossos}

\begin{proposition}
  Let $L$ be FPLC.
  Then $\Gamma_L$ embeds a elicitable finite property.
\end{proposition}
\begin{proof}
  \raf{In cvx-fin-notes}
\end{proof}

As every finite property is embedded by an FPLC loss, it remains only to ask how many dimensions are required.
This motivates the definition of $d$-FPLC-embeddable.

\begin{definition}
  Property $\gamma:\P\toto\R$ is $d$-FPLC-embeddable if there is some FPLC loss $L:\reals^d\times\Y\to\reals$ such that $\Gamma_L$ embeds $\gamma$.
\end{definition}

\begin{example}
  \raf{Some examples...}
\end{example}

\section{Embedding Dimension}

\begin{proposition}
  Let $\gamma:\Delta(\Y)\toto\R$ be a elicitable finite property for $n=|\Y|$.
  Then $\gamma$ is $(n-1)$-FPLC-embeddable.
\end{proposition}
\begin{proof}
  \raf{In cvx-fin-notes}  
\end{proof}

\raf{Do we also have $d$-embeddable $\implies$ $d$-FPLC-embeddable if $\gamma$ is finite?  That seems harder to show; maybe it was on our list once.}\jessie{Not as hard to show if embeddable $\iff$ FPLC}

\begin{proposition}
  Let $\gamma:\Delta(\Y)\toto\R$ be a finite property. \raf{Do we need elicitable?}
  Then $\gamma$ is $1$-FPLC-embeddable if and only if $\gamma$ is oriented.
\end{proposition}
\begin{proof}
  \raf{In cvx-fin-notes}
\end{proof}

\begin{theorem}
  Let $\gamma:\Delta(\Y)\toto\R$ be an elicitable finite property.
  Then $\gamma$ is $d$-FPLC-embeddable if and only if [[AWESOME CHARACTERIZATION]].
\end{theorem}
\begin{proof}
  [[AWESOME PROOF]]
\end{proof}

Implications for complexity: certainly $\gamma$ is $d$-FPLC-embeddable $\implies$ $\elic_\FPLC(\gamma) \leq d$.
\raf{If $\gamma$ is elicitable, I think this is an $\iff$, right?}
If $\gamma$ is finite but not elicitable, it will not be FPLC-embeddable for any dimension, but could still have finite FPLC elicitation complexity.
E.g.\ 
\jessie{Example?  This statement's a little surprising to me since the FPLC $L$ would elicit some other property $\Gamma$ and $\gamma$ would also embed some finite elicitable property $\zeta$... but if $\strip(\gamma) \neq \strip(\zeta) = \trim(\Gamma)$ then how does $\Gamma$ embed $\gamma$?  Maybe I'm too fixed on FPLCs...}

\end{document}
%%% Local Variables:
%%% mode: latex
%%% TeX-master: t
%%% End:
