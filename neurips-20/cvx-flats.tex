\documentclass{article}

% if you need to pass options to natbib, use, e.g.:
%     \PassOptionsToPackage{numbers, compress}{natbib}
% before loading neurips_2020

% ready for submission
% \usepackage{neurips_2020}

% to compile a preprint version, e.g., for submission to arXiv, add add the
% [preprint] option:
%     \usepackage[preprint]{neurips_2020}

% to compile a camera-ready version, add the [final] option, e.g.:
%     \usepackage[final]{neurips_2020}

% to avoid loading the natbib package, add option nonatbib:
\usepackage{neurips_2020}

\usepackage[utf8]{inputenc} % allow utf-8 input
\usepackage[T1]{fontenc}    % use 8-bit T1 fonts
\usepackage{hyperref}       % hyperlinks
\usepackage{url}            % simple URL typesetting
\usepackage{booktabs}       % professional-quality tables
\usepackage{amsfonts}       % blackboard math symbols
\usepackage{nicefrac}       % compact symbols for 1/2, etc.
\usepackage{microtype}      % microtypography
\usepackage{amsmath, amsfonts, amssymb, amsthm}

\usepackage{color}
\definecolor{darkblue}{rgb}{0.0,0.0,0.2}
\hypersetup{colorlinks,breaklinks,
	linkcolor=darkblue,urlcolor=darkblue,
	anchorcolor=darkblue,citecolor=darkblue}
\usepackage{wrapfig}
\usepackage{subcaption}
\usepackage[colorinlistoftodos,textsize=tiny]{todonotes} % need xargs for below
%\usepackage{accents}
\usepackage{bbm}
\usepackage{xspace}

\newcommand{\Comments}{1}
\newcommand{\mynote}[2]{\ifnum\Comments=1\textcolor{#1}{#2}\fi}
\newcommand{\mytodo}[2]{\ifnum\Comments=1%
	\todo[linecolor=#1!80!black,backgroundcolor=#1,bordercolor=#1!80!black]{#2}\fi}
\newcommand{\raf}[1]{\mynote{green}{[RF: #1]}}
\newcommand{\raft}[1]{\mytodo{green!20!white}{RF: #1}}
\newcommand{\jessie}[1]{\mynote{purple}{[JF: #1]}}
\newcommand{\jessiet}[1]{\mytodo{purple!20!white}{JF: #1}}
\newcommand{\proposedadd}[1]{\mynote{orange}{#1}}
\newcommand{\bo}[1]{\mynote{blue}{[Bo: #1]}}
\newcommand{\botodo}[1]{\mytodo{blue!20!white}{[Bo: #1]}}
\newcommand{\btw}[1]{\mytodo{gray!20!white}{[BTW: #1]}}%TURN OFF FOR NOW \mytodo{gray}{#1}}
\ifnum\Comments=1               % fix margins for todonotes
\setlength{\marginparwidth}{1in}
\fi

\newcommand{\reals}{\mathbb{R}}
\newcommand{\posreals}{\reals_{>0}}%{\reals_{++}}
\newcommand{\simplex}{\Delta_\Y}
\newcommand{\relint}[1]{\mathrm{relint}(#1)}
\newcommand{\prop}[1]{\mathrm{prop}[#1]}
\newcommand{\elic}{\mathrm{elic}}
\newcommand{\eliccvx}{\mathrm{elic}_\mathrm{cvx}}
\newcommand{\elicpoly}{\mathrm{elic}_\mathrm{pcvx}}
\newcommand{\elicembed}{\mathrm{elic}_\mathrm{embed}}
\newcommand{\ccdim}{\mathrm{ccdim}}
\newcommand{\codim}{\mathrm{codim}}
\newcommand{\supp}{\mathrm{supp}}
\newcommand{\spn}{\mathrm{span}}
\newcommand{\propdis}{\mu}

\newcommand{\C}{\mathcal{C}}
\newcommand{\D}{\mathcal{D}}
\newcommand{\E}{\mathbb{E}}
\newcommand{\F}{\mathcal{F}}
\newcommand{\I}{\mathcal{I}}
\newcommand{\N}{\mathcal{N}}
\newcommand{\R}{\mathcal{R}}
\renewcommand{\P}{\mathcal{P}}
\renewcommand{\S}{\mathcal{S}}
\newcommand{\U}{\mathcal{U}}
\newcommand{\X}{\mathcal{X}}
\newcommand{\Y}{\mathcal{Y}}


\newcommand{\risk}[1]{#1^*}
\newcommand{\inprod}[2]{\langle #1, #2 \rangle}
\newcommand{\toto}{\rightrightarrows}

\newtheorem{theorem}{Theorem}
\newtheorem{lemma}{Lemma}
\newtheorem{proposition}{Proposition}
\newtheorem{corollary}{Corollary}
\newtheorem{conjecture}{Conjecture}
\newtheorem{definition}{Definition}
\newtheorem{assumption}{Assumption}
\newtheorem{remark}{Remark}


\DeclareMathOperator*{\argmax}{arg\,max}
\DeclareMathOperator*{\argmin}{arg\,min}
\DeclareMathOperator*{\arginf}{arg\,inf}
\DeclareMathOperator*{\sgn}{sgn}

\title{Indirect elicitation as a necessary condition for consistent surrogate losses}

% The \author macro works with any number of authors. There are two commands
% used to separate the names and addresses of multiple authors: \And and \AND.
%
% Using \And between authors leaves it to LaTeX to determine where to break the
% lines. Using \AND forces a line break at that point. So, if LaTeX puts 3 of 4
% authors names on the first line, and the last on the second line, try using
% \AND instead of \And before the third author name.

\author{%
  Jessie Finocchiaro\\
%  Department of Computer Science\\
  CU Boulder\\
  \texttt{jefi8453@colorado.edu} 
  % examples of more authors
  \And
   Rafael Frongillo\\
%   Department of Computer Science\\
   CU Boulder\\
  % Address \\
   \texttt{raf@colorado.edu} 
   \And
   Bo Waggoner\\
%   Department of Computer Science\\
   CU Boulder \\
  % Address \\
   \texttt{bwag@colorado.edu} 
  % \And
  % Coauthor \\
  % Affiliation \\
  % Address \\
  % \texttt{email} \\
  % \And
  % Coauthor \\
  % Affiliation \\
  % Address \\
  % \texttt{email} \\
}



\begin{document}

\maketitle

\begin{abstract}
%	For finite prediction tasks, one can often naturally model their specific problem with a discrete loss.
%	However, these discrete losses are often hard to optimize, so one often seeks a surrogate losses yielding desirable statistical guarantees with respect to the original problem.
	In prediction tasks, one typically seeks to minimize \emph{empirical risk} by learning a hypothesis function making predictions from a given feature vector about a desired future outcome.
	This hypothesis is learned by minimizing a loss function over a set of labeled training data.
	However, as not all loss functions are easily optimized, one often wishes to optimize a \emph{surrogate loss} yielding the same statistical guarantees as the original.
	In this paper, we are specifically interested in the construction of \emph{consistent} surrogate losses; particularly those that are  ``efficient'' in the dimension of their input.
	We extend the general notion of consistency to properties, so that we may understand when one can construct a convex, consistent surrogate without requiring an original loss, which are uncommon for continuous prediction tasks (for example, the median).
	This allows us to observe that indirect property elicitation is necessary to construct a consistent surrogate, which allows us to apply property elicitation complexity results to the minimal dimension of a consistent surrogate.
	Moreover, when surrogate losses are convex, we state a new lower bound on the dimension of a convex surrogate by using tools from property elicitation.
\end{abstract}

\section{Introduction}\label{sec:intro}
In supervised machine learning, one often wants to make a prediction about future outcomes by training a classifier to minimize the average empirical loss of a labeled training set, where the loss is determined by the task at hand.
For example, 0-1 loss is often desired for classification tasks.
However, finite losses are typically difficult to optimize, so we construct a continuous \emph{surrogate} loss that one can more easily optimize.
In particular, we want our surrogate loss to yield bounds on the excess risk so that optimal predictions for the surrogate can be linked to optimal predictions for an original (possibly discrete) loss with bounded excess error, regardless of the underlying data distribution.
Moreover, in continuous prediction settings, one often starts with a statistic of the data they would like to estimate, but it is often unclear how to construct a a loss that is consistent for the given statistic, or property.
For example, if one wants to estimate the $\alpha$-quantile, they can do so by minimizing pinball loss over their data, but it is often unclear \emph{why} pinball loss is good for this prediction task.
Throughout this paper, the statistical guarantees we desire are \emph{consistency}, as it is the only condition that guarantees good bounds on excess surrogate loss. \jessiet{Raf, maybe fill in here?}
In particular, we are specifically interested in consistent \emph{and convex} surrogates in which any sequence of hypotheses converging to the optimal prediction can be linked to a sequence that converges to either the correct prediction for the discrete loss or statistic value, depending on the setting. 
In Theorem~\ref{thm:consistent-implies-indir-elic}, we show indirect property elicitation is a necessary condition for consistency so that we can use a tool based on property elicitation to bound the minimum dimension needed to construct a consistent convex surrogate.
Previous lower bounds on the dimension of a consistent convex surrogate for a given task have been considered in one of four settings: first, previous results diverge where one are given a target loss or simply a statistic of the data they wish to estimate.
Second, previous results typically hold only under a finite prediction setting or a continuous estimation setting.
Theorem~\ref{thm:cvx-flats}, however, is general enough to yield bounds across all four settings.

We typically use convex surrogate loss functions that take some input $u \in \reals^d$ and measure error against the observed outcome $y \in \Y$.
The value $d$ provides some notion of efficiency for the surrogate, as low-dimensional convex losses can improve the efficiency of the optimization algorithm.
\cite{frongillo2015elicitation} pose an open question regarding a general characterization of the efficiency of \emph{convex elicitable properties}, which we show is bounded by finding the minimum dimension $d$ for which a consistent, convex loss can be constructed for the task at hand.

In Section~\ref{sec:related-work}, we review previous work and introduce the two main concepts used: \emph{consistency} and \emph{property elicitation}.
We draw connections between the two in Section~\ref{sec:consis-implies-indir} by showing that if a loss and link are consistent with respect to a target loss or the property it elicits, then it must also indirectly elicit the same property.
In Section~\ref{sec:char-convex}, we present our main result (Theorem~\ref{thm:cvx-flats}), a corollary of which characterizes the existence of consistent convex surrogates in a given dimension $d$.
In Section~\ref{sec:finite-calib}, we relate our setting to the study of \emph{convex calibration dimension} introduced by~\cite{ramaswamy2016convex} by focusing on the finite prediction setting.
Our Theorem~\ref{thm:cvx-flats} generalizes their bound on convex calibration dimension by observing that the \emph{subspace of feasible dimensions} constructed in their bound is just one case of a flat produced by our Theorem~\ref{thm:cvx-flats}.
To see this we use our results to generalize and simplify the proof of~\cite[Theorem 16]{ramaswamy2016convex}.
In continuous settings, one often has a prediction task for which they want to construct a consistent convex loss, rather than an original loss.
In Section~\ref{sec:contin-consis}, we address the construction of convex consistent losses by applying the main insights from Theorem~\ref{thm:cvx-flats} to this setting, answering the open question posed by~\cite[Section 8]{frongillo2018elicitation}.

\subsection{Notation}
In this paper, we take the outcome set $\Y$ and consider $\simplex$ to be the simplex over $\Y$.
For $p \in \simplex$, let $\supp(p)$ be the support of $p$.
%discrete land
We use $\R$ to denote a report set; it is not necessary for $\Y = \R$.  
For example, in ranking problems, $\R$ may be all $n!$ permutations over the $n$ outcomes forming $\Y$.
A loss $\ell : \R \times \Y \to \reals_+$ is \emph{discrete} if $|\R| < \infty$, and in general, we use $\ell$ to denote our \emph{target loss} for which we want to construct a consistent surrogate $L$.
For a distribution $p \in \simplex$, we denote the expected discrete loss for report $r$ to be $\E_{Y \sim p} \ell(r, Y) := \ell(r; p)$.
We denote the surrogate loss $L : \reals^d \times \Y \to \reals_+$, and typically denote a surrogate report with $u$.

Throughout this paper, we use tools from \emph{property elicitation} to understand the existence of consistent surrogate functions; a property is simply a function mapping distributions over the outcome simplex to reports.
Introduced in Section~\ref{subsec:properties}, a property $\Gamma: \simplex \to 2^\R \setminus \emptyset$ is denoted $\Gamma:\simplex \toto \R$.
We call a property \emph{set-valued} if there is a $p \in \simplex$ so that $|\Gamma(p)| > 1$, and \emph{single-valued} otherwise.
We use $\Gamma := \prop{L}$ to denote that $\Gamma$ is the (unique) property elicited by $L$.

Throughout this paper, we focus on characteristics of \emph{convex} surrogate losses.
When we say a loss $L(\cdot; p)$ is \emph{convex}, we mean that it is convex in its first argument for every $p \in \simplex$, which is equivalent to saying $L$ is convex in the report for every outcome $y \in \Y$.
Moreover, the regret of a loss $L(u; p)$ is the excess loss over the optimal; that is, $R_L(u,p) := L(u,p) - \inf_{u^*} L(u^*, p)$.

\section{Background and Related work}\label{sec:related-work}

\subsection{Consistency and calibration for convex losses}\label{subsec:convex-surrogates}

For discrete prediction problems, one often is able to intuitively start with a target loss $\ell$ in mind: for classification, 0-1 loss comes to mind; for rankings, a borda-like loss that assigns error based on one's distance from the winner's predicted ranking.
However, when one starts with a (often discrete) target loss they want to minimize, it is often a computationally hard problem.
For this reason, we use surrogate losses, but desire consistency to guarantee the surrogate ``corresponds'' to the original loss, or property we want to predict, as is more common in the the continuous estimation setting.
However, a surrogate loss is no good on its own; one needs a link function to map from the surrogate prediction space back to the \emph{correct} prediction in the original prediction space.
This notion of a correct surrogate is typically captured by the notion of \emph{consistency}, introduced formally in Definitions~\ref{def:consistent-ell} and~\ref{def:consistent-prop}.
\emph{Calibration} was the primary tool to study consistency until~\cite{agarwal2015consistent} introduced the use of property elicitation to study consistency in certain situations.

\begin{definition}[Calibrated: Finite predictions]\label{def:calibrated-finite}
	Let $\ell : \R \times \Y \to \reals_+$ be a discrete target loss eliciting the property $\gamma$.
	A surrogate loss $L : \reals^d \times \Y \to \reals_+$  and link $\psi:\reals^d \to \R$ pair $(L, \psi)$ is \emph{calibrated} with respect to $\ell$ if 
	\begin{equation}\label{eq:calibration}
	\forall p \in \simplex: \inf_{u \in \reals^d : \psi(u) \not \in \gamma(p)} L(u;p) > \inf_{u \in \reals^d} L(u;p)~.~
	\end{equation}
\end{definition}

\cite{zhang2004statistical,lin2004note,bartlett2006convexity,tewari2007consistency} form a characterization of consistent and calibrated surrogates for classification problems.
In particular,~\cite{bartlett2006convexity} show that if a convex surrogate is differentiable, minimal at $0$, and nonnegative, there is a consistent surrogate for binary classification problems, and~\cite{tewari2007consistency} generalizes this result for multiclass classification. 
\cite{ramaswamy2016convex} further show necessary and sufficient conditions for finite prediction problems to be consistent that can be applied to general discrete losses.
\cite{steinwart2007compare} generalizes the study of consistent losses from the finite prediction setting and characterizes different types of loss functions, relating excess risk bounds, consistency, and calibration, giving proofs for various classes of surrogate losses (i.e. margin-based, distance-based, supervised, unsupervised, etc.)
See~\cite[Chapter 2]{steinwart2008support} for further discussion of these loss functions.
In Appendix~\ref{app:calibration}, we give a more general definition of calibration which is equivalent to Definition~\ref{def:calibrated-finite} in finite outcome settings, but generalizes to infinite outcome settings.


\subsection{Property elicitation}\label{subsec:properties}
In Section~\ref{sec:consis-implies-indir} we show that indirect property elicitation is a necessary condition for constructing a consistent surrogate loss.
Property elicitation of a single property is well-understood through \cite{savage1971elicitation,osband1985information-eliciting,lambert2008eliciting, lambert2009eliciting, lambert2018elicitation}.
Recently, the elicitation of multiple properties simultaneously and of high-dimensional properties was studied by~\cite{frongillo2015vector-valued,fissler2015higher,frongillo2018elicitation}.
In particular, these works study the \emph{minimum} dimension of a real-valued loss needed to indirectly elicit (see Definition~\ref{def:indirectly-elicits} below) a property or vector of properties by directly eliciting other (possibly vector-valued) properties.
Moreover,~\cite{agarwal2015consistent} is the first to our knowledge to formally relate property elicitation to the consistency of a surrogate loss.

Informally, a property is a function mapping a probability distribution $p$ over the outcome set $\Y$ to the ``suggested report'' given $p$.
\begin{definition}[Property, elicits, level set]
	\jessiet{Don't like the notation here... fix later.}
	A \emph{property} is a (possibly set-valued) function $\Gamma : \simplex \toto \R$ mapping distributions to reports.
	A loss $L : \R \times \Y \to \reals_+$ \emph{elicits} the property $\Gamma$ if,
	\begin{equation}
	\forall p \in \simplex, \;\; \Gamma(p) = \argmin_{u \in \R}L(u;p)
	\end{equation}
	Moreover, we call a \emph{level set} $\Gamma_r := \{p \in \simplex : r \in \Gamma(p)\}$ to be the set of distributions for which reporting $u$ minimizes the expected loss of the loss eliciting $\Gamma$.
\end{definition}

Note, moreover, that we call a property $\gamma: \simplex \toto \R$ \emph{finite} if $|\R| < \infty$.
Without much loss of generality, we assume that finite properties are \emph{non-redundant}, meaning that for each $\gamma_r$, there is no $\gamma_{r'}$ such that $\gamma_{r'} \subseteq \gamma_r$.

\cite{finocchiaro2018convex} are among the first to consider a characterization of \emph{convex} elicitable properties, in which they find that all continuous, nowhere locally constant elicitable properties (in finite outcome settings) are elicitable by a \emph{convex} loss.
However, their assumptions are more restrictive than ours; they only consider losses defined on the real line.% and assume the properties to be identifiable: an assumption we do not need.

Sometimes we can ``indirectly'' elicit a property by taking a transformation of some other property that might be easier to elicit.

\begin{definition}[Indirect Elicitation]\label{def:indirectly-elicits}
	A loss and link $(L, \psi)$ \emph{indirectly elicits} a property $\gamma:\simplex \toto \R'$ if $L$ elicits a property $\Gamma: \simplex \toto \R$ such that for all $r \in \R$, we have $\Gamma_r \subseteq \gamma_{\psi(r)}$.
	
	Moreover, $L$ indirectly elicits $\gamma$ if such a link $\psi$ exists.
\end{definition}

While there are a few possible definitions of indirect elicitation, this one is the most applicable for our setting because when we consider set-valued properties, we do not have that \emph{all} optimal reports for $\Gamma$ must be linked back to \emph{all} optimal reports for $\gamma$.
Instead, we lighten this restriction to say that any optimal report for $\Gamma$ must be linked to \emph{an} optimal report for $\gamma$, and must be done in a consistent manner.

\subsection{Complexity notions}\label{subsec:complexity}
When $\R \subseteq \reals^d$ for some $d \in \mathbb{Z}$, \cite{frongillo2018elicitation} introduce the notion of \emph{(convex) elicitation complexity}, which uses $d$ to measure the ``efficiency'' of a loss $L$ eliciting $\Gamma$.
However, their notion of complexity relies not just on the property $\Gamma$, but instead on any property \emph{indirectly elicited} by $L$.
\begin{definition}[Convex Elicitation Complexity]
	The \emph{convex elicitation complexity} of a property $\eliccvx(\Gamma)$ is the minimum dimension $d$ such that there is a convex loss $L : \reals^d \times \Y \to \reals$ indirectly eliciting $\Gamma$.
\end{definition}

\cite{ramaswamy2016convex} additionally introduce a notion of \emph{convex calibration dimension} for discrete losses similar to elicitation complexity mentioned above; we discuss this further in Section~\ref{sec:finite-calib}.
Their convex calibration dimension results yield consistent surrogates for finite prediction problems such as hierarchical classification from~\cite{ramaswamy2015hierarchical} and classification with an abstain option studied by~\cite{ramaswamy2018consistent}.
However, we show that our Theorem~\ref{thm:cvx-flats} can be used to derive their convex calibration bounds, and even derive tighter bounds on convex calibration dimension in some instances.

\begin{definition}[Convex Calibration Dimension]
	The \emph{convex calibration dimension} $\ccdim(\ell)$ of a discrete loss $\ell$ is the minimum dimension $d$ such that there is a convex loss $L: \reals^d \times \Y \to \reals$ and link $\psi$ such that $L$ is calibrated with respect to $\ell$.
\end{definition}

We can additionally relate convex calibration dimension of a loss to the convex elicitation complexity of the loss it elicits.
\begin{proposition}
	Consider the discrete loss $\ell : \R \times \Y \to \reals_+$ and $\gamma:= \prop{\ell}$.
	Then $\eliccvx(\gamma) \leq \ccdim(\ell)$.
\end{proposition}
\begin{proof}
	Let $d$ be the convex calibration dimension of $\ell$, and $(L, \psi)$ be calibrated with respect to $\ell$ with $L$ convex.
	Then $L$ indirectly elicits $\gamma$ by Lemma~\ref{lem:calib-implies-indir}.
\end{proof}

\cite{finocchiaro2019embedding} presents the notion of a surrogate loss \emph{embedding} a discrete loss, typically by a polyhedral (piecewise linear and convex) surrogate.
Moreover,~\cite{finocchiaro2020embedding} introduces the notion of \emph{embedding dimension}, which is a lower bound on both convex elicitation complexity of finite properties and convex calibration dimension of discrete losses.
It is an open problem to understand if these efficiency definitions are generally equivalent. \jessie{Move this sentence to conclusions?}


\section{Consistency implies indirect elicitation}\label{sec:consis-implies-indir}
While the subtleties of \emph{consistency} vary slightly within the literature, we use two definitions that have the same implications, but one may be more appropriate than the other in a given context.

\begin{definition}[Consistent: loss]\label{def:consistent-ell}
	A loss and link $(L,\psi)$ are consistent with respect to an original loss $\ell$ if, for all distributions $D$ over input and label spaces $\X \times\Y$, and for all sequences of measurable hypothesis functions $\{f_m : \X \to \R\}$,
	\begin{align*}
	\E_D L(f_m(X), Y) \to \inf_f \E_D L(f(X), Y) &\implies \E_D \ell((\psi \circ f_m)(X), Y) \to \inf_f \E_D \ell((\psi \circ f)(X), Y)~.~
	\end{align*}
\end{definition}
\bo{I guess we need the inf to be over a certain class of functions...the left and right side expectations need to be well-defined.}

For continuous properties, such as the expected value, variance, and entropy, asking for a consistent surrogate with respect to an original \emph{loss} may not make sense when we do not have one.
In this setting, we also define consistency with respect to a property that one wishes to estimate. 

\begin{definition}[Consistent: property]\label{def:consistent-prop}
	Suppose we are given a loss $L : \R \times \Y \to \reals$, link function $\psi: \R \to \R'$, and property $\gamma:\simplex \toto \R'$.
	Moreover, let $\propdis : \R' \times \simplex \to \reals_+$ any function where $\propdis(r,p) = 0 \iff r \in \gamma(p)$.
	
	We say $(L, \psi)$ is \emph{$\mu$-consistent with respect to the property} $\gamma$ if, for all $D$ over $\X \times \Y$ with marginal distributions $D_x$, and for all sequences of measurable functions $\{f_m: \X \to \R\}$, 
	\begin{equation}
	\E_{D} L(f_m(X), Y) \to \inf_f \E_{D} L( f(X), Y) \implies \E_X \propdis(\psi \circ f_m(X), D_X) \to 0~.~
	\end{equation}
	
	We simply say the pair is consistent with respect to $\gamma$ if there is a $\propdis$ such that $(L,\psi)$ are $\propdis$-consistent with respect to $\gamma$.
\end{definition}

Theorem~\ref{thm:consistent-implies-indir-elic} shows that consistency, either with respect to a target loss $\ell$ or a property $\gamma$, implies indirect elicitation.

\jessie{This section definitely needs some work because I'm not sure what is staying and what's going to the Appendix.}
\jessie{This proof uses calibration via Definition~\ref{def:calibrated-general}.}
\begin{proposition}\label{prop:consistent-implies-calibrated}
	If a loss and link $(L, \psi)$ are consistent with respect to a loss $\ell$, then they are calibrated with respect to $\ell$.
	\jessiet{Probably appendix later; also, should this be a lemma instead of a proposition?}
  \raft{Do we define ``continuous loss''?  Maybe ``(not necessarily discrete)'' is clearer.}
\end{proposition}
\begin{proof}
	We show the contrapositive.
	If $(L, \psi)$ are not calibrated with respect to $\ell$, then there is a sequence $\{u_m\}$ such that $R_L(u_m; p) \to 0$ but $R_\ell(\psi(u_m); p) \not \to 0$ via Lemma~\ref{lem:calib-converging-regrets}.
	Suppose $D \sim \X \times\Y$ has only one $x \in \X$ with $Pr_D(X = x) > 0$ so that $p := D_x$ and $\E_D f(X,Y) = \E_p f(x, Y)$.
	Consider any sequence of functions $\{f_m\} \to f$ with $f_m(x) = u_m$ for all $f_m$.
  \raft{Type error with $D=D_x$; see AA15 for how they phrase \jessie{better?}}
	Now we have $\E_D L(f_m(X), Y) \to \inf_f \E_D L(f(X), Y)$, but $\E_D \ell(\psi \circ f(X), Y) \not \to \inf_f \E_D \ell(\psi \circ f(X), Y)$, and therefore $(L, \psi)$ is not calibrated with respect to $\ell$. 
\end{proof}

\begin{lemma}\label{lem:calib-implies-indir}
	If a surrogate and link $(L, \psi)$ are calibrated with respect to a loss $\ell:\R \times\Y \to \reals$, then $L$ indirectly elicits the property $\gamma := \prop{\ell}$.
\end{lemma}
\begin{proof}
	Let $\Gamma$ be the unique property directly elicited by $L$, and fix $p \in \simplex$ with $u$ such that $p \in \Gamma_u$.
	As $p \in \Gamma_u$, then $\zeta(L(u;p) - \risk{L}(p)) = \zeta(0) = 0$, we observe the bound $\ell(\psi(u); p) \leq \risk{\ell}(p)$.
	We also have $\ell(\psi(u); p) \geq \risk{\ell}(p)$ by definition of $\risk{\ell}$, so we must have $\ell(\psi(u);p) = \risk{\ell}(p) = \ell(\gamma(p); p)$, and therefore, $p \in \gamma_{\psi(u)}$.
	Thus, we have $\Gamma_u \subseteq \gamma_{\psi(u)}$, so $L$ indirectly elicits $\gamma$.
\end{proof}

\begin{lemma}\label{lem:consistent-loss-implies-prop}
	Suppose we are given a surrogate loss $L$ and link $\psi$, and target loss $\ell$.
	Consider $\propdis: \R \times \simplex \to \reals_+$ with $\mu(r,p) := R_\ell(r,p)$.
	$(L, \psi)$ are consistent with respect to $\ell$ if and only if they are $\propdis$-consistent with respect to $\gamma := \prop{\ell}$.
	\bo{Change to if and only if. Given $L,\psi$, define $\mu$, then consistent iff consistent.}\jessie{Thoughts?}
\end{lemma}
\begin{proof}
	First, observe that $\propdis(r,p) = 0 \iff r \in \gamma(p)$, since $r \in \gamma(p)$ implies $\ell(r;p) = \risk{\ell}(p)$, and if $r \not \in \gamma(p)$, then $\ell(r;p) > \risk{\ell}(p)$.
	
	Now suppose $(L, \psi)$ are consistent with respect to $\ell$, and take any sequence $\{f_m\}$ so that, for all $D$ over $\X \times \Y$, we have $\E_D L(f_m(X)) \to \inf_f \E_D L(f(X), Y)$.
	This implies 
	\begin{align*}
	&\; \E_D \ell(\psi \circ f_m(X), Y)\to \inf_f \E_D \ell(\psi \circ f(X), Y) \\
	&\iff \E_X R_\ell(\psi \circ f_m(X), D_X) \to 0\\
	&\iff \E_X \propdis(\psi \circ f_m(X), D_X) \to 0~.~
	\end{align*}
\end{proof}

This result allows us to conceptualize the relationship between consistency and properties, and it suggests that our consistency with respect to a property is the correct notion to consider.


\begin{theorem}\label{thm:consistent-implies-indir-elic}
	If a surrogate and link pair $(L, \psi)$ is consistent with respect to a property $\gamma$ or loss $\ell$ eliciting $\gamma$, then $(L, \psi)$ indirectly elicits $\gamma$.
\end{theorem}
\begin{proof}
%  \raft{Maybe ref the consistency definition here explicitly, at the top and the end}
	As consistency with respect to $\ell$ implies consistency with respect to $\gamma := \prop{\ell}$ (Lemma~\ref{lem:consistent-loss-implies-prop}), it suffices to show the result for consistency with respect to a property $\gamma$ (Definition~\ref{def:consistent-prop}).
  
	We show the contrapositive; suppose $(L, \psi)$ does not indirectly elicit $\gamma$, and take $\Gamma := \prop{L}$.
%  \raft{Oh good!  I was going to say we should use this way of stating indirect elicitation (``$(L, \psi)$ indirectly elicits $\gamma$'') when we define it above, and use it.  We can then also define the version without $\psi$ easily. \jessie{Changed the definition of indirect elicitation and this Theorem statement.}}
	Then there is a distribution $p \in \simplex$ so that $u \in \Gamma(p)$ but $\psi(u) \not \in \gamma(p)$.

	Consider the constant sequence $\{u_m\}$ with $u_m = u$ for all $m$, and take $D \sim \X \times\Y$ with only one $x \in \X$ with $Pr_D(X = x) > 0$ so that $p := D_x$ and $\E_D g(X,Y) = \E_p g(x, Y)$ for all functions $g : \X \times \Y \to \reals$, and any sequence $\{f_m\}$ so that $f_m(x) = u_m$ for all $m$.
  \raft{Check $D$ wording here too \jessie{better?}}
  	Since $\{u_m\}$ is a constant sequence, we observe $\E_D L(f_m(X), Y) = \inf_f L(f(X),Y)$ for all $m$ since $u \in \Gamma(p)$.
	This yields $\E_D L(f_m(X), Y) \to \inf_f L(f(X),Y)$.
%  \raft{Maybe clearer to first say $\E_D L(f_m(X), Y) = \inf_f L(f(X),Y)$ for all $m$, and then the easy implication of the convergence \jessie{Done}}
	However, we have $\E_X \propdis(\psi \circ f_m(X), D_X) = \propdis(\psi(u_m), p) \not \to 0$ as $\propdis(\psi(u), p) \neq 0$ (since $\psi(u) \not \in \gamma(p)$) and the sequence is constant.
%  \raft{Almost -- just clarify that this is the expectation over $D$ too by construction \jessie{Kinda done?}}
	Thus, we observe $(L, \psi)$ is not consistent with respect to $\gamma$ (Definition~\ref{def:consistent-prop}).
\end{proof}

\section{Characteristics of Consistent Convex Surrogates}\label{sec:char-convex}

When one restricts their focus to \emph{convex} surrogates, we know that all minima are global, and we then have $\vec 0  \in \partial L(u, y)$ if and only if $u$ minimizes $L$ in its first component.
This observation allows us to consider subgradient sets of the loss at a fixed distribution $p$ as the weighted Minkowski sums of subgradient sets for the loss on each outcome.
Fixing a distribution $p$, we can then construct a set of $n$ witnesses in $\partial L(u,y)$ for each $y$ whose $p$-weighted minkowski sum is $\vec 0$.
The nullspace of the matrix of these witnesses yields a flat, formally introduced below, of codimension $d$ intersecting our fixed distribution $p$ and the level set(s) $\gamma_r$ containing $p$.

\begin{definition}[Flat]
	A \emph{flat} $F$ of $\reals^n$ is an affine subspace of $\reals^n$.
	In other words, $F\subseteq \reals^n$ can be written $F=\{x\in\reals^n : Wx + b = 0\}$ for some $W\in\reals^{d\times n}$ and $b\in\reals^d$ where $d\leq n$.
	The codimension of the flat $F$ relative to $\reals^n$ is given by $\mathrm{rank}(W)$, and the dimension of $F$ is $n - \mathrm{rank}(W)$.
\end{definition}

Now we can consider specific characteristics of convex losses in order to understand the convex elicitation complexity of a given property.

\begin{theorem}\label{thm:cvx-flats}
	Suppose we are given a property $\gamma$ and distribution $p \in \simplex$.
	For all $r\in\gamma(p)$, if there is no flat $F$ of codimension at most $d$ containing $p$ so that $F \cap \simplex \subseteq \gamma_r$, then there is no convex surrogate loss $L : \reals^d \times \Y \to \reals$ that indirectly elicits $\gamma$.\jessie{Do we still want to state it as the contrapositive?  That's where the bound comes from, but we use the original direction for all of the intuitive explanation.}
\end{theorem}
\begin{proof}[Proof]
	We prove the contrapositive: if there is a surrogate loss $L$ indirectly eliciting $\gamma$, then for all $p \in \simplex$ and $r \in \R$ such that $r \in \gamma(p)$, there is an $d$-codimensional flat $F$ such that $F \cap \simplex \subseteq \gamma_r$ and $p \in F$.
	
	Suppose we have $L:\reals^d \times \Y \to \reals$ eliciting $\Gamma$ and link $\psi : \reals^d \to \R$ such that for all $u \in \reals^d$, we have $\Gamma_u \subseteq \gamma_{\psi(u)}$, i.e., $L$ indirectly elicits $\gamma$.
	Fix $p \in \simplex$, and take any $u \in \Gamma(p)$ such that $\psi(u) = r \in \gamma(p)$.
	We know that $u \in \Gamma(p) \iff \vec 0 \in \partial L(u; p) \iff \vec 0 \in \oplus_y p_y \partial L(u,y)$, where $\oplus$ denotes the nonnegative weighted Minkowski sum~\cite[Theorem 4.1.1]{hiriart2012fundamentals}.  (Observe that in order for $L$ to be convex, $L(\cdot, y)$ must be convex for all $y \in \Y$.) 
	
	Observe that the Minkowski sum $\oplus_y p_y \partial L(u,y) = \{p_1 x_1 + \ldots + p_n x_n : x_1 \in \partial L(u,1), \ldots, x_n \in \partial L(u,n) \}$.
	In order for $\vec 0$ to be in this $p$-weighted Minkowski sum, we then have that each choice of $x_i$ is a flat for $\partial L(u, i)$.
	Moreover, with $p$ fixed, we know there is a vector of witnesses $W = [x_1, \ldots, x_n] \in \reals^{d\times n}$ such that $\sum_y p_y x_y = \vec 0$.
	Now we can take $F := \ker(W)$, whose rank is at most $d$.

	We also claim that $F \cap \simplex \subseteq \gamma_r$.
	To see this, consider $q \in F$, but $q \not \in \gamma_r$.
	That would mean that $Wq + b = \vec 0$ , which in turn implies $\vec 0 \in \partial L(u;q)$, implying $q \in \Gamma_u$.
	Therefore, $q \not \in \gamma_r$ would contradict $L$ indirectly eliciting $\gamma$.
\end{proof}

Since Theorem~\ref{thm:consistent-implies-indir-elic} says that consistency implies indirect elicitation, we can now apply results on convex elicitation complexity, like the result above, to understand the minimal dimension required to construct a consistent surrogate loss.
Thus, elicitation complexity greater than $d$ implies no indirect elicitation of the property of interest (or target loss eliciting such a property), which in turn implies there is no $d$-dimensional consistent surrogate with respect to a given loss or property of interest.

\section{Discrete-valued predictions}\label{sec:finite-calib}

\bo{Where do we discuss the chain from consistency to indirect elicitation ... how much to talk about our general calibration definition, and general consistency definitions?}

When one is given a discrete target loss,~\cite{ramaswamy2016convex} give lower bounds on the convex calibration dimension, or equivalently the dimension of a convex and consistent surrogate, by constructing what they call the subspace of feasible dimensions and giving bounds in terms of its dimension.
\begin{definition}[Subspace of feasible directions]
	Define the \emph{subspace of feasible directions} $\S_\C(p)$ of a convex set $\C \subseteq \reals^n$ at a point $p \in \C$ as the subspace $\S_\C(p) = \{ v \in \reals^n : \exists \epsilon_0 > 0 $ such that $p + \epsilon v \in \C \; \forall \epsilon \in (-\epsilon_0,\epsilon_0) \}$.
%  \raf{Let's simplify to $\epsilon \in (-\epsilon_0,\epsilon_0)$}
\end{definition}

We regain the bounds provided by~\cite[Theorem 16]{ramaswamy2016convex}, but have to do a little bit of groundwork to translate from our Theorem~\ref{thm:cvx-flats} and their result, restated in Corollary~\ref{cor:fsd-bound}.
First, in Lemma~\ref{lem:feas-sub-is-a-flat}, we show that, for a fixed $p$ and report $r$ so that $r \in \gamma(p)$, the feasible subspace $\S_{\gamma_r}(p)$ is a flat satisfying the requirements of Theorem~\ref{thm:cvx-flats}.
Therefore, we have the dimension of $\S_{\gamma_r}(p)$ being upper bounded by $d$, at least for $p \in \relint{\simplex}$. 
\begin{lemma}\label{lem:feas-sub-is-a-flat}
	Suppose we have the finite elicitable property $\gamma$ and distribution $p \in \relint{\simplex}$ with $r \in \gamma(p)$.
	If $F$ is a flat in $\mathrm{affhull}(\simplex)$ containing $p$ such that $F \cap \simplex \subseteq \gamma_r$, then $F - p$ is a subspace contained in $\S_{\gamma_r}(p)$.
%  \raf{I think you want $p$ in the interior of the simplex for now}
\end{lemma}
\begin{proof}
  Observe $F-p$ is a subspace as it is a linear shift of $F$, which is a linear subspace by definition of a flat and the fact that it contains $\vec 0$.
  Now consider $v \in F - p$.
  Since $p \in \relint{\simplex}$, there is an open ball of radius $\epsilon$ in the affine hull of $\simplex$ so that for all $q \in B(p, \epsilon)$, we have $q \in \simplex$.
  In particular, take $\alpha = \epsilon / 2$, and we observe $p \pm \alpha v \in B(p, \epsilon)$, and therefore $p \pm \alpha v \in \simplex$.
  Moreover, by the assumption $v \in F - p$, we also have $p \pm \alpha v \in \gamma_r$. 
  Since level sets of elicitable properties are convex (\cite{lambert2009eliciting}) this is true for all $\alpha' \leq \alpha$.
  Therefore, we observe $v \in S_{\gamma_r}(p)$, so $F-p \subseteq S_{\gamma_r}(p)$.
%  First, if $p+v \in \gamma_r$ and $p -v \in \gamma_r$, then we have $v \in \S_{\gamma_r}(p)$ with $\epsilon_0 = 1$ as level sets of elicitable properties are convex by~\cite{lambert2009eliciting}.
%  If either $p + v$ or $p - v \not \in \gamma_r$, it must be because the term is out of the simplex by definition of $F$.
%  
%  However, if both $p + \alpha^+ v$ and $p - \alpha^- v \in \simplex$ for some $\alpha^\pm \in (0,1)$, then $v \in \S_{\gamma_r}(p)$ with $\epsilon_0 = \min(\alpha^+, \alpha^-)$.
%  As $p \in \relint{\simplex}$, there is always such an $\epsilon_0$; if there were not, then we would observe $v \not \in F - p$.
%  Therefore, we have $v \in F - p \implies v \in \S_{\gamma_r}(p)$, so $(F - p) \subseteq \S_{\gamma_r}(p)$.
\end{proof}

The next result helps us generalize Lemma~\ref{lem:feas-sub-is-a-flat} to any $p$; not just to $p \in \relint{\simplex}$.
\newcommand{\simplexp}{\Delta_{\Y'}}
\begin{lemma}\label{lem:p-boundary-fsd}
	For any $p \in \simplex$ and $r$ such that $p \in \gamma_r$, take $\Y' := \supp(p)$.
	Define $\gamma' : \simplex \toto \R$ with $\gamma' : q \mapsto \gamma(q) \cap \simplexp$.
	Then $\dim(\S_{\gamma_r}(p)) = \dim(\S_{\gamma'_r}(p))$.
\end{lemma}
\begin{proof}
	Consider the ambient space of both $\S_{\gamma_r}(p)$ and $\S_{\gamma'_r}(p)$ is $\reals^\Y$.
	We trivially have $\dim(\S_{\gamma_r}(p)) \geq \dim(\S_{\gamma'_r}(p))$ since $\gamma'$ is simply $\gamma$ projected down to an affine subspace of $\reals^\Y$.
	
	Now to see $\dim(\S_{\gamma_r}(p)) \leq \dim(\S_{\gamma'_r}(p))$, it suffices to show subset inclusion.
	Take some $v \not \in \S_{\gamma'_r}(p)$.
	Observe that $\gamma'_r = \gamma_r \cap \simplexp$, so if $q^\pm := p \pm \epsilon v \not \in \gamma'_r$ for all $\epsilon > 0$, it is either because one of $q^+$ or $q^-$ is not in $\gamma_r$ or because either $q^\pm \not \in \simplexp$.
	The first case can be seen easily by the definition of $\gamma'$, and the latter can be seen because leaving $\simplexp$ means one of $q^\pm$ also is not in $\simplex$, and therefore not in $\gamma_r$ as $\gamma_r \subseteq \simplex$.
	Thus $\S_{\gamma_r}(p) = \S_{\gamma'_r}(p)$.
	%commented out 05.26.2020
	%	For simplicity, consider $\epsilon = \epsilon_0 / 2$.
%	Consider $P := \spn(\{e_y : y \in \supp(p) \})$.
%	\begin{align*}
%	\dim(\S'_{\gamma_r}(p)) &= \dim(\S_{\gamma_r}(p)) + \dim(P) - \dim(\S_{\gamma_r}(p) + P)
%	\end{align*}
%	If $\dim(P) = \dim(\S_{\gamma_r}(p) + P)$, the claim holds.
%	Be definition of $P$, we know $\dim(P) = \|p\|_0$, so we can show $\dim(\S_{\gamma_r}(p) + P) = \|p\|_0$.		
%	One way to do this is to show $\S_{\gamma_r}(p) \subseteq P$.
%	
%	Suppose there was some $v \not\in P$ such that $v \in \S_{\gamma_r}(p)$.
%	That means there is some $i$ so that $i \not \in \supp(p)$ (i.e. $p_i = 0$) but $v_i \neq 0$. \jessiet{Check this line.}
%	However, if $v \in \S_{\gamma_r}(p)$, then there is some $\epsilon_0 > 0$ so that $p + \epsilon v$ and $p - \epsilon v \in \gamma_r \subseteq \simplex$ for all $\epsilon \in (0, \epsilon_0)$.
%	For simplicity, consider $\epsilon = \epsilon_0 / 2$.
%	Since $p_i = 0$, this gives $(p+\epsilon v)_i = (\epsilon v)_i$, and likewise for $(p - \epsilon v)_i$.
%	As neither $\epsilon$ nor $v_i$ are $0$, one of these terms must be negative, and therefore not in the simplex.
%	As $\gamma_r \subseteq \simplex$, we then have $v \not \in \S_{\gamma_r}(p)$.
%	Therefore, $\S_{\gamma_r}(p) \subseteq P$, and $\dim(\S_{\gamma_r}(p) + P) = \dim(P) = \|p\|_0$.
\end{proof}

We can now understand the subspace of feasible directions as a special case of a flat described by Theorem~\ref{thm:cvx-flats}.
The following statement from~\cite{ramaswamy2016convex} provides an upper bound on the convex calibration dimension of a given discrete loss, and now follows as a Corollary of our Theorem~\ref{thm:cvx-flats} and Lemma~\ref{lem:feas-sub-is-a-flat}.

\begin{corollary}[\cite{ramaswamy2016convex}]\label{cor:fsd-bound}
	Suppose we are given a discrete loss $\ell:\R \to \reals^\Y_+$ eliciting $\gamma$.
	Take $p \in \simplex$ and $r \in \R$ such that $p \in \gamma_r$.
	\begin{equation}
	\ccdim(\ell) \geq \|p\|_0 - \dim(\S_{\gamma_r}(p)) - 1~.~
	\end{equation}
\end{corollary}
\begin{proof}
	Let $L : \reals^d \times \Y \to \reals$ be a calibrated surrogate for $\ell$, and consider $\Y' := \supp(p)$ and what happens when we restrict $L$ and $\ell$ to only the outcomes in $\Y'$.
	Take $L' := L|_{\Y'}$ and $\ell' := \ell|_{\Y'}$.
	
	First, observe $L'$ (eliciting $\Gamma$) indirectly elicits $\gamma' := \prop{\ell'}$ since, for all $p \in \simplex$, and therefore all $p \in \Delta_{\Y'}$, we have $p \in \Gamma_u \implies p \in \gamma_{\psi(u)}$, and $\Gamma(p) = \Gamma'(p)$ and $\gamma(p) = \gamma'(p)$ for $p \in \Delta_{\Y'}$.
	This same observation can be used to observe that $L'$ is also calibrated with respect to $\ell'$ as the calibration bound holds for all $p \in \simplex$, and therefore for all $p \in \Delta_{\Y'}$ by equality of $\gamma$ and $\gamma'$ for $p \in \Delta_{\Y'}$.
	
	As $L'$ is calibrated with respect to $\ell'$ and indirectly elicits $\gamma' := \prop{\ell'}$, then by Theorem~\ref{thm:cvx-flats}, we know there exists a flat $F'$ of codimension $d$ relative to $\reals^{\Y'}$ so that $\dim(F') = \|p\|_0 - d - 1$.
	Moreover, as $d$ is the maximal codimension of such a flat, we show in Lemma~\ref{lem:feas-sub-is-a-flat} that $\S_{\gamma_r}(p)$ is one such flat, and we have $\dim(\S_{\gamma_r}(p)) \geq \dim(F') = \|p\|_0 - d- 1$.
	Lemma~\ref{lem:p-boundary-fsd} then states $\dim(\S_{\gamma'_r}(p)) = \dim(\S_{\gamma_r}(p))$, and from there we observe the result.
%	First, for intuition, consider $p \in \relint{\simplex}$. 
%	Theorem~\ref{thm:cvx-flats} implies that there exists a flat $F'$ of affine dimension at least $n-d-1$.
%	Lemma~\ref{lem:feas-sub-is-a-flat} then says that $\dim(\S_{\gamma_r}(p)) \geq \dim(F') \geq n-d-1 \implies d \geq n - \dim(\S_{\gamma_r}(p)) - 1$.
%
%	
%	If we can show that $L' := L|_{\Y'}$ is calibrated with respect to $\ell' := \R \times \Y' \to \reals$ with $\ell(r,y) = \ell'(r,y)$ for all $r \in \R$ and $y \in \Y'$ and indirectly elicits $\gamma' := \prop{\ell'}$, then we observe the existence of a flat $F^*$ of codimension $d$ (in $\reals^{\Y'}$) so that $\dim(F^*) = \|p\|_0 - d - 1$ by Theorem~\ref{thm:cvx-flats}.
%	Then we can use Lemma~\ref{lem:p-boundary-fsd} to observe $\dim(\S_{\gamma'_r}(p)) = \dim(\S_{\gamma_r}(p))$, and thus the result holds.
%	
%	Now to see that $L'$ is calibrated with respect to $\ell'$, consider that $\gamma(p) = \gamma'(p)$ for all $p \in \Delta^{\Y'}$,
%	\begin{align*}
%		\forall p \in \Delta_{\Y'}:  \inf_{u \in \reals^d : \psi(u) \not \in \gamma'(p)} L'(u;p) = \inf_{u \in \reals^d : \psi(u) \not \in \gamma(p)} L(u;p) > \inf_{u \in \reals^d} L(u;p) = \inf_{u \in \reals^d} L'(u;p)~.~
%	\end{align*}
%	Therefore, we have $L'$ calibrated with respect to $\ell'$.
%
%	Moreover, we want to show $L'$ indirectly elicits $\gamma'$ so we can apply Theorem~\ref{thm:cvx-flats}.
%	First, observe that $L'$ elicits $\Gamma' : p \mapsto \Gamma(p)$ for all $p \in \Delta_{\Y'}$ 
%	For all $p \in \simplex$, we have $u \in \Gamma(p) \implies \psi(u) \in \gamma(p)$.
%	As $\Delta_{\Y'} \subseteq \simplex$, this is in particular true for all $p \in \Delta_{\Y'}$.
%	Therefore, for all $p \in \Delta_{\Y'}$, we have $\Gamma'_u = \Gamma_u \cap \Delta_{\Y'} \subseteq \gamma_{\psi(u)} \cap \Delta_{\Y'} = \gamma'_{\psi(u)}$, and thus $L'$ indirectly elicits $\gamma'$.
\end{proof}


\section{Continuous-valued predictions}\label{sec:contin-consis}

Theorem~\ref{thm:cvx-flats} also addresses one major open question from~\cite{frongillo2015elicitation,frongillo2018elicitation}\jessiet{which citation?}: \emph{what are lower bounds on convex elicitation complexity?}
Since proposition~\ref{prop:consistent-implies-indir-elic-prop} allows us to use convex elicitation complexity as a tool to understand efficiency of consistent convex surrogates for a given property, which is more often what is given in a continuous prediction setting.

For example, when one wants to learn an $\alpha$-quantile, we start with the property rather than a loss.
In the literature, when one wants to learn a quantile, pinball loss $L(r,y) = (r-y)(\mathbbm{1}_{r \geq y} - \alpha)$ typically appears without explanation or justification.
Elicitation allows us to understand why pinball loss is consistent for learning a quantile as the pinball loss elicits the $\alpha$-quantile.

\newcommand{\lbar}{\underline{L}} % couldn't do L* while proofreading...
\newcommand{\iden}{\mathrm{iden}}
\newcommand{\Var}{\mathrm{Var}}

\begin{lemma}[\cite{frongillo2018elicitation}]
  \label{lem:elic-complex-bayes-concave}
  Suppose the loss $L$ elicits $\Gamma:\simplex\to\R$.
  Let $\lbar$ be the Bayes risk of $L$.
  Then for any $p,p'\in\simplex$ with $\Gamma(p)\neq\Gamma(p')$, we have $\lbar(\lambda p + (1-\lambda) p') > \lambda \lbar(p) + (1-\lambda) \lbar(p')$ for all $\lambda\in(0,1)$.
\end{lemma}

\begin{definition}
  A property $\Gamma:\simplex\to\R$ is \emph{$d$-identifiable} if its level sets are flats of co-dimension at most $d$ intersected with $\simplex$.
  \raft{Update to be consistent with our definitions / notation / conventions.  I'd vote for using co-dimension since it makes so much of the analysis easier...  Could make more formal too.}
  We write $\iden(\Gamma) = \min\{d \in \mathbb{N} : \Gamma\text{ is $d$-identifiable}\}$.
\end{definition}
\jessiet{Flats or unions of flats?}

\raf{A modification of the argument of \citet[Corollary 7]{frongillo2018elicitation}}

\begin{theorem}
  \label{thm:bayes-risk-lower-bound}
  Let $L$ elicit some identifiable $\Gamma:\simplex\to\reals^d$ with $\iden(\Gamma)=d$.
  If $\lbar$ is non-constant on every non-singleton level set of $\Gamma$, then $\eliccvx(\lbar) \geq \min(d+1,n-1)$.
\end{theorem}
\begin{proof}
  Suppose for a contradiction that we have some convex-elicitable property $\hat\Gamma:\simplex\to\reals^d$ and link $\psi : \reals^d \to \reals$ such that $\lbar = \psi \circ \hat\Gamma$.
  The condition $\iden(\Gamma)=d$ implies that $\Gamma$ is $d$-identifiable but not $(d-1)$-identifiable, and thus there must be some level set $\Gamma_r$ which is a flat of dimension $n-d-1$.
  \raft{What I mean is you can't describe $\Gamma_r$ using a flat of another dimension; revisit once we settle terminology}

  The proof of \citet[Theorem 4]{frongillo2018elicitation} argues that $\hat\Gamma$ must refine $\Gamma$, in the sense that every level set of $\hat\Gamma$ is contained in a level set of $\Gamma$; for completeness we give the argument here.
  Suppose for a contradiction that we have $p,p'$ with $\hat\Gamma(p)=\hat\Gamma(p')$ but $\Gamma(p) \neq \Gamma(p')$.
  As $\lbar = \psi \circ \hat\Gamma$, we also have $\lbar(p) = \lbar(p')$.
  Letting $p'' = \tfrac 1 2 p +  \tfrac 1 2 p'$, Lemma~\ref{lem:elic-complex-bayes-concave} would then give us $\lbar(p'') >  \tfrac 1 2 \lbar(p) +  \tfrac 1 2 \lbar(p') = \lbar(p)$.
  By \citet{osband1985providing}, the level sets $\hat\Gamma_{\hat r}$ are convex, giving $\hat\Gamma(p'') = \hat\Gamma(p)$, which would imply $\lbar(p'')=\lbar(p)$, contradicting $\lbar = \psi \circ \hat\Gamma$.
  We conclude $\hat\Gamma$ must refine $\Gamma$, and in particular, we must have $\hat\Gamma_{\hat r} \subseteq \Gamma_r$ for some $\hat r\in\hat\Gamma(\simplex)$.
  % The proof of \cite[Theorem 4]{frongillo2018elicitation} argues that $\hat\Gamma$ must refine $\Gamma$, in the sense that for all $\hat r \in \hat\Gamma(\simplex)$ we have $\hat\Gamma_{\hat r} \subseteq \Gamma_r$ for some $r\in\reals^k$.
  
  By Theorem~\ref{thm:cvx-flats}, as $\hat\Gamma$ is $d$-convex elicitable $\hat\Gamma_{\hat r}$ is the union of flats of dimension $n-d-1$ interesected with $\simplex$.
  \raft{Should flesh out this next step; this is the key step}
  As $\hat\Gamma_{\hat r} \subseteq \Gamma_r$ and $\Gamma_r$ is a flat of dimension $n-d-1$ interesected with $\simplex$, we conclude $\hat\Gamma_{\hat r} = \Gamma_r$.
  Since by assumption $\lbar$ is non-constant on $\Gamma_r = \hat\Gamma_{\hat r}$, we have distributions $p,p' \in\hat\Gamma_{\hat r}$ with $\lbar(p)\neq\lbar(p')$, which again contradicts $\lbar = \psi \circ \hat\Gamma$.
  \raft{We actually needed $\Gamma_r$ to be a non-singleton for this last step; otherwise we need to conclude the $n-1$ case; revisit after flat/dimension stuff}
\end{proof}


\subsection{Examples \raf{all informal for now}}

Both these results exactly match what Ian and I showed for the class of identifiable properties, which is super interesting...
\begin{itemize}
\item Variance: $\iden(\E Y) = 1$, and over $\simplex$ every non-singleton level set of $\E Y$ has different variances, so the theorem gives $\eliccvx(\Var) \geq 2$.  We then have the matching upper bound since $L(r,y) = (r_1-y)^2 + (r_2-y^2)^2$ is convex in $r$, and we can link to $\Var$ from there.
\item Entropy and norms: In fact, \emph{any} strictly convex/concave function $G$ of $p$ will do here.  You can write it as the Bayes risk (or negative Bayes risk) of a strictly proper scoring rule, which elicits the full distribution $\Gamma(p) = p$.  Since $\iden(\Gamma) = n-1$ since you don't actually need the $n$th coordinate, you have $\eliccvx(G) = n-1$; the upper bound comes from the fact that you can elicit the full distribution with a convex loss and just compute $G$.
\end{itemize}


\section{Conclusions and future work}\label{sec:conclusions}
In this work, we show that indirect property elicitation can be studied as a necessary condition for the existence of a consistent surrogate loss (Theorem~\ref{thm:consistent-implies-indir-elic}).
Furthermore, we introduce a new lower bound (Theorem~\ref{thm:cvx-flats}) on the dimension of a consistent convex loss that is generally applicable and extends previous results from both the discrete prediction and continuous estimation settings.
Bounding the prediction dimension on convex surrogate losses can yield possibly significant improvements in the complexity of the optimization algorithm as a whole.

While this work tightens bounds on the dimension of consistent surrogate losses, it does not completely characterize the dimensionality of a convex surrogate for a given problem.
One might be able to further tighten these bounds by property elicitation by studying monotonicity and adjacency of level sets.
In finite predictions, these bounds might also be tightened if the equivalence of convex calibration dimension and embedding dimension of~\cite{finocchiaro2020embedding} is shown; the current embedding dimension bounds are not tight either, but structure is imposed by considering the embedding framework.
Lastly, the practical reason why consistency is desired is to ensure the guarantee of empirical risk minimization (ERM) rates; however, the relationship between ERM rates and property elicitation has not been studied.


\newpage

\section*{Broader Impact}
This work is entirely theoretical, thus the opportunity for broader impacts is indirect, but still present.
Understanding the complexity and efficiency of consistent, convex surrogates for different target losses and properties we wish to learn can lead to understanding the tradeoffs that are often made in ad-hoc surrogates that are not consistent.
These ad-hoc surrogates are consistent with respect to \emph{some property}, although it may not be the one we wish to elicit, either directly or indirectly.
As a general prediction task arises, one might be able to use property elicitation to understand what question they are actually answering (in the best base, with sufficient data that matches the real-world data distribution) by minimizing their surrogate loss and how it differs from the question they are actually trying to answer.

\begin{ack}
Nishant, Adam
\end{ack}

\bibliographystyle{plainnat}
\bibliography{diss,extra}

\newpage
\appendix
\section{A general notion of calibration}\label{app:calibration}
For general settings, we introduce a new notion of calibration that is a special case of calibration as introduced by~\cite[Chapter 3]{steinwart2008support}.
We will show that in finite prediction settings, it is equivalent to the more commonplace definition given in Definition~\ref{def:calibrated-finite}.
Therefore, we use this more general definition of calibration when proving statements about the relationship between consistency, calibration, and indirect elicitation.

\begin{definition}[Calibrated]\label{def:calibrated-general}
	A loss $L:\reals^d \times \Y \to \reals$ is \emph{calibrated} with respect to a loss $\ell : \R \times \Y \to \reals$ eliciting the property $\gamma$ if there is a link $\psi : \reals^d \to \R$ such that, for all distributions $p \in \simplex$, there exists a function $\zeta : \reals_+ \to \reals_+$ with $\zeta$ continuous at $0^+$ and $\zeta(0) = 0$ such that for all $u \in \reals^d$, we have
	\begin{equation}\label{eq:calibrated-general}
	\ell( \psi(u); p) - \risk{\ell}(p)  \leq \zeta \left(  L(u;p) - \risk{L}(p) \right)~.~
	\end{equation}
\end{definition}

Consider the following four conditions: Suppose we are given $\zeta:\reals_+ \to \reals_+$.
\begin{enumerate}
	\item [A] $\zeta$ satisfies $\zeta : 0 \mapsto 0$ and is continuous at $0$.
	\item [B] $\epsilon_m \to 0 \implies \zeta(\epsilon_m) \to 0$.
	\item [C] Given $\zeta:\reals \to \reals_+$, for all $u \in \reals^d$, $R_\ell(\psi(u); p) \leq \zeta(R_L(u;p))$.
	\item [D] For all $p \in \simplex$ and sequences $\{u_m\}$ so that $R_L(u_m; p) \to 0$, we have $R_\ell(\psi(u_m); p) \to 0$.
\end{enumerate}
The existence of a function $\zeta$ so that $(A \wedge C)$ defines calibration as in Definition~\ref{def:calibrated-general}, and we show $A \iff B$ in Lemma~\ref{lem:continuous-iff-limits}.  
Lemma~\ref{lem:calib-converging-regrets} shows calibration if and only if $D$, which yields a condition equivalent to calibration without dependence the function $\zeta$.

\begin{proposition}
	When $\R$ is finite, a loss and link $(L, \psi)$ are calibrated with respect to a target loss $\ell$ via Definition~\ref{def:calibrated-general} if and only if calibration via Definition~\ref{def:calibrated-finite}.
\end{proposition}
\begin{proof}
	\jessiet{Old proof commented out; simplified using Lemma~\ref{lem:calib-converging-regrets}}
$\implies$
	We prove the contrapositive; if $(L, \psi)$ is not calibrated with respect to $\ell$ by Definition~\ref{def:calibrated-finite}, then it is not calibrated via Definition~\ref{def:calibrated-general} either.
	If $(L, \psi)$ are not calibrated with respect to $\ell$ by Definition~\ref{def:calibrated-finite}, then there is a $p \in \simplex$ so that $\inf_{u : \psi(u) \not \in \gamma(p)} L(u;p) = \inf_u L(u; p)$.
	Thus there is a sequence $\{u_m\}$ so that $\lim_{m \to \infty} \psi(u_m) \not \in \gamma(p)$ and $L(u_m; p) \to \risk{L}(p)$.  
	Now we have $R_L(u_m; p \to 0$ but $R_\ell(\psi(u_m); p) \not \to 0$, so by Lemma~\ref{lem:calib-converging-regrets}, we contradict calibration by Definition~\ref{def:calibrated-general}.
%	
% Commented out 05.19.2020 for easier proof if Lemma 5 is true.
%	We prove the contrapositive; if $(L, \psi)$ is not calibrated with respect to $\ell$ by Definition~\ref{def:calibrated-finite}, then it is not calibrated via Definition~\ref{def:calibrated-general} either.
%	
%	Suppose there was a distribution $p \in \simplex$ so that $\inf_{u : \psi(u) \not \in \gamma(p)} L(u;p) = \inf_{u} L(u;p)$.
%	There must then be a sequence $\{u_m\} \to u$ so that $\lim_{m \to \infty} \psi(u_m) \not \in \gamma(p)$ and $L(u_m; p) \to \risk{L}(p)$.
%	
%	This consequently implies $R_L(u_m;p) \to 0$ as $L(u_m; p) \to \risk{L}(p)$, but as $\ell(\psi(u_m); p) \not \to \risk{\ell}(p)$ (if it did converge, then we would have $\psi(u_m) \to r \in \gamma(p)$), so $R_\ell(\psi(u_m); p) \not \to 0$.
%	Thus, by Lemma~\ref{lem:calib-converging-regrets}, we have no calibration via Definition~\ref{def:calibrated-general}.  

$\impliedby$
Suppose there was a function $\zeta$ satisfying the bound in Equation~\eqref{eq:calibrated-general} for a fixed distribution $p \in \simplex$.
Observe the bound in Equation~\eqref{eq:calibration} can be written as $R_L(u,p) > 0$ for all $p \in \simplex$ and $u$ so that $\psi(u)$ is bounded away from $\gamma(p)$. \jessie{details correct??}

By Equation~\eqref{eq:calibrated-general}, for any sequence $\{u_m\}$ so that $\psi(u_m) \not \to \gamma(p)$, we have must have $\zeta(R_\ell(\psi(u_m), p)) \not \to 0$ as we would otherwise contradict the bound in Equation~\eqref{eq:calibrated-general} since $R_\ell(\psi(u), p) \not \to 0$. 
Therefore $R_L(u_m, p) \not \to 0$; thus, the strict inequality holds.
\end{proof}

The following Lemma shows that conditions $A$ and $B$ are equivalent, so that we can using condition $B$ in lieu of condition $A$ in the proof of Lemma~\ref{lem:calib-converging-regrets}
\begin{lemma}\label{lem:continuous-iff-limits}
	A function $\zeta:\reals \to \reals$ is continuous at $0$ and $\zeta(0) = 0$ if and only if the sequence $\{u_m\} \to 0 \implies \zeta(u_m) \to 0$.
	\jessie{$A \iff B$}
\end{lemma}
\begin{proof}
	$\implies$ Suppose we have a sequence $\{u_m\} \to 0$.
	By continuity, we have $\lim_{u_m \to 0}\zeta(u_m) = \zeta(0) = 0$, so $\zeta(u_m) \to 0$.
	
	$\impliedby$ Suppose $\zeta(0) \neq 0$ but $\zeta$ was continuous at $0$.
	The constant sequence $\{u_m\} = 0$ then converges to $0$, but as $\zeta$ is continuous at $0$, we must have $\lim_{m \to \infty}\zeta(u_m) = \zeta(0) \neq 0$, so $\zeta(u_m) \not \to 0$.
	
	Now suppose $\zeta(0) = 0$ but $\zeta$ was not continuous at $0$.
	There must be a sequence $\{u_m\} \to 0$ so that $\lim_{m \to \infty}\zeta(u_m) \neq \zeta(0) = 0$, so $\zeta(u_m) \not \to 0$.
\end{proof}

Lemma~\ref{lem:calib-converging-regrets} now gives a condition equivalent to calibration without requiring one to already have a function $\zeta$ in mind.
\begin{lemma}\label{lem:calib-converging-regrets}
	A continuous surrogate and link $(L,\psi)$ are calibrated (via definition~\ref{def:calibrated-general}) with respect to $\ell$ if and only if, for all $p \in \simplex$ and sequences $\{u_m\}$ so that $R_L(u_m; p) \to 0$, we have $R_\ell(\psi(u_m); p) \to 0$.
	\jessie{$(A \wedge C) \iff D$}
\end{lemma}
\begin{proof}
\jessie{$(A \wedge C) \implies D$}
	$\implies$ Take a sequence $\{u_m\}$ so that $R_L(u_m;p) \to 0$.
	Since $\zeta(0) = 0$ and $\zeta$ is continuous at $0$, we have $\zeta(R_L(u_m;p)) \to 0$.
	As the bound from Equation~\eqref{eq:calibrated-general} is satisfied for all $u \in \reals^d$ by assumption, we observe
	\begin{align*}
	\forall m, \; &0 \leq R_\ell(\psi(u_m); p) \leq \zeta(R_L(u_m;p))\\
	\implies &0 \leq \lim_{m \to \infty} R_\ell(\psi(u_m); p) \leq \lim_{m \to \infty} \zeta(R_L(u_m;p)) = 0\\
	\implies &0 = \lim_{m\to\infty} R_\ell(\psi(u_m); p) ~.~
	\end{align*}
	
	
	$\impliedby$ 
\jessie{$D \implies (A \wedge C)$}
	Fix $p \in \simplex$, and consider $\zeta(c) := \sup_{u: R_L(u;p) \leq c} R_\ell(\psi(u); p)$.  
	We will show $R_L(u_m; p) \to 0 \implies R_\ell(\psi(u_m); p) \to 0$ gives calibration via the function $\zeta$ constructed above. 
	With $\zeta$ as constructed, we should that the bound in equation~\eqref{eq:calibrated-general} is satisfied for all $u \in \reals^d$ and apply Lemma~\ref{lem:continuous-iff-limits} to observe that if there is a sequence $\{\epsilon_m\} \to 0$ so that $\zeta(\epsilon_m) \not \to 0$, it is because $R_L(u_m, p) \not \to 0 \not \implies R_\ell(\psi(u_m), p) \to 0$.
	

\jessie{D $\implies$ C}
Now, we observe that the bound in Equation~\eqref{eq:calibrated-general} is satisfied for all $u \in \reals^d$ by construction of $\zeta$.
Let $S(v) := \{u' \in \reals^d : R_L(u';p) \leq R_L(v,p) \}$.
Showing $R_\ell(\psi(u);p) \leq \sup_{u' \in S(u)} R_\ell(\psi(u') ; p)$ for all $u \in \reals^d$ gives the condition $C$.
As $u$ is in the space over which the surpremum is being taken (as $R_L(u;p) \leq R_L(u;p)$), we then have calibration by definition of the supremum.

\jessie{Not $B$ leads to contradiction of $D$.}
Now suppose there exists a sequence $\{\epsilon_m\} \to 0$ so that $\zeta(\epsilon_m) \not \to 0$.
Consider $S(\epsilon) = \{u \in \reals^d : R_L(u,p) \leq \epsilon\}$.

\begin{align*}
\epsilon_1 \leq \epsilon_2 &\implies S(\epsilon_1) \subseteq S(\epsilon_2)\\
&\implies \zeta(\epsilon_1) \leq \zeta(\epsilon_2)~.~
\end{align*}
Now suppose there exists a sequence $\{u_m\}$ so that $R_L(u_m, p) \to 0$.
Then for all $\epsilon > 0$, there exists a $m' \in \mathbb{N}$ so that $R_L(u_m, p) < \epsilon$ for all $m \geq m'$.
Since this is true for all $\epsilon$, we have $S(\epsilon)$ nonempty for all $\epsilon > 0$, and therefore $\zeta(c)$ is finite for all $c > 0$.
Now if $\zeta(\epsilon_m) \not \to 0$, it must be because $R_\ell(\psi(u_m), p) \not \to 0$ for some sequence converging to zero surrogate regret, and therefore we contradict the statement $R_L(u_m, p) \to 0 \implies R_\ell(\psi(u_m), p) \to 0$.

Moreover, we argue that such a sequence of $\{u_m\}$ with converging surrogate regret always exists by continuity and boundedness from below \jessiet{really just need lower semi-continuity and boundedness from below} of the surrogate loss, since we can take the constant sequence at the (attained) infimum.
%
%\jessie{$D \implies A$, but actually $\lnot B \wedge C \implies \lnot D$}
%Fix $p \in \simplex$.
%Suppose we have a sequence $\{\epsilon_m\}$ so that $\epsilon_m \to 0$, but $\zeta(\epsilon_m) \not \to 0$.
%If $L$ is continuous over the reals \jessiet{Need this, right?}, then for each $m$, we can construct a subsequence $\{u^j_m\}$ so that $R_L(u^j_m, p) \to \epsilon_m$ for all $m$.
%We can now construct the sequence $\{\epsilon'_m\}$ with $\epsilon'_m = \lim_{j \to \infty} R_L(u^j_m, p)$.
%This sequence converges to $0$, but we have $\lim_{m \to \infty} \zeta(\epsilon'_m) = \lim_{m \to \infty} \lim_{j \to \infty} \zeta(R_L(u^j_m, p)) \not \to 0$. 
%As $R_\ell(\psi(u_m); p) \leq R_L(u_m; p)$ for all $m$, we then have this being true in the limit.
%For this sequence, this then gives a loose bound of $0 \leq \lim_{m\to\infty} R_\ell(\psi(u_m); p) \leq c$. 
%	
\end{proof}

\section{Omitted Proofs}


\end{document}

%%% Local Variables:
%%% mode: latex
%%% TeX-master: t
%%% End:
