\documentclass{article}

% if you need to pass options to natbib, use, e.g.:
%     \PassOptionsToPackage{numbers, compress}{natbib}
% before loading neurips_2020

% ready for submission
% \usepackage{neurips_2020}

% to compile a preprint version, e.g., for submission to arXiv, add add the
% [preprint] option:
%     \usepackage[preprint]{neurips_2020}

% to compile a camera-ready version, add the [final] option, e.g.:
%     \usepackage[final]{neurips_2020}

% to avoid loading the natbib package, add option nonatbib:
\usepackage{neurips_2020}

\usepackage[utf8]{inputenc} % allow utf-8 input
\usepackage[T1]{fontenc}    % use 8-bit T1 fonts
\usepackage{hyperref}       % hyperlinks
\usepackage{url}            % simple URL typesetting
\usepackage{booktabs}       % professional-quality tables
\usepackage{amsfonts}       % blackboard math symbols
\usepackage{nicefrac}       % compact symbols for 1/2, etc.
\usepackage{microtype}      % microtypography
\usepackage{amsmath, amsfonts, amssymb, amsthm}

\usepackage{color}
\definecolor{darkblue}{rgb}{0.0,0.0,0.2}
\hypersetup{colorlinks,breaklinks,
	linkcolor=darkblue,urlcolor=darkblue,
	anchorcolor=darkblue,citecolor=darkblue}
\usepackage{wrapfig}
\usepackage{subcaption}
\usepackage[colorinlistoftodos,textsize=tiny]{todonotes} % need xargs for below
%\usepackage{accents}
\usepackage{bbm}
\usepackage{xspace}

\newcommand{\Comments}{1}
\newcommand{\mynote}[2]{\ifnum\Comments=1\textcolor{#1}{#2}\fi}
\newcommand{\mytodo}[2]{\ifnum\Comments=1%
	\todo[linecolor=#1!80!black,backgroundcolor=#1,bordercolor=#1!80!black]{#2}\fi}
\newcommand{\raf}[1]{\mynote{green}{[RF: #1]}}
\newcommand{\raft}[1]{\mytodo{green!20!white}{RF: #1}}
\newcommand{\jessie}[1]{\mynote{purple}{[JF: #1]}}
\newcommand{\jessiet}[1]{\mytodo{purple!20!white}{JF: #1}}
\newcommand{\proposedadd}[1]{\mynote{orange}{#1}}
\newcommand{\bo}[1]{\mynote{blue}{[Bo: #1]}}
\newcommand{\botodo}[1]{\mytodo{blue!20!white}{[Bo: #1]}}
\newcommand{\btw}[1]{\mytodo{gray!20!white}{[BTW: #1]}}%TURN OFF FOR NOW \mytodo{gray}{#1}}
\ifnum\Comments=1               % fix margins for todonotes
\setlength{\marginparwidth}{1in}
\fi

\newcommand{\reals}{\mathbb{R}}
\newcommand{\posreals}{\reals_{>0}}%{\reals_{++}}
\newcommand{\simplex}{\Delta_\Y}
\newcommand{\prop}[1]{\mathrm{prop}[#1]}
\newcommand{\eliccvx}{\mathrm{elic}_\mathrm{cvx}}
\newcommand{\elicpoly}{\mathrm{elic}_\mathrm{pcvx}}
\newcommand{\elicembed}{\mathrm{elic}_\mathrm{embed}}
\newcommand{\ccdim}{\mathrm{ccdim}}

\newcommand{\C}{\mathcal{C}}
\newcommand{\D}{\mathcal{D}}
\newcommand{\E}{\mathbb{E}}
\newcommand{\F}{\mathcal{F}}
\newcommand{\I}{\mathcal{I}}
\newcommand{\N}{\mathcal{N}}
\newcommand{\R}{\mathcal{R}}
\renewcommand{\S}{\mathcal{S}}
\newcommand{\U}{\mathcal{U}}
\newcommand{\X}{\mathcal{X}}
\newcommand{\Y}{\mathcal{Y}}


\newcommand{\risk}[1]{\underline{#1}}
\newcommand{\inprod}[2]{\langle #1, #2 \rangle}
\newcommand{\toto}{\rightrightarrows}

\newtheorem{theorem}{Theorem}
\newtheorem{lemma}{Lemma}
\newtheorem{proposition}{Proposition}
\newtheorem{corollary}{Corollary}
\newtheorem{conjecture}{Conjecture}
\newtheorem{definition}{Definition}
\newtheorem{assumption}{Assumption}


\DeclareMathOperator*{\argmax}{arg\,max}
\DeclareMathOperator*{\argmin}{arg\,min}
\DeclareMathOperator*{\arginf}{arg\,inf}
\DeclareMathOperator*{\sgn}{sgn}

\title{cvx-flats}

% The \author macro works with any number of authors. There are two commands
% used to separate the names and addresses of multiple authors: \And and \AND.
%
% Using \And between authors leaves it to LaTeX to determine where to break the
% lines. Using \AND forces a line break at that point. So, if LaTeX puts 3 of 4
% authors names on the first line, and the last on the second line, try using
% \AND instead of \And before the third author name.

\author{%
	Peter Bartlett\\
	UC Berkeley\\
	\texttt{bartlett@cs.berkeley.edu}
	\And
  Jessie Finocchiaro\\
%  Department of Computer Science\\
  CU Boulder\\
  \texttt{jefi8453@colorado.edu} 
  % examples of more authors
  \And
   Rafael Frongillo\\
%   Department of Computer Science\\
   CU Boulder\\
  % Address \\
   \texttt{raf@colorado.edu} 
   \And
   Bo Waggoner\\
%   Department of Computer Science\\
   CU Boulder \\
  % Address \\
   \texttt{bwag@colorado.edu} 
  % \And
  % Coauthor \\
  % Affiliation \\
  % Address \\
  % \texttt{email} \\
  % \And
  % Coauthor \\
  % Affiliation \\
  % Address \\
  % \texttt{email} \\
}



\begin{document}

\maketitle

\begin{abstract}
%	For finite prediction tasks, one can often naturally model their specific problem with a discrete loss.
%	However, these discrete losses are often hard to optimize, so one often seeks a surrogate losses yielding desirable statistical guarantees with respect to the original problem.
	In prediction tasks, one typically seeks to minimize \emph{empirical risk} by learning a hypothesis function making predictions from a given feature vector about a desired future outcome.
	This hypothesis is learned by minimizing a loss function over a set of labeled training data.
	However, as not all loss functions are easily optimized, one often wishes to optimize a \emph{surrogate loss} yielding the same statistical guarantees as the original.
	In this paper, we are specifically interested in the construction of \emph{consistent} surrogate losses that are ``efficient'' in the dimension of their input.
	We observe a connection between consistent surrogates and indirect property elicitation, which allows us to apply property elicitation complexity results to the minimal dimension of a consistent surrogate.
	In particular, we characterize the minimal input dimension needed in order to construct a \emph{convex}, consistent surrogate using tools from property elicitation.
\end{abstract}

\section{Introduction}\label{sec:intro}

In supervised machine learning, one often wants to make a prediction about future outcomes by training a classifier to minimize the average \emph{empirical loss} of a labeled training set, where the loss is determined by the task at hand.
For example, 0-1 loss is often desired for classification tasks.
However, discontinuous, finite losses are difficult to optimize, so we typically construct a \emph{surrogate} loss that one can more easily optimize.
We are specifically interested in \emph{consistent} and \emph{convex} surrogates yielding the same statistical guarantees as optimizing an original, discrete loss.
\jessie{......}

\subsection{Notation}
In this paper, we take the outcome set $\Y$ with $|\Y| = n < \infty$.
Let $\simplex$ be the $(n-1)$ dimensional simplex over the outcomes in $\Y$.

%discrete land
Let $\R$ be a finite report set and $\Y$ a finite outcome set.
Note that we do not necessarily have $\Y = \R$.  
For example, in ranking problems, $\R$ may be all $n!$ permutations over the $n$ outcomes forming $\Y$.
We consider a discrete loss $\ell : \R \to \reals_+^\Y$ to map a discrete prediction to a real-valued punishment for error on each possible outcome.
We denote $\ell(r)_y$ as the $y^{th}$ element of the vector $\ell(r)$, and consider this to be the punishment for reporting $r$ if outcome $y$ is realized.
For a distribution $p \in \simplex$, we denote the expected discrete loss for report $r$ to be $\inprod{p}{\ell(r)}$.
When considering surrogate losses, we denote the loss by $L : \reals^d \to \reals_+^\Y$, and typically denote a surrogate report with $u$.

We sometimes override the notation $\R$ to refer to a generic report set.
Throughout this paper, we use tools from \emph{property elicitation} to understand the existence of consistent surrogate functions.
Introduced in Section~\ref{subsec:properties}, a (set-valued) property $\Gamma: \simplex \to 2^\R \setminus \emptyset$ is denoted $\Gamma:\simplex \toto \R$.
We denote a finite property elicited by a discrete loss $\ell$ by $\gamma := \prop{\ell}$.
Additionally, we let $\Gamma := \prop{L}$ be the ``surrogate'' property elicited by the surrogate loss $L$.
Again, we sometimes override notation to let $\Gamma$ be a generic property; when $\Gamma: \simplex \toto \R$, we refer to both a generic property and report set.

\section{Related work}\label{sec:related-work}
\subsection{Consistency and calibration for convex losses}\label{subsec:convex-surrogates}
\cite{bartlett2006convexity,tewari2007consistency} for a characterization of consistent, calibrated surrogates for classification problems.

\begin{definition}[Consistent]
	Suppose $f_m : \X \to \R$ is the hypothesis function learned by minimizing empirical training loss over $m$ training examples.
	For a distribution $D$ over input and label spaces $\X \times\Y$, we say $\{f_m\}$ is \emph{consistent} with respect to a loss $L: \R \to \reals^\Y_+$ if
	\begin{align*}
	\E_D L(f_n(X), Y) &\to \inf_f \E_D L(f(X), Y)~.~
	\end{align*}
	Moreover, we say a loss and link $(L,\psi)$ are consistent with respect to a discrete loss $\ell$ if 
	\begin{align*}
		\E_D L(f_n(X), Y) \to \inf_f \E_D L(f(X), Y) &\implies \E_D \ell(\psi \circ f_n(X), Y) \to \inf_f \E_D \ell(\psi \circ f(X), Y)~.~
	\end{align*}
\end{definition}



\cite{frongillo2015elicitation} introduces \emph{Elicitation Complexity}. \jessie{\ldots}

\cite{finocchiaro2019embedding} presents the notion of a surrogate loss \emph{embedding} a discrete loss-- typically by a polyhedral (piecewise linear and convex) surrogate.
For all discrete losses, $\eliccvx(\ell) \leq \elicpoly(\ell) \leq \elicembed(\ell)$.
However, one open question that remains from their work is if these bounds are equal. \jessie{\ldots}

\subsection{Property elicitation}\label{subsec:properties}

\begin{definition}[Property, elicits, level set]
	\jessiet{Don't like the notation here... fix later.}
	A \emph{property} is a set-valued function $\Gamma : \simplex \toto \R$ mapping distributions to reports.
	A loss $L : \R \to \reals_+^\Y$ \emph{elicits} the property $\Gamma$ if,
	\begin{equation}
	\forall p \in \simplex, \;\; \Gamma(p) = \argmin_{r \in \R}\inprod{p}{L(r)}
	\end{equation}
	Moreover, we call a \emph{level set} $\Gamma_r := \{p \in \simplex : r \in \Gamma(p)\}$ to be the set of distributions for which reporting $u$ minimizes the expected loss of the loss eliciting $\Gamma$.
\end{definition}
Note, moreover, that we call a property $\gamma: \simplex \toto \R$ \emph{finite} if $|\R| < \infty$.
Without much loss of generality, we assume that finite properties are \emph{non-redundant}, meaning that for each $\gamma_r$, there is no $\gamma_{r'}$ such that $\gamma_{r'} \subseteq \gamma_r$.

\begin{definition}\label{def:indirectly-elicits}
	A loss $L$ \emph{indirectly elicits} a property $\Gamma':\simplex \toto \R'$ if it elicits a property $\Gamma: \simplex \toto \R$ such that there is a function $\psi:\R \to \R'$ such that for all $r \in \R$, we have $\Gamma_r \subseteq \Gamma'_{\psi(r)}$.
\end{definition}


\section{Implications of property elicitation for consistency}
\begin{proposition}\label{prop:consistent-implies-indir-elic}
	If a surrogate and link $(L, \varphi)$ is consistent with respect to $\ell$, then $L$ indirectly elicits $\gamma := \prop{\ell}$.
\end{proposition}

\begin{proposition}
	Consider the discrete loss $\ell : \R \to \reals^\Y_+$ and $\gamma:= \prop{\ell}$.
	Then $\eliccvx(\gamma) \leq \ccdim(\ell)$.
\end{proposition}


\section{Characteristics of convex surrogates}
\begin{definition}[Identifiable]
	A property $\Gamma: \simplex \to \R$ is \emph{identifiable} if there is a function $V: \R \to \reals^\Y$ such that 
	\begin{equation*}
	\forall p \in \simplex, \; r \in \Gamma(p) \iff V(r) = \vec 0~.~
	\end{equation*}
\end{definition}

\jessie{Note from 2018 neurips paper that 1-elicitable iff 1-convex elicitable for continuous properties and how this differs.}

\begin{theorem}
	Convex elicitable properties are kind of identifiable \jessie{Formalize}
\end{theorem}

\begin{theorem}
	Suppose we are given a discrete loss $\ell:\R \to \reals^\Y_+$ eliciting $\gamma:= \prop{\ell}$.
	Fix $p \in \simplex$, and let $r$ be the report such that $p \in \gamma_r$.  
	If there is not $(n - d - 1)$-dimensional flat $F$ such that $F \subseteq \gamma_r$ and $p \in F$, then there is no surrogate loss $L : \reals^d \to \reals^\Y_+$ and link $\psi:\reals^d \to \R$ such that $(L, \psi)$ is calibrated with respect to $\ell$.
	Moreover, this must hold for all $p \in \simplex$ and $r \in \gamma(p)$.
\end{theorem}

\section{Finite outcome settings and finite properties}
\begin{definition}[Calibrated]
	Let $\ell : \R \to \reals^\Y_+$ be a discrete loss eliciting the property $\gamma$.
	A surrogate loss $L : \reals^d \to \reals^\Y_+$ is \emph{calibrated} with respect to $\ell$ if there exists a link function $\psi: \reals^d \to \R$ such that
	\begin{equation}
	\forall p \in \simplex: \inf_{u \in \reals^d : \psi(u) \not \in \gamma(p)} \inprod{p}{L(u)} > \inf_{u \in \reals^d} \inprod{p}{L(u)}~.~
	\end{equation}
\end{definition}


\cite{ramaswamy2016convex} introduces \emph{Convex Calibration Dimension}, which studies the minimal dimension input $d$ to construct a convex surrogate $L : \reals^d \to \reals^\Y_+$ that is calibrated with respect to a given discrete loss $\ell$.

\begin{theorem}[\cite{ramaswamy2016convex}]
	If $\Y$ is finite, then the surrogate loss $L:\reals^d \to \reals^\Y_+$ is calibrated with respect to $\ell: \R \to \reals^\Y_+$ if and only if there exists a link function $\psi : \reals^d \to \R$ such that for all distributions $D$ on $\X \times\Y$ and all sequences of (vector) functions $f_m : \X \to \reals^d$,
	\begin{equation*}
	\E_D L(f_m(X), Y) \to \inf_f \E_P L(f(X), Y) \implies \E_D \ell(\psi  \circ f_m(X), Y) \to \inf_f \E_D \ell(\psi \circ f(X), Y)~.~
	\end{equation*}
\end{theorem}
In words, we have that a surrogate and link are calibrated with respect to a discrete loss if and only if any consistent sequence of hypotheses for the surrogate, when linked, is consistent for the discrete loss.

As calibration does not rely on the representation space $\X$, we can ignore this and focus instead on calibration; we focus on \jessie{marginal} probability distributions $p \in \simplex$, agnostic to $\X$.


The following is a statement from~\cite{ramaswamy2016convex} providing an upper bound on the convex calibration dimension of a given discrete loss.
However, using tools from property elicitation, we can simplify their proof, which relies on considering sequences of losses that approach the infimum and taking $\epsilon$-subdifferentials of the surrogate loss.

\begin{theorem}[\cite{ramaswamy2016convex}]
	Suppose we are given a discrete loss $\ell:\R \to \reals^\Y_+$ eliciting $\gamma$.
	Take $p \in \simplex$ and $r \in \R$ such that $p \in \gamma_r$.
	Then \jessie{Need to define FSD $\mu_{\gamma_r}$}
	\begin{equation}
	\ccdim(\ell) \geq \|p\|_0 - \mu_{\gamma_r}(p) - 1~.~
	\end{equation}
\end{theorem}
\begin{proof}
	\jessie{Abbreviated FSD proof}
\end{proof}

\jessie{Abstain example... maybe... hopefully...}

\newpage

\section*{Broader Impact}
\jessie{Now required for NeurIPS, but doesn't count towards page limit.}

\begin{ack}
All the thanks
\end{ack}

\bibliographystyle{plainnat}
\bibliography{diss,extra}

\end{document}
