\documentclass{article}

% if you need to pass options to natbib, use, e.g.:
%     \PassOptionsToPackage{numbers, compress}{natbib}
% before loading neurips_2020

% ready for submission
% \usepackage{neurips_2020}

% to compile a preprint version, e.g., for submission to arXiv, add add the
% [preprint] option:
%     \usepackage[preprint]{neurips_2020}

% to compile a camera-ready version, add the [final] option, e.g.:
%     \usepackage[final]{neurips_2020}

% to avoid loading the natbib package, add option nonatbib:
\usepackage{neurips_2020}

\usepackage[utf8]{inputenc} % allow utf-8 input
\usepackage[T1]{fontenc}    % use 8-bit T1 fonts
\usepackage{hyperref}       % hyperlinks
\usepackage{url}            % simple URL typesetting
\usepackage{booktabs}       % professional-quality tables
\usepackage{amsfonts}       % blackboard math symbols
\usepackage{nicefrac}       % compact symbols for 1/2, etc.
\usepackage{microtype}      % microtypography
\usepackage{amsmath, amsfonts, amssymb, amsthm}

\usepackage{color}
\definecolor{darkblue}{rgb}{0.0,0.0,0.2}
\hypersetup{colorlinks,breaklinks,
	linkcolor=darkblue,urlcolor=darkblue,
	anchorcolor=darkblue,citecolor=darkblue}
\usepackage{wrapfig}
\usepackage{subcaption}
\usepackage[colorinlistoftodos,textsize=tiny]{todonotes} % need xargs for below
%\usepackage{accents}
\usepackage{bbm}
\usepackage{xspace}

\newcommand{\Comments}{1}
\newcommand{\mynote}[2]{\ifnum\Comments=1\textcolor{#1}{#2}\fi}
\newcommand{\mytodo}[2]{\ifnum\Comments=1%
	\todo[linecolor=#1!80!black,backgroundcolor=#1,bordercolor=#1!80!black]{#2}\fi}
\newcommand{\raf}[1]{\mynote{green}{[RF: #1]}}
\newcommand{\raft}[1]{\mytodo{green!20!white}{RF: #1}}
\newcommand{\jessie}[1]{\mynote{purple}{[JF: #1]}}
\newcommand{\jessiet}[1]{\mytodo{purple!20!white}{JF: #1}}
\newcommand{\proposedadd}[1]{\mynote{orange}{#1}}
\newcommand{\bo}[1]{\mynote{blue}{[Bo: #1]}}
\newcommand{\botodo}[1]{\mytodo{blue!20!white}{[Bo: #1]}}
\newcommand{\btw}[1]{\mytodo{gray!20!white}{[BTW: #1]}}%TURN OFF FOR NOW \mytodo{gray}{#1}}
\ifnum\Comments=1               % fix margins for todonotes
\setlength{\marginparwidth}{1in}
\fi

\newcommand{\reals}{\mathbb{R}}
\newcommand{\posreals}{\reals_{>0}}%{\reals_{++}}
\newcommand{\simplex}{\Delta_\Y}
\newcommand{\prop}[1]{\mathrm{prop}[#1]}
\newcommand{\relint}[1]{\mathrm{relint}(#1)}
\newcommand{\elic}{\mathrm{elic}}
\newcommand{\eliccvx}{\mathrm{elic}_\mathrm{cvx}}
\newcommand{\elicpoly}{\mathrm{elic}_\mathrm{pcvx}}
\newcommand{\elicembed}{\mathrm{elic}_\mathrm{embed}}
\newcommand{\ccdim}{\mathrm{ccdim}}
\newcommand{\codim}{\mathrm{codim}}
\newcommand{\supp}{\mathrm{supp}}
\newcommand{\spn}{\mathrm{span}}
\newcommand{\propdis}{\mu}

\newcommand{\C}{\mathcal{C}}
\newcommand{\D}{\mathcal{D}}
\newcommand{\E}{\mathbb{E}}
\newcommand{\F}{\mathcal{F}}
\newcommand{\I}{\mathcal{I}}
\newcommand{\N}{\mathcal{N}}
\renewcommand{\P}{\mathcal{P}}
\newcommand{\R}{\mathcal{R}}
\renewcommand{\S}{\mathcal{S}}
\newcommand{\U}{\mathcal{U}}
\newcommand{\X}{\mathcal{X}}
\newcommand{\Y}{\mathcal{Y}}


\newcommand{\risk}[1]{\underline{#1}}
\newcommand{\inprod}[2]{\langle #1, #2 \rangle}
\newcommand{\toto}{\rightrightarrows}

\newtheorem{theorem}{Theorem}
\newtheorem{lemma}{Lemma}
\newtheorem{proposition}{Proposition}
\newtheorem{corollary}{Corollary}
\newtheorem{conjecture}{Conjecture}
\newtheorem{definition}{Definition}
\newtheorem{assumption}{Assumption}
\newtheorem{remark}{Remark}


\DeclareMathOperator*{\argmax}{arg\,max}
\DeclareMathOperator*{\argmin}{arg\,min}
\DeclareMathOperator*{\arginf}{arg\,inf}
\DeclareMathOperator*{\sgn}{sgn}

\title{Indirect elicitation as a necessary condition for consistent surrogate losses}

% The \author macro works with any number of authors. There are two commands
% used to separate the names and addresses of multiple authors: \And and \AND.
%
% Using \And between authors leaves it to LaTeX to determine where to break the
% lines. Using \AND forces a line break at that point. So, if LaTeX puts 3 of 4
% authors names on the first line, and the last on the second line, try using
% \AND instead of \And before the third author name.

\author{%
  Jessie Finocchiaro\\
%  Department of Computer Science\\
  CU Boulder\\
  \texttt{jefi8453@colorado.edu} 
  % examples of more authors
  \And
   Rafael Frongillo\\
%   Department of Computer Science\\
   CU Boulder\\
  % Address \\
   \texttt{raf@colorado.edu} 
   \And
   Bo Waggoner\\
%   Department of Computer Science\\
   CU Boulder \\
  % Address \\
   \texttt{bwag@colorado.edu} 
  % \And
  % Coauthor \\
  % Affiliation \\
  % Address \\
  % \texttt{email} \\
  % \And
  % Coauthor \\
  % Affiliation \\
  % Address \\
  % \texttt{email} \\
}



\begin{document}

\maketitle

\begin{abstract}
	\jessie{in other doc}
\end{abstract}

\section{Introduction}\label{sec:intro}

In supervised machine learning, one often wants to make a prediction about future outcomes by training a classifier to minimize the average empirical loss of a labeled training set, where the loss is determined by the task at hand.
For example, 0-1 loss is often desired for classification tasks.
However, finite losses are typically difficult to optimize, so we construct a continuous \emph{surrogate} loss that one can more easily optimize.
In particular, we want our surrogate loss to yield bounds on the excess risk so that optimal predictions for the surrogate can be linked to optimal predictions for an original (possibly discrete) loss with bounded excess error, regardless of the underlying data distribution.
Moreover, in continuous prediction settings, one often starts with a statistic of the data they would like to estimate, but it is often unclear how to construct a a loss that is consistent for the given statistic, or property.
For example, if one wants to estimate the $\alpha$-quantile, they can do so by minimizing pinball loss over their data, but it is often unclear \emph{why} pinball loss is good for this prediction task.
Throughout this paper, the statistical guarantees we desire are \emph{consistency}, as it is the only condition that guarantees good bounds on excess surrogate loss. \jessiet{Raf, maybe fill in here?}
In particular, we are specifically interested in consistent \emph{and convex} surrogates in which any sequence of hypotheses converging to the optimal prediction can be linked to a sequence that converges to either the correct prediction for the discrete loss or statistic value, depending on the setting. 

We typically use convex surrogate loss functions that take some input $u \in \reals^d$ and measure error against the observed outcome $y \in \Y$.
The value $d$ provides some notion of efficiency for the surrogate, as low-dimensional convex losses can improve the efficiency of the optimization algorithm.
\cite{frongillo2015elicitation} pose an open question regarding a general characterization of the efficiency of \emph{convex elicitable properties}, which we show is bounded by finding the minimum dimension $d$ for which a consistent, convex loss can be constructed for the task at hand.

In Section~\ref{sec:related-work}, we review previous work and introduce the two main concepts used: \emph{consistency} and \emph{property elicitation}.
We draw connections between the two in Section~\ref{sec:consis-implies-indir} by showing that if a loss and link are consistent with respect to a target loss or the property it elicits, then it must also indirectly elicit the same property.
In Section~\ref{sec:char-convex}, we present our main result (Theorem~\ref{thm:cvx-flats}), a corollary of which characterizes the existence of consistent convex surrogates in a given dimension $d$.
In Section~\ref{sec:finite-calib}, we relate our setting to the study of \emph{convex calibration dimension} introduced by~\cite{ramaswamy2016convex} by focusing on the finite prediction setting.
Our Theorem~\ref{thm:cvx-flats} generalizes their bound on convex calibration dimension by observing that the \emph{subspace of feasible dimensions} constructed in their bound is just one case of a flat produced by our Theorem~\ref{thm:cvx-flats}.
To see this we use our results to generalize and simplify the proof of~\cite[Theorem 16]{ramaswamy2016convex}.
In continuous settings, one often has a prediction task for which they want to construct a consistent convex loss, rather than an original loss.
In Section~\ref{sec:contin-consis}, we address the construction of convex consistent losses by applying the main insights from Theorem~\ref{thm:cvx-flats} to this setting, answering the open question posed by~\cite[Section 8]{frongillo2018elicitation}.


\subsection{Notation}
\jessie{$\supp$, $\spn$}
In this paper, we take the outcome set $\Y$ and consider $\simplex$ to be the simplex over $\Y$. 

%discrete land
Let $\R$ be a report set and $\Y$ a outcome set; it is not necessary for $\Y = \R$.  
For example, in ranking problems, $\R$ may be all $n!$ permutations over the $n$ outcomes forming $\Y$.
A loss $\ell : \R \times \Y \to \reals_+$ is a \emph{discrete loss} if $|\R| < \infty$, and in general, we use $\ell$ to denote our \emph{target loss} for which we want to construct a consistent surrogate.
For a distribution $p \in \simplex$, we denote the expected discrete loss for report $r$ to be $\E_{Y \sim p} \ell(r, Y) := \ell(r; p)$.
When considering surrogate losses, we denote the loss by $L : \reals^d \times \Y \to \reals_+$, and typically denote a surrogate report with $u$.

Throughout this paper, we use tools from \emph{property elicitation} to understand the existence of consistent surrogate functions; a property is simply a function mapping distributions over the outcome simplex to reports.
Introduced in Section~\ref{subsec:properties}, a property $\Gamma: \simplex \to 2^\R \setminus \emptyset$ is denoted $\Gamma:\P \toto \R$.
We call a property \emph{set-valued} if there is a $p \in \P$ so that $|\Gamma(p)| > 1$, and \emph{single-valued} otherwise.
We use $\Gamma := \prop{L}$ to denote that $\Gamma$ is the (unique) property elicited by $L$.

Throughout this paper, we focus on characteristics of \emph{convex} surrogate losses.
Here, when we say a loss $L(\cdot; p)$ is \emph{convex}, we mean that it is convex in its first argument for every $p \in \simplex$, which is equivalent to saying $L$ is convex in the report for every outcome $y \in \Y$.
Moreover, the regret of a loss $L(u; p)$ is the excess loss over the optimal.  
That is, $R_L(u,p) := L(u,p) - \inf_{u^*} L(u^*, p)$.

\section{Related work}\label{sec:related-work}
\subsection{Consistency and calibration for convex losses}\label{subsec:convex-surrogates}
When one starts with a discrete loss they want to minimize, it is often a computationally hard problem.
For this reason, we use surrogate losses, but desire consistency to guarantee the surrogate ``corresponds'' to the original loss (or property we want to predict).
However, a surrogate loss is no good on its own; one needs a link function to map from the surrogate prediction space back to the original prediction space.
In particular, we would like to map back to the \emph{correct} prediction that would have been given if we had directly optimized the original problem.
This notion of a correct surrogate is captured by two slightly distinct notions introduced here: \emph{consistency} and \emph{calibration}.

\cite{zhang2004statistical,lin2004note,bartlett2006convexity,tewari2007consistency} form a characterization of consistent and calibrated surrogates for classification problems.
In particular,~\cite{bartlett2006convexity} show that if a convex surrogate is differentiable, minimal at $0$, and nonnegative, there is a consistent surrogate for binary classification problems, and~\cite{tewari2007consistency} generalizes this result for multiclass classification. 
\cite{ramaswamy2016convex} further show necessary and sufficient conditions for finite prediction problems to be consistent that can be applied to general discrete losses.
They additionally introduce a notion of \emph{convex calibration dimension} for discrete losses similar to elicitation complexity mentioned above; we discuss this further in Section~\ref{sec:finite-calib}.
Their convex calibration dimension results yield consistent surrogates for finite prediction problems such as hierarchical classification from~\cite{ramaswamy2015hierarchical} and classification with an abstain option studied by~\cite{ramaswamy2018consistent}.

\cite{steinwart2007compare} generalizes the study of consistent losses from the finite prediction setting and characterizes different types of loss functions, relating excess risk bounds, consistency, and calibration, giving proofs for various classes of surrogate losses (i.e. margin-based, distance-based, supervised, unsupervised, etc.)
See~\cite[Chapter 2]{steinwart2008support} for further discussion of these loss functions.



%One way to prove consistency of a surrogate loss  with respect to an original loss is to show that one has satisfied the following \emph{excess risk bound}.
%\begin{definition}[Excess risk bound]
%	A surrogate loss and link pair $(L,\psi)$ satisfies the \emph{excess risk bound} with respect to a loss $\ell$ if there exists an increasing function $\zeta : \reals \to \reals$ that is continuous at $0$ with $\zeta(0) = 0$ so that for all $f:\X \to \R$ and data distributions $D$ over $\X \times \Y$, we have
%	\begin{multline}
%	\E_{(X,Y) \sim D} \ell(\psi \circ f(X), Y) - \inf_{f^*} \E_{(X,Y) \sim D} \ell(\psi \circ f^*(X), Y) \\ 
%	\leq \zeta \left( \E_{(X,Y) \sim D} L(f(X), Y) - \inf_{f^*} \E_{(X,Y) \sim D} L(f^*(X), Y) \right)
%	\end{multline}
%\end{definition}



In finite settings, calibration as given in Definition~\ref{def:calibrated-finite} has been shown to be equivalent to consistency~\citet{bartlett2006convexity,tewari2007consistency,ramaswamy2016convex,ramaswamy2018consistent}.
\begin{definition}[Calibrated: Finite predictions]\label{def:calibrated-finite}
	Let $\ell : \R \times \Y \to \reals_+$ be a discrete loss eliciting the property $\gamma$.
	A surrogate loss $L : \reals^d \times \Y \to \reals_+$ is \emph{calibrated} with respect to $\ell$ if there exists a link function $\psi: \reals^d \to \R$ such that
	\begin{equation}\label{eq:calibration}
	\forall p \in \simplex: \inf_{u \in \reals^d : \psi(u) \not \in \gamma(p)} L(u;p) > \inf_{u \in \reals^d} L(u;p)~.~
	\end{equation}
\end{definition}
In Appendix~\ref{app:calibration}, we give a more general definition of calibration which implies this definition in finite outcome settings.

In a finite setting with a given discrete loss, we can now as the question of \emph{how efficiently} a convex and calibrated surrogate can be constructed.
To formalize this notion of efficiency, we consider \emph{convex calibration dimension.} 
\begin{definition}[Convex Calibration Dimension;~\cite{ramaswamy2016convex}]
	The \emph{convex calibration dimension} $\ccdim(\ell)$ of a discrete loss $\ell$ is the minimum dimension $d$ such that there is a convex loss $L: \reals^d \times \Y \to \reals$ and link $\psi$ such that $L$ is calibrated with respect to $\ell$.
\end{definition}

\subsection{Property elicitation}\label{subsec:properties}
In this paper, we use tools from property elicitation to study the construction of consistent surrogate losses.
Property elicitation of a single property is well-understood through \cite{savage1971elicitation,osband1985information-eliciting,lambert2008eliciting, lambert2009eliciting, lambert2018elicitation}.
Recently, the elicitation of multiple properties was studied by~\cite{frongillo2015vector-valued,frongillo2015elicitation,fissler2015higher}.
In particular, these works study the \emph{minimum} dimension of a real-valued loss needed to elicit a property or vector of properties.
Moreover,~\cite{agarwal2015consistent} is the first to our knowledge to formally relate property elicitation to the consistency of a surrogate loss.


Informally, a property is a function mapping a probability distribution $p$ over the outcome set $\Y$ to the ``suggested report'' given $p$.

\begin{definition}[Property, elicits, level set]
	A \emph{property} is a (possibly set-valued) function $\Gamma : \simplex \toto \R$ mapping distributions to reports.
	A loss $L : \R \times \Y \to \reals_+$ \emph{elicits} the property $\Gamma$ if,
	\begin{equation}
	\forall p \in \simplex, \;\; \Gamma(p) = \argmin_{r \in \R}\inprod{p}{L(r)}
	\end{equation}
	Moreover, we call a \emph{level set} $\Gamma_r := \{p \in \P : r \in \Gamma(p)\}$ to be the set of distributions for which reporting $u$ minimizes the expected loss of the loss eliciting $\Gamma$.
\end{definition}
Note, moreover, that we call a property $\gamma: \simplex \toto \R$ \emph{finite} if $|\R| < \infty$.
Without much loss of generality, we assume that finite properties are \emph{non-redundant}, meaning that for each $\gamma_r$, there is no $\gamma_{r'}$ such that $\gamma_{r'} \subseteq \gamma_r$.

\cite{finocchiaro2018convex} are among the first to consider a characterization of \emph{convex} elicitable properties, in which they find that all continuous, nowhere locally constant elicitable properties (in finite outcome settings) are elicitable by a \emph{convex} loss.
However, their assumptions are more restrictive than ours; they only consider losses defined on the real line (see Section~\ref{subsec:elic-cplx} below for our generalization) and assume the properties to be identifiable: an assumption we do not need here.
\cite{frongillo2018elicitation} then presents a more general characterization of convex elicitable properties that are identifiable, but we lift this assumption with with our main results in Theorem~\ref{thm:cvx-flats}.

When $\R \subseteq \reals^d$ for some $d \in \mathbb{Z}$, \cite{frongillo2015elicitation} introduce the notion of \emph{(convex) elicitation Complexity}, which uses $d$ to measure the ``efficiency'' of a loss $L$ eliciting $\Gamma$.
However, their notion of complexity relies not just on the property $\Gamma$, but instead on any property \emph{indirectly elicited} by $L$.

\begin{definition}[Indirect Elicitation]\label{def:indirectly-elicits}
	A loss $L$ \emph{indirectly elicits} a property $\gamma:\P \toto \R'$ if it elicits a property $\Gamma: \P\toto \R$ such that there is a function $\psi:\R \to \R'$ such that for all $r \in \R$, we have $\Gamma_r \subseteq \gamma_{\psi(r)}$.
\end{definition}
While there are a few possible definitions of indirect elicitation, this one is the most applicable for our setting because when we consider set-valued properties, we do not have that \emph{all} optimal reports for $\Gamma$ must be linked back to \emph{all} optimal reports for $\gamma$.
Instead, we loosen this restriction to say that any optimal report for $\Gamma$ must be linked to \emph{an} optimal report for $\gamma$, and must be done in a consistent manner.

When properties are vector-valued (i.e.$\, \Gamma: \simplex \to \reals^d$),~\cite{frongillo2015elicitation} introduce the notion of \emph{convex elicitation complexity}.

\begin{definition}[Convex Elicitation Complexity]
	The \emph{convex elicitation complexity} of a property $\eliccvx(\Gamma)$ is the minimum dimension $d$ such that there is a convex loss $L : \reals^d \times \Y \to \reals$ indirectly eliciting $\Gamma$.
\end{definition}

In Proposition~\ref{prop:consistent-implies-indir-elic}, we relate indirect elicitation to the consistency of a surrogate loss.


\section{Consistency implies indirect elicitation}\label{sec:consis-implies-indir}
While the subtleties of \emph{consistency} vary slightly with each work, we use two definitions that have the same implications, but one may be more appropriate in a given context.

\begin{definition}[Consistent: loss]
	Suppose $f_m : \X \to \R$ is the hypothesis function learned by minimizing empirical training loss over $m$ labeled examples.
	A loss and link $(L,\psi)$ are consistent with respect to an original loss $\ell$ if, for all distributions $D$ over input and label spaces $\X \times\Y$, 
	\begin{align*}
	\E_D L(f_m(X), Y) \to \inf_{f \text{ msbl}} \E_D L(f(X), Y) &\implies \E_D \ell(\psi \circ f_m(X), Y) \to \inf_{f \text{ msbl}} \E_D \ell(\psi \circ f(X), Y)~.~
	\end{align*}
\end{definition}


For continuous properties, such as the expected value, variance, and entropy, asking for a consistent surrogate with respect to an original \emph{loss} may not make sense when we do not have one.
In this setting, we also define consistency with respect to a property that one wishes to estimate. 
\begin{definition}[Consistent: property]
	Blah blah
\end{definition}


Propositions~\ref{prop:consistent-implies-indir-elic-loss} and~\ref{prop:consistent-implies-indir-elic-prop} below allows us to understand the connection between property elicitation and consistent losses.
Moreover, combining Proposition~\ref{prop:consistent-implies-indir-elic} and Theorem~\ref{thm:cvx-flats}, we provide new conditions for finding lower bounds on convex calibration dimension for finite prediction settings in Section~\ref{sec:finite-calib}.
In continuous prediction settings, we can generalize Theorem~\ref{thm:cvx-flats}\jessie{hopefully...} in order to answer an open question about convex elicitation complexity posed by~\cite{frongillo2015elicitation}.

\begin{proposition}\label{prop:consistent-implies-calibrated}
	If a loss and link $(L, \psi)$ are consistent with respect to a loss $\ell$, then they are calibrated with respect to $\ell$.
	\jessiet{Probably appendix later; also, should this be a lemma instead of a proposition?}
	\raft{Do we define ``continuous loss''?  Maybe ``(not necessarily discrete)'' is clearer.}
\end{proposition}

\begin{lemma}\label{lem:calib-implies-indir}
	If a surrogate and link $(L, \psi)$ are calibrated with respect to a loss $\ell:\R \times\Y \to \reals$, then $L$ indirectly elicits the property $\gamma := \prop{\ell}$.
\end{lemma}
\begin{lemma}\label{lem:consistent-loss-implies-prop}
	If $(L, \psi)$ are consistent with respect to $\ell$, then they are consistent with respect to $\gamma := \prop{\ell}$.
\end{lemma}

This result allows us to apply the upper bounds from~\cite{frongillo2015elicitation} when trying to find the minimal dimension consistent surrogate for a given loss.
In particular, when $\Y$ is finite and we restrict $L$ to be convex, this translates to applications to \emph{convex calibration dimension} of~\cite{ramaswamy2016convex}.

\begin{theorem}\label{thm:consistent-implies-indir-elic}
	If a surrogate and link pair $(L, \psi)$ is consistent with respect to a property $\gamma$ or loss $\ell$ eliciting $\gamma$, then $(L, \psi)$ indirectly elicits $\gamma$.
\end{theorem}

\section{Characteristics of Consistent Convex Surrogates}

When one restricts their focus to \emph{convex} surrogates, we know that all minima are global, and we then have $\vec 0  \in \partial L(u, y)$ if and only if $u$ minimizes $L$ in its first component.
This observation allows us to consider subgradient sets of the loss at a fixed distribution $p$ as the weighted Minkowski sums of subgradient sets for the loss on each outcome.
We use this observation to construct a flat that is contained in the subgradient set of $L(r;p)$ and contains $\vec 0$ to yield a new bound on convex elicitation complexity.

\begin{definition}[Flat]
	blah blah
\end{definition}
The codimension of the flat $F$ is given by $f := \mathrm{rank}(W)$.

Now we can consider specific characteristics of convex losses in order to understand the convex elicitation complexity of a given property.

\begin{theorem}\label{thm:cvx-flats}
	Suppose we are given a property $\gamma$ and distribution $p \in \simplex$.
	For all $r\in\gamma(p)$, if there is no $(n - d-1)$-dimensional flat $F$ containing $p$ so that $F \cap \simplex \subseteq \gamma_r$, then there is no convex surrogate loss $L : \reals^d \times \Y \to \reals$ that indirectly elicits $\gamma$.
\end{theorem}

Since Propositions~\ref{prop:consistent-implies-indir-elic-loss} and~\ref{prop:consistent-implies-indir-elic-prop} say that consistency implies indirect elicitation, we then have elicitation complexity greater than $d$ implies no indirect elicitation via a $d$-dimensional property, which in turn implies there is no $d$-dimensional consistent surrogate for a loss or property of interest.

\section{Finite report and outcome settings}\label{sec:finite-calib}

\begin{definition}[Subspace of feasible directions]
	Define the \emph{subspace of feasible directions} $\S_\C(p)$ of a convex set $\C \subseteq \reals^n$ at a point $p \in \C$ as the subspace $\S_\C(p) = \{ v \in \reals^n : \exists \epsilon_0 > 0 $ such that $p + \epsilon v \in \C \; \forall \epsilon \in (-\epsilon_0,\epsilon_0) \}$.
	%  \raf{Let's simplify to $\epsilon \in (-\epsilon_0,\epsilon_0)$}
\end{definition}

\begin{lemma}\label{lem:feas-sub-is-a-flat}
	Suppose we have the finite elicitable property $\gamma$ and distribution $p \in \relint{\simplex}$ with $r \in \gamma(p)$.
	If $F$ is a flat containing $p$ such that $F \cap \simplex \subseteq \gamma_r$, then $F - p$ is a subspace contained in $\S_{\gamma_r}(p)$.
	%  \raf{I think you want $p$ in the interior of the simplex for now}
\end{lemma}
\begin{proof}
	To spell it out, observe $F-p$ is a subspace as it is a linear shift of $F$, which is a linear subspace by definition of a flat and the fact that it contains $\vec 0$.
	Now consider $v \in F - p$.
	Since $p \in \relint{\simplex}$, there is an open ball of radius $\epsilon$ in the affine hull of $\simplex$ so that for all $q \in B(p, \epsilon)$, we have $q \in \simplex$.
	In particular, take $\alpha = \epsilon / 2$, and we observe $p \pm \alpha v \in B(p, \epsilon)$, and therefore $p \pm \alpha v \in \simplex$.
	Moreover, by the assumption $v \in F - p$, we also have $p \pm \alpha v \in \gamma_r$. 
	Since level sets of elicitable properties are convex (\cite{lambert2009eliciting}) this is true for all $\alpha' \leq \alpha$.
	Therefore, we observe $v \in S_{\gamma_r}(p)$, so $F-p \subseteq S_{\gamma_r}(p)$.
	%  First, if $p+v \in \gamma_r$ and $p -v \in \gamma_r$, then we have $v \in \S_{\gamma_r}(p)$ with $\epsilon_0 = 1$ as level sets of elicitable properties are convex by~\cite{lambert2009eliciting}.
	%  If either $p + v$ or $p - v \not \in \gamma_r$, it must be because the term is out of the simplex by definition of $F$.
	%  
	%  However, if both $p + \alpha^+ v$ and $p - \alpha^- v \in \simplex$ for some $\alpha^\pm \in (0,1)$, then $v \in \S_{\gamma_r}(p)$ with $\epsilon_0 = \min(\alpha^+, \alpha^-)$.
	%  As $p \in \relint{\simplex}$, there is always such an $\epsilon_0$; if there were not, then we would observe $v \not \in F - p$.
	%  Therefore, we have $v \in F - p \implies v \in \S_{\gamma_r}(p)$, so $(F - p) \subseteq \S_{\gamma_r}(p)$.
\end{proof}


%\begin{lemma}\label{lem:feas-sub-is-a-flat}
%	Suppose we have the finite property $\gamma$ and distribution $p \in \simplex$.
%	For ant $r$ such that $r \in \gamma(p)$, the subspace $F := \mu_{\gamma_r}(p) + p $ is a flat, $F$ contains $p$, and $F \cap \simplex \subseteq \gamma_r$.
%\end{lemma}
%\begin{proof}
%	Write the level set $\gamma_r = \{q \in \reals^n : A_1 q \leq b_1, \; A_2 q \leq b_2, \; A_3 q \leq b_3 \}$, where $A_1 p \leq b_1, A_2 p < b_2,$ and $A_3 p = b_3$.
%	
%	We want to show three things: first, $F$ is a flat as it is $\ker([A_1 ; A_3])$ by~\cite{ramaswamy2016convex}.
%	Second, $p \in F + p$ by construction, since $\vec 0 \in F$.
%	(Consider that $p \in \gamma_r$, so with $epsilon_0 = 1$, we have $p + \epsilon\vec 0 = p \in \gamma_r$ for all $\epsilon \in (0, 1)$; thus, $\vec 0 \in F$.)
%	
%	Third, we want to show $(F + p) \cap \simplex \subseteq \gamma_r$.
%	Take some $q := p + \epsilon v$ for $v \in F$.
%	By construction of $F$, we have $q \in F \implies q \in \simplex$, so $F + p \subseteq \simplex \implies F + p \cap \simplex = F + p$.
%	Thus it just remains to be shown that $F + p \subseteq \gamma_r$.
%	Since $v\in F \implies q \in F + p \implies p + \epsilon v \in \gamma_r$ by construction, so we have $F+p \subseteq \gamma_r$.	
%\end{proof}

%\begin{lemma}\label{lem:p-boundary-fstar}
%	For a given property $\gamma$, fix $p \in \simplex$ and $r \in\R$ so that $r \in \gamma(p)$.
%	Consider the flat $F'$ containing $p$ of dimension at least $n-d-1$.
%	The flat $F^* = F' \cap \spn(\{e_y : y \in \supp(p)\})$ has dimension $\dim(F') - (n - \|p\|_0)$.
%\end{lemma}
%\begin{proof}
%	Consider $P := \spn(\{e_y : y \in \supp(p) \})$.
%	\begin{align*}
%	\dim(F^*) &= \dim(F') + \|p\|_0 - \dim(F' + P)
%	\end{align*}
%	The claim then holds if $n = \dim(F' + P)$.
%	\jessie{This is where I'm stuck??}
%\end{proof}

\begin{lemma}\label{lem:p-boundary-fsd}
	For any $p \in \simplex$ and $r$ such that $p \in \gamma_r$, define $\gamma'$ to be $\gamma$ restricted to the simplex $\Delta_{\supp(p)}$.
	Then $\S_{\gamma_r}(p) = \S_{\gamma'_r}(p)$.
\end{lemma}


The following result from~\cite{ramaswamy2016convex} allows us to use calibration as a tool to study consistency in finite prediction settings, where both $\R$ and $\Y$ are finite sets.
In finite prediction tasks, we often want a consistent surrogate with respect to an original loss, so we focus on this notion of consistency here.

\begin{corollary}[\cite{ramaswamy2016convex}]\label{thm:calib-iff-consistent}
	If $\Y$ is finite, then the surrogate loss $L:\reals^d \to \reals^\Y_+$ is calibrated with respect to $\ell: \R \times \Y \to \reals_+$ if and only if there exists a link function $\psi : \reals^d \to \R$ such that for all distributions $D$ on $\X \times\Y$ and all sequences of (vector) functions $f_m : \X \to \reals^d$,
	\begin{equation*}
	\E_D L(f_m(X), Y) \to \inf_f \E_P L(f(X), Y) \implies \E_D \ell(\psi  \circ f_m(X), Y) \to \inf_f \E_D \ell(\psi \circ f(X), Y)~.~
	\end{equation*}
\end{corollary}
In words, a surrogate and link are calibrated with respect to a discrete loss if and only if any consistent sequence of hypotheses for the surrogate, when linked, is consistent for the discrete loss.

As calibration does not rely on the representation space $\X$, we can ignore this and focus instead on calibration; we focus on probability distributions $p \in \simplex$ agnostic to $\X$.

\begin{lemma}
	Suppose we are given a discrete loss $\ell : \R \times\Y \to \reals_+$ eliciting the property $\gamma$.
	If $(L, \psi)$ is calibrated with respect to $\ell$, then $L$ indirectly elicits $\gamma$.
\end{lemma}
\jessiet{Direct proof}

\cite{ramaswamy2016convex} give lower bounds on convex calibration dimension, which in turn yield lower bounds on the dimension of a consistent convex surrogate.
The main tool in their bound is to consider the \emph{subspace of feasible directions}, which we show in Lemma~\ref{lem:feas-sub-is-a-flat} is an example of one possible flat arising from Theorem~\ref{thm:cvx-flats}.
Lemma~\ref{lem:feas-sub-is-a-flat} then yields the bound of~\cite[Theorem 16]{ramaswamy2016convex} as a corollary.

\begin{definition}[Subspace of feasible directions]
	Define the \emph{subspace of feasible directions} $\S_\C(p)$ of a convex set $\C \subseteq \reals^n$ at a point $p \in \C$ as the subspace $\S_\C(p) = \{ v \in \reals^n : \exists \epsilon_0 > 0 $ such that $p + \epsilon v \in \C$ and $p - \epsilon v \in \C \; \forall \epsilon \in (0,\epsilon_0) \}$ .
\end{definition}

\begin{lemma}\label{lem:feas-sub-is-a-flat}
	Suppose we have the finite property $\gamma$ and distribution $p \in \simplex$ with $r \in \gamma(p)$.
	If $F$ is a flat containing $p$ such that $F \cap \simplex \subseteq \gamma_r$, then $F - p$ is a subspace contained in $\S_{\gamma_r}(p)$.
\end{lemma}


The following is a statement from~\cite{ramaswamy2016convex} providing an upper bound on the convex calibration dimension of a given discrete loss, which now follows as a Corollary of our Theorem~\ref{thm:cvx-flats} and Lemma~\ref{lem:feas-sub-is-a-flat}.

\begin{corollary}[\cite{ramaswamy2016convex}]
	Suppose we are given a discrete loss $\ell:\R \to \reals^\Y_+$ eliciting $\gamma$.
	Take $p \in \simplex$ and $r \in \R$ such that $p \in \gamma_r$.
	\begin{equation}
	\ccdim(\ell) \geq \|p\|_0 - \dim(\S_{\gamma_r}(p)) - 1~.~
	\end{equation}
\end{corollary}

\subsection{Elicitation complexity and convex calibration dimension}
We can additionally relate convex calibration dimension of a loss to the convex elicitation complexity of the loss it elicits.

\begin{proposition}
	Consider the discrete loss $\ell : \R \times \Y \to \reals_+$ with $\gamma:= \prop{\ell}$.
	Then $\eliccvx(\gamma) \leq \ccdim(\ell)$.
\end{proposition}

\cite{finocchiaro2019embedding} presents the notion of a surrogate loss \emph{embedding} a discrete loss, typically by a polyhedral (piecewise linear and convex) surrogate.
Moreover,~\cite{coltpaper} introduces the notion of \emph{embedding dimension}, which is a lower bound on both convex elicitation complexity of finite properties and convex calibration dimension of discrete losses.
However, it is an open problem to understand if these efficiency definitions are generally equivalent.

\section{Continuous Properties}\label{sec:contin-consis}
We address one major open question from~\cite{frongillo2015elicitation} by a generalization of Theorem~\ref{thm:cvx-flats}: \emph{what are lower bounds on convex elicitation complexity?}
Proposition~\ref{prop:consistent-implies-indir-elic-prop} allows us to use convex elicitation complexity as a tool to understand efficiency of consistent convex surrogates for a given property, which is more often what is given in a continuous prediction setting.

For example, when one wants to learn an $\alpha$-quantile, we start with the property rather than a loss.
In the literature, when one wants to learn a quantile, pinball loss $L(r,y) = (r-y)(\mathbbm{1}_{r \geq y} - \alpha)$ typically appears without explanation or justification.
Elicitation allows us to understand why pinball loss is consistent for learning a quantile as the pinball loss elicits the $\alpha$-quantile.



\newpage

\section*{Broader Impact}
\jessie{Now required for NeurIPS, but doesn't count towards page limit.}

\begin{ack}
All the thanks
\end{ack}

\bibliographystyle{plainnat}
\bibliography{diss,extra}

\newpage
\appendix
\section{A case for general calibration}

In finite settings, consistency and calibration of a surrogate are equivalent under mild assumptions.
Here, we generalize the definition of calibration to extend to continuous prediction settings.
\begin{definition}[Calibrated]\label{def:calibrated-general}
	A loss $L:\reals^d \times \Y \to \reals$ is \emph{calibrated} with respect to a property $\gamma : \simplex \toto \R$ elicited by the loss $\ell$ if there is a link $\psi : \reals^d \to \R$ such that, for all distributions $p \in \simplex$, there exists an increasing function $\zeta : \reals \to \reals$ with $\zeta$ continuous at $0$ and $\zeta(0) = 0$ such that for all $u \in \reals^d$, we have
	\begin{equation}
	\ell( \psi(u); p) - \ell(\gamma(p); p)  \leq \zeta \left(  L(u;p) - L(\Gamma(p); p) \right)~.~
	\end{equation}
\end{definition}


\begin{proposition}\label{prop:calib-defs-equiv}
	When $\R$ is finite, calibration of a property via Definition~\ref{calibrated-general} implies calibration via Definition~\ref{def:calibration-finite}.
\end{proposition}

\end{document}
