\documentclass[12pt]{article}
\usepackage[utf8]{inputenc}
\usepackage{mathtools, amsmath, amsthm, amssymb, graphicx, mathrsfs, verbatim}
%\usepackage[thmmarks, thref, amsthm]{ntheorem}
\usepackage{color}
\usepackage{wrapfig}
\usepackage{subcaption}
\usepackage[colorinlistoftodos,textsize=tiny]{todonotes} % need xargs for below
%\usepackage{accents}
\usepackage{bbm}
\usepackage{xspace}
\usepackage[margin=1.25in]{geometry}

\newcommand{\Comments}{1}
\newcommand{\mynote}[2]{\ifnum\Comments=1\textcolor{#1}{#2}\fi}
\newcommand{\mytodo}[2]{\ifnum\Comments=1%
  \todo[linecolor=#1!80!black,backgroundcolor=#1,bordercolor=#1!80!black]{#2}\fi}
\newcommand{\raf}[1]{\mynote{green}{[RF: #1]}}
\newcommand{\raft}[1]{\mytodo{green!20!white}{RF: #1}}
\newcommand{\jessie}[1]{\mynote{purple}{[JF: #1]}}
\newcommand{\jessiet}[1]{\mytodo{purple!20!white}{JF: #1}}
\ifnum\Comments=1               % fix margins for todonotes
  \setlength{\marginparwidth}{1in}
\fi


\newcommand{\reals}{\mathbb{R}}
\newcommand{\posreals}{\reals_{>0}}%{\reals_{++}}

% alphabetical order, by convention
\newcommand{\D}{\mathcal{D}}
\newcommand{\E}{\mathbb{E}}
\newcommand{\F}{\mathcal{F}}
\newcommand{\I}{\mathcal{I}}
\renewcommand{\P}{\mathcal{P}}
\newcommand{\R}{\mathcal{R}}
\newcommand{\X}{\mathcal{X}}
\newcommand{\Y}{\mathcal{Y}}


\newcommand{\inter}[1]{\mathring{#1}}%\mathrm{int}(#1)}
\newcommand{\cl}[1]{\text{cl}}
%\newcommand{\expectedv}[3]{\overline{#1}(#2,#3)}
\newcommand{\expectedv}[3]{\E_{Y\sim{#3}} {#1}(#2,Y)}
\newcommand{\toto}{\rightrightarrows}
\newcommand{\trim}{\mathrm{trim}}
\newcommand{\fplc}{finite-piecewise-linear and convex\xspace} %xspace for use in text
\newcommand{\conv}{\mathrm{conv}}
\newcommand{\ones}{\mathbbm{1}}
\newcommand{\im}{\text{im}}
\newcommand{\strip}{\text{strip}}


\DeclareMathOperator*{\argmax}{arg\,max}
\DeclareMathOperator*{\argmin}{arg\,min}
\DeclareMathOperator*{\arginf}{arg\,inf}
\DeclareMathOperator*{\sgn}{sgn}

\newtheorem{theorem}{Theorem}
\newtheorem{lemma}{Lemma}
\newtheorem{proposition}{Proposition}
\newtheorem{definition}{Definition}
\newtheorem{corollary}{Corollary}
\newtheorem{conjecture}{Conjecture}




\begin{document}

\section{Notation and Definitions}

Let $\Y$ be a finite outcome space, with $n:=|\Y|$, and $\Delta(\Y)$ be the set of probability distributions over $\Y$.

We define the \emph{level set} of a property at report value $r\in\R$ to be the set $\Gamma_r := \{p\in\P : r \in \Gamma(p)\}$.
Additionally, we can consider an unlabeled level set $\theta$ as this same set of distributions, but not associated with the report $r$.

The set $\Theta \subseteq 2^{\Delta(\Y)}$ is an unlabeled property, and finite if $|\Theta| < \infty$.
Observe that $\Theta$ is a set of level sets that divide the simplex, and not a function.

\begin{definition}
Let $\Gamma$ be a labeled property.
Define $\text{strip}(\Gamma) := \{ \Gamma_r : r \in \R \}$.
Informally, strip removes labels from the property.
\end{definition}

We will use the following notation: $\Gamma' = \Gamma \cap \R'$ is the property $\Gamma' : p \to \Gamma(p) \cap \R'$.

\begin{definition}
  Consider two labeled properties $\Gamma'$ and $\Gamma$.
  We say $\Gamma' < \Gamma : \Delta(\Y) \to \R$ if you get $\Gamma'$ by removing reports from $\Gamma$.
  (i.e. $\Gamma' = \R' \cap \Gamma$ for some $\R' \subset \R$)
\end{definition}

\begin{definition}
A property $\Gamma$ is \emph{finite} if $\im(\Gamma)$ is a finite set.
%\jessie{Take 2:}  A property $\Gamma$ is \emph{finite} if there are only finitely many full-dimensional level sets.
\end{definition}

\begin{lemma}\label{conj:closed-level-sets}
  For any $r\in\R$, the level set $\Gamma_r = \{ p \in \Delta(\Y) : \Gamma(p) = r \}$ is closed.
  i.e. $\Gamma_r = \text{cl} (\Gamma_r)$.
\end{lemma}
\begin{proof}
  \emph{Sketch:  } Since property values are the subgradient sets of the overarching Scoring Rule and subgradient sets are closed, the level sets of $\Gamma$ are closed.
\end{proof}

%\begin{definition}
%$trim(\Gamma_L)$ is the property elicited by the loss $L|_{\R'}$, where $\R' = \{r \in \R : \Gamma_r^\circ \neq \emptyset \}$. 
%\end{definition}
%\jessie{This definition has nothing to do with $\Gamma$ but $\Gamma \cap \R$ doesn't typecheck to me... and the previous definition doesn't specifcy $\R'$}


Notes before starting:
\begin{itemize}
\item (Assuming the metric space is $(\P = \Delta(\Y), \rho)$, where $\rho$ is Euclidean distance, although I don't think the choice of metric matters?)
\item The interior of a level set $X$ is defined as $\inter{X} = \{p \in X: x > 0,  \forall q \in B(x,p), \argmin_rE_p L(r,Y) = \argmin_r E_q L(r,Y)\}$
\item A level set is \emph{full-dimensional} if dim(affine hull($\Gamma_r$)) is equal to dim($\P$). \jessie{I think...}
\item A full-dimensional level set has non-empty interior.
\end{itemize}

\section{Conjecture}


\begin{conjecture}\label{conj:intersect-lev-sets}
  Let the property $\Gamma$ have a finite set $\Theta$ of full-dimensional level sets, so that for all $\theta \in \Theta$,  there is a unique optimal report $r_\theta$ for all $p \in \inter{\theta}$.
  Consider $\R' := \{ r_\theta\}_{\theta \in \Theta}$.
  Then for every $r \in \R$, there exist a finite set $R \in 2^{\R'}$ so that $\Gamma_r = \bigcap_{r' \in R} \Gamma_{r'}$.
\end{conjecture}


\section{Formal Statements and modifications needed to prove}


\begin{lemma}
  Let $L: \R \to \reals^\Y$ elicit a nondegenerate property $\Gamma$.
There exists an unlabeled property $\Theta$ so that for all nondegenerate, nonredundant $\Gamma' < \Gamma$, we have $\strip(\Gamma') = \Theta := \text{trim}(\Gamma)$.
\end{lemma}

\begin{proof}
\jessie{To show:  For any nonredundant, nondegenerate $\Gamma' < \Gamma$, that $\strip(\Gamma') = \strip(\Gamma)$}.

Consider the nonredundant, nondegenerate properties $\Gamma' < \Gamma$.
As $\strip$ is well-defined, clearly $\strip(\Gamma) = \Theta$ is a well-defined set of level sets.

We can define them as $\Gamma' = \Gamma \cap \R'$ for some $\R' \subseteq \R$ and similarly $\tilde \Gamma = \Gamma \cap \tilde \R$ for $\tilde \R \subseteq \R$.
Then we can write $\Theta' := \strip(\Gamma') = \{\Gamma_r : r \in \R' \}$ and $\Theta := \strip(\tilde \Gamma) = \{ \Gamma_{\tilde r} : \tilde r \in \tilde \R \}$.

Consider $\theta \in \Theta$.
Then there is some $r \in \R$ such that $r\in\Gamma(p)\cap \R \implies r \in \Gamma(p)$ for all $p \in \theta$.
(Note:  If $r \in \R'$, then, trivially, $r \in \Gamma(p) \cap \R'$ for all $p \in \theta$.)
If $r \not \in \R'$, then as $\Gamma'$ is nondegenerate, there exists some $r' \in \R'$ so that $r' \in \Gamma'(p) = \Gamma(p) \cap \R'$ for all $p \in \theta$.
Therefore, $\theta \in \Theta'$,
Since $\theta \in \Theta \implies \theta \in \Theta'$, we know $\Theta \subseteq \Theta'$.

As $\R' \subseteq \R$, it is easy to see  $\Theta' \subseteq \Theta$, and therefore $\Theta = \Theta'$.

Therefore, for all nonredundant, non-degenerate properties $\Gamma' < \Gamma$, we observe an unlabeled property $\trim(\Gamma) := \Theta = \strip(\Gamma')$.
\end{proof}

\subsection{Main statement}

Let $\Gamma$ be a finite, non-redundant convex elicitable property.

\begin{enumerate}
\item The property $\Gamma$ has a finite set $\Theta$ of full-dimensonal level sets, and for all $\theta \in \Theta$, there exists an $r_\theta \in \R$ such that $\Gamma(p) = \{r_\theta\}$ for all $p \in \inter{\theta}$.
\item $\text{strip}(\Gamma)$ is finite.
\end{enumerate}

\begin{proposition}\label{prop:optimal-reports-per-level-set}
1 $\iff$ 2
\end{proposition}



Notes:
In order to prove this, we need to assume the property is non-redundant.


\begin{proof}
Let $L$ be the loss function eliciting the property $\Gamma$.

$1 \implies 2$. 
% \emph{Assumptions used: non-redundancy}


Consider $\strip(\Gamma) = \{\Gamma_r : r \in \R \}$.
Let us construct the finite set $\R' = \{r_\theta \}_{\theta \in \Theta}$.
Clearly, $\R' \subseteq \R$.

We can rewrite $\strip(\Gamma) = \{\Gamma_{r'} : r' \in \R' \} \cup \{\Gamma_{r} : r \in \R \setminus \R' \}$.
As the first set is finite, we just need to show $\{\Gamma_{r} : r \in \R \setminus \R' \}$ is finite in order to show $\strip(\Gamma)$ is finite.

Let us consider some $r\in\R \setminus \R'$.
The level set $\Gamma_r = \{p : \Gamma(p) = r\}$ by definition.
As $r \not\in \R'$, the interior is empty; that is, $\inter{\Gamma_r} = \emptyset$.
% For any $p \in \Gamma_r$, then, there is some $q \in B(\epsilon, p)$ such that $\E_q L(r', Y) < \E_q L(r,y)$ for some $r' \in \R'$.
% As $r = \argmin_{\tilde{r}} \E_p L(\tilde{r}, Y)$, then we observe $\E_p L(r, Y) = \E_p L(r', Y) = \min_{\tilde{r}}\E_p L(\tilde{r}, Y)$ for $p \in \Gamma_r$. 
% Thus, $\Gamma_r \subseteq \Gamma_{r'}$.
By Conjecture \ref{conj:intersect-lev-sets}, there exist a set of finite reports $R \in 2^{\R'}$ such that $\bigcap_{r'\in R} \Gamma_{r'} = \Gamma_r$.
As there are only finitely many combinations of reports $R$ in the finite set $\R'$, then the set $\{ \Gamma_r : r \in \R \setminus \R' \} =  \{ \bigcap_{r_1 \in R} \Gamma_{r_1} : R \in 2^{\R'} \}$ is finite.


The union of a two finite sets is then finite, so $\strip(\Gamma) = \{ \Gamma_{r'} : r' \in \R' \} \cup \{ \Gamma_r : r \in \R \setminus \R' \}$ is finite.



% Since $\Gamma$ is finite, we know it only has finitely many FDLS, and since $\Gamma$ is convex, these level sets are convex.
% On the interior of these level sets, we know there is a unique optimal report.
% As $\R' = \bigcup_{\Gamma_r \text{is FDLS}}r$, $\R'$ is finite as the finite union of finite sets is finite.
% By Lemma 2 from original writeup, or new definition of trim, we know that $L|_\R$ elicits trim($\Gamma$).
% As the new report set is finite, there many only be a finite number of minimizing reports.

% Consider the full-dimensional level sets of $\Gamma$: as $\R' \subseteq \R$, then when $L$ is restricted to $\R'$, for any $r \in \R$, there must be some $r' \in \R'$ such that $\Gamma_{r} \subseteq \Gamma_{r'}$, as if .



%Since $\Gamma$ is a finite property, there are a finite number of (non-empty) level sets.
%As there is only one optimal report per (interior of) level set, there are a finite of minimizing reports.
%Let this finite set be $\tilde{\R}$, and observe $\tilde{R}$ is finite.
%By definition, $\tilde{\R} \subseteq \R$.
%As $\im(\Gamma') = \im(\Gamma) \cap \R = \tilde{\R} \cap \R'$ \footnote{Needs showing}, then the finite intersection of a finite set and another set is finite, so $\im(\Gamma')$ is finite.


\bigskip
$2 \implies 1$.  
\emph{Assumptions used: Non-redundancy, nondegeneracy of $\Gamma$}

Let $\Theta := \strip(\Gamma)$ be a finite set.
By construction, $\Theta$ is a set of level sets defining the property $\Gamma$.
As $\Theta$ is finite, $\Gamma$ then only has finitely many level sets, and therefore, finitely many \emph{full-dimensional} level sets.
For any full-dimensional level set $\theta \in \Theta$, as $\Gamma$ is non-degenerate (i.e. $\forall p \in \Delta(\Y), \Gamma(p) \neq \emptyset$), there is some report $r_\theta$ so that $r_\theta \in \Gamma(p)$ for all $p \in \theta$.
Therefore, we are left to show that $r_\theta$ is the unique value of the property $\Gamma(p)$ for all $p \in \inter{\theta}$.

For contradiction, suppose there was some $s \in \R$ such that $\E_p L(s,Y) = \E_p L(r_\theta, Y)$ for some $p \in \inter{\theta}$.
%\jessie{(Note that if $E_p L(s,Y) < \E_p L(r_\theta, Y)$, then $r_\theta \not\in \Gamma(p)$, so we reduce to an equality.)}

As $p \in \inter{\theta}$, we then know $\E_q L(r_\theta, Y) \leq \E_q L(s,Y)$ for all $q \in B(\epsilon, p)$.
Therefore, $\Gamma_s \subseteq \inter{\Gamma_{r_\theta}} \subseteq \Gamma_{r_\theta}$.
As there are no dominated reports (by non-redundancy of $\Gamma$), this implies $\Gamma_s = \Gamma_{r_\theta}$.
Therefore, we can see that $r_\theta = s$, since otherwise we would contradict nonredundancy of $\Gamma$.
As $r_\theta = s$, we can then say $r_\theta$ is the unique property value on $\inter{\theta}$.
Therefore, $\Gamma(p) = \{r_\theta\}$ for all $p \in \inter{\theta}$.

As the full-dimensional level sets in $\Theta$ are finite and the optimal report is unique on the interior of level sets, $2 \implies 1$.

\end{proof}

\end{document}
%%% Local Variables:
%%% mode: latex
%%% TeX-master: t
%%% End:

