\documentclass[12pt]{article}
\usepackage[utf8]{inputenc}
\usepackage{mathtools, amsmath, amsthm, amssymb, graphicx, mathrsfs, verbatim}
%\usepackage[thmmarks, thref, amsthm]{ntheorem}
\usepackage{color}
\usepackage{wrapfig}
\usepackage{subcaption}
\usepackage[colorinlistoftodos,textsize=tiny]{todonotes} % need xargs for below
%\usepackage{accents}
\usepackage{bbm}
\usepackage{xspace}
\usepackage[margin=1.25in]{geometry}

\usepackage[colorlinks=true,breaklinks=true,bookmarks=true,urlcolor=blue,
     citecolor=blue,linkcolor=blue,bookmarksopen=false,draft=false]{hyperref}

\newcommand{\Comments}{1}
\newcommand{\mynote}[2]{\ifnum\Comments=1\textcolor{#1}{#2}\fi}
\newcommand{\mytodo}[2]{\ifnum\Comments=1%
  \todo[linecolor=#1!80!black,backgroundcolor=#1,bordercolor=#1!80!black]{#2}\fi}
\newcommand{\raf}[1]{\mynote{green}{[RF: #1]}}
\newcommand{\raft}[1]{\mytodo{green!20!white}{RF: #1}}
\newcommand{\jessie}[1]{\mynote{purple}{[JF: #1]}}
\newcommand{\jessiet}[1]{\mytodo{purple!20!white}{JF: #1}}
\newcommand{\btw}[1]{\mytodo{gray!20!white}{BTW: #1}}
\ifnum\Comments=1               % fix margins for todonotes
  \setlength{\marginparwidth}{1in}
\fi


\newcommand{\reals}{\mathbb{R}}
\newcommand{\posreals}{\reals_{>0}}%{\reals_{++}}

% alphabetical order, by convention
\newcommand{\D}{\mathcal{D}}
\newcommand{\E}{\mathbb{E}}
\newcommand{\F}{\mathcal{F}}
\newcommand{\I}{\mathcal{I}}
\renewcommand{\P}{\mathcal{P}}
\newcommand{\R}{\mathcal{R}}
\newcommand{\X}{\mathcal{X}}
\newcommand{\Y}{\mathcal{Y}}


\newcommand{\inter}[1]{\mathring{#1}}%\mathrm{int}(#1)}
\newcommand{\cl}[1]{\text{cl}(#1)}
%\newcommand{\expectedv}[3]{\overline{#1}(#2,#3)}
\newcommand{\expectedv}[3]{\E_{Y\sim{#3}} {#1}(#2,Y)}
\newcommand{\toto}{\rightrightarrows}
\newcommand{\trim}{\mathrm{trim}}
\newcommand{\fplc}{finite-piecewise-linear and convex\xspace} %xspace for use in text
\newcommand{\conv}{\mathrm{conv}}
\newcommand{\ones}{\mathbbm{1}}
\newcommand{\aff}{\text{aff}}
\newcommand{\im}{\text{im}}
\newcommand{\strip}{\mathrm{strip}}
\newcommand{\card}{\textbf{card}}

\DeclareMathOperator*{\argmax}{arg\,max}
\DeclareMathOperator*{\argmin}{arg\,min}
\DeclareMathOperator*{\arginf}{arg\,inf}
\DeclareMathOperator*{\sgn}{sgn}

\newtheorem{theorem}{Theorem}
\newtheorem{lemma}{Lemma}
\newtheorem{proposition}{Proposition}
\newtheorem{definition}{Definition}
\newtheorem{corollary}{Corollary}
\newtheorem{conjecture}{Conjecture}
\newtheorem{notation}{Notation}




\begin{document}

\section{Notation and Definitions}


A set $\Theta \subseteq 2^{\R}$ is an unlabeled property, and finite if $|\Theta| < \infty$.
Note that $\R$ finite implies $\Theta$ is finite.
Observe that $\Theta$ is a set of level sets, and not a function.


\begin{definition}
Let $\Gamma:\P\toto\R$.
Define $\text{strip}(\Gamma) := \{ \Gamma_r : r \in \R \}$.
\end{definition}
Informally, strip removes labels from the property.
\btw{Strip includes the boundaries of level sets / lower dimensional level sets.}

\begin{notation}
Let $\gamma: \P \toto\R'$ and $\Gamma: \P \toto\R$.
We say $\gamma \preceq \Gamma$ if for all $p \in \P$, we observe $\gamma(p) = \Gamma(p) \cap \R'$.
\end{notation}

%\begin{definition}
%	\jessie{Not sure we need this, but I feel like it might be useful in the future for showing uniqueness up to a relabeling.}
%	Let $\gamma$ and $\zeta$ be set-valued properties.
%	We say $\gamma~\approx~\zeta$ if $\strip(\gamma)~=~\strip(\zeta)$.
%\end{definition}

%Observe that for every set of properties $\gamma: \P \toto \R'$ and $\zeta: \P \toto \R$ such that $\gamma \approx \zeta$, there is a bijection $\phi:\R' \to \R$ so that $\phi(\gamma(p)) = \zeta(p)$ \footnote{Here, we slightly abuse notation and, given the finite set $S$, write $\phi(S) := \{\phi(x) : x \in S \}$.	($\phi$ is real valued, but we might be taking $\phi$ on every element of a set.)} for all $p \in \P$.
%\jessie{In hindsight-- not sure on this because of the extra dimension embedding counter example...}



\begin{definition}
	The elicitable property $\Gamma:\P\toto \R$ is an \emph{embedding} of $\gamma: \P \toto \R'$ if there exists an injection $\varphi:\R' \to \R$ such that $\varphi(r) \in \Gamma(p) \iff r \in \gamma(p)$. %there exists a property $\zeta : \P \toto \R$ so that $\zeta \preceq \Gamma$ and $\gamma \approx \zeta$.
\end{definition}

Note that we can take $\varphi^{-1}(r') = \{ r \in \R: \varphi(r) = r' \}$.
That is, we define the inverse of $\varphi$ as a possibly set-valued function.

\begin{definition}
A property $\Gamma: \P \to \R$ is \emph{finite} if $\R$ is finite. % $\im(\Gamma) := \{r\in\Gamma(p) : p\in\P, r \in \R \} = \bigcup_{p\in\P} \Gamma(p)$ is a finite set.
\end{definition}


\begin{definition}
	A level set $\theta$ is \emph{full-dimensional} if it has nonempty interior.
\end{definition}



\section{Conjectures and Lemmas on the way}

\begin{conjecture}\label{conj:gam-prime-exists}
	Let $\Gamma:\P \toto \R$ be a nondegenerate elicitable property with a finite set of full-dimensional level sets $\Theta$ so that $\bigcup_{\theta \in \Theta}\theta = \P$.
	There exists a nondegenerate, nonredundant property $\gamma: \P \toto \R'$ such that $\gamma \preceq \Gamma$. 
\end{conjecture}
\begin{proof}
\jessie{Assumptions used: Finite FDLS}

	
	Construct the finite set $\R' = \{[r] \in \Gamma(p) : p \in \inter{\theta} \}_{\theta \in \Theta}$, where $[r]$ is one consistent report for all the reports $r, r'$ so that $\Gamma_r = \Gamma_{r'}$. 
	\jessie{Double check/clarify the statement of $\R'$.}
	(If $\R'$ is not finite, choose just one $r \in \R$ so that $\theta = \Gamma_r$.)
	Defining the property $\gamma : p \mapsto \Gamma(p) \cap \R'$, we see that $\gamma \preceq \Gamma$.
	%Equivalently, we can observe for all $p \in \P$, that $\gamma(p) = \{ [r] : r \in \Gamma(p) \cap \R' \}$. %\jessie{Don't think this is necessary...?}
	We want to show that the constructed $\gamma$ is nondegenerate and nonredundant.

	As $\Gamma$ is nondegenerate, there is some report $r \in \R$ such that $r \in \Gamma(p)$.
	Therefore, we know that $\gamma(p) = [r]$ for all $p\in \P$.
	Since $[r]$ is well-defined for every report $r \in \R$, we know $\gamma$ is nondegenerate.

	Now to see $\gamma$ is nonredundant, first consider two level sets $r_1, r_2$ such that $\Gamma_{r_1} = \Gamma_{r_2}$.
	In this case, $[r_1] = [r_2] = r'$, so $\gamma_{r'} = \gamma_{r'}$ does not lead to a redundant report.
	
	To see there are no reports $r, r'$ so that $\gamma_{r} \subsetneq \gamma_{r'}$, we use contradiction.
	Suppose there are reports $r, r' \in \R'$ so that $r \neq r'$ and $\gamma_{r'} \subsetneq \gamma_{r}$.
	By construction of $\R'$, we know both $\gamma_{r'}$ and $\gamma_r$ are full dimensional.
	Therefore, if both $r$ and $r'$ are optimal on a set of positive measure, we contradict convexity of the scoring rule $G$, since there is a bijection between the subgradients of $G$ and reports in $\R'$ (once we reduce to equivalence classes-- see Theorem~\ref{thm:bij-loss-to-score}).
	As $G$ is convex, it must be differentiable almost everywhere, but for every distribution in $\Gamma_{r'}$, the subgradient of $G$ is set-valued, and therefore $G$ is not differentiable on a set of positive Lebesgue measure in the simplex. \jessie{Waved my hands at the bijection bit, come back and clarify.}
	Therefore, the redundancy of $\gamma$ contradicts the elicitability of $\gamma$, so we conclude $\gamma$ is nonredundant. 

%	\jessie{This next paragraph is definitely incomplete (HOLE)}
%	If $\gamma$ is redundant, then remove any redundant reports from $\R'$.
%	Any level set with a redundant report must either (a.) equal to another level set or (b.) a proper subset of another level set. 
%	We know that (a.) does not happen from the construction of $\R'$ using only one equivalent report per level set.
%	To see that (b.) does not happen, suppose for contradiction that $r_1, r_2 \in \R'$ so that $\gamma_{r_1}\subsetneq \gamma_{r_2}$.
%	Then we know that $\gamma_{r_1}$ must not be full dimensional \jessie{by argument from scoring rule $G$} and therefore, $\gamma_{r_1}$ must have empty interior.
%	However, if $\gamma_{r_1} = \Gamma_{r_1}$ has empty interior, then $r_1\not\in \R'$, so we observe a contradiction.
%	Thus, $\gamma$ is nonredundant.
	
	Therefore, our constructed $\gamma$ is nondegenerate, nonredundant, and $\gamma \preceq \Gamma$.
	
\end{proof}


\begin{lemma}\label{lem:define-trim}
	Let $\Gamma:\P\toto\R$  be a nondegenerate, elicitable property with a finite set of level sets $\Theta$ so that $\bigcup_{\theta \in \Theta} \theta = \P$.
	There exists an unlabeled property $\Theta$ so that for all nondegenerate, nonredundant $\gamma \preceq \Gamma$, we have $\strip(\gamma) = \Theta := \trim(\Gamma)$.
\end{lemma}

\begin{proof}
	Let $\gamma:\P \toto \bar\R$ and $\Gamma' : \P \toto \R'$ be nondegenerate, nonredundant properties so that $\gamma \preceq \Gamma$ and $\Gamma'\preceq \Gamma$.
	We want to show $\strip(\gamma) = \strip(\Gamma')$.
	For each $\bar r \in \bar\R$, there is a set of reports in $S \subseteq \R$ so that there is a full-dimensional level set $\theta = \bigcap_{s\in S}\Gamma_s$ and $\bar r = [s]$ for every $s \in S$.
	For the same set of reports $S$, we can equivalently choose a different representative report $r' \in \R'$ so that $r' = [s]$ for every $s \in S$, since $\Gamma' \preceq \Gamma$.
	Therefore, there is a bijection $\varphi : \bar \R \to \R'$ so that $\varphi(\gamma(p)) = \Gamma'(p)$\footnote{Abusing notation for set $X$, $\varphi(X) = \{ \varphi(x) : x\in X \}$  } for all $p \in \P$.
	Therefore, the level sets of each report are the same since they are restrictions of the same property, so $\strip(\gamma) = \{ \gamma_r : r \in \bar \R \} = \{ \Gamma_r : r \in \bar \R \} = \{ \Gamma_{\varphi(r)} : r \in \bar \R \} = \{ \Gamma_r : r \in \R' \} = \strip(\Gamma')$.
\end{proof}




\begin{conjecture}\label{conj:trim-full-dim}
	Consider a nondegenerate, elicitable property $\Gamma$ with finitely many level sets.
	Every $\theta \in \Theta := \trim(\Gamma)$ is full-dimensional  
\end{conjecture}

\begin{proof}[Proof Sketch]
	\jessie{Uses: Conjecture~\ref{conj:gam-prime-exists}.}
	
	We define $\trim(\Gamma)$ in terms of the property $\gamma: \P \toto \R'$ such that $\gamma \preceq \Gamma$ and $\R'$ is as constructed in Conjecture~\ref{conj:gam-prime-exists}.
	For each $\theta \in \trim(\Gamma) = \strip(\gamma)$, there must be a report $r \in \R'$ such that $\theta = \gamma_r = \Gamma_{r}$.
	By construction, $r \in \R' \implies \inter{\gamma_r} = \inter{\theta} \neq \emptyset$.
	Since $\theta$ has nonempty interior, then $\theta$ is full dimensional. 
	
\end{proof}


\begin{conjecture}\label{conj:lev-sets-subsets}
	Let the property $\Gamma: \P \toto \reals^d$ be convex elicitable and the nondegenerate, nonredundant property $\gamma: \P \toto \R$ so that $\gamma \preceq \Gamma$.
	Then the level sets of $\Gamma$ are subsets of the level sets of $\gamma$.
	%\jessie{Statement needs to be changed so instead of assuming embedding, just want $\gamma \preceq \Gamma$.}
\end{conjecture}

\begin{proof}
	It is sufficient to show $\theta \in \strip(\Gamma) \implies \theta \in \strip(\gamma)$.
	For $\theta \in \strip(\Gamma)$, there is some report $r \in \reals^d$ so that $r \in \Gamma(p)$ for every $p \in \theta$.
	Now consider the report $r' = [r]$.
	By definition, $r \in \Gamma(p) \implies [r] = r' \in \gamma(p)$.
	Therefore, there is some $r' \in \R$ so that $p \in \Gamma_r \implies p \in \Gamma_{[r]} = \gamma_{r'}$.
%	For every $r \in \reals^d$, there is some $r' \in \R$ so that $ p \in \theta = \Gamma_r = \{ p \in \P : r \in \Gamma(p) \} \subseteq \Gamma_{r'} \implies p \in \Gamma_{r'}$ in two cases.
%	
%	First, if $r \not \in \R$, it is a redundant report, so its level set is a subset or equal to another level set for a report in $\R$.
%	In the second case, for $r \in \R$, we know $p \in \Gamma_{r} \implies p \in \{ p \in \P : r \in \Gamma(p) \cap \R \} = \{ p \in \P : r \in \gamma(p) \} = \gamma_{r}$. 
\end{proof}




\section{Defining Embeddability}


\begin{proposition}\label{prop:optimal-reports-per-level-set}
  Let $\Gamma:\P\toto\reals^d$ be a non-degenerate (convex) elicitable property.

  The following are equivalent:
  \begin{enumerate}
  \item There is a nondegenerate, finite property $\gamma:\P\toto\R'$ such that there is an injection $\varphi:\R'\to\reals^d$ such that $r\in\gamma(p) \iff \varphi(r) \in \Gamma(p)$. (i.e. $\Gamma$ embeds finite $\gamma$).  
  \item There is a finite set of full dimensional level sets $\Theta$ of $\Gamma$ that union to $\P$.
  \item $\strip(\Gamma)$ is finite.
  \item $\trim(\Gamma)$ is finite.
  \item $\trim(\Gamma)$ is finite and elicitable.
  \end{enumerate}
\end{proposition}

\begin{proof}
We proceed in a cycle to show $4 \implies 1 \implies 5 \implies 2 \implies 3 \implies 4$.

\begin{enumerate}


\item[$4 \implies 1$] 
Construct $\R' = \{ [r] \in \Gamma(p): p \in \inter{\theta} \}_{\theta\in\Theta}$ as in Conjecture~\ref{conj:gam-prime-exists}.
Let $\varphi : \R' \to \reals^d$ be the identity and $\gamma := \Gamma|_{\R'}$.
Since $\trim(\Gamma)$ is finite, we then know $\R'$ is finite, and so $\gamma$ is a finite property.
We can then easily see that $r \in \gamma(r) \implies r \in \Gamma(p)$ as $\R' \subseteq \reals^d$.
Additionally, for any $r \in \R'$ so that $r \in \Gamma(p)$, we know that $r \in \Gamma|_{\R'} = \gamma$.
Therefore, $\Gamma$ embeds $\Gamma|_{\R'} = \gamma$.


%Since $\Gamma$ has a finite set of full-dimensional level sets comprising $\trim(\Gamma)$ that union to $\P$ (by nondegeneracy of $\Gamma$), we can apply Conjecture~\ref{conj:gam-prime-exists} to observe there is a nondegenerate, nonredundant $\gamma \preceq \Gamma$, and we show $\Gamma$ embeds $\gamma$.
%First, we know $\gamma$ is finite as there is only one report $r \in \R'$ per full dimensional level set, of which we know there are finitely many since $\trim(\Gamma)$ is finite (as each level set in $\trim$ is full dimensional from Conjecture~\ref{conj:trim-full-dim}).

%For every $r' \in \R'$, let $\varphi(r') = \{ r\in\reals^d : r' = [r] \}$.
%Then clearly for all $r \in \Gamma(p)$, we know $[r] = r' \in \gamma(p)$.
%Additionally, we know that if $r' \in \gamma(p)$, then every equivalent report $r$ such that $[r] = r'$ as in $\Gamma(p)$, so $\Gamma$ embeds $\gamma$.
%\jessie{Circular logic...}

\item [$1 \implies 5$]
We know $\strip(\gamma) = \trim(\Gamma)$, and as $\gamma$ is finite, so is $\strip(\gamma)$ since strip is generated from the (finite) report set.
Additionally, we know that $L|_{\R'}$ elicits $\trim(\Gamma)$ from \emph{cvx-fin-notes.tex}.

\item [$5 \implies 2$]
Since $\Gamma$ is nondegenerate and every $\theta \in \trim(\Gamma)$ is full dimensional (Conjecture~\ref{conj:trim-full-dim}), it is clear that the union of the (full-dimensional) level sets of $\trim(\Gamma)$ union to $\P$ by nondegneracy of $\Gamma$.


\item [$2 \implies 3$]
	We know there is a nonredundant, nondegenerate property $\gamma : \P \toto \R$ so that $\gamma \preceq \Gamma$ and therefore $\strip(\gamma) = \trim(\Gamma) = \Theta$ by Conjecture~\ref{conj:gam-prime-exists} and the definition of trim (Lemma~\ref{lem:define-trim}).
	We then want to show $\strip(\Gamma) = \strip(\gamma) \bigcup S$, where $S$ is some finite set.
	Consider the level set $\Gamma_{r'}$ for $r' \in \reals^d$.
	If $r' \in \R$, then $\Gamma_{r'} \in \strip(\gamma)$.
	If $r' \not \in \R$, then there is a report $r \in \R$ so that the level set $\Gamma_{r'} \subseteq \Gamma_r = \gamma_r$ by Conjecture~\ref{conj:lev-sets-subsets}, and we know that $\Gamma_{r'}$ has empty interior by the construction of $\R$ and nonredundancy of $\gamma$, as $\{p \in \P: \{r_1, r_2\} \in \gamma(p), [r_1] \neq [r_2] \}$ must have measure $0$ by the convexity of $G$.
	\jessie{Waving my hands at this again, but same argument as the proof of Conjecture 1.}

			
	We know this only happens at lower dimensional level sets $\theta$ since the optimal report is unique on the interior of level sets as $\gamma$ is nonredundant.
	That is, for all level sets $\theta$ and $p \in \inter \theta, |\partial G(p)| = 1$.
	Since $\theta$ is not full dimensional and there is only a finite set of full dimensional level sets at union to $\P$, we know there is a finite set of full dimensional level sets $\{\theta_i\}_{i=1}^m$ in the $\epsilon$-ball around each $p \in \theta$ so that for any $p_i \in \inter{\theta_i}$, we observe $\partial G(p_i) \neq \partial G(p_j)$ for all $i \neq j$ and there is a bijection $\varphi$ so that $\varphi(\partial G(p_i)) = \gamma(p_i) = \{r_i\}$ for all $1 \leq i \leq m$.
	Additionally, as level sets are closed, we claim $p \in \gamma_{r_i} \implies p \in \Gamma_{r_i}$ for all $1 \leq i \leq m$.
	Therefore, $p \in \theta \implies p \in \bigcap_{r' \in \{r_i\}_{i=1}^m} \gamma_{r'}\implies p \in \bigcap_{r' \in \{r_i\}_{i=1}^m} \Gamma_{r'}$ by closure of level sets for some $\{r_i\}_{i=1}^m \subseteq \R'$.
	As $\R'$ is finite, so is the power set $2^{\R'}$, so the set $S = \{\Gamma_r : r \in \reals^d \setminus \R \}$ is finite.
	Therefore $\strip(\Gamma) = \strip(\gamma) \bigcup S$ is finite. 
		

\item [$3 \implies 4$]
Since $\trim(\Gamma) \subseteq \strip(\Gamma)$, then the strip being finite implies the trim is finite.


\end{enumerate} 

\end{proof}



\section{Lemmas that probably aren't necessary to show in a paper, but for the sake of convincing myself}

\begin{lemma}\label{lem:closed-level-sets}
	Let $\Gamma$ be elicitable.
	For any $r\in\R$, the level set $\Gamma_r = \{ p \in \Delta(\Y) : \Gamma(p) = r \}$ is closed.
\end{lemma}
\begin{proof}[Proof sketch]
	Let $L$ elicit $\Gamma$.
	The function $G(p) = -\min_{r\in\R} \E_p L(r,Y) = -\E_p L(\Gamma(p),Y)$ (taking any $r\in\Gamma(p)$ in case of ties) is convex.
	There exists $\D \subseteq \partial G$ and some bijection $\varphi : \R \to \D$ with $\Gamma(p) = \varphi^{-1}(\D\cap \partial G_p)$.
	Thus, $\Gamma_r = \{p : r\in\Gamma(p)\} = \{p : \varphi(r) \in \D\cap\partial G_p\} = \{p : \varphi(r) \in \partial G_p\}$, which is closed by convex analysis.
	See [Raf/Ian\footnote{\url{https://arxiv.org/pdf/1211.3043.pdf}} (Cor 3.11), with their Thm 3.5 and Rockefellar Thm 24.4]  
\end{proof}


\begin{definition}\label{def:unlabeled-nonredundant}
	An unlabeled property $\Theta$ is \emph{nonredundant} if for there are no full dimensional level sets $\theta_1, \theta_2 \in \Theta$ so that $\theta_1 \subseteq \theta_2$.
\end{definition}


\begin{lemma}\label{lem:nonempty-inter-iff-aff-ind}
	Let $\theta$ be a [closed] convex level set.
	$\theta$ is full dimensional $\iff$ there are distributions $p_1, p_2, \ldots, p_n \in \theta$ so that $p_1, p_2, \ldots, p_n$ are affinely independent.
\end{lemma}
\begin{proof}
	Recall that $n := |\Delta(\Y)|$, so we say the simplex is $n$-dimensional.
	\begin{enumerate}
		\item [$\implies$]
		Assume the level set $\theta$ is full dimensional.
		Then there exists a $p \in \inter{\theta}$ and for every $q \in B(\epsilon, p)$, we know that $q \in \theta$.
		Take $q_i = p + \epsilon e_i$\footnote{$e_i$ is the vector with $1$ in the $i^{th}$ component, and is $0$ elsewhere} in each of the $n$ dimensions, and since $\theta$ is full dimensional and closed, we know each $q_i \in\theta$.
		As $\{e_i\}_{i=1}^n$ is affinely independent (it forms the standard basis for $\reals^n$), so then is $\{q_i\}_{i=1}^n$ as we are invariant to additive and multiplicative shifts.		
		
		\item [$\impliedby$]
		Let the set of distributions $Q := \{q_i\}_{i=1}^n$ be affinely independent, where each $q_i \in \theta$.
		Since the set $\theta$ is convex, then the convex hull $\conv(Q) \subseteq \theta$.
		It is sufficient to show $\conv(Q)$ is full dimensional.
		Let $q_1, q_2 \in Q$ so that $q_1 \neq q_2$.
		As $p = \sum_{i=1}^n \frac{1}{n}q_i \in \conv(Q)$, as is every $q \in B(\epsilon, p)$ since they are also convex combinations of elements of $Q$, we can see $\emptyset \neq \inter{\conv(Q)} \subseteq \inter{\theta}$, so we conclude $\theta$ is full-dimensional.
	\end{enumerate}
\end{proof}

\begin{theorem}[F/K 2014] \label{thm:bij-loss-to-score}
	Let $\Gamma: \P \toto\R$ be nonredundant and elicited by $\Gamma$-regular \jessiet{Can we assume this with convexity?} affine loss $A: \R \times \P \to \reals$ be given.
	Then $L$ elicits $\Gamma$ if and only if there exists some convex $G: \conv(\P) \to \reals$ with $G(\P) \subseteq \reals$, some $\D \subseteq \partial G$, and a bijection $\varphi : \R \to \D$ so that $\Gamma(t) = \varphi^{-1}(\D \cap \partial G_t)$.
\end{theorem}



\end{document}
%%% Local Variables:
%%% mode: latex
%%% TeX-master: t
%%% End:

