\documentclass[12pt]{article}
\usepackage[utf8]{inputenc}
\usepackage{mathtools, amsmath, amsthm, amssymb, graphicx, mathrsfs, verbatim}
%\usepackage[thmmarks, thref, amsthm]{ntheorem}
\usepackage{color}
\usepackage{wrapfig}
\usepackage{subcaption}
\usepackage[colorinlistoftodos,textsize=tiny]{todonotes} % need xargs for below
%\usepackage{accents}
\usepackage{bbm}
\usepackage{xspace}
\usepackage[margin=1.25in]{geometry}

\usepackage[colorlinks=true,breaklinks=true,bookmarks=true,urlcolor=blue,
     citecolor=blue,linkcolor=blue,bookmarksopen=false,draft=false]{hyperref}

\newcommand{\Comments}{1}
\newcommand{\mynote}[2]{\ifnum\Comments=1\textcolor{#1}{#2}\fi}
\newcommand{\mytodo}[2]{\ifnum\Comments=1%
  \todo[linecolor=#1!80!black,backgroundcolor=#1,bordercolor=#1!80!black]{#2}\fi}
\newcommand{\raf}[1]{\mynote{green}{[RF: #1]}}
\newcommand{\raft}[1]{\mytodo{green!20!white}{RF: #1}}
\newcommand{\jessie}[1]{\mynote{purple}{[JF: #1]}}
\newcommand{\jessiet}[1]{\mytodo{purple!20!white}{JF: #1}}
\newcommand{\btw}[1]{\mytodo{gray!20!white}{BTW: #1}}
\ifnum\Comments=1               % fix margins for todonotes
  \setlength{\marginparwidth}{1in}
\fi


\newcommand{\reals}{\mathbb{R}}
\newcommand{\posreals}{\reals_{>0}}%{\reals_{++}}

% alphabetical order, by convention
\newcommand{\D}{\mathcal{D}}
\newcommand{\E}{\mathbb{E}}
\newcommand{\F}{\mathcal{F}}
\newcommand{\I}{\mathcal{I}}
\renewcommand{\P}{\mathcal{P}}
\newcommand{\R}{\mathcal{R}}
\newcommand{\X}{\mathcal{X}}
\newcommand{\Y}{\mathcal{Y}}


\newcommand{\inter}[1]{\mathring{#1}}%\mathrm{int}(#1)}
\newcommand{\cl}[1]{\text{cl}(#1)}
%\newcommand{\expectedv}[3]{\overline{#1}(#2,#3)}
\newcommand{\expectedv}[3]{\E_{Y\sim{#3}} {#1}(#2,Y)}
\newcommand{\toto}{\rightrightarrows}
\newcommand{\trim}{\mathrm{trim}}
\newcommand{\fplc}{finite-piecewise-linear and convex\xspace} %xspace for use in text
\newcommand{\conv}{\mathrm{conv}}
\newcommand{\ones}{\mathbbm{1}}
\newcommand{\aff}{\text{aff}}
\newcommand{\im}{\text{im}}
\newcommand{\strip}{\mathrm{strip}}
\newcommand{\card}{\textbf{card}}

\DeclareMathOperator*{\argmax}{arg\,max}
\DeclareMathOperator*{\argmin}{arg\,min}
\DeclareMathOperator*{\arginf}{arg\,inf}
\DeclareMathOperator*{\sgn}{sgn}

\newtheorem{theorem}{Theorem}
\newtheorem{lemma}{Lemma}
\newtheorem{proposition}{Proposition}
\newtheorem{definition}{Definition}
\newtheorem{corollary}{Corollary}
\newtheorem{conjecture}{Conjecture}
\newtheorem{notation}{Notation}




\begin{document}

\section{Notation and Definitions}


A set of level sets $\Theta$ is an unlabeled property, and finite if $|\Theta| < \infty$.

\begin{definition}
Let $\Gamma:\P\toto\R$.
Define $\text{strip}(\Gamma) := \{ \Gamma_r : r \in \R \}$.
\end{definition}
Informally, strip removes labels from the property.
\btw{Strip includes the boundaries of level sets / lower dimensional level sets.}

\begin{notation}
Let $\gamma: \P \toto\R$ and $\Gamma: \P \toto\reals^d$.
We say $\gamma \preceq \Gamma$ if for all $p \in \P$, we observe $\gamma(p) = \Gamma(p) \cap \R$.
\end{notation}

\begin{definition}
	The elicitable property $\Gamma:\P\toto \reals^d$ is an \emph{d-embedding} of $\gamma: \P \toto \R$ if there exists an injection $\varphi:\R \to \reals^d$ such that $\varphi(r) \in \Gamma(p) \iff r \in \gamma(p)$. 
\end{definition}

%Note that we can take $\varphi^{-1}(r') = \{ r \in \R: \varphi(r) = r' \}$.
%That is, we define the inverse of $\varphi$ as a possibly set-valued function.

\begin{definition}
	A level set $\theta$ is \emph{full-dimensional} if it has nonempty interior.
\end{definition}


\section{Lemmas on the way}

\begin{lemma}\label{lem:gam-prime-exists}
	Let $\Gamma:\P \toto \reals^d$ be a nondegenerate elicitable property with a finite set of full-dimensional level sets $\Theta$ so that $\bigcup_{\theta \in \Theta}\theta = \P$.
	There exists a nondegenerate, nonredundant property $\gamma: \P \toto \R$ such that $\gamma \preceq \Gamma$. 
\end{lemma}
\begin{proof}
	Construct the finite set $\R = \{[u] \in \Gamma(p) : p \in \inter{\theta} \}_{\theta \in \Theta}$, where $[u]$ is one consistent report for all the reports $r, r'$ such that $\Gamma_r = \Gamma_{r'}$. 
	Defining the property $\gamma : p \mapsto \Gamma(p) \cap \R$, we see that $\gamma \preceq \Gamma$.
	We want to show that the constructed $\gamma$ is nondegenerate and nonredundant.

	As $\Gamma$ is nondegenerate, there is some report $u \in \reals^d$ such that $u \in \Gamma(p)$ for all $p\in\P$, so $[u] \in \gamma(p)$ and is well-defined for all $p \in \P$.
	Thus, $\gamma$ is nondegenerate.

	To see $\gamma$ is nonredundant, first consider two reports $u_1, u_2$ such that $\Gamma_{u_1} = \Gamma_{u_2}$.
	In this case, $[u_1] = [u_2] = r$, so $\gamma_{r} = \gamma_{r}$ does not lead to a redundant report.
	
	\jessiet{This is the part I'm least confident about... see Lemma~\ref{lem:nonempty-inter-iff-aff-ind}}
	Now, suppose there were reports $r, r' \in \R$ so that $\gamma_{r} \subsetneq \gamma_{r'}$.
	By construction of $\R$, both $\gamma_{r'}$ and $\gamma_r$ are full dimensional.
	For each elicitable property, there is a convex scoring rule $G:\Delta_\Y \to \reals$ with a bijection between the subgradients of $G$ and reports in $\R$ (one we reduce to equivalent reports-- see Theorem~\ref{thm:bij-loss-to-score}).
	As $G$ is convex, it must be differentiable almost everywhere. 
	However, for every distribution $p \in \gamma_{r'} = \gamma_{r} \cap \gamma_{r'}$, the subgradient of $\partial G(p)$ is set-valued, and therefore $G$ is not differentiable, on this set of positive Lebesgue measure in the simplex. 
	Contradicting the elicitability of $\gamma$, we conclude $\gamma$ is nonredundant. 

	Therefore, our constructed $\gamma$ is nondegenerate, nonredundant, and $\gamma \preceq \Gamma$.
\end{proof}


\begin{lemma}\label{lem:define-trim}
	Let $\Gamma:\P\toto\reals^d$  be a nondegenerate, elicitable property with a finite set of level sets $\Theta$ so that $\bigcup_{\theta \in \Theta} \theta = \P$.
	There exists an unlabeled property $\Theta$ so that for all nondegenerate, nonredundant $\gamma \preceq \Gamma$, we have $\strip(\gamma) = \Theta := \trim(\Gamma)$.
\end{lemma}

\begin{proof}
	Let $\gamma:\P \toto \R$ and $\Gamma' : \P \toto \R'$ be nondegenerate, nonredundant properties so that $\gamma \preceq \Gamma$ and $\Gamma'\preceq \Gamma$.
	We want to show $\strip(\gamma) = \strip(\Gamma')$ in order to conclude that $\trim(\Gamma)$ is unique.
	For each $r \in \R$, there is a set of reports in $S \subseteq \reals^d$ so that there is a full-dimensional level set $\theta = \bigcap_{u\in S}\Gamma_s$ and $r = [u]$ for every $u \in S$.
	For the same set of reports $S$, we can equivalently choose a (possibly different) representative report $r' \in \R'$ so that $r' = [u]$ for every $u \in S$, since $\Gamma' \preceq \Gamma$.
	
	There is a bijection $\varphi : \R \to \R'$ so that $\varphi(\gamma(p)) = \Gamma'(p)$ for all $p \in \P$.
	Thus, the level sets of each report are the same since they are restrictions of the same property, as $\strip(\gamma) = \{ \gamma_r : r \in \R \} = \{ \Gamma_r : r \in \R \} = \{ \Gamma_{\varphi(r)} : r \in \R \} = \{ \Gamma_r : r \in \R' \} = \strip(\Gamma')$.
\end{proof}

\begin{lemma}\label{lem:lev-sets-subsets}
	Let the property $\Gamma: \P \toto \reals^d$ be elicitable and the nondegenerate, nonredundant property $\gamma: \P \toto \R$ so that $\gamma \preceq \Gamma$.
	Then the level sets of $\Gamma$ are subsets of the level sets of $\gamma$.
\end{lemma}

\begin{proof}
	It is sufficient to show $\theta \in \strip(\Gamma) \implies \theta \in \strip(\gamma)$.
	Given $\theta \in \strip(\Gamma)$, there is some report $u \in \reals^d$ so that $u \in \Gamma(p)$ for every $p \in \theta$.
	For every $r \in \R$, recall that there is an $u \in \reals^d$ so that $r = [u]$.
	By definition, $u \in \Gamma(p) \implies [u] = r \in \gamma(p)$.
	Therefore, there is some $r \in \R$ so that $p \in \Gamma_u \implies p \in \Gamma_{[u]} = \gamma_{r}$.
\end{proof}

\section{Defining Embeddability}

\begin{proposition}\label{prop:optimal-reports-per-level-set}
  Let $\Gamma:\P\toto\reals^d$ be a non-degenerate (convex) elicitable property.

  The following are equivalent:
  \begin{enumerate}
  \item There is a nondegenerate, finite property $\gamma:\P\toto\R$ such that there is an injection $\varphi:\R\to\reals^d$ such that $r\in\gamma(p) \iff \varphi(r) \in \Gamma(p)$. (i.e. $\Gamma$ embeds finite $\gamma$).  
  \item $\trim(\Gamma)$ is finite.     
  \item There is a finite set of full-dimensional level sets $\Theta$ of $\Gamma$ that union to $\P$.

  \end{enumerate}
\end{proposition}

\begin{proof}
We proceed in a cycle to show $1 \implies 2 \implies 3 \implies 1 $.
\begin{enumerate}


\item [$1 \implies 2$]

We know $\strip(\gamma) = \trim(\Gamma)$ (since $\varphi$ is an injection, and there is some intermediate property $\Gamma'$ such that $\Gamma' \preceq \Gamma$ and with a bijection $\phi:\Gamma'(p) \mapsto \gamma(p)$, so $\strip(\Gamma') = \strip(\gamma)$.) 
As $\gamma$ is finite, so is $\strip(\gamma)$ since strip is generated from the (finite) report set.

\item [$2 \implies 3$]
Since $\Gamma$ is nondegenerate and every $\theta \in \trim(\Gamma)$ is full dimensional, it is clear that the union of the (full-dimensional) level sets of $\trim(\Gamma)$ union to $\P$ by nondegneracy of $\Gamma$.
	

\item[$3 \implies 1$]
Construct $\R := \{[u] \in \Gamma(p) : p \in \inter{\theta} \}_{\theta \in \Theta}$.
The property $\gamma: p \mapsto \Gamma(p) \cap\R$ is then finite as $\R$ is finite.
Taking $\varphi$ to be the identity, $\Gamma$ embeds this finite $\gamma$.
\end{enumerate} 

\end{proof}



\section{Lemmas probably not necessary for the paper}

\begin{lemma}\label{lem:nonempty-inter-iff-aff-ind}
	Let $\theta$ be a [closed] convex level set.
	The following are equivalent:
	\begin{enumerate}
		\item $\theta$ is full-dimensional
		\item $m(\theta) > 0$
		\item There are distributions $p_1, p_2, \ldots, p_n \in \theta$ so that $p_1, p_2, \ldots, p_n$ are affinely independent.
	\end{enumerate}
\end{lemma}
\begin{proof}
	Recall that $n := |\Delta_{\Y}|$, so we say the simplex is $n$-dimensional.
	\begin{enumerate}
		\item [$1 \implies 2$]
		A convex set that is full-dimensional contains the $\epsilon$-ball, since it has nonempty interior, and this ball has positive measure.
		
		\item [$2 \implies 3$]
		A convex set of positive measure contains the $\epsilon$-ball, which is the superset of the convex hull of a $n$ affinely independent points, and thus contains those points.
		
		\item [$3 \implies 1$]
		Let the set of distributions $Q := \{q_i\}_{i=1}^n$ be affinely independent, where each $q_i \in \theta$.
		Since the set $\theta$ is convex, then $\conv(Q) \subseteq \theta$.
		Let $q_1, q_2 \in Q$ so that $q_1 \neq q_2$.
		As $p = \sum_{i=1}^n \frac{1}{n}q_i \in \conv(Q)$, as is every $q \in B(\epsilon, p)$ since they are also convex combinations of elements of $Q$, we can see $\emptyset \neq \inter{\conv(Q)} \subseteq \inter{\theta}$, so we conclude $\theta$ is full-dimensional.
	\end{enumerate}
\end{proof}

\begin{theorem}[F/K 2014] \label{thm:bij-loss-to-score}
	Let $\Gamma: \P \toto\R$ be nonredundant and elicited by $\Gamma$-regular \jessiet{Can we assume this with convex elicitability? I read $\Gamma$-regular to mean that $L(r,p) < \infty$ for all $r \in \R$.} affine loss $L: \R \times \P \to \reals$ be given.
	Then $L$ elicits $\Gamma$ if and only if there exists some convex $G: \conv(\P) \to \reals$ with $G(\P) \subseteq \reals$, some $\D \subseteq \partial G$, and a bijection $\varphi : \R \to \D$ so that $\Gamma(t) = \varphi^{-1}(\D \cap \partial G_t)$.
\end{theorem}

\newpage
\section{Definitions for paper}
\begin{definition}
	Let $\gamma:\Delta_\Y \toto \R$ and $\Gamma:\Delta_\Y \toto \R'$ be elicitable properties.
	The property $\gamma$ refines $\Gamma$ if, for all $r \in \R$, there is an $r' \in \R'$ so that $\gamma_r \subseteq \Gamma_{r'}$.
	[Additionally, there is at least one $r \in \R$ and $r' \in \R'$ such that $\gamma_r \subsetneq \Gamma_{r'}$.]
\end{definition}

\end{document}
%%% Local Variables:
%%% mode: latex
%%% TeX-master: t
%%% End:

