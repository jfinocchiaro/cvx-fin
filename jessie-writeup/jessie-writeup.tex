\documentclass[12pt]{article}
\usepackage[utf8]{inputenc}
\usepackage{mathtools, amsmath, amsthm, amssymb, graphicx, mathrsfs, verbatim}
%\usepackage[thmmarks, thref, amsthm]{ntheorem}
\usepackage{color}
\usepackage{wrapfig}
\usepackage{subcaption}
\usepackage[colorinlistoftodos,textsize=tiny]{todonotes} % need xargs for below
%\usepackage{accents}
\usepackage{bbm}
\usepackage{xspace}
\usepackage[margin=1.25in]{geometry}

\usepackage[colorlinks=true,breaklinks=true,bookmarks=true,urlcolor=blue,
     citecolor=blue,linkcolor=blue,bookmarksopen=false,draft=false]{hyperref}

\newcommand{\Comments}{1}
\newcommand{\mynote}[2]{\ifnum\Comments=1\textcolor{#1}{#2}\fi}
\newcommand{\mytodo}[2]{\ifnum\Comments=1%
  \todo[linecolor=#1!80!black,backgroundcolor=#1,bordercolor=#1!80!black]{#2}\fi}
\newcommand{\raf}[1]{\mynote{green}{[RF: #1]}}
\newcommand{\raft}[1]{\mytodo{green!20!white}{RF: #1}}
\newcommand{\jessie}[1]{\mynote{purple}{[JF: #1]}}
\newcommand{\jessiet}[1]{\mytodo{purple!20!white}{JF: #1}}
\ifnum\Comments=1               % fix margins for todonotes
  \setlength{\marginparwidth}{1in}
\fi


\newcommand{\reals}{\mathbb{R}}
\newcommand{\posreals}{\reals_{>0}}%{\reals_{++}}

% alphabetical order, by convention
\newcommand{\D}{\mathcal{D}}
\newcommand{\E}{\mathbb{E}}
\newcommand{\F}{\mathcal{F}}
\newcommand{\I}{\mathcal{I}}
\renewcommand{\P}{\mathcal{P}}
\newcommand{\R}{\mathcal{R}}
\newcommand{\X}{\mathcal{X}}
\newcommand{\Y}{\mathcal{Y}}


\newcommand{\inter}[1]{\mathring{#1}}%\mathrm{int}(#1)}
\newcommand{\cl}[1]{\text{cl}}
%\newcommand{\expectedv}[3]{\overline{#1}(#2,#3)}
\newcommand{\expectedv}[3]{\E_{Y\sim{#3}} {#1}(#2,Y)}
\newcommand{\toto}{\rightrightarrows}
\newcommand{\trim}{\mathrm{trim}}
\newcommand{\fplc}{finite-piecewise-linear and convex\xspace} %xspace for use in text
\newcommand{\conv}{\mathrm{conv}}
\newcommand{\ones}{\mathbbm{1}}
\newcommand{\im}{\text{im}}
\newcommand{\strip}{\text{strip}}


\DeclareMathOperator*{\argmax}{arg\,max}
\DeclareMathOperator*{\argmin}{arg\,min}
\DeclareMathOperator*{\arginf}{arg\,inf}
\DeclareMathOperator*{\sgn}{sgn}

\newtheorem{theorem}{Theorem}
\newtheorem{lemma}{Lemma}
\newtheorem{proposition}{Proposition}
\newtheorem{definition}{Definition}
\newtheorem{corollary}{Corollary}
\newtheorem{conjecture}{Conjecture}
\newtheorem{notation}{Notation}




\begin{document}

\section{Notation and Definitions}


Let $\Y$ be a finite outcome space, with $n:=|\Y|$, and $\Delta(\Y)$ be the set of probability distributions over $\Y$.

We define the \emph{level set} of a property at report value $r\in\R$ to be the set $\Gamma_r := \{p\in\P : r \in \Gamma(p)\}$.
Additionally, we can consider an unlabeled level set $\theta$ as this same set of distributions, but not associated with the report $r$.

The set $\Theta \subseteq 2^{\Delta(\Y)}$ is an unlabeled property, and finite if $|\Theta| < \infty$.
Observe that $\Theta$ is a set of level sets that divide the simplex, and not a function.


\begin{definition}
Let $\Gamma$ be a labeled property.
Define $\text{strip}(\Gamma) := \{ \Gamma_r : r \in \R \}$.
Informally, strip removes labels from the property.
\jessie{Strip includes the boundaries of level sets / lower dimensional level sets.}
\end{definition}

\begin{notation}
We will use the following notation: $\Gamma' = \Gamma \cap \R'$ is the property $\Gamma' : p \mapsto \Gamma(p) \cap \R'$.  
\end{notation}

\begin{notation}
  Consider two labeled properties $\Gamma' : \Delta(\Y) \toto \R$ and $\Gamma : \Delta(\Y) \toto \R$.
  We say $\Gamma' \leq \Gamma$ if you get $\Gamma'$ by removing reports from $\Gamma$.
  (i.e. $\Gamma' = \R' \cap \Gamma$ for some $\R' \subseteq \R$)
\end{notation}

\begin{definition}
A property $\Gamma$ is \emph{finite} if $\im(\Gamma) := \{r\in\Gamma(p) : p\in\Delta(\Y)\} = \bigcup_{p\in\Delta(\Y)} \Gamma(p)$ is a finite set.
\end{definition}

\begin{definition}\label{def:unlabeled-nonredundant}
  An unlabeled property $\Theta$ is \emph{nonredundant} if for any full dimensional $\theta_1, \theta_2 \in \Theta$ we observe $\theta_1 \cap \theta_2 = \emptyset$ almost everywhere.
\end{definition}



Other notes before starting:
\begin{itemize}
\item (Assuming the metric space is $(\Delta(\Y), \rho)$, where $\rho$ is Euclidean distance, although I don't think the choice of metric matters?).
We just use this looking at interiors of level sets.
\item The measure $\mu$ is arbitrary, but why not use Lebesgue measure on $\Delta(\Y) \subset \reals^{|\Delta(\Y)|-1}$
\item The interior of a level set $X$ is defined as $\inter{X} = \{p \in X: \epsilon > 0,  \forall q \in B(\epsilon,p), \argmin_rE_p L(r,Y) = \argmin_r E_q L(r,Y)\}$
\item Define the boundary $bd(X) := cl(X) \setminus \inter{X}$.
\item A level set is \emph{full-dimensional} if dim(affine hull($\Gamma_r$)) is equal to dim($\Delta(\Y)$). 
\item A full-dimensional level set has non-empty interior.
\end{itemize}

\section{Conjectures (to prove)}

\begin{conjecture}\label{conj:gam-prime-exists}
	Let $\Gamma$ be a nondegenerate elicitable property \jessie{with a finite set of level sets.}
	There exists a nondegenerate, nonredundant property $\Gamma'$ such that $\Gamma' \leq \Gamma$ and $\trim(\Gamma) = \strip(\Gamma')$.  
\end{conjecture}
\begin{proof}
\emph{Assumptions used: Finite level sets.}

	To start, construct $\R' \subseteq \R := \{ r \in \R : \inter{\Gamma_r} \neq \emptyset \}$.
	Clearly, $\R'$ is finite.
	Define the property $\Gamma' = \Gamma\cap \R'$.
	We want to show that the constructed $\Gamma'$ is nondegenerate and nonredundant.
	As $\Gamma$ is nondegenerate, there is some report $r \in \R$ such that $r \in \Gamma(p)$.
	If $p \in \inter{\Gamma}_r$, then by construction of $\R'$, we see $r \in \R'$.
	
	Thus, we just need concern ourselves with $p \not \in \inter{\Gamma_r}$.
	As $\Gamma$ is nondegenerate, there are then $n$ level sets $\{\Gamma_{r_i}\}_{i=1}^n, \in \R$ in the $\epsilon$-ball around $p$.
	We claim some $r_i$ is in $\R'$.
	Since there are only a finite set of level sets, at least one in the $\epsilon$-ball centered on $p$ is full dimensional.
	Therefore, there is some $r_i \in \R'$ so that $p \in cl(\Gamma_{r_i})$, and as level sets are closed by Lemma~\ref{lem:closed-level-sets}, we then observe $p \in \Gamma_{r_i}$.
	Therefore $\Gamma'$ is nondegenerate.
	
	Now to show $\Gamma'$ is nonredundant.
	For contradiction, suppose $\Gamma'$ is redundant.
	Then there are $r_1, r_2 \in \R'$ with $r_1 \neq r_2$ so that either $\Gamma'_{r_1} \subsetneq \Gamma'_{r_2}$ or $\Gamma'_{r_1} = \Gamma'_{r_2}$.
	
	First, suppose $\Gamma'_{r_1} = \Gamma'_{r_2}$
	Since we can think as the property values as labelings of the subgradient sets (denoted by $\partial$) of the the concave Bayes Risk $G(p)$.
	\jessie{Need source?}
	As $G$ is concave, it is differentiable almost everywhere, which means that for everywhere but a $\mu$-null set, $|\partial G(p)| = 1$.
	Then $\Gamma'(p) = \partial G(p) \implies |\Gamma'(p)| = 1$ almost everywhere, which-- along with the fact that the level sets $\Gamma'_{r_1}$ and $\Gamma'_{r_2}$ are full dimensional (and therefore have positive measure since they are convex), contradicts the claim $r_1 \neq r_2$.
	Thus we conclude there aren't reports $r_1, r_2 \in \R'$ so that $r_1 \neq r_2$ and $\Gamma'_{r_1} = \Gamma'_{r_2}$.
	
%	By construction of $\R'$, we know both $\Gamma'_{r_1}$ and  $\Gamma'_{r_2}$ are full dimensional.
%	If $\Gamma'_{r_1} = \Gamma'_{r_2} = \Gamma'_{r_1} \cap \Gamma'_{r_2}$ are full dimensional, this contradicts something?
	
	Otherwise suppose $\Gamma'_{r_1} \subsetneq \Gamma'_{r_2}$.
	As $\Gamma'_{r_1}$ is full-dimensional, so is $\Gamma'_{r_1} \cap \Gamma'_{r_2} = \Gamma'_{r_1}$.
	Again, since we know that $|\Gamma'(p)| = 1$ almost everywhere, and as the intersection of the two level sets is full-dimensional, this again contradicts the differentiability of $G$ almost everywhere, since the intersection of the level sets has positive measure in the simplex.
	Therefore, $\Gamma'$ is nonredundant.

		
	Finally, by definition of $\trim(\Gamma)$, we conclude $\Gamma'$ is a nondegenerate, nonredundant property such that $\trim(\Gamma) = \strip(\Gamma')$.
	
\end{proof}


\begin{conjecture}\label{conj:trim-full-dim}
	Every level set of $\Theta := \trim(\Gamma)$ is full-dimensional  
\end{conjecture}

\begin{proof}[Proof Sketch]
	\emph{Uses: Conjecture~\ref{conj:gam-prime-exists}.  Claim: $\strip(\Gamma')$ is nonredundant.}
	
	In order to understand $\trim(\Gamma)$ is, we use the $\Gamma' \leq \Gamma$ constructed in Conjecture~\ref{conj:gam-prime-exists}, and let $\R'$ be the set such that $\Gamma' = \Gamma\cap \R'$.
	First, we see $\Theta = \strip(\Gamma')$ is nonredundant in the unlabeled sense as $\Gamma'$ is nonredundant in the labeled set from Conjecture~\ref{conj:nonredundant-defs-equivalent}, since we know $\Gamma'$ is nonredundant-- and well-defined from Conjecture~\ref{conj:gam-prime-exists}.
	\jessie{To show.}
	
	
	
	For contradiction:
	Let $\theta \in \Theta$ not full-dim and non-redundant (whatever that means for an unlabeled property-- see definition \ref{def:unlabeled-nonredundant} section for proposed definition).
	
	We now show that every level set $\theta \in \Theta = \strip(\Gamma') = \trim(\Gamma)$ is full dimensional.
	
	We proceed in 2 cases for a level set $\theta \in \Theta$:
	\begin{enumerate}
		\item For all $\theta' \in \Theta$, $\theta \cap \theta' = \emptyset$.
		We know the level set $\theta$ is closed by Lemma~\ref{lem:closed-level-sets}
		By definition of $\R'$, there is some $p \in \inter{\theta}$ so that for $r_\theta = \Gamma'(p)$, there is a ball $B(\epsilon, p)$ centered around $p$ so that for every $q \in B(\epsilon, p)$, $r_\theta \in \Gamma'(q)$.
%		(In fact, as $\Theta$ is nonredundant, we claim $r_\theta = \Gamma'(q)$. \jessie{Need to show?})
		Therefore, the level set $\Gamma_{r_\theta} = \theta$ is full dimensional as it has nonempty interior.
		
		Intuitively, this can be seen since the interior of $\theta$ is nonempty by nondegeneracy of $\Gamma'$.
		
		\bigskip
		\hrule
		Other notes from meeting:\\
		Use closure and nondegeneracy
		
		Look at subset topology? ($\tau_{\R'} := \{\R' \cap U : U \in \tau \}$.)
		
		\hrule
		\item Exists $\theta' \in \Theta$, $\theta \cap \theta' \neq \emptyset$.
		
		By nonredundancy of $\Gamma'$, we then observe nonredundancy of $\Theta$, we know that $\mu(\theta \cap \theta') = 0$.
		We want to show that $\theta$ is full dimensional.
		\jessie{TODO}
		
		\bigskip
		\hrule
		Other notes from meeting:\\
		Look at expected loss/score of $\Gamma'$.
		\hrule
	\end{enumerate}
	
	Therefore, we have a contradiction and conclude $\theta \in \trim(\Gamma)$ is full dimensional. 
\end{proof}

\begin{conjecture}\label{conj:nonredundant-defs-equivalent}
	An elicitable property with finite level sets $\Gamma$ is nonredundant (in the labeled definition) $\iff$ $\strip(\Gamma)$ is nonredundant (in the unlabeled definition.)
\end{conjecture}
\begin{proof}
	We show both directions.\\
	\begin{enumerate}
		\item [$\implies$]
		Let the property $\Gamma:\P \to \R$ be nonredundant.
		Then for any two reports $r, r' \in \R$ with $r \neq r'$, the level sets $\theta := \Gamma_r \not \subseteq \Gamma_{r'}$ and $\theta' := \Gamma_{r'} \not \subseteq \Gamma_r$.
		In what follows, $G : \P \to \R$ is the Bayes Risk, or optimal report given a distribution $p$.
		Each report $r$ corresponds to a labeling of the subgradient sets of $G$.
		
		\jessie{Doesn't really use the assumption at all... uses concavity of Bayes Risk.}
		
		For notation, we write the set $S := \{p \in \P : |\partial(G(p))| > 1\}$
		
		Then consider the set $\theta \cap \theta'$.
		

%		Suppose for contradiction that $\inter{(\theta \cap \theta')} \neq \emptyset$.
%		Then there is some $s \in \inter{(\theta \cap \theta')}$ so that $p \in B(\epsilon, s) \implies p \in \inter{(\theta \cap \theta')}$.
%		As the ball $B(\epsilon s)$ is convex with nonempty interior, it has positive measure, and by monotonicity of a measure, we say $0 < \mu(B(\epsilon, s)) \leq \mu(\inter{(\theta \cap \theta')})$.
		
		Since the property $\Gamma$ is elicitable, the Bayes Risk is strictly convex-- meaning it is differentiable almost everywhere, thus $\mu(S) = 0$.
		
		Since we assume $r \neq r'$, we observe that $(\theta\cap\theta') \subseteq S$.
		By monotonicity of $\Gamma$, since $(\theta\cap\theta') \subseteq S$, we know $\mu(\theta \cap \theta') \leq \mu(S) = 0$.
		
%		This, however, contradicts the positive measure of the interior.
%		Therefore, it must be the case that the interior has Lebesgue measure $0$.

		Since $0 \leq \mu(\theta \cap \theta') \leq \mu(S) = 0$, we observe the unlabeled definition of nonredundancy.
		
		
		\item[$\impliedby$] 
		Say that for any two full dimensional level sets $\theta, \theta' \in \Theta := \strip(\Gamma)$ that $\mu(\theta \cap \theta') = 0$.
		Then there are reports $r, r' \in \R$ such that $\theta = \Gamma_r$ and $\theta' = \Gamma_{r'}$.
		As $\mu(\Gamma_r \cap \Gamma_{r'}) = 0 \implies$ the interior of $\Gamma_r \cap \Gamma_{r'}$ is empty.
		Therefore, the intersection of the two level sets is not full dimensional.
		
		As $\theta, \theta'$ are both full dimensional, than their intersection not being full dimensional implies they only share optimal reports on the boundary. \jessie{To cite?}
		
		Therefore, we can see that $\theta \not \subseteq \theta'$ as $\theta \neq \theta' \cap \theta$, nor are they equal as their intersection is lower dimensional.
		
		Therefore, $\strip(\Gamma)$ is nonredundant $\implies \Gamma$ is nonredundant.
	\end{enumerate}
	
\end{proof}


\section{Lemmas}

\begin{lemma}\label{lem:closed-level-sets}
	Let $\Gamma$ be elicitable.
	For any $r\in\R$, the level set $\Gamma_r = \{ p \in \Delta(\Y) : \Gamma(p) = r \}$ is closed.
\end{lemma}
\begin{proof}[Proof sketch]
	Let $L$ elicit $\Gamma$.
	The function $G(p) = -\min_{r\in\R} \E_p L(r,Y) = -\E_p L(\Gamma(p),Y)$ (taking any $r\in\Gamma(p)$ in case of ties) is convex.
	There exists $\D \subseteq \partial G$ and some bijection $\varphi : \R \to \D$ with $\Gamma(p) = \varphi^{-1}(\D\cap \partial G_p)$.
	Thus, $\Gamma_r = \{p : r\in\Gamma(p)\} = \{p : \varphi(r) \in \D\cap\partial G_p\} = \{p : \varphi(r) \in \partial G_p\}$, which is closed by convex analysis -- use \url{https://arxiv.org/pdf/1211.3043.pdf}.
	See [Raf/Ian, above (Cor 3.11) with their Thm 3.5 and Rockefellar 24.4]  
\end{proof}

\begin{lemma}\label{lem:intersect-lev-sets}
  Let the elicitable property $\Gamma$ have a finite set $\Theta$ of full-dimensional level sets, so that for all $\theta \in \Theta$,  there is a unique optimal report $r_\theta$ for all $p \in \inter{\theta}$.
  Consider the finite set $\R' := \{ r_\theta\}_{\theta \in \Theta}$.
  Then for every $r \in \R$, there exists a finite set $R \subseteq \R'$ so that $\Gamma_r = \bigcap_{r' \in R} \Gamma_{r'}$.
\end{lemma}
%\jessie{To show: can find set $R$ s.t. $\Gamma_r = \bigcap_{r' \in R} \Gamma_{r'}$, and $R$ is finite.}
\begin{proof}
  \emph{Facts used: Level sets are closed (Lemma \ref{lem:closed-level-sets}), non-degeneracy of $\Gamma$.}\\

  Note:    As $\Gamma$ is non-degenerate, we know for all $p \in \P$ that $|\Gamma(p)| \geq 1$.
  For any $p \in \P$, if $|\Gamma(p)| = 1$, then $\Gamma(p) = r \in \R'$, so we can choose $S := \{r\} \subseteq \R'$ so that $\Gamma_r = \cap_{r \in S} \Gamma_r$.
  
  In general, if $\Gamma(p)$ is set-valued, there is some collection $S \subseteq \R$ so that $p \in \bigcap_{r \in S} \Gamma_r$.
  We want to find set a \emph{finite} set $S \subseteq \R'$ so that the previous statement is true.
  ($S$ is not necessarily finite, but we want to show there is a finite $S$ so that's true.)

Construct the finite set $S_r := \{ r_i \in \R' : \forall q \in \Gamma_{r}, \exists p \in \Gamma_{r_i} \text{ s.t. } p \in B(\epsilon, q) \}$.
In words, $S_r$ is the set of reports that are uniquely optimal for some distribution in the $\epsilon$ ball around any $p \in \Gamma_r$.
$S_r \subseteq \R'$ by construction, so $S_r$ is finite. 
We then just want to show the equivalence of $\Gamma_r$ and $\cap_{r' \in S_r}\Gamma_{r'}$.

($\subseteq$)  We want to show $\Gamma_r \subseteq \cap_{r' \in S_r}\Gamma_{r'}$
Let $q \in \Gamma_r$.
If $q \in \Gamma_r$, then for every $r_i \in S_r$, there is some $p \in \inter{\Gamma_{r_i}}$ so that $p \in B(\epsilon, q)$.
Since $q$ is $\epsilon$-close to $p$, then we observe $q \in cl(\Gamma_{r_i})$.
\jessie{A bit hand-wavvy}

As level sets are closed (by Lemma \ref{lem:closed-level-sets}), $\Gamma_{r_i} = cl(\Gamma_{r_i})$, so $q \in \Gamma_{r_i}$.
%Thus as $q \in \Gamma_r \implies q \in \Gamma_{r_i}$ for all $r_i \in S_r$, $q \in \Gamma_r \implies q \in \cap_{r_i \in S} \Gamma_{r_i}$.
Thus as $q \in \Gamma_r \implies q \in \Gamma_{r_i}$ for all $r_i \in S_r$, we observe $\Gamma_r \subseteq \cap_{r_i \in S}\Gamma_{r_i}$.

\bigskip 
($\supseteq$)  Now suppose the distribution $t \in \cap_{r_i \in S_r}\Gamma_{r_i}$.
We want to show $t \in \Gamma_r$.

As $t \in \cap_{r_i \in S_r}\Gamma_{r_i}$, we know for every $r_i \in S_r$ that $t \in \Gamma_{r_i}$.

Since each $r_i \in S_r$, that means that for every $q \in \Gamma_r$, there is a distribution $p \in \Gamma_{r_i}$ that is in the $\epsilon$-ball $B$ around $q$.
As this is true for every $r_i \in S_r$, the distribution $t \in B(\epsilon, q)$ for some $q \in \Gamma_r$. 
Therefore, as level sets are closed, $t \in B(\epsilon, q) \implies t \in cl(\Gamma_r) \implies t \in \Gamma_r$.
Thus, $\cap_{r_i \in S_r}\Gamma_{r_i} \subseteq \Gamma_r$.

Therefore, by constructing the finite set $S_r$, we observe there is a finite set $S_r := R \subseteq \R' \subseteq \R$ such that for every $r \in \R$, we observe $\Gamma_r = \cap_{r_i \in R}\Gamma_{r_i}$.

\end{proof}


\begin{lemma}\label{lem:define-trim}
	Let $L: \R \to \reals^\Y$ elicit a nondegenerate property $\Gamma$.
	There exists an unlabeled property $\Theta$ so that for all nondegenerate, nonredundant $\Gamma' \leq \Gamma$, we have $\strip(\Gamma') = \Theta := \text{trim}(\Gamma)$.
\end{lemma}

\begin{proof}
	
	Consider the nonredundant, nondegenerate properties $\Gamma' \leq \Gamma$.
	As $\strip$ is well-defined, clearly $\Theta := \strip(\Gamma)$ is a well-defined set of level sets (property).
	
	We can define a nondegenerate $\Gamma' = \Gamma \cap \R'$ for some $\R' \subseteq \R$.
	Then we can write $\Theta' := \strip(\Gamma') = \{\Gamma_r : r \in \R' \}$.
	Our hope is to show $\Theta = \Theta'$.
	
	As $\R' \subseteq \R$, it is easy to see  $\Theta' \subseteq \Theta$, and therefore it is just left to show $\Theta \subseteq \Theta'$.
	
	Consider $\theta \in \Theta$.
	Then there is some $r \in \R$ such that $r\in\Gamma(p)\cap \R \implies r \in \Gamma(p)$ for all $p \in \theta$.
	
	We now want to show that there is some $r' \in \R'$ such that $r' \in \Gamma(p)$ for all $p \in \theta$.
	As $\Gamma'$ is nondegenerate, there exists some $r' \in \R'$ so that $r' \in \Gamma'(p) = \Gamma(p) \cap \R'$ for some $p \in \inter{\theta}$.
	Therefore, $\Gamma'_{r'} \subseteq \theta$.
	As $\Gamma'$ is elicitable, its level sets are convex, as is $\theta$.
	As no full-dimensional level set is the subset of another, it must either be the case that $\Gamma'_{r'}$ is not full dimensional or $\Gamma'_{r'} = \theta$.
	As we consider $r' \in \Gamma'(p)$ for some $p \in \inter{\theta}$, we know the first is not true, so it must be the case that $\Gamma'_{r'} = \theta$.
	Therefore, $r' \in \Gamma(p)$ for all $p \in \theta$.
	
	Therefore, $\theta \in \Theta'$,
	Since $\theta \in \Theta \implies \theta \in \Theta'$, we know $\Theta \subseteq \Theta'$.
	
	As $\Theta' \subseteq \Theta$, we conclude $\Theta = \Theta'$.
	
	Therefore, for all nonredundant, non-degenerate properties $\Gamma' \leq \Gamma$, we observe an unlabeled property $\trim(\Gamma) := \Theta = \strip(\Gamma')$.
	
\end{proof}


\section{Main statement}

Let $\Gamma$ be a finite, non-degenerate convex elicitable property.

\begin{enumerate}
\item The property $\Gamma$ has a finite set $\Theta$ of full-dimensonal level sets, and for all $\theta \in \Theta$, there exists an $r_\theta \in \R$ such that $\Gamma(p) = \{r_\theta\}$ for all $p \in \inter{\theta}$.
\item $\trim(\Gamma)$ is finite.  \raf{was strip}
\end{enumerate}

\begin{proposition}\label{prop:optimal-reports-per-level-set}
1 $\iff$ 2
\end{proposition}



Notes:
In order to prove this, we need to assume the property is non-redundant.


\begin{proof}
Let $L$ be the loss function eliciting the property $\Gamma$.

\begin{enumerate}
\item [$1 \implies 2$]
Consider the finite set $\R' := \{r_\theta\}_{\theta \in \Theta}$.
Clearly, $\R' \subseteq \R$.
We construct the property $\Gamma' := \Gamma \cap \R'$ so that $\Gamma' \leq \Gamma$.
If we can show $\Gamma'$ is nondegenerate and nonredundant, we can then show $\strip(\Gamma')$ is finite, which is $\trim(\Gamma)$ by definition.

Since there is an optimal report $r_\theta \in \Gamma'(p)$ for every $p \in \inter{\theta}$ and level sets are closed, there is then an optimal report $r_\theta \in \Gamma'(p)$ for every $p \in \theta$.
Therefore, $\Gamma'$ is nondegenerate.

To see $\Gamma'$ is nonredundant, consider the assumption of uniqueness of the optimal report on the interior of each level set.
Thus, a report could only be redundant on the boundary of a level set.
As $\R'$ is constructed to only contained optimal reports for full dimensional level sets, any $r$ such that $\Gamma_r$ is not full dimensional is not an element of $\R'$.
For any $r, r' \in \R'$ with $r \neq r'$, we can see that $\Gamma_r \neq \Gamma_{r'}$ as both level sets are full dimensional and  on the (nonempty) interior of each level set, the respective report us uniquely optimal by assumption.
To see (WLOG) $\Gamma_{r'} \not \subset \Gamma_r$, consider that since $\Gamma_r$ and $\Gamma_{r'}$ are full-dimensional, $\Gamma_{r'} \subseteq \Gamma_r \implies \Gamma_r \cap \Gamma_{r'} = \Gamma_{r'}$ is full dimensional, and we would observe $ r \in \Gamma(p)$ for all $p \in \inter{\Gamma_{r'}}$, contradicting unique optimality of the report $r'$ on the interior. 
We conclude $\Gamma'$ is nonredundant.


Consider $\strip(\Gamma') = \{\Gamma_r : r \in \R' \}$.
As $\R'$ is finite, so is $\strip(\Gamma')$.
Since $\Gamma'$ is a  nondegenerate, nonredundant property such that $\Gamma' \leq \Gamma$, we invoke the definition of trim to say $\trim(\Gamma) = \strip(\Gamma')$, so as $\strip(\Gamma')$ is finite, we conclude $\trim(\Gamma)$ is finite.


%We can rewrite $\strip(\Gamma) = \{\Gamma_{r'} : r' \in \R' \} \cup \{\Gamma_{r} : r \in \R \setminus \R' \}$.
%As the first set is finite, we just need to show $\{\Gamma_{r} : r \in \R \setminus \R' \}$ is finite in order to show $\strip(\Gamma)$ is finite.
%
%Let us consider some $r\in\R \setminus \R'$.
%The level set $\Gamma_r = \{p : \Gamma(p) = r\}$ by definition.
%As $r \not\in \R'$, the interior is empty; that is, $\inter{\Gamma_r} = \emptyset$.
%% For any $p \in \Gamma_r$, then, there is some $q \in B(\epsilon, p)$ such that $\E_q L(r', Y) < \E_q L(r,y)$ for some $r' \in \R'$.
%% As $r = \argmin_{\tilde{r}} \E_p L(\tilde{r}, Y)$, then we observe $\E_p L(r, Y) = \E_p L(r', Y) = \min_{\tilde{r}}\E_p L(\tilde{r}, Y)$ for $p \in \Gamma_r$. 
%% Thus, $\Gamma_r \subseteq \Gamma_{r'}$.
%By Conjecture \ref{lem:intersect-lev-sets}, there exist a set of finite reports $R \subseteq \R'$ such that $\bigcap_{r'\in R} \Gamma_{r'} = \Gamma_r$.
%As the set $\R'$ is finite, then the power set $2^{\R'}$ is finite.
%Thus, the set $\{ \Gamma_r : r \in \R \setminus \R' \} =  \{ \bigcap_{r_1 \in R} \Gamma_{r_1} : R \subseteq \R' \}$ is finite.
%
%
%The union of a two finite sets is then finite, so $\trim(\Gamma) \subseteq \strip(\Gamma) = \{ \Gamma_{r'} : r' \in \R' \} \cup \{ \Gamma_r : r \in \R \setminus \R' \}$ is finite.



\item[$2 \implies 1$]  
\emph{Assumptions used: Nondegeneracy, elicitability of $\Gamma$}

Let $\Theta := \trim(\Gamma)$ be a finite set.
By Conjecture~\ref{conj:gam-prime-exists}, there is then a nonredundant, nondegenerate property $\Gamma'$ such that $\strip(\Gamma') = \trim(\Gamma)$.
Let $\R' \subseteq \R$ be the finite set used to define the property $\Gamma'$.

As $\Theta$ is finite and $\trim(\Gamma)$ is the set of full dimensional level sets of $\Gamma$, the finite number of full-dimensional level sets follows from the assumption and definitions combined.

Consider a level set $\theta\in \Theta$.
There is a report $r \in \R'$ so that $\Gamma'(p) = \{r\}$ for all $p \in \inter{\theta}$-- this follows from nonredundancy of $\Gamma'$.

Therefore if the function $\trim(\Gamma)$ is well-defined and finite, then the labeled property $\Gamma$ has a finite set of full-dimensional level sets and for every $\theta \in \Theta$, there is a unique optimal report for every distribution $p \in \inter{\theta}$.


%Let $\Theta := \trim(\Gamma)$ be a finite set.
%By construction, $\Theta$ is a set of full dimensional level sets defining the property $\Gamma$.
%Thus, $\Gamma$ then only has finitely many \emph{full-dimensional} level sets.
%Then there is a finite set $\R' \subseteq \R$ defined by the property $\Gamma' \leq \Gamma$ where $\strip(\Gamma') = \trim(\Gamma)$.\jessie{Make sure I can claim this.}
%
%Consider a level set $\theta \in \Theta$.
%We want to show there is a unique $r \in \R'$ such that $\Gamma(p) = \{r\}$ for all $p \in \inter{\theta}$.
%
%Take the finite set $\R' \subseteq \R$ such that the nondegenerate, nonredundant property $\Gamma' \leq \Gamma$.
%Fix some level set $\theta \in \Theta$.
%For any $p \in \inter{\theta}$, there is then a report $r \in \R'$ such that  $r \in \Gamma(p)$ by nondegeneracy of $\Gamma'$.
%We then want to show $r$ is unique.
%
%For contradiction, consider some $r' \in \R'$ such that $r \neq r'$.
%As $\trim(\Gamma) = \strip(\Gamma')$ contains only full dimensional level sets, then $\Gamma_r$ and $\Gamma_{r'}$ are both full dimensional, as is their intersection $\Gamma_r \cap \Gamma_{r'}$.
%Since we define $\Gamma'$ in a manner so that for all $r \in \R' \implies \Gamma_r = \Gamma'_r$, we then say $\Gamma_r \cap \Gamma_{r'} = \Gamma'_r \cap \Gamma'_{r'}$, which contradicts the nonredundancy of $\Gamma'$.
%Therefore we observe a contradiction, and conclude $r = r'$.
%Therefore, if $\trim(\Gamma)$ is finite, then the property $\Gamma$ has a finite set (namely, $\trim(\Gamma)$) of full dimensional level sets so that for every $\theta \in \Theta$, there is a unique optimal report $r$ for every $p \in \inter{\theta}$.


\end{enumerate} 

\end{proof}

\section{To Do after meeting 9.24.18}



look at Rockefellar- algebraic interior
Appendix of complexity paper- the Bayes Risk must be strictly concave.


Order of To-Dos -- ordered by dependency
\begin{enumerate}
\item Show there exists a nondeg, nonred $\Gamma' \leq \Gamma$ that $\trim(\Gamma) = \strip(\Gamma')$. (Conjecture~\ref{conj:gam-prime-exists}) -- Done
\item Show every level set of the trim is Full Dimensional (Conjecture~\ref{conj:trim-full-dim}).
\item Proof of $2 \implies 1$-- Done-ish.
\end{enumerate}

\end{document}
%%% Local Variables:
%%% mode: latex
%%% TeX-master: t
%%% End:

