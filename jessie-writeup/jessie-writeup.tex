\documentclass[12pt]{article}
\usepackage[utf8]{inputenc}
\usepackage{mathtools, amsmath, amsthm, amssymb, graphicx, mathrsfs, verbatim}
%\usepackage[thmmarks, thref, amsthm]{ntheorem}
\usepackage{color}
\usepackage{wrapfig}
\usepackage{subcaption}
\usepackage[colorinlistoftodos,textsize=tiny]{todonotes} % need xargs for below
%\usepackage{accents}
\usepackage{bbm}
\usepackage{xspace}
\usepackage[margin=1.25in]{geometry}

\usepackage[colorlinks=true,breaklinks=true,bookmarks=true,urlcolor=blue,
     citecolor=blue,linkcolor=blue,bookmarksopen=false,draft=false]{hyperref}

\newcommand{\Comments}{1}
\newcommand{\mynote}[2]{\ifnum\Comments=1\textcolor{#1}{#2}\fi}
\newcommand{\mytodo}[2]{\ifnum\Comments=1%
  \todo[linecolor=#1!80!black,backgroundcolor=#1,bordercolor=#1!80!black]{#2}\fi}
\newcommand{\raf}[1]{\mynote{green}{[RF: #1]}}
\newcommand{\raft}[1]{\mytodo{green!20!white}{RF: #1}}
\newcommand{\jessie}[1]{\mynote{purple}{[JF: #1]}}
\newcommand{\jessiet}[1]{\mytodo{purple!20!white}{JF: #1}}
\ifnum\Comments=1               % fix margins for todonotes
  \setlength{\marginparwidth}{1in}
\fi


\newcommand{\reals}{\mathbb{R}}
\newcommand{\posreals}{\reals_{>0}}%{\reals_{++}}

% alphabetical order, by convention
\newcommand{\D}{\mathcal{D}}
\newcommand{\E}{\mathbb{E}}
\newcommand{\F}{\mathcal{F}}
\newcommand{\I}{\mathcal{I}}
\renewcommand{\P}{\mathcal{P}}
\newcommand{\R}{\mathcal{R}}
\newcommand{\X}{\mathcal{X}}
\newcommand{\Y}{\mathcal{Y}}


\newcommand{\inter}[1]{\mathring{#1}}%\mathrm{int}(#1)}
\newcommand{\cl}[1]{\text{cl}}
%\newcommand{\expectedv}[3]{\overline{#1}(#2,#3)}
\newcommand{\expectedv}[3]{\E_{Y\sim{#3}} {#1}(#2,Y)}
\newcommand{\toto}{\rightrightarrows}
\newcommand{\trim}{\mathrm{trim}}
\newcommand{\fplc}{finite-piecewise-linear and convex\xspace} %xspace for use in text
\newcommand{\conv}{\mathrm{conv}}
\newcommand{\ones}{\mathbbm{1}}
\newcommand{\im}{\text{im}}
\newcommand{\strip}{\text{strip}}


\DeclareMathOperator*{\argmax}{arg\,max}
\DeclareMathOperator*{\argmin}{arg\,min}
\DeclareMathOperator*{\arginf}{arg\,inf}
\DeclareMathOperator*{\sgn}{sgn}

\newtheorem{theorem}{Theorem}
\newtheorem{lemma}{Lemma}
\newtheorem{proposition}{Proposition}
\newtheorem{definition}{Definition}
\newtheorem{corollary}{Corollary}
\newtheorem{conjecture}{Conjecture}
\newtheorem{notation}{Notation}




\begin{document}

\section{Notation and Definitions}

Let $\Y$ be a finite outcome space, with $n:=|\Y|$, and $\Delta(\Y)$ be the set of probability distributions over $\Y$.

We define the \emph{level set} of a property at report value $r\in\R$ to be the set $\Gamma_r := \{p\in\P : r \in \Gamma(p)\}$.
Additionally, we can consider an unlabeled level set $\theta$ as this same set of distributions, but not associated with the report $r$.

The set $\Theta \subseteq 2^{\Delta(\Y)}$ is an unlabeled property, and finite if $|\Theta| < \infty$.
Observe that $\Theta$ is a set of level sets that divide the simplex, and not a function.

\begin{definition}
Let $\Gamma$ be a labeled property.
Define $\text{strip}(\Gamma) := \{ \Gamma_r : r \in \R \}$.
Informally, strip removes labels from the property.
\end{definition}

\begin{notation}
We will use the following notation: $\Gamma' = \Gamma \cap \R'$ is the property $\Gamma' : p \mapsto \Gamma(p) \cap \R'$.  
\end{notation}


\begin{notation}
  Consider two labeled properties $\Gamma' : \Delta(\Y) \toto \R$ and $\Gamma : \Delta(\Y) \toto \R$.
  We say $\Gamma' \leq \Gamma$ if you get $\Gamma'$ by removing reports from $\Gamma$.
  (i.e. $\Gamma' = \R' \cap \Gamma$ for some $\R' \subseteq \R$)
\end{notation}

\begin{definition}
A property $\Gamma$ is \emph{finite} if $\im(\Gamma) := \{r\in\Gamma(p) : p\in\Delta(\Y)\} = \bigcup_{p\in\Delta(\Y)} \Gamma(p)$ is a finite set.
%\jessie{Take 2:}  A property $\Gamma$ is \emph{finite} if there are only finitely many full-dimensional level sets.
\end{definition}

\begin{lemma}\label{conj:closed-level-sets}
  Let $\Gamma$ be elicitable.
  For any $r\in\R$, the level set $\Gamma_r = \{ p \in \Delta(\Y) : \Gamma(p) = r \}$ is closed.
\end{lemma}
\begin{proof}[Proof sketch]
  Let $L$ elicit $\Gamma$.
  The function $G(p) = -\min_{r\in\R} \E_p L(r,Y) = -\E_p L(\Gamma(p),Y)$ (taking any $r\in\Gamma(p)$ in case of ties) is convex.
  There exists $\D \subseteq \partial G$ and some bijection $\varphi : \R \to \D$ with $\Gamma(p) = \varphi^{-1}(\D\cap \partial G_p)$.
  Thus, $\Gamma_r = \{p : r\in\Gamma(p)\} = \{p : \varphi(r) \in \D\cap\partial G_p\} = \{p : \varphi(r) \in \partial G_p\}$, which is closed by convex analysis -- use \url{https://arxiv.org/pdf/1211.3043.pdf}.
	See [Raf/Ian, above (Cor 3.11) with their Thm 3.5 and Rockefellar 24.4]  
\end{proof}


Notes before starting:
\begin{itemize}
\item (Assuming the metric space is $(\Delta(\Y), \rho)$, where $\rho$ is Euclidean distance, although I don't think the choice of metric matters?).
We just use this looking at interiors of level sets.
\item The interior of a level set $X$ is defined as $\inter{X} = \{p \in X: \epsilon > 0,  \forall q \in B(\epsilon,p), \argmin_rE_p L(r,Y) = \argmin_r E_q L(r,Y)\}$
\item A level set is \emph{full-dimensional} if dim(affine hull($\Gamma_r$)) is equal to dim($\Delta(\Y)$). 
\item A full-dimensional level set has non-empty interior.
\end{itemize}

\section{Conjecture turned Lemma}


\begin{lemma}\label{lem:intersect-lev-sets}
  Let the elicitable property $\Gamma$ have a finite set $\Theta$ of full-dimensional level sets, so that for all $\theta \in \Theta$,  there is a unique optimal report $r_\theta$ for all $p \in \inter{\theta}$.
  Consider the finite set $\R' := \{ r_\theta\}_{\theta \in \Theta}$.
  Then for every $r \in \R$, there exists a finite set $R \subseteq \R'$ so that $\Gamma_r = \bigcap_{r' \in R} \Gamma_{r'}$.
\end{lemma}
\jessie{To show: can find set $R$ s.t. $\Gamma_r = \bigcap_{r' \in R} \Gamma_{r'}$, and $R$ is finite.}
\begin{proof}
  \emph{Facts used: Level sets are closed, non-degeneracy of level sets}\\

\hrule
\bigskip
  Note:    As $\Gamma$ is non-degenerate, we know for all $p \in \P$ that $|\Gamma(p)| \geq 1$.
  For any $p \in \P$, if $|\Gamma(p)| = 1$, then $\Gamma(p) = r \in \R'$, so we can choose $S := \{r\} \subseteq \R'$ so that $\Gamma_r = \cap_{r \in S} \Gamma_r$.
  
  In general, if $\Gamma(p)$ is set-valued, there is some collection $S \subseteq \R$ so that $p \in \bigcap_{r \in S} \Gamma_r$.
  We want to find set a \emph{finite} set $S \subseteq \R'$ so that the previous statement is true.
  ($S$ is not necessarily finite, but we want to show there is a finite $S$ so that's true.)

Construct the finite set $S_r := \{ r_i \in \R' : \forall q \in \Gamma_{r}, \exists p \in \Gamma_{r_i} \text{ s.t. } p \in B(\epsilon, q) \}$.
In words, $S_r$ is the set of reports that are uniquely optimal for some distribution in the $\epsilon$ ball around any $p \in \Gamma_r$.
$S_r \subseteq \R'$ by construction, so $S_r$ is finite. 
We then just want to show the equivalence of $\Gamma_r$ and $\cap_{r' \in S_r}\Gamma_{r'}$.

($\subseteq$)  We want to show $\Gamma_r \subseteq \cap_{r' \in S_r}\Gamma_{r'}$
Let $q \in \Gamma_r$.
If $q \in \Gamma_r$, then for every $r_i \in S_r$, there is some $p \in \inter{\Gamma_{r_i}}$ so that $p \in B(\epsilon, q)$.
Since $q$ is $\epsilon$-close to $p$, then we observe $q \in cl(\Gamma_{r_i})$.
\jessie{A bit hand-wavvy}

As level sets are closed (by Lemma \ref{conj:closed-level-sets}), $\Gamma_{r_i} = cl(\Gamma_{r_i})$, so $q \in \Gamma_{r_i}$.
%Thus as $q \in \Gamma_r \implies q \in \Gamma_{r_i}$ for all $r_i \in S_r$, $q \in \Gamma_r \implies q \in \cap_{r_i \in S} \Gamma_{r_i}$.
Thus as $q \in \Gamma_r \implies q \in \Gamma_{r_i}$ for all $r_i \in S_r$, we observe $\Gamma_r \subseteq \cap_{r_i \in S}\Gamma_{r_i}$.

\bigskip 
($\supseteq$)  Now suppose the distribution $t \in \cap_{r_i \in S_r}\Gamma_{r_i}$.
We want to show $t \in \Gamma_r$.

As $t \in \cap_{r_i \in S_r}\Gamma_{r_i}$, we know for every $r_i \in S_r$ that $t \in \Gamma_{r_i}$.

Since each $r_i \in S_r$, that means that for every $q \in \Gamma_r$, there is a distribution $p \in \Gamma_{r_i}$ that is in the $\epsilon$-ball $B$ around $q$.
As this is true for every $r_i \in S_r$, the distribution $t \in B(\epsilon, q)$ for some $q \in \Gamma_r$. 
Therefore, as level sets are closed, $t \in B(\epsilon, q) \implies t \in cl(\Gamma_r) \implies t \in \Gamma_r$.
Thus, $\cap_{r_i \in S_r}\Gamma_{r_i} \subseteq \Gamma_r$.

Therefore, by constructing the finite set $S_r$, we observe there is a finite set $S_r := R \subseteq \R' \subseteq \R$ such that for every $r \in \R$, we observe $\Gamma_r = \cap_{r_i \in R}\Gamma_{r_i}$.

\end{proof}


\section{Formal Statements and modifications needed to prove}


\begin{lemma}\label{lem:define-trim}
  Let $L: \R \to \reals^\Y$ elicit a nondegenerate property $\Gamma$.
There exists an unlabeled property $\Theta$ so that for all nondegenerate, nonredundant $\Gamma' \leq \Gamma$, we have $\strip(\Gamma') = \Theta := \text{trim}(\Gamma)$.
\end{lemma}

\begin{proof}

Consider the nonredundant, nondegenerate properties $\Gamma' \leq \Gamma$.
As $\strip$ is well-defined, clearly $\Theta := \strip(\Gamma)$ is a well-defined set of level sets (property).

We can define a nondegenerate $\Gamma' = \Gamma \cap \R'$ for some $\R' \subseteq \R$.
Then we can write $\Theta' := \strip(\Gamma') = \{\Gamma_r : r \in \R' \}$.
Our hope is to show $\Theta = \Theta'$.

As $\R' \subseteq \R$, it is easy to see  $\Theta' \subseteq \Theta$, and therefore it is just left to show $\Theta \subseteq \Theta'$.

Consider $\theta \in \Theta$.
Then there is some $r \in \R$ such that $r\in\Gamma(p)\cap \R \implies r \in \Gamma(p)$ for all $p \in \theta$.

We now want to show that there is some $r' \in \R'$ such that $r' \in \Gamma(p)$ for all $p \in \theta$.
As $\Gamma'$ is nondegenerate, there exists some $r' \in \R'$ so that $r' \in \Gamma'(p) = \Gamma(p) \cap \R'$ for some $p \in \inter{\theta}$.
Therefore, $\Gamma'_{r'} \subseteq \theta$.
As $\Gamma'$ is elicitable, its level sets are convex, as is $\theta$.
As no full-dimensional level set is the subset of another, it must either be the case that $\Gamma'_{r'}$ is not full dimensional or $\Gamma'_{r'} = \theta$.
As we consider $r' \in \Gamma'(p)$ for some $p \in \inter{\theta}$, we know the first is not true, so it must be the case that $\Gamma'_{r'} = \theta$.
Therefore, $r' \in \Gamma(p)$ for all $p \in \theta$.

Therefore, $\theta \in \Theta'$,
Since $\theta \in \Theta \implies \theta \in \Theta'$, we know $\Theta \subseteq \Theta'$.

As $\Theta' \subseteq \Theta$, we conclude $\Theta = \Theta'$.

Therefore, for all nonredundant, non-degenerate properties $\Gamma' \leq \Gamma$, we observe an unlabeled property $\trim(\Gamma) := \Theta = \strip(\Gamma')$.

\end{proof}


\subsection{Main statement}

Let $\Gamma$ be a finite, non-degenerate convex elicitable property.

\begin{enumerate}
\item The property $\Gamma$ has a finite set $\Theta$ of full-dimensonal level sets, and for all $\theta \in \Theta$, there exists an $r_\theta \in \R$ such that $\Gamma(p) = \{r_\theta\}$ for all $p \in \inter{\theta}$.
\item $\trim(\Gamma)$ is finite.  \raf{was strip}
\end{enumerate}

\begin{proposition}\label{prop:optimal-reports-per-level-set}
1 $\iff$ 2
\end{proposition}



Notes:
In order to prove this, we need to assume the property is non-redundant.

As $\Gamma \leq \Gamma$, then by Lemma \ref{lem:define-trim}, $\trim(\Gamma) = \strip(\Gamma)$.


\begin{proof}
Let $L$ be the loss function eliciting the property $\Gamma$.

$1 \implies 2$. 
% \emph{Assumptions used: non-redundancy}


Consider $\strip(\Gamma) = \{\Gamma_r : r \in \R \}$.
Let us construct the finite set $\R' = \{r_\theta \}_{\theta \in \Theta}$.
Clearly, $\R' \subseteq \R$.

We can rewrite $\strip(\Gamma) = \{\Gamma_{r'} : r' \in \R' \} \cup \{\Gamma_{r} : r \in \R \setminus \R' \}$.
As the first set is finite, we just need to show $\{\Gamma_{r} : r \in \R \setminus \R' \}$ is finite in order to show $\strip(\Gamma)$ is finite.

Let us consider some $r\in\R \setminus \R'$.
The level set $\Gamma_r = \{p : \Gamma(p) = r\}$ by definition.
As $r \not\in \R'$, the interior is empty; that is, $\inter{\Gamma_r} = \emptyset$.
% For any $p \in \Gamma_r$, then, there is some $q \in B(\epsilon, p)$ such that $\E_q L(r', Y) < \E_q L(r,y)$ for some $r' \in \R'$.
% As $r = \argmin_{\tilde{r}} \E_p L(\tilde{r}, Y)$, then we observe $\E_p L(r, Y) = \E_p L(r', Y) = \min_{\tilde{r}}\E_p L(\tilde{r}, Y)$ for $p \in \Gamma_r$. 
% Thus, $\Gamma_r \subseteq \Gamma_{r'}$.
By Conjecture \ref{lem:intersect-lev-sets}, there exist a set of finite reports $R \subseteq \R'$ such that $\bigcap_{r'\in R} \Gamma_{r'} = \Gamma_r$.
As the set $\R'$ is finite, then the power set $2^{\R'}$ is finite.
Thus, the set $\{ \Gamma_r : r \in \R \setminus \R' \} =  \{ \bigcap_{r_1 \in R} \Gamma_{r_1} : R \subseteq \R' \}$ is finite.


The union of a two finite sets is then finite, so $\trim(\Gamma) = \strip(\Gamma) = \{ \Gamma_{r'} : r' \in \R' \} \cup \{ \Gamma_r : r \in \R \setminus \R' \}$ is finite.



% Since $\Gamma$ is finite, we know it only has finitely many FDLS, and since $\Gamma$ is convex, these level sets are convex.
% On the interior of these level sets, we know there is a unique optimal report.
% As $\R' = \bigcup_{\Gamma_r \text{is FDLS}}r$, $\R'$ is finite as the finite union of finite sets is finite.
% By Lemma 2 from original writeup, or new definition of trim, we know that $L|_\R$ elicits trim($\Gamma$).
% As the new report set is finite, there many only be a finite number of minimizing reports.

% Consider the full-dimensional level sets of $\Gamma$: as $\R' \subseteq \R$, then when $L$ is restricted to $\R'$, for any $r \in \R$, there must be some $r' \in \R'$ such that $\Gamma_{r} \subseteq \Gamma_{r'}$, as if .



%Since $\Gamma$ is a finite property, there are a finite number of (non-empty) level sets.
%As there is only one optimal report per (interior of) level set, there are a finite of minimizing reports.
%Let this finite set be $\tilde{\R}$, and observe $\tilde{R}$ is finite.
%By definition, $\tilde{\R} \subseteq \R$.
%As $\im(\Gamma') = \im(\Gamma) \cap \R = \tilde{\R} \cap \R'$ \footnote{Needs showing}, then the finite intersection of a finite set and another set is finite, so $\im(\Gamma')$ is finite.


\bigskip
$2 \implies 1$.  
\emph{Assumptions used: Nondegeneracy, elicitability of $\Gamma$}

Let $\Theta := \trim(\Gamma)$ be a finite set.
By construction, $\Theta$ is a set of level sets defining the property $\Gamma$.
As $\Theta$ is finite, $\Gamma$ then only has finitely many level sets, and therefore, finitely many \emph{full-dimensional} level sets.
We construct the set $\R' := \{r \in \R : \{r\} = \argmin_r \E_p L(r,Y) \text{ for } p \in \inter{\theta}; \theta \in \Theta \}$.
Consider a full-dimensional level set $\theta = \Gamma_r \in \Theta$ for some $r \in \R'$.
As $\Theta$ is finite, we just want to show that there is a unique optimal report on the interior of each $\theta \in \Theta$.
As $\trim(\Gamma) = \strip(\Gamma')$ for the constructed $\Gamma'$, we just need to show $\Gamma'$ is nondegenerate, as uniqueness of the optimal report on each interior follows from the construction of $\R'$.

Suppose there was a distribution $p \in \inter{\theta}$ for some $\theta \in \Theta$ such that $\Gamma'(p) = \emptyset$.
Then $\Gamma(p)$ is set-valued, since $\Gamma$ is nondegenerate, and every $r \in \Gamma(p) \not\in \R'$.
Suppose there were some $r_1, r_2 \in \R$ (this generalizes) so that $r_1, r_2 \not \in \R'$ but $\{r_1, r_2\} = \Gamma(p)$ for some $p \in \inter{\theta}$.
Then (WLOG) $\Gamma_{r_1} = \theta$ and we know $\Gamma_{r_2} \subseteq \Gamma_{r_1} = \theta$.
As $r_1 \not\in \R$, for every $p \in \inter{\theta}$ we know $r_1$ must not be the unique minimizer.
If $\Gamma_{r_2} \subseteq \Gamma_{r_1}$, then the property $\tilde \Gamma := \Gamma \cap (\R' \cup \{r_1\})$ is nonempty at the distribution $p$ in question and $\strip(\Gamma') = \strip(\tilde \Gamma)$.


If these possible empty values of $\Gamma'$ at multiple distributions, the same argument need only be applied finitely many times, so there is a nondegenerate $\tilde \Gamma$ such that $\strip(\tilde \Gamma) = \trim(\Gamma)$ has a unique optimal report on the interior of each level set.


Thus, 2$\implies$ 1.

\jessie{Ran out of time, but not satisfied with this completely.}


%For contradiction, suppose there was some $s \in \R$ such that $\E_p L(s,Y) = \E_p L(r_\theta, Y)$ for some $p \in \inter{\theta}$.
%
%As $p \in \inter{\theta}$, we then know $\E_q L(r_\theta, Y) \leq \E_q L(s,Y)$ for all $q \in B(\epsilon, p)$.
%Therefore, $\Gamma_s \subseteq \inter{\Gamma_{r_\theta}} \subseteq \Gamma_{r_\theta}$.
%As there are no dominated reports (by non-redundancy of $\Gamma$), this implies $\Gamma_s = \Gamma_{r_\theta}$.
%Therefore, we can see that $r_\theta = s$, since otherwise we would contradict non-redundancy of $\Gamma$.
%As $r_\theta = s$, we can then say $r_\theta$ is the unique property value on $\inter{\theta}$.
%Therefore, $\Gamma(p) = \{r_\theta\}$ for all $p \in \inter{\theta}$.
%
%As the full-dimensional level sets in $\Theta$ are finite and the optimal report is unique on the interior of level sets, $2 \implies 1$.

\end{proof}

\end{document}
%%% Local Variables:
%%% mode: latex
%%% TeX-master: t
%%% End:

