\documentclass[12pt]{article}
\usepackage[utf8]{inputenc}
\usepackage{mathtools, amsmath, amsthm, amssymb, graphicx, mathrsfs, verbatim}
%\usepackage[thmmarks, thref, amsthm]{ntheorem}
\usepackage{color}
\usepackage{wrapfig}
\usepackage{subcaption}
\usepackage[colorinlistoftodos,textsize=tiny]{todonotes} % need xargs for below
%\usepackage{accents}
\usepackage{bbm}
\usepackage{xspace}
\usepackage[margin=1.25in]{geometry}

\usepackage[colorlinks=true,breaklinks=true,bookmarks=true,urlcolor=blue,
     citecolor=blue,linkcolor=blue,bookmarksopen=false,draft=false]{hyperref}

\newcommand{\Comments}{1}
\newcommand{\mynote}[2]{\ifnum\Comments=1\textcolor{#1}{#2}\fi}
\newcommand{\mytodo}[2]{\ifnum\Comments=1%
  \todo[linecolor=#1!80!black,backgroundcolor=#1,bordercolor=#1!80!black]{#2}\fi}
\newcommand{\raf}[1]{\mynote{green}{[RF: #1]}}
\newcommand{\raft}[1]{\mytodo{green!20!white}{RF: #1}}
\newcommand{\jessie}[1]{\mynote{purple}{[JF: #1]}}
\newcommand{\jessiet}[1]{\mytodo{purple!20!white}{JF: #1}}
\newcommand{\btw}[1]{\mytodo{gray!20!white}{BTW: #1}}
\ifnum\Comments=1               % fix margins for todonotes
  \setlength{\marginparwidth}{1in}
\fi


\newcommand{\reals}{\mathbb{R}}
\newcommand{\posreals}{\reals_{>0}}%{\reals_{++}}

% alphabetical order, by convention
\newcommand{\D}{\mathcal{D}}
\newcommand{\E}{\mathbb{E}}
\newcommand{\F}{\mathcal{F}}
\newcommand{\I}{\mathcal{I}}
\renewcommand{\P}{\mathcal{P}}
\newcommand{\R}{\mathcal{R}}
\newcommand{\X}{\mathcal{X}}
\newcommand{\Y}{\mathcal{Y}}

\newcommand{\simplex}{\Delta_{\Y}}

\newcommand{\inter}[1]{\mathring{#1}}%\mathrm{int}(#1)}
\newcommand{\cl}[1]{\text{cl}(#1)}
%\newcommand{\expectedv}[3]{\overline{#1}(#2,#3)}
\newcommand{\expectedv}[3]{\E_{Y\sim{#3}} {#1}(#2,Y)}
\newcommand{\toto}{\rightrightarrows}
\newcommand{\trim}{\mathrm{trim}}
\newcommand{\fplc}{finite-piecewise-linear and convex\xspace} %xspace for use in text
\newcommand{\conv}{\mathrm{conv}}
\newcommand{\ones}{\mathbbm{1}}
\newcommand{\aff}{\text{aff}}
\newcommand{\im}{\text{im}}
\newcommand{\strip}{\mathrm{strip}}
\newcommand{\card}{\textbf{card}}

\DeclareMathOperator*{\argmax}{arg\,max}
\DeclareMathOperator*{\argmin}{arg\,min}
\DeclareMathOperator*{\arginf}{arg\,inf}
\DeclareMathOperator*{\sgn}{sgn}

\newtheorem{theorem}{Theorem}
\newtheorem{lemma}{Lemma}
\newtheorem{proposition}{Proposition}
\newtheorem{definition}{Definition}
\newtheorem{corollary}{Corollary}
\newtheorem{conjecture}{Conjecture}
\newtheorem{notation}{Notation}




\begin{document}

\section{Terminology}
	Given a vector of normals $V$, the polytope $P^V = \{ x \in \reals^3 : x \cdot v_i \leq 1, \text{ for all } 1 \leq i \leq n \}$ is the feasible polytope corresponding to the abstain cell of the abstain property formed by the set of normals $V$.
	
	The $i^{th}$ feasible polytope is $P_i^V = P^V \cap \{x \in \reals^3 : x \cdot v_i \leq -1 \}$ and the $i^{th}$ face of $P^V$ is $F_i^V = P^V \cap \{ x \in \reals^3 : v_i \cdot x = 1 \}$.
	If $F^V_i$ is the $i^{th}$ face of $P^V$, then $F^V_{-i}$ is the face $F^V_j$ such that $v_j = -v_i$.

\section{Proofs and conjectures}

\begin{lemma}
	For the abstain ($\alpha = 1/2$) property and $P^V$ as above, if $v_i \in V$, then $-v_i \in V$.
\end{lemma}
\begin{proof}
	In google drive on Dec 27 page.
\end{proof}

\begin{conjecture}\label{conj:each-face-facet}
	For all $i$, $F_i^V$ must be a facet.
		\jessie{We can't say this holds true in general (example: trying to embed abstain on 4 outcomes in $\reals^3$), but can use it in this $n=9, x \in \reals^3$ case.}
		
		Take 2: If, for $n = 9$, the abstain property was $3$-embeddable, every face $F_i^V$ would be a facet.
\end{conjecture}
\begin{proof}[Proof sketch of take 2]
	WLOG, let $F_1^V$ be the ``farthest'' face, and $v_1 = -v_9$.
	Suppose $F_1^V = P_9^V$ was not a facet.
	Then neither is $F_9^V$.
	As every $F_i^V$ for $2 \leq i \leq 8$ must be ``between $F_1^V$ and $F_9^V$'', then the abstain property restricted to outcomes $i \in [2,8]$ must be 2-embeddable (by the thread stemming from Raf's conversation with Peter Bartlett).
	As we know abstain with $n=7$ is not 2-embeddable, we reach a contradiction.
	
	The same logic follows for $F_2^V$ and $F_8^V$ to be facets, since abstain on $n=5$ is not $2$-embeddable.
	The faces $\{F_i^V \}_{i=3}^7$ must additionally be facets since there is no way to make contact with the first four established facets without being a facet itself.

	Therefore, if abstain on $9$ outcomes is $3$-embeddable, we know $F_i^V$ must be a facet for all $i$.
\end{proof}


\begin{conjecture}\label{conj:intersect-Pi-not-facet}
	Let $P^V$ be the convex polytope corresponding to the abstain cell of the abstain property.
	For all $i \neq j$, we know $P^V_i \cap P^V_j$ is not a facet of $P^V$.
\end{conjecture}
\begin{proof}
	We know $x \in P^V_i \implies x \cdot v_i \leq -1$ and as $-v_i \in V$, then $x \in P^V \implies x \cdot -v_i \leq 1 \implies x \cdot v_i \geq -1$, and thus $x \in P^V_i \iff x \cdot v_i = -1$.
	Additionally, we know for $i \neq j$ that 
	\begin{align*}
	P_i^V \cap P_j^V &= P^V \cap \{x : x \cdot v_i \leq -1 \} \cap \{x : x \cdot v_j \leq -1 \}\\
	&= P^V \cap \{x : x \cdot v_i = -1 \} \cap \{x : x \cdot v_j = -1 \}
	\end{align*} 
	which is a facet of $P^V$ if and only if $v_i = v_j$.
	As we assume $i \neq j$ and the normals are in general position, then for $i \neq j$, we can see that $v_i \neq v_j \implies P^V_i \cap P_j^V$ is not a facet of $P^V$.
\end{proof}



Note that any face $F_j^V$ of a convex polytope $P^V$ can be described as the convex hull of some subset $A_j$ of the points $A$ describing $P^V$.
(i.e. $A$ is the finite set such that $P^V = \conv(A)$.)
\begin{conjecture}\label{conj:convex-hull-subset-Fi}
	Let $v_i$ define the $P_i^V$ farthest from the origin.
	For $i \neq j$, let us denonte $S_j := (A_j \cap A_i) \cup (A_j \cap A_{-i})$.
	Then $\conv(S_j) = F_j$
\end{conjecture}
\begin{proof}
We just want to show that $A_j = (A_i \cap A_j) \cup (A_{-i} \cap A_j)$.

Since every other face must be ``between'' $P_i^V$ and $P_{-i}^V$, we know that $A = A_i \cup A_{-i}$.
If there is another $x \in A$ such that this is not true, it is either inside or outside the convex hull of $A_i \cup A_{-i}$.
If it is in the convex hull, then we can ignore it to observe the same polytope is constructed as the convex hull of $A \setminus \{x\}$.
If it is outside the convex hull of the elements of $A$, then we contradict the fact that $F_i^V$ and $F_{-i}^V$ are the ``farthest'' faces of $P^V$.
\jessie{Not 100\% sure the previous statement is true, at least for this reason...}
Therefore, as $A = A_i \cup A_{-i}$ and $A_j \subseteq A$, then 
\begin{align*}
(A_i \cap A_j) \cup (A_{-i} \cap A_j) &= A_j \cap (A_i \cup A_{-i})\\
&= A_j \cap A\\
&= A_j
\end{align*}
and therefore the face $F_j^V$ must be the convex hull of the points in the finite set $A_j = \conv(S_j)$.

\end{proof}

\begin{conjecture}
	There is no closed convex polytope $P^V$ in $\reals^3$ with more than 8 facets such that for all $i$ and all $j \neq i$,there exists a vertex $x \in F_j^V$ such that $x \in P^V_i$.
\end{conjecture}
\begin{proof}[Proof sketch]
	Since each $F_i^V$ must be a facet, and there must be a unique face of intersection for each $P_j^V \cap F_i^V$ for each $j \neq i$, we know that $F_i^V$ must have at least 4 edges and vertices since there must be at least 7 ($n-2$) points of contact on 9 outcomes. 
	(If $n > 9$, the same idea holds.)
	
	\jessie{Don't like my explanation of this... coming back to it now that I'm thinking about the proof in a new way.}
	If $F_1^V$ and $F_9^V$ are even-sided polygons, then the faces $F_k^V$ that intersect on vertices of $F_1^V$ and $F_9^V$ are not facets, contradicting conjecture~\ref{conj:each-face-facet}.
	If each was a facet, that would mean the intersection between two $P_j^V$ and $P_k^V$ is a facet, contradicting conjecture~\ref{conj:intersect-Pi-not-facet}.
	
	If $F_1^V$ and $F_9^V$ are odd sided polygons, then each $F_i^V$ for $1 < i < 9$ must only have 3 edges as to not contradict conjecture~\ref{conj:intersect-Pi-not-facet}.
	Since $F_2^V$ has at most 3 edges and vertices, there are not enough faces for each $P_i^V$ (with $i \neq 2$) to uniquely intersect a face of $F_2^V$.
\end{proof}

\begin{conjecture}
	The abstain property is not 3-embeddable.
\end{conjecture}
\begin{proof}
	Using the above Conjecture and Raf's iff statement in the google drive.
\end{proof}

\newpage
\begin{theorem}
	For the abstain property ($\alpha = 1/2$) on $n$ outcomes to be $d$-embeddable, there must be a convex polytope in $\reals^d$ with $n$ faces such that each of the $n$ faces has $(n-1)$ total subfaces.
	(Note this is a necessary, but not sufficient condition.)
\end{theorem}

Note that like Conjecture~\ref{conj:each-face-facet}, each of the faces should be a facet if we were trying to embed abstain on 9 outcomes in $\reals^3$.

To show this is not true for $d=3$ and $n=9$, consider the normals $v_1$ and $v_9$-- the corresponding faces $F^V_1$ and $F_9^V$ must be facets in order for $P^V$ to be a convex polytope.
In order for $F_1^V$ to have $9$ total faces (including $0$ faces, $1$ faces, etc.), and maintain $P^V$ as a convex polytope, $F_1^V$ must have at least 4 edges and vertices.
However, in $\reals^3$, in order for $F_j^V$ to have at least four edges and vertices, then it must be the convex hull of at least $4$ points in $\reals^3$.
This is only true if both $F^V_1$ and $F_9^V$ both have $n-1$ edges, but this construction fails as there is not enough contact between faces.
The ``sawtooth'' approach, however, does not satisfy this necessary condition.

\newpage
\begin{lemma}
	Consider $\Gamma:\simplex \toto \reals^2$ embedding a finite property $\gamma: \simplex \toto \R$, and $S := \{u \in \reals^2 : \Gamma_u \neq \emptyset\}$.
	
	If $\Gamma$ is $2$-embeddable, then there for any line $l \in \reals^2$, the property $\Gamma|_{S\cap l}$ is $1$-embeddable.
\end{lemma}


\jessie{Does this not work for 6 as well?}
\begin{theorem}
	The abstain($\alpha$) property with $\alpha > 1/2$ and $|\Y| = 6$ is not $2$-embeddable.
\end{theorem}
\begin{proof}
  We use the fact that abstain($\alpha > 1/2$)  with $|\Y| = 3$ is not $1$-embeddable.
  Additionally, for any embedding $\R'$, there is line $\l$ such that $\Gamma|_{S \cap l}$ is reduced to abstain($\alpha > 1/2$)  with $|\Y| = 3$.
	
  Let $\Gamma:\simplex \toto \R'$ embed the finite property $\gamma$, which is the 
  For each outcome $i \neq a$, we know $\conv(\{i,a\}) \subseteq S$.
	In fact, $S = \cup_{i \in \R'} \conv(\{i,a\})$.
	
  The set $S$ is then a ``star'' in $\reals^2$.
  (Note that if $a$ is not embedded in the convex hull of the reports, then there is a line $l$ that intersects $\conv(\{i,a\}$ for all reports $i$.)
	
  Since each report $i$ must be embedded at least $\epsilon$ far away from $a$, consider $\theta_{i,j}$ to be the angle between embedded reports $i$ and $j$, treating $a$ as the origin, without loss of generality.
	
  Since $\sum_{i=1}^6 \theta_{i,i+2} =720$, there must be some angles $\theta_{i,i+1} + \theta_{i+1,i+2} = \theta_{i,i+2} \leq 120^o < 180^o$.
  The line $l$ can then intersect $\conv(\{i,a\})$, $\conv(\{i+1, a\})$, and $\conv(\{i+2, a\})$ as it can be thought of as having an angle of $180^o$.


  Therefore, for any $2$-embedding of abstain($\alpha > 1/2$) with $n \geq 6$, there is a reduction to a $1$-embedding of abstain($\alpha > 1/2$) for $n = 3$.
  Since this above is not $1$-embeddable, we conclude that abstain($\alpha > 1/2$) with $n \geq 6$ is not $2$-embeddable.
\end{proof}


\end{document}
%%% Local Variables:
%%% mode: latex
%%% TeX-master: t
%%% End:

