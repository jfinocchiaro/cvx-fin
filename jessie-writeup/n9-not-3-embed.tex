\documentclass[12pt]{article}
\usepackage[utf8]{inputenc}
\usepackage{mathtools, amsmath, amsthm, amssymb, graphicx, mathrsfs, verbatim}
%\usepackage[thmmarks, thref, amsthm]{ntheorem}
\usepackage{color}
\usepackage{wrapfig}
\usepackage{subcaption}
\usepackage[colorinlistoftodos,textsize=tiny]{todonotes} % need xargs for below
%\usepackage{accents}
\usepackage{bbm}
\usepackage{xspace}
\usepackage{tikzsymbols}
\usepackage[margin=1.25in]{geometry}

\usepackage[colorlinks=true,breaklinks=true,bookmarks=true,urlcolor=blue,
     citecolor=blue,linkcolor=blue,bookmarksopen=false,draft=false]{hyperref}

\newcommand{\Comments}{1}
\newcommand{\mynote}[2]{\ifnum\Comments=1\textcolor{#1}{#2}\fi}
\newcommand{\mytodo}[2]{\ifnum\Comments=1%
  \todo[linecolor=#1!80!black,backgroundcolor=#1,bordercolor=#1!80!black]{#2}\fi}
\newcommand{\raf}[1]{\mynote{green}{[RF: #1]}}
\newcommand{\raft}[1]{\mytodo{green!20!white}{RF: #1}}
\newcommand{\jessie}[1]{\mynote{purple}{[JF: #1]}}
\newcommand{\jessiet}[1]{\mytodo{purple!20!white}{JF: #1}}
\newcommand{\btw}[1]{\mytodo{gray!20!white}{BTW: #1}}
\ifnum\Comments=1               % fix margins for todonotes
  \setlength{\marginparwidth}{1in}
\fi


\newcommand{\reals}{\mathbb{R}}
\newcommand{\posreals}{\reals_{>0}}%{\reals_{++}}

% alphabetical order, by convention
\newcommand{\D}{\mathcal{D}}
\newcommand{\E}{\mathbb{E}}
\newcommand{\F}{\mathcal{F}}
\newcommand{\I}{\mathcal{I}}
\renewcommand{\P}{\mathcal{P}}
\newcommand{\R}{\mathcal{R}}
\newcommand{\X}{\mathcal{X}}
\newcommand{\Y}{\mathcal{Y}}

\newcommand{\simplex}{\Delta_{\Y}}

\newcommand{\inter}[1]{\mathring{#1}}%\mathrm{int}(#1)}
\newcommand{\cl}[1]{\text{cl}(#1)}
%\newcommand{\expectedv}[3]{\overline{#1}(#2,#3)}
\newcommand{\expectedv}[3]{\E_{Y\sim{#3}} {#1}(#2,Y)}
\newcommand{\inprod}[2]{\langle #1, #2 \rangle}
\newcommand{\toto}{\rightrightarrows}
\newcommand{\trim}{\mathrm{trim}}
\newcommand{\fplc}{finite-piecewise-linear and convex\xspace} %xspace for use in text
\newcommand{\conv}{\mathrm{conv}}
\newcommand{\ones}{\mathbbm{1}}
\newcommand{\aff}{\text{aff}}
\newcommand{\im}{\text{im}}
\newcommand{\strip}{\mathrm{strip}}
\newcommand{\card}{\textbf{card}}

\DeclareMathOperator*{\argmax}{arg\,max}
\DeclareMathOperator*{\argmin}{arg\,min}
\DeclareMathOperator*{\arginf}{arg\,inf}
\DeclareMathOperator*{\sgn}{sgn}

\newtheorem{theorem}{Theorem}
\newtheorem{lemma}{Lemma}
\newtheorem{proposition}{Proposition}
\newtheorem{definition}{Definition}
\newtheorem{corollary}{Corollary}
\newtheorem{conjecture}{Conjecture}
\newtheorem{notation}{Notation}




\begin{document}

\section{Terminology}
	Given a vector of normals $V$, the polytope $P^V = \{ x \in \reals^3 : x \cdot v_i \leq 1, \text{ for all } 1 \leq i \leq n \}$ is the feasible polytope corresponding to the abstain cell of the abstain property formed by the set of normals $V$.
	
	The $i^{th}$ feasible polytope is $P_i^V = P^V \cap \{x \in \reals^3 : x \cdot v_i \leq -1 \}$ and the $i^{th}$ face of $P^V$ is $F_i^V = P^V \cap \{ x \in \reals^3 : v_i \cdot x = 1 \}$.
	If $F^V_i$ is the $i^{th}$ face of $P^V$, then $F^V_{-i}$ is the face $F^V_j$ such that $v_j = -v_i$.

\section{Proofs and conjectures}

\begin{lemma}
	For the abstain ($\alpha = 1/2$) property and $P^V$ as above, if $v_i \in V$, then $-v_i \in V$.
\end{lemma}
\begin{proof}
	In google drive on Dec 27 page.
\end{proof}

\begin{conjecture}\label{conj:each-face-facet}
	For all $i$, $F_i^V$ must be a facet.
		\jessie{We can't say this holds true in general (example: trying to embed abstain on 4 outcomes in $\reals^3$), but can use it in this $n=9, x \in \reals^3$ case.}
		
		Take 2: If, for $n = 9$, the abstain property was $3$-embeddable, every face $F_i^V$ would be a facet.
\end{conjecture}
\begin{proof}[Proof sketch of take 2]
	WLOG, let $F_1^V$ be the ``farthest'' face, and $v_1 = -v_9$.
	Suppose $F_1^V = P_9^V$ was not a facet.
	Then neither is $F_9^V$.
	As every $F_i^V$ for $2 \leq i \leq 8$ must be ``between $F_1^V$ and $F_9^V$'', then the abstain property restricted to outcomes $i \in [2,8]$ must be 2-embeddable (by the thread stemming from Raf's conversation with Peter Bartlett).
	As we know abstain with $n=7$ is not 2-embeddable, we reach a contradiction.
	
	The same logic follows for $F_2^V$ and $F_8^V$ to be facets, since abstain on $n=5$ is not $2$-embeddable.
	The faces $\{F_i^V \}_{i=3}^7$ must additionally be facets since there is no way to make contact with the first four established facets without being a facet itself.

	Therefore, if abstain on $9$ outcomes is $3$-embeddable, we know $F_i^V$ must be a facet for all $i$.
\end{proof}


\begin{conjecture}\label{conj:intersect-Pi-not-facet}
	Let $P^V$ be the convex polytope corresponding to the abstain cell of the abstain property.
	For all $i \neq j$, we know $P^V_i \cap P^V_j$ is not a facet of $P^V$.
\end{conjecture}
\begin{proof}
	We know $x \in P^V_i \implies x \cdot v_i \leq -1$ and as $-v_i \in V$, then $x \in P^V \implies x \cdot -v_i \leq 1 \implies x \cdot v_i \geq -1$, and thus $x \in P^V_i \iff x \cdot v_i = -1$.
	Additionally, we know for $i \neq j$ that 
	\begin{align*}
	P_i^V \cap P_j^V &= P^V \cap \{x : x \cdot v_i \leq -1 \} \cap \{x : x \cdot v_j \leq -1 \}\\
	&= P^V \cap \{x : x \cdot v_i = -1 \} \cap \{x : x \cdot v_j = -1 \}
	\end{align*} 
	which is a facet of $P^V$ if and only if $v_i = v_j$.
	As we assume $i \neq j$ and the normals are in general position, then for $i \neq j$, we can see that $v_i \neq v_j \implies P^V_i \cap P_j^V$ is not a facet of $P^V$.
\end{proof}



Note that any face $F_j^V$ of a convex polytope $P^V$ can be described as the convex hull of some subset $A_j$ of the points $A$ describing $P^V$.
(i.e. $A$ is the finite set such that $P^V = \conv(A)$.)
\begin{conjecture}\label{conj:convex-hull-subset-Fi}
	Let $v_i$ define the $P_i^V$ farthest from the origin.
	For $i \neq j$, let us denonte $S_j := (A_j \cap A_i) \cup (A_j \cap A_{-i})$.
	Then $\conv(S_j) = F_j$
\end{conjecture}
\begin{proof}
We just want to show that $A_j = (A_i \cap A_j) \cup (A_{-i} \cap A_j)$.

Since every other face must be ``between'' $P_i^V$ and $P_{-i}^V$, we know that $A = A_i \cup A_{-i}$.
If there is another $x \in A$ such that this is not true, it is either inside or outside the convex hull of $A_i \cup A_{-i}$.
If it is in the convex hull, then we can ignore it to observe the same polytope is constructed as the convex hull of $A \setminus \{x\}$.
If it is outside the convex hull of the elements of $A$, then we contradict the fact that $F_i^V$ and $F_{-i}^V$ are the ``farthest'' faces of $P^V$.
\jessie{Not 100\% sure the previous statement is true, at least for this reason...}
Therefore, as $A = A_i \cup A_{-i}$ and $A_j \subseteq A$, then 
\begin{align*}
(A_i \cap A_j) \cup (A_{-i} \cap A_j) &= A_j \cap (A_i \cup A_{-i})\\
&= A_j \cap A\\
&= A_j
\end{align*}
and therefore the face $F_j^V$ must be the convex hull of the points in the finite set $A_j = \conv(S_j)$.

\end{proof}

\begin{conjecture}
	There is no closed convex polytope $P^V$ in $\reals^3$ with more than 8 facets such that for all $i$ and all $j \neq i$,there exists a vertex $x \in F_j^V$ such that $x \in P^V_i$.
\end{conjecture}
\begin{proof}[Proof sketch]
	Since each $F_i^V$ must be a facet, and there must be a unique face of intersection for each $P_j^V \cap F_i^V$ for each $j \neq i$, we know that $F_i^V$ must have at least 4 edges and vertices since there must be at least 7 ($n-2$) points of contact on 9 outcomes. 
	(If $n > 9$, the same idea holds.)
	
	\jessie{Don't like my explanation of this... coming back to it now that I'm thinking about the proof in a new way.}
	If $F_1^V$ and $F_9^V$ are even-sided polygons, then the faces $F_k^V$ that intersect on vertices of $F_1^V$ and $F_9^V$ are not facets, contradicting conjecture~\ref{conj:each-face-facet}.
	If each was a facet, that would mean the intersection between two $P_j^V$ and $P_k^V$ is a facet, contradicting conjecture~\ref{conj:intersect-Pi-not-facet}.
	
	If $F_1^V$ and $F_9^V$ are odd sided polygons, then each $F_i^V$ for $1 < i < 9$ must only have 3 edges as to not contradict conjecture~\ref{conj:intersect-Pi-not-facet}.
	Since $F_2^V$ has at most 3 edges and vertices, there are not enough faces for each $P_i^V$ (with $i \neq 2$) to uniquely intersect a face of $F_2^V$.
\end{proof}

\begin{conjecture}
	The abstain property is not 3-embeddable.
\end{conjecture}
\begin{proof}
	Using the above Conjecture and Raf's iff statement in the google drive.
\end{proof}

\newpage
\begin{theorem}
	For the abstain property ($\alpha = 1/2$) on $n$ outcomes to be $d$-embeddable, there must be a convex polytope in $\reals^d$ with $n$ faces such that each of the $n$ faces has $(n-1)$ total subfaces.
	(Note this is a necessary, but not sufficient condition.)
\end{theorem}

Note that like Conjecture~\ref{conj:each-face-facet}, each of the faces should be a facet if we were trying to embed abstain on 9 outcomes in $\reals^3$.

To show this is not true for $d=3$ and $n=9$, consider the normals $v_1$ and $v_9$-- the corresponding faces $F^V_1$ and $F_9^V$ must be facets in order for $P^V$ to be a convex polytope.
In order for $F_1^V$ to have $9$ total faces (including $0$ faces, $1$ faces, etc.), and maintain $P^V$ as a convex polytope, $F_1^V$ must have at least 4 edges and vertices.
However, in $\reals^3$, in order for $F_j^V$ to have at least four edges and vertices, then it must be the convex hull of at least $4$ points in $\reals^3$.
This is only true if both $F^V_1$ and $F_9^V$ both have $n-1$ edges, but this construction fails as there is not enough contact between faces.
The ``sawtooth'' approach, however, does not satisfy this necessary condition.

\newpage
\begin{lemma}
	Consider $\Gamma:\simplex \toto \reals^2$ embedding a finite property $\gamma: \simplex \toto \R$, and $S := \{u \in \reals^2 : \Gamma_u \neq \emptyset\}$.
	
	If $\Gamma$ is $2$-embeddable, then there for any line $l \in \reals^2$, the property $\Gamma|_{S\cap l}$ is $1$-embeddable.
\end{lemma}


\jessie{Does this not work for 6 as well?}
\begin{theorem}
	The abstain($\alpha$) property with $\alpha > 1/2$ and $|\Y| = 6$ is not $2$-embeddable.
\end{theorem}
\begin{proof}
  We use the fact that abstain($\alpha > 1/2$)  with $|\Y| = 3$ is not $1$-embeddable.
  Additionally, for any embedding $\R'$, there is line $\l$ such that $\Gamma|_{S \cap l}$ is reduced to abstain($\alpha > 1/2$)  with $|\Y| = 3$.
	
  Let $\Gamma:\simplex \toto \R'$ embed the finite property $\gamma$, which is the 
  For each outcome $i \neq a$, we know $\conv(\{i,a\}) \subseteq S$.
	In fact, $S = \cup_{i \in \R'} \conv(\{i,a\})$.
	
  The set $S$ is then a ``star'' in $\reals^2$.
  (Note that if $a$ is not embedded in the convex hull of the reports, then there is a line $l$ that intersects $\conv(\{i,a\}$ for all reports $i$.)
	
  Since each report $i$ must be embedded at least $\epsilon$ far away from $a$, consider $\theta_{i,j}$ to be the angle between embedded reports $i$ and $j$, treating $a$ as the origin, without loss of generality.
	
  Since $\sum_{i=1}^6 \theta_{i,i+2} =720$, there must be some angles $\theta_{i,i+1} + \theta_{i+1,i+2} = \theta_{i,i+2} \leq 120^o < 180^o$.
  The line $l$ can then intersect $\conv(\{i,a\})$, $\conv(\{i+1, a\})$, and $\conv(\{i+2, a\})$ as it can be thought of as having an angle of $180^o$.


  Therefore, for any $2$-embedding of abstain($\alpha > 1/2$) with $n \geq 6$, there is a reduction to a $1$-embedding of abstain($\alpha > 1/2$) for $n = 3$.
  Since this above is not $1$-embeddable, we conclude that abstain($\alpha > 1/2$) with $n \geq 6$ is not $2$-embeddable.
\end{proof}


\newpage
\section{Post-submission thoughts}
\subsection{The mode}
\jessiet{For all the $\leq$ statements in here, there might be a reason they would have to be equalities, but I haven't been able to think it fully through/prove it, so they are in here as inequalities for now.}
\begin{conjecture}
	The normals constructing a tight H-rep for a cell for the mode must be linearly independent
\end{conjecture}
\begin{proof}
  Consider the matrix describing the cell for outcome $n$...
	
  $B^{(n)} := \begin{pmatrix}
  -1 & 0 & 0 & \ldots & 1 \\
  0 & -1 & 0 & \ldots & 1 \\
  0 & 0 & -1 & \ldots &  1\\
  0 & 0 & \ldots &-1 &1\\
  \end{pmatrix}$
	

  Now for each column $B_i^{(n)}$ of $B^{(n)}$, we must have a set of normals such that for each normal $j$, we must have $\inprod{V_j}{x} = B_{i,j}$ in order for the halfspace representation to be tight.
  Let us write the polytope $T_i := \{x \in \reals^d : Vx \leq B_i \}$ as the feasible polytope for the H-rep of the property.
  In order for $\{x \in T_n : \inprod{V_j}{x} = 1 \}$ to be nonempty for all $V_j$, we assert the normals $V_j$ must all be linearly independent.
  
	
%	The first $n-1$ columns of $B^{(n)}$ form the standard basis in $\reals^{n-1}$.

  To see this, first consider that if $V_2 = -V_1$ , then one of these constraints for the polytope $T_n$ is redundant.
  Therefore these normals do not form a tight $H$-rep since for (WLOG) $V_2$, the halfspace represented by $\{x : \inprod{V_2}{x} \leq 1 \}$ is a (strict since none of the normals can be $0$) subset of the halfspace $\{x : \inprod{V_1}{x} \leq 1 \}$.
	
  To generalize this to linear independence of the normals, suppose $V_{\Laughey} = \sum_{i \neq \Laughey} \lambda_i V_i$ so that $V_{\Laughey}$ is a linear combination of other normals.
%  
  The tightness condition, for the columns being indexed by $i$ and rows indexed by $j$ is as follows:
  For all $i \in \{1, 2, \ldots, n\}$ and $j \in \{1, 2, \ldots, k = n-1\}$ there must be an $x \in T_i$ so that 
  \begin{align*}
 \inprod{V_j}{x} = 
  \begin{cases}
  1 & i = n\\
  -1& i = j\\
  0 & i \neq j, i \neq n
  \end{cases}
  \end{align*}
  
  
  Given the tightness constraints, if $V_{\Laughey}$ is a linear combination of $V_i$s, then $\inprod{V_{\Laughey}}{x} = \sum_{i\neq \Laughey} \inprod{\lambda_i V_i}{x} \leq 0 + 0 + \ldots + 0 + \lambda_n = \lambda_n$.
  	
  If $\lambda_n= -1$, we contradict the tightness of $T_n$ shown above. 
  Otherwise, we contradict the tightness of $T_i$ since there is no constraint such that $\inprod{V_{\Laughey}}{x} = -1$ given the tightness of the other constraints. 
  
\end{proof}

\begin{corollary}
  The mode is not $(n-2)$ embeddable
\end{corollary}
\begin{proof}
  In order to have a tight $H$-rep, we need normals that are linearly independent.
  Since the normals must be linearly independent, it must be true that for $V \in \reals^{k \times d}$ that $k \leq d$.
  However, since $k = n-1 \leq d$, we know that the embedding dimension must be at least $n-1$.
\end{proof}

\subsection{Abstain}
Let's think about abstain for a second in a similar mindset.
\begin{theorem}
	The normals constructing a tight H-rep for the abstain(1/2) property on $n$ outcomes \jessie{must sum to $\vec{0}$ and} [[COOL CHARACTERIZATION]]
\end{theorem}
\begin{proof}
  Consider the matrix $B^{(abs)} \in \reals^{k \times n}$ (where $k = n$) describing the abstain cell...
	
  $B^{(abs)} := \begin{pmatrix}
  -1 & 1 & 1 & \ldots & 1\\
  1 & -1 & 1 & \ldots & 1\\
  1 & 1 & -1 & \ldots & 1\\
  1 & 1 & \ldots & 1 & -1\\
  \end{pmatrix}$
	
  The tightness constraint for the abstain cell is for all columns of $B^{(abs)}$, indexed by $i$, and rows of $V$, indexed by $j$, we know there must be an $x \in T_i$ such that 
  \begin{align*}
  \inprod{V_j}{x} = 
  \begin{cases}
  -1 & i = j\\
  1& o/w\\
  \end{cases}
  \end{align*}
	  
  When a normal $V_{\Laughey}$ can be written as a linear combination of other normals $\sum_{j \neq \Laughey} \lambda_j V_j$, then we know $\inprod{V_{\Laughey}}{x} = \sum_{j \neq \Laughey} \inprod{V_j}{x}$.
  Further, by the requirement that $x \in T_i$, we know that
  \begin{align*}
  	\inprod{V_{\Laughey}}{x} &= \sum_{j \neq \Laughey} \inprod{V_{\Laughey}}{x}\\
  	&\leq \lambda_1 + \lambda_2 + \ldots + (-\lambda_i) + \lambda_{i+1} + \ldots + \lambda_n\\
  \end{align*}
	  
  For $i = \Laughey$, we know that this tightness constraint is then $-1 \leq \sum_{j \neq i} \lambda_j$.
  Therefore, there must be some $\lambda_\ell \leq -1$ for the tightness constraint to be satisfied.
  However, when $i = \ell$, then we must have $V_\ell$ being able to be written as some linear combination of normals where $V_\ell = -\frac{\lambda_1}{\lambda_\ell} V_1 + \ldots + \frac{1}{\lambda_\ell} V_{\Laughey} + \ldots + (- \frac{\lambda_n}{\lambda_\ell}) V_n$.
  Since $\lambda_\ell \leq -1$, we know that $\frac{1}{\lambda_\ell} =: \lambda^*_{\Laughey} \in (0, -1]$.
  \end{proof}
  
  \begin{corollary}
    The abstain(1/2) property is \jessie{$\lceil \log_2 n \rceil$ embeddable, but no less?}
  \end{corollary}	
  
  \subsection{Generalizing abstain}
  However, we know that we have to use the same set of normal for all $T_i$ for $i \in \{1,2, \ldots, n\}$.
  Therefore, if we want to write any $V_j$ as a linear combination of other normals, we can use the structure of all $T_i$ to note the possible relationships between normals.
  
  For example, for $x \in T_1$, we know that by the tightness constraints
  \begin{align*}
  -1 = \inprod{V_1}{x} &= \inprod{\lambda_2 V_2}{x} + \inprod{\lambda_3 V_3}{x}\\ 
  &= \lambda_2 \inprod{V_2}{x} + \lambda_3 \inprod{V_3}{x}\\
  &\geq \lambda_2 + \lambda_3
  \end{align*}
  
  Similarly for $x \in T_1$, we know
  \begin{align*}
  1 = \inprod{V_2}{x} &= \inprod{\lambda_1 V_1}{x} + \inprod{\lambda_3 V_3}{x}\\ 
  &= \lambda_1 \inprod{V_1}{x} + \lambda_3 \inprod{V_3}{x}\\
  &\geq -\lambda_1 + \lambda_3
  \end{align*}
  and so on for the tightness constraint on $V_3$ and $T_1$.
  
  Combining the inequalities, this yields
  \[
  \begin{bmatrix}
      0 & 1 & 1      \\
      -1 & 0 & 1 \\
      -1 & 1 & 0      
  \end{bmatrix}
  \begin{bmatrix}
      \lambda_1\\
      \lambda_2\\
      \lambda_3
  \end{bmatrix}
  \leq 
  \begin{bmatrix}
      -1\\
      1 \\
      1
  \end{bmatrix} 
  \]
  
  We can then repeat this process for $T_2$ and $T_3$ (since all three require the same set of normals) to observe
  
    \[
    \begin{bmatrix}
        0 & 1 & 1      \\
        -1 & 0 & 1 \\
        -1 & 1 & 0 \\ \hline
        0 & -1 & 1\\
        1 & 0 & 1\\
        1 & -1 & 0\\ \hline
        0 & 1 & -1\\
        1 & 0 &-1 \\
        1 & 1 & 0     
    \end{bmatrix}
    \begin{bmatrix}
        \lambda_1\\
        \lambda_2\\
        \lambda_3
    \end{bmatrix}
    \leq 
    \begin{bmatrix}
        -1\\
        1 \\
        1 \\ \hline
        1 \\
        -1 \\
        1\\ \hline
        1\\
        1\\
        -1\\
    \end{bmatrix} 
    \]
    
    We then want to find a feasible solution for this polytope.
    (It seems like $-\ones$ should be a feasible solution for all $n$, but I haven't proven it.)
    
    Since we know have a set of ``weights'' that show the relationship between normals, we want to consider that we then have $V_j = \sum_{\ell \neq j}\lambda_\ell V_\ell.$
    Therefore, we can set up a system (using our feasible solution $\lambda = -\ones$) by stating $V_1 = -V_2 - V_3$, $V_2 = -V_1 - V_3$, etc.
    
    Therefore, we have the system
    \[
    \begin{bmatrix}
        0 & -1 & -1\\
        -1 & 0 & -1 \\
        -1 & -1 & 0
    \end{bmatrix}
    \begin{bmatrix}
        V_1\\
        V_2\\
        V_3
    \end{bmatrix}
    =
    \begin{bmatrix}
        V_1\\
        V_2 \\
        V_3 
    \end{bmatrix} 
    \]
    From here, we can find (any) eigenvector corresponding to the eigenvalue $1$, and imposes some structure on the normals.
    
    For example, we know that $-V_1 = V_2$ by taking the Hadamard product of the eigenvector (corresponding to eigenvalue $1$) and the normals $V$.
    
    Once we fix $V_1$ and $V_2$, we can then look at the rest of the normals composing $V$ through the same system equality and conclude that the sum of the rest of the normals should be $\vec{0}$.
    Since $V_1 = -V_2$, their sum is also $\vec{0}$, and thus the sum of all the normals must be $\vec{0}$.
    
    However, a quick counterexample shows that this is not sufficient.
    Consider the matrix $V = [(0,1), (1,0), (0,-1), (-1/2, \sqrt{3} / 2), (-1/2, -\sqrt{3} / 2)]$ whose polar is given in Figure~\ref{fig:sum-to-0-gen-pos-ce}.
    
	\begin{figure}
	\centering
	\includegraphics[width=0.7\linewidth]{./sum-to-0-gen-pos-ce}
	\caption{Counterexample where the normals sum to $\vec{0}$ but the abstain property is not embeddable in $\reals^2$.}
	\label{fig:sum-to-0-gen-pos-ce}
	\end{figure}

	The normals are in general position and sum to $\vec{0}$, but this arrangement cannot be used to embed abstain in $\reals^2$.
	
	Playing around with this example, $T_i$ is not empty for all $i$, but fails to form a tight H-rep because of the adjacency requirements, which follow from the total number of $a$-faces ($a \in \{0, 1, 2, \ldots, d-1\}$) of each $T_i$.
	For example, here, two of the faces are lines, which have 4 faces in $\reals^2$, and therefore cannot make contact (i.e. have nonempty intersection with) all $T_i^j := T_i \cap \{x : \inprod{V_j}{x} = b_{i,j} \}$.
	(I'm using some implicit assumption that each $T_i^j$ is a face of $T_i$, which I believe should be true, and I think is in An Introduction to Convex Polytopes\footnote{\url{https://www.fmf.uni-lj.si/~lavric/brondsted.pdf}})
	
	
\subsection*{General geometric thoughts}
Thoughts:
\begin{itemize}
\item Can we write $B$ and $V$ so that $T = V^\circ$ and each $b_i \preceq \ones$? (componentwise inequality)?
\item There has to be some restriction on the normals stronger than them being in general position, but not as strong as them being affinely independent.
\item The matrix $V = [(0,1), (1,0), (0,-1), (-1/2, \sqrt{3} / 2), (-1/2, -\sqrt{3} / 2)]$ shows that general position and summing to $\vec{0}$ is not sufficient for the abstain property.
\item This polytope doesn't work because of something about its geometric structure.
\end{itemize}

\begin{definition}
A matrix $V$ of normals is \emph{nonredundant} \jessiet{Maybe use a different term since nonredundancy is also used in reference to properties} if \jessie{\ldots}
\end{definition}

\begin{conjecture}
The matrix $V \in \reals^{k \times d}$ forms a tight H-representation if and only if (a.) Each $T_i$ has at least $k$ faces, (b.) For all $j = \{1,2,\ldots, k\}$, the polytope $T_i^j$ is nonempty, and (c.) for all $i \in \{1,2,\ldots, n\}$ we have $|\{T_i^j\}_{j=1}^k| = k$.  That is, fixing any $T_i$, we observe $T_i^j$ is unique.
\end{conjecture}

\end{document}
%%% Local Variables:
%%% mode: latex
%%% TeX-master: t
%%% End:

