%\newtheorem{theorem}{Theorem}
%\newtheorem{lemma}{Lemma}
%\newtheorem{fact}{Fact}
%\newtheorem{observation}{Observation}
%\newtheorem{proposition}{Proposition}
%\newtheorem{corollary}{Corollary}
%
%\newtheorem*{theorem*}{Theorem}
%\newtheorem*{lemma*}{Lemma}
%\newtheorem*{proposition*}{Proposition}

%\theoremstyle{definition}\newtheorem{definition}{Definition}
%\theoremstyle{definition}\newtheorem{assumption}{Assumption}

\newcommand{\Comments}{1}
\newcommand{\mynote}[2]{\ifnum\Comments=1\textcolor{#1}{#2}\fi}
\newcommand{\mytodo}[2]{\ifnum\Comments=1%
  \todo[linecolor=#1!80!black,backgroundcolor=#1,bordercolor=#1!80!black]{#2}\fi}
\newcommand{\raf}[1]{\mynote{green}{[RF: #1]}}
\newcommand{\raft}[1]{\mytodo{green!20!white}{RF: #1}}
\newcommand{\proposedadd}[1]{\mynote{orange}{#1}}
\newcommand{\bo}[1]{\mynote{blue}{[Bo: #1]}}
\newcommand{\botodo}[1]{\mytodo{blue!20!white}{[Bo: #1]}}
\newcommand{\btw}[1]{\mytodo{gray!20!white}{BTW: #1}}%TURN OFF FOR NOW \mytodo{gray}{#1}}
%\ifnum\Comments=1               % fix margins for todonotes
%  \setlength{\marginparwidth}{1in}
%\fi


\DeclareMathOperator*{\argmax}{arg\,max}
\DeclareMathOperator*{\argmin}{arg\,min}
\DeclareMathOperator*{\arginf}{arg\,inf}

\newcommand{\reals}{\mathbb{R}}
\newcommand{\posreals}{\reals_{>0}}%{\reals_{++}}
\newcommand{\dom}{\mathrm{dom}}
\newcommand{\defeq}{\doteq}%\vcentcolon=} % define equals

\newcommand{\prop}[1]{\mathrm{prop}[#1]}

\newcommand{\simplex}{\Delta_\Y}

% alphabetical order, by convention
\newcommand{\A}{\mathcal{A}}
\newcommand{\C}{\mathcal{C}}
\newcommand{\D}{\mathcal{D}}
\DeclareMathOperator{\E}{\mathbb{E}}  % expectation
%\newcommand{\E}{\mathbb{E}}
\newcommand{\F}{\mathcal{F}}
\newcommand{\I}{\mathcal{I}}
\newcommand{\N}{\mathcal{N}}
\renewcommand{\P}{\mathcal{P}}
\newcommand{\Q}{\mathcal{Q}}
\newcommand{\R}{\mathcal{R}}
\newcommand{\Sc}{\mathcal{S}}
\newcommand{\U}{\mathcal{U}}
\newcommand{\X}{\mathcal{X}}
\newcommand{\Y}{\mathcal{Y}}


\newcommand{\risk}[1]{\underline{#1}}
\newcommand{\inprod}[2]{\langle #1, #2 \rangle}%\mathrm{int}(#1)}
\newcommand{\inter}[1]{\mathring{#1}}%\mathrm{int}(#1)}
%\newcommand{\expectedv}[3]{\overline{#1}(#2,#3)}
\newcommand{\expectedv}[3]{\E_{Y\sim{#3}} {#1}(#2,Y)}
\newcommand{\toto}{\rightrightarrows}
\newcommand{\strip}{\mathrm{strip}}
\newcommand{\trim}{\mathrm{trim}}
\newcommand{\fplc}{finite-piecewise-linear and convex\xspace} %xspace for use in text
\newcommand{\conv}{\mathrm{conv}}
\newcommand{\indopp}{\bar{\mathbbm{1}}}
\newcommand{\ones}{\mathbbm{1}}
% \DeclarePairedDelimiter\ceil{\lceil}{\rceil}

\newcommand{\regret}[3]{R_{#1}(#2,#3)}
\newcommand{\Reg}{\mathrm{Regret}}

\newcommand{\Alg}{\mathrm{Alg}}

