\documentclass[anon]{colt2020} % Anonymized submission
% \documentclass{colt2020} % Include author names

% The following packages will be automatically loaded:
% amsmath, amssymb, natbib, graphicx, url, algorithm2e

\title[Embedding dimension]{Embedding Dimension of Polyhedral Losses}
\usepackage{times}

\usepackage{lmodern}
\usepackage{hyperref}       % hyperlinks  %[implicit=false, bookmarks=false]
\usepackage{url}            % simple URL typesetting
\usepackage{booktabs}       % professional-quality tables
\usepackage{amsfonts}       % blackboard math symbols
\usepackage{nicefrac}       % compact symbols for 1/2, etc.
\usepackage{microtype}      % microtypography
%\usepackage[margin=1.0in]{geometry}

\usepackage{mathtools, amsmath, amssymb, graphicx, verbatim}
%\usepackage{amsthm}
%\usepackage[thmmarks, thref, amsthm]{ntheorem}
\usepackage{color}
\definecolor{darkblue}{rgb}{0.0,0.0,0.2}
\hypersetup{colorlinks,breaklinks,
	linkcolor=darkblue,urlcolor=darkblue,
	anchorcolor=darkblue,citecolor=darkblue}
\usepackage{wrapfig}
\usepackage{subcaption}
\usepackage[colorinlistoftodos,textsize=tiny]{todonotes} % need xargs for below
%\usepackage{accents}
\usepackage{bbm}
\usepackage{xspace}

%\usetikzlibrary{calc}
\newcommand{\Comments}{1}
\newcommand{\mynote}[2]{\ifnum\Comments=1\textcolor{#1}{#2}\fi}
\newcommand{\mytodo}[2]{\ifnum\Comments=1%
	\todo[linecolor=#1!80!black,backgroundcolor=#1,bordercolor=#1!80!black]{#2}\fi}
\newcommand{\raf}[1]{\mynote{green}{[RF: #1]}}
\newcommand{\raft}[1]{\mytodo{green!20!white}{RF: #1}}
\newcommand{\jessie}[1]{\mynote{purple}{[JF: #1]}}
\newcommand{\jessiet}[1]{\mytodo{purple!20!white}{JF: #1}}
\newcommand{\bo}[1]{\mynote{blue}{[Bo: #1]}}
\newcommand{\botodo}[1]{\mytodo{blue!20!white}{[Bo: #1]}}
\newcommand{\btw}[1]{\mytodo{orange!20!white}{BTW: #1}}
\ifnum\Comments=1               % fix margins for todonotes
\setlength{\marginparwidth}{1in}
\fi


\newcommand{\reals}{\mathbb{R}}
\newcommand{\posreals}{\reals_{>0}}%{\reals_{++}}
\newcommand{\dom}{\mathrm{dom}}
\newcommand{\epi}{\text{epi}}
\newcommand{\relint}{\mathrm{relint}}
\newcommand{\prop}[1]{\Gamma[#1]}
\newcommand{\eliccts}{\mathrm{elic}_\mathrm{cts}}
\newcommand{\eliccvx}{\mathrm{elic}_\mathrm{cvx}}
\newcommand{\elicpoly}{\mathrm{elic}_\mathrm{pcvx}}
\newcommand{\elicembed}{\mathrm{elic}_\mathrm{embed}}

\newcommand{\cell}{\mathrm{cell}}

\newcommand{\abstain}[1]{\mathrm{abstain}_{#1}}
\newcommand{\mode}{\mathrm{mode}}

\newcommand{\simplex}{\Delta_\Y}

% alphabetical order, by convention
\newcommand{\C}{\mathcal{C}}
\newcommand{\D}{\mathcal{D}}
\newcommand{\E}{\mathbb{E}}
\newcommand{\F}{\mathcal{F}}
\renewcommand{\H}{\mathcal{H}}
\newcommand{\N}{\mathcal{N}}
\newcommand{\I}{\mathcal{I}}
\newcommand{\R}{\mathcal{R}}
\newcommand{\T}{\mathcal{T}}
\newcommand{\U}{\mathcal{U}}
\newcommand{\V}{\mathcal{V}}
\newcommand{\X}{\mathcal{X}}
\newcommand{\Y}{\mathcal{Y}}
\renewcommand{\P}{\mathcal{P}}

\newcommand{\risk}[1]{\underline{#1}}
\newcommand{\inprod}[2]{\langle #1, #2 \rangle}%\mathrm{int}(#1)}
\newcommand{\inter}[1]{\mathrm{int}(#1)}%\mathrm{int}(#1)}
%\newcommand{\expectedv}[3]{\overline{#1}(#2,#3)}
\newcommand{\expectedv}[3]{\E_{Y\sim{#3}} {#1}(#2,Y)}
\newcommand{\toto}{\rightrightarrows}
\newcommand{\strip}{\mathrm{strip}}
\newcommand{\trim}{\mathrm{trim}}
\newcommand{\fplc}{finite-piecewise-linear and convex\xspace} %xspace for use in text
\newcommand{\conv}{\mathrm{conv}}
\newcommand{\indopp}{\bar{\mathbbm{1}}}
\newcommand{\ones}{\mathbbm{1}}
\DeclarePairedDelimiter\ceil{\lceil}{\rceil}

\newcommand{\Ind}[1]{\mathbf{1}\{#1\}}

\DeclareMathOperator*{\argmax}{arg\,max}
\DeclareMathOperator*{\argmin}{arg\,min}
\DeclareMathOperator*{\arginf}{arg\,inf}
\DeclareMathOperator*{\sgn}{sgn}

%\newtheorem{theorem}{Theorem}
%\newtheorem{lemma}{Lemma}
%\newtheorem{proposition}{Proposition}
%\newtheorem{definition}{Definition}
%\newtheorem{corollary}{Corollary}
%\newtheorem{conjecture}{Conjecture}
\newtheorem{condition}{Condition}
\newtheorem{claim}{Claim}
% Use \Name{Author Name} to specify the name.
% If the surname contains spaces, enclose the surname
% in braces, e.g. \Name{John {Smith Jones}} similarly
% if the name has a "von" part, e.g \Name{Jane {de Winter}}.
% If the first letter in the forenames is a diacritic
% enclose the diacritic in braces, e.g. \Name{{\'E}louise Smith}

% Two authors with the same address
% \coltauthor{\Name{Author Name1} \Email{abc@sample.com}\and
%  \Name{Author Name2} \Email{xyz@sample.com}\\
%  \addr Address}

% Three or more authors with the same address:
% \coltauthor{\Name{Author Name1} \Email{an1@sample.com}\\
%  \Name{Author Name2} \Email{an2@sample.com}\\
%  \Name{Author Name3} \Email{an3@sample.com}\\
%  \addr Address}

% Authors with different addresses:
\coltauthor{%
 \Name{Jessie Finocchiaro} \Email{jefi8453@colorado.edu}\\
 \addr CU Boulder
 \AND
 \Name{Rafael Frongillo} \Email{Raf@colorado.edu}\\
 \addr CU Boulder
 \AND
 \Name{Bo Waggoner} \Email{bwag@colorado.edu}\\
 \addr CU Boulder%
}

\begin{document}

\maketitle

\begin{abstract}%
  We far away from this.%
\end{abstract}

\begin{keywords}%
  Calibrated surrogates, convex surrogates, proper scoring rules%
\end{keywords}

\section{Introduction}
To do...

\section{Related work and background}

\section{Property elicitation tools}

\section{Embedding dimension}

\section{1d-characterization}
\label{sec:1d}

In the previous section, we saw that every finite elicitable property is embeddable.
We now turn to a more fine-grained question: given a finite elicitable property $\gamma$, how many dimensions $d$ are required to embed $\gamma$ in $\reals^d$?
We call the minimum such $d$ the \emph{embedding dimension} of $\gamma$.
\raft{Do we want to call this ``embedding complexity''?}
We begin with $d=1$, or real-valued embeddings.
As we will see, the notion of \emph{orderable} properties, defined below, will play a central role.
In particular, a finite elicitable property is 1-embeddable if and only if it is orderable.
We then go on to draw tight connections between orderability and other elicitation complexity notions, such as via general convex losses or continuous properties, yielding a thorough understanding of the real-valued case.

Let us first define what it means for a finite property to be orderable.
\begin{definition}
	A finite property $\gamma:\simplex\toto\R$ is \emph{orderable} if there is an enumeration $\R = \{r_1,\ldots,r_k\}$ such that for all $i\in\{1,\ldots,k-1\}$, the intersection $\gamma_{r_i} \cap \gamma_{r_{i+1}}$ is a hyperplane intersected with $\simplex$.
\end{definition}

\citet{lambert2018elicitation} shows that a finite property has a \emph{order-sensitive} loss, meaning one assigning higher loss to ``farther'' reports (with respect to a given order), if and only if it is orderable.
We will see that, in fact, a finite property is 1-embeddable if and only if it is orderable.
Both the forward direction and its converse will make use of a related definition: we say a property is \emph{monotone} if, roughly speaking, it can be defined in terms of (possibly infinitely many) hyperplanes whose normal vectors have coefficients which are monotone in the report value.

First, the forward direction.

\begin{proposition}\label{prop:orderable-embed}
	Every orderable property $\gamma$ is 1-embeddable.
\end{proposition}

The converse of Proposition~\ref{prop:orderable-embed}, that every 1-embeddable property is orderable, follows from a much broader statement, given below as Proposition~\ref{prop:indirect-orderable}: finite elicitable properties which are indirectly elicited by 1-dimensional convex losses must be orderable.
Together with Proposition~\ref{prop:embed-link}, which shows that embeddings give rise to calibrated links, we have in particular that 1-embeddable finite properties are orderable.

\begin{proposition}\label{prop:indirect-orderable}
	If convex $L : \reals \to \reals^\Y$ indirectly elicits a finite elicitable property $\gamma$, then $\gamma$ is orderable.
\end{proposition}

The proof of Proposition~\ref{prop:indirect-orderable} follows from Lemma~\ref{lem:prop-L-monotone} in Appendix~\ref{app:dimension-1}, which states that the property elicited by any convex loss must be monotone in the sense above.
(The functions $a,b$ are constructed in terms of the subgradients of the loss, thus guaranteeing monotonicity of their coefficients.)
The result then follows from the equivalence of monotonicity and orderability for finite properties (Lemma~\ref{lem:orderable-monotone}).

As a corollary of Proposition~\ref{prop:indirect-orderable} and Theorem~\ref{thm:polyhedral-embed}, we now also see that polyhedral losses embed orderable properties.

\begin{corollary}\label{cor:embed-orderable}
	Every polyhedral $L : \reals \to \reals^\Y$ embeds an orderable property.
\end{corollary}

We now see that orderability characterizes 1-embeddable properties, and have started to see that the concept is central even when considering indirect elicitation.
In particular, a conclusion of the previous two results is that if a finite elicitable property has elicitation complexity 1 with respect to convex or polyhedral losses, it must be orderable.
This in turn implies that these elicitation complexity classes coincide in dimension 1.
We now formalize this observation, and add one more complexity class: that of continuous real-valued non-locally-constant properties.

\begin{theorem}\label{thm:1d-tfae}
	Let $\gamma$ be a finite elicitable property.
	The following are equivalent.
	\begin{enumerate}\setlength{\itemsep}{0pt}
		\item $\gamma$ is orderable.
		\item $\elicembed(\gamma)=1$. ($\gamma$ is 1-embeddable.)
		\item $\elicpoly(\gamma)=1$. ($\gamma$ is indirectly elicitable via a polyhedral loss $L:\reals\to\reals^\Y$.)
		\item $\eliccvx(\gamma)=1$. ($\gamma$ is indirectly elicitable via a convex loss $L:\reals\to\reals^\Y$.)
		\item $\eliccts(\gamma)=1$. ($\gamma$ is indirectly elicitable via a real-valued continuous non-locally-constant property.)
	\end{enumerate}
\end{theorem}

The proof, given in Appendix~\ref{app:dimension-1}, only shows the equivalence 1 $\Leftrightarrow$ 5, as we have already established the equivalence of the first four statements: $1 \Rightarrow 2,3,4$ from Proposition~\ref{prop:orderable-embed}, as the proof gives a loss which is polyhedral (and convex), and Proposition~\ref{prop:indirect-orderable} shows $2,3,4 \Rightarrow 1$, as each of these cases would give a convex loss indirectly eliciting a finite elicitable property.
Note that while~\citet{finocchiaro2018convex} gives a similar statement to $1\Leftrightarrow 5$, namely that continuous non-locally-constant real-valued properties are elicitable if and only if they are convex elicitable, their proof relies on several regularity conditions which we do not require; we instead give a direct proof.

From Theorem~\ref{thm:1d-tfae}, we can already prove lower bounds on embedding dimension, as well as the other complexity classes mentioned.
Here are two simple examples.
\begin{corollary}
	Let $n\geq 3$ and $\alpha\in(0,1)$.  Then the following are bounded below by 2:
	$\elicembed(\abstain{\alpha})$,
	$\elicpoly(\abstain{\alpha})$,
	$\eliccvx(\abstain{\alpha})$,
	$\eliccts(\abstain{\alpha})$,
	$\elicembed(\mode)$,
	$\elicpoly(\mode)$,
	$\eliccvx(\mode)$,
	$\eliccts(\mode)$.
\end{corollary}
\jessiet{Can we separate the abstain and mode by a new line so they kind of line up?  Or just write it a more elegant way than exhaustive?}
\begin{proof}
	The cell adjacency graph of an orderable property must be a path, but for the mode it is the complete graph on $n$ nodes, and for $\abstain{\alpha}$ a star graph on $n+1$ nodes, which are not paths when $n \geq 3$.
\end{proof}

\section{Higher dimensions}

\subsection{Polytope fun}

\subsection{Quadratic feasibility program}


% Acknowledgments---Will not appear in anonymized version
\acks{We thank a bunch of people.}

\bibliography{refs}

\newpage
\appendix

\section{1-dimensional characterization proofs}
\begin{lemma}\label{lem:orderable-monotone}
	A finite property is orderable if and only if it is monotone.
\end{lemma}
\begin{proof}
	Let $\gamma:\simplex\toto\R$ be finite and monotone.
	Then we can use the total ordering of $\R$ to write $\R = \{r_1,\ldots,r_k\}$ such that $r_i < r_{i+1}$ for all $i \in \{1,\ldots,k-1\}$.
	We now have $\gamma_{r_i} \cap \gamma_{r_{i+1}} = \{p\in\simplex : \inprod{a(r_{i+1})}{p} \leq 0 \leq \inprod{b(r_i)}{p} \}$.
	If this intersection is empty, then there must be some $p$ with $\inprod{b(r_i)}{p} < 0$ and $\inprod{a(r_{i+1})}{p} > 0$; by monotonicity, no earlier or later reports can be in $\gamma(p)$, so we see that $\gamma(p) = \emptyset$, a contradiction.
	Thus the intersection is nonempty, and as we also know $b(r_i) \leq a(r_{i+1})$ we conclude $b(r_i) = a(r_{i+1})$, and the intersection is the hyperplane defined by $b(r_i) = a(r_{i+1})$.
	
	For the converse, let $\gamma:\simplex\toto\R$ be finite and orderable.
	From~\cite[Theorem 4]{lambert2018elicitation}, we have positively-oriented normals $v_i\in\reals^\Y$ for all $i \in \{1,\ldots,k-1\}$ such that $\gamma_{r_i} \cap \gamma_{r_{i+1}} = \{p\in\simplex : \inprod{v_i}{p} = 0\}$, and moreover, for all $i \in \{2,\ldots,k-1\}$, we have $\gamma_{r_i} = \{p\in\simplex : \inprod{v_{i-1}}{p} \leq 0 \leq \inprod{v_i}{p}\}$, while $\gamma_{r_1} = \{p\in\simplex : 0 \leq \inprod{v_1}{p} \}$ and $\gamma_{r_k} = \{p\in\simplex : \inprod{v_{k-1}}{p} \leq 0\}$.
	\jessiet{Should the sign on $\gamma_{r_k}$ be changed?}
	From the positive orientation of the $v_i$, we have for all $p\in\simplex$ that $\sgn(\inprod{v_i}{p})$ is monotone in $i$.
	In particular, it must be that for all $y$, $\sgn((v_i)_y)$ is monotone in $i$, taking the distribution with all weight on outcome $y$.
	% (Similarly, if $\gamma(\ones_y) = \{r_j,r_{j+1}\}$, then $(v_i)_y = v_i \cdot \ones_y < 0$ for $i < j$, $(v_j)_y = 0$, and $(v_i)_y > 0$ for $i > j$.)
	\raft{This observation could use a better proof.}
	
	For all $i\in\{2,\ldots,k-1\}$, we wish to find $\alpha_i \geq 0$ such that $v_{i-1} \leq \alpha_i v_i$ (component-wise).
	\jessiet{How should $w$ be interpreted?}
	To this end, fix $i$ and let $v = v_{i-1}, w = v_{i}, \alpha=\alpha_i$.
	By the above, there is no $y\in\Y$ with $v_y > 0 > w_y$, so the condition $v \leq \alpha w$ is equivalent to the following:
	
	\begin{enumerate}
		\item[(i)] For all $y$ with $v_y,w_y < 0$, $\alpha \leq v_y/w_y$.
		\item[(ii)] For all $y$ with $v_y,w_y > 0$, $\alpha \geq v_y/w_y$.
	\end{enumerate}
	Defining $\Y^- = \{y\in\Y: v_i,w_y < 0\}, \Y^+ = \{y\in\Y: v_i,w_y > 0\}$, these conditions are in turn equivalent to $\min_{y\in\Y^-} v_y/w_y \geq \max_{y\in\Y^+} v_y/w_y$, with the usual conventions $\min \emptyset = \infty,\; \max \emptyset = -\infty$.
	Suppose this inequality were not satisfied.
	Then we would have $y \in \Y^-,y'\in\Y^+$ such that $0 < v_y/w_y < w_{y'}/v_y$, which would in turn imply $|v_y|/v_{y'} < |w_y| / w_{y'}$.
	Letting $c = \tfrac 1 2 \left(|w_y| / w_{y'} + |v_y|/v_{y'}\right)$ and taking $p$ to be the distribution with weight $1/(1+c)$ on $y$ and $c/(1+c)$ on $y'$, we see that
	\begin{align*}
	\inprod{v}{p} &= \frac 1 {1+c} \left(v_y + \tfrac 1 2 (|w_y| / w_{y'} + |v_y|/v_{y'})v_{y'}\right) > \frac 1 {1+c} \left(v_y + (|v_y|/v_{y'})v_{y'}\right) = 0
	\\
	\inprod{w}{p} &= \frac 1 {1+c} \left(w_y + \tfrac 1 2 (|w_y| / w_{y'} + |v_y|/v_{y'})v_{y'}\right) < \frac 1 {1+c} \left(w_y + (|w_y|/w_{y'})w_{y'}\right) = 0~,
	\end{align*}
	thus violating the observation that $\sgn(\inprod{v_i}{p})$ is monotone in $i$ (recall that $v = v_{i-1}, w = v_{i}$).
	
	We can now construct the $a,b$ in Definition~\ref{def:monotone-prop}.
	Letting
	\[\alpha_i = \tfrac 1 2 \left(\min_{y\in\Y^-(i)} (v_{i-1})_y/(v_{i})_y + \max_{y\in\Y^+(i)} (v_{i-1})_y/(v_{i})_y\right)~,\] the preceding argument shows that $v_{i-1} \leq \alpha_i v_{i}$ for all $i \in \{2,\ldots,k-1\}$.
	We now set $b(r_i) = a(r_{i+1}) = (\prod_{j=2}^i \alpha_j) v_i$ for $i\in\{1,\ldots,k-1\}$, as well as $a(r_1) = -\max_{y\in\Y} |(v_1)_y|\ones$ and $b(r_k) = \max_{y\in\Y} |a(r_k)_y|\ones$, where $\ones\in\reals^\Y$ denotes the all-ones vector.
	
	The above establishes the second condition of monotone properties in Definition~\ref{def:monotone-prop}.
	To see the first condition, note that we have already established it for $i\in\{2,\ldots,k-1\}$.
	For $i=1,k$, we merely observe that $\inprod{a(r_1)}{p} \leq 0$ and $\inprod{b(r_k)}{p} \geq 0$ for all $p\in\simplex$.
\end{proof}

\raft{The following statement is true I believe, but low priority: ``An elicitable property $\Gamma:\simplex\toto\reals$ is convex elicitable (elicited by a convex $L : \reals \to \reals^\Y$) if and only if it is monotone.''  Start of the proof commented out.  Just need to show that $b$ is the upper limit of $a$ and $a$ the lower of $b$; should follow from elicitability of $\Gamma$.}
\begin{lemma}\label{lem:prop-L-monotone}
	For any convex $L : \reals \to \reals^\Y_+$, the property $\prop{L}$ is monotone.
\end{lemma}
\begin{proof}
	If $L$ is convex and elicits $\Gamma$, let $a,b$ be defined by $a(r)_y = \partial_- L(r)_y$ and $b(r) = \partial_+ L(r)_y$, that is, the left and right derivatives of $L(\cdot)_y$ at $r$, respectively.
	Then $\partial L(r)_y = [a(r)_y,b(r)_y]$.
	We now have $r \in \prop{L}(p) \iff 0 \in \partial \inprod{p}{L(r)} \iff \inprod{a(r)}{p} \leq 0 \leq \inprod{b(r)}{p}$, showing the first condition.
	The second condition follows as the subgradients of $L$ are monotone functions (see e.g.~\citet[Theorem 24.1]{rockafellar1997convex}).
	% Conversely, given such an $a,b$, we appeal to~\citet[Theorem 24.2]{rockafellar1997convex}, which gives us that $L(u)_y := \int_0^u a(u)_y$ is convex, and
\end{proof}

\newcommand{\Pbar}{\overline P}
\begin{lemma}\label{lem:pbar}
	Let $\gamma:\simplex\toto\R$ be a finite elicitable property, and suppose there is a calibrated link $\psi$ from an elicitable $\Gamma$ to $\gamma$.
	For each $r\in\R$, define $P_r = \bigcup_{u\in\psi^{-1}(r)} \Gamma_u \subseteq \simplex$, and let $\Pbar_r$ denote the closure of the convex hull of $P_r$.
	Then $\gamma_r = \Pbar_r$ for all $r\in\R$.
\end{lemma}
\begin{proof}
	As $P_r \subseteq \gamma_r$ by the definition of calibration, and $\gamma_r$ is closed and convex, we must have $\Pbar_r \subseteq \gamma_r$.
	Furthermore, again by calibration of $\psi$, we must have $\bigcup_{r\in\R} P_r = \bigcup_{u\in\reals} \Gamma_u = \simplex$, and thus $\bigcup_{r\in\R} \Pbar_r = \simplex$ as well.
	Suppose for a contradiction that $\gamma_r \neq \Pbar_r$ for some $r\in\R$.
	From Lemma~\ref{lem:finite-full-dim}, $\gamma_r$ has nonempty interior, so we must have some $p\in\inter\gamma_r \setminus \Pbar_r$.
	But as $\bigcup_{r'\in\R} \Pbar_{r'} = \simplex$, we then have some $r'\neq r$ with $p\in\Pbar_{r'} \subseteq \gamma_{r'}$.
	By Theorem~\ref{thm:aurenhammer}, the level sets of $\gamma$ form a power diagram, and in particular a cell complex, so we have contradicted point (ii) of Definition~\ref{def:cell-complex}: the relative interiors of the faces must not be disjoint.
	Hence, for all $r\in\R$ we have $\gamma_r = \Pbar_r$.
\end{proof}

We now prove the remaining results.
%Proposition~\ref{prop:indirect-orderable}, which states the following: if convex $L : \reals \to \reals^\Y$ indirectly elicits a finite elicitable property $\gamma$, then $\gamma$ is orderable.
\begin{proof}[of Proposition~\ref{prop:indirect-orderable}]
	Let $\gamma:\simplex\toto\R$.
	From Lemma~\ref{lem:prop-L-monotone}, $\Gamma := \prop{L}$ is monotone.
	Let $\psi:\reals\to\R$ be the calibrated link from $\Gamma$ to $\gamma$.
	From Lemma~\ref{lem:pbar}, we have $\Pbar_r = \gamma_r$ for all $r\in\R$, where $\Pbar_r$ is the closure of the convex hull of $\bigcup_{u\in\psi^{-1}(r)} \Gamma_u$.
	
	As $\Gamma$ is monotone, we must have $a,b : \R\to\reals^\Y$ such that $\Pbar_r = \{p\in\simplex : \inprod{a(r)}{p} \leq 0 \leq \inprod{b(r)}{p} \}$.
	(Take $a(r)_y = \inf_{u\in\psi^{-1}(r)} a(u)_y$ and $b(r)_y = \sup_{u\in\psi^{-1}(r)} b(u)_y$.)
	Now taking $p_r\in\inter\gamma_r$ and picking $u_r \in \Gamma(p_r)$, we order $\R = \{r_1,\ldots,r_k\}$ so that $u_{r_i} < u_{r_{i+1}}$ for all $i\in\{1,\ldots,k-1\}$.
	(The $u_{r_i}$ must all be distinct, as we chose $p_r$ so that $\gamma(p_r) = \{r\}$, so $\psi(u_{r_i}) = r_i$ for all $i$.)
	
	Let $i\in\{1,\ldots,k-1\}$.
	By monotonicity of $\Gamma$, we must have $a(r_i) \leq b(r_i) \leq a(r_{i+1}) \leq b(r_{i+1})$.
	As $\bigcup_{r\in\R} \Pbar_r = \bigcup_{r\in\R} \gamma_r = \simplex$, we must therefore have $b(r_i) = a(r_{i+1})$.
	Finally, we conclude $\gamma_{r_i} \cap \gamma_{r_{i+1}} = \{p\in\simplex : \inprod{b(r_i)}{p} = 0\}$.
	As these statements hold for all $i\in\{1,\ldots,k-1\}$, $\gamma$ is orderable.
\end{proof}

\newcommand{\floor}[1]{\lfloor #1\rfloor}
\begin{proof}[of Theorem~\ref{thm:1d-tfae}]
	As remarked before the theorem statement, it remains only to show $1 \iff 5$.
	\raft{This got a tad hand-wavy toward the end, but I think it's not too bad.}
	For the forward direction, by Lemma~\ref{lem:orderable-monotone}, we have $v(0),\ldots,v(k)\in\reals^\Y$ such that the coefficients $v(i)_y$ are monotone in $i$ for all $y\in\Y$, and the following two conditions hold: (1) \jessiet{Change to (i) and (ii) for consistency?} for all $i\in\{1,\ldots,k\}$ we have $\gamma_{r_i} = \{p\in\simplex : \inprod{v(i-1)}{p} \leq 0 \leq \inprod{v(i)}{p} \}$, and (2) for all $i\in\{1,\ldots,k-1\}$ we have $\gamma_{r_i} \cap \gamma_{r_{i+1}} = \{p\in\simplex : \inprod{v(i)}{p} = 0\}$.
	Letting $\floor{u}$ denote the floor function, and $\mathrm{mod}(u) := u - \floor{u}$, we extend the above definition to $v:[0,k]\to\reals^\Y$ given by
	\begin{align*}
	v(u) = (1-\mathrm{mod}(u))v(\floor{u}) + \mathrm{mod}(u)v(\floor{u}+1)~,
	\end{align*}
	which is continuous by construction.
	Moreover, $v(\cdot)_y$ is monotone for all $y\in\Y$.
	As the level sets $\gamma_r$ are full-dimensional (Lemma~\ref{lem:finite-full-dim}), for all $p\in\simplex$, there is a unique $u\in[0,k]$ such that $\inprod{v(u)}{p} = 0$.
	\jessiet{Why is the above statement true on boundaries of level sets?}
	We conclude that the property $\Gamma:\simplex\to[0,k]$ given by $\Gamma_u = \{p\in\simplex : \inprod{v(u)}{p} = 0\}$ is a single-valued property, which continuous as $v$ is continuous.
	Finally, the link $\psi(u) = \floor{u}+1$ is calibrated from $\Gamma$ to $\gamma$.
	
	For the converse, let $\Gamma:\simplex\to\reals$ be a continuous, non-locally-constant, elicitable property, with a calibrated link $\psi:\reals\to\R$ to $\gamma$.
	By~\cite{lambert2018elicitation,steinwart2014elicitation}, we have some continuous $v:\Gamma(\simplex)\to\reals^\Y$ such that for all $u\in\Gamma(\simplex)$ we have $\Gamma_u = \{p\in\simplex : \inprod{v(u)}{p} = 0\}$.
	
	Let $c_r = \inf\psi^{-1}(r)$ and $d_r = \sup \psi^{-1}(r)$ for all $r\in\R$, and let $I_r = (c_r,d_r)$.
	By continuity, $\Gamma^{-1}(I_r)$ is an open set, and the definition of calibration, we have $\Gamma^{-1}(I_r) \subseteq \gamma_r$; we conclude $\Gamma^{-1}(I_r) \subseteq \inter\gamma_r$. \jessiet{Why $\subseteq$ and not $=$?}
	Again by continuity of $\Gamma$, we must have $\Gamma(\gamma_r) = [c_r,d_r] := \overline I_r$.
	
	As the interiors of the level sets of $\gamma$ are disjoint (Theorem~\ref{thm:aurenhammer}), the intervals $\{I_r:r\in\R\}$ must also be disjoint.
	Define an ordering $\R = \{r_1,\ldots,r_k\}$ so that $I_{r_1} < \cdots < I_{r_k}$.
	Then for all $i\in\{1,\ldots,k-1\}$ we must have $d_{r_i} = c_{r_{i+1}}$; otherwise, the point $u = \tfrac 1 2 (d_{r_i} + c_{r_{i+1}})$ would not belong to any $\overline I_r$, and as $\Gamma(\simplex) = \Gamma(\bigcup_r\gamma_r) = \bigcup_r \overline I_r$, this would violate continuity of $\Gamma$.
	We conclude that $\gamma_{r_i}\cap\gamma_{r_{i+1}} = \Gamma_{d_i} = \{p\in\simplex : \inprod{v(d_i)}{p} = 0\}$ for all $i\in\{1,\ldots,k-1\}$.
\end{proof}

\end{document}
