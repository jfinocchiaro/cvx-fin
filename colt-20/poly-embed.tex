\documentclass[anon]{colt2020} % Anonymized submission
% \documentclass{colt2020} % Include author names

% The following packages will be automatically loaded:
% amsmath, amssymb, natbib, graphicx, url, algorithm2e

\title[Embedding dimension]{Embedding Dimension of Polyhedral Losses}
\usepackage{times}

\usepackage{lmodern}
\usepackage{hyperref}       % hyperlinks  %[implicit=false, bookmarks=false]
\usepackage{url}            % simple URL typesetting
\usepackage{booktabs}       % professional-quality tables
\usepackage{amsfonts}       % blackboard math symbols
\usepackage{nicefrac}       % compact symbols for 1/2, etc.
\usepackage{microtype}      % microtypography
%\usepackage[margin=1.0in]{geometry}

\usepackage{mathtools, amsmath, amssymb, graphicx, verbatim}
%\usepackage{amsthm}
%\usepackage[thmmarks, thref, amsthm]{ntheorem}
\usepackage{color}
\definecolor{darkblue}{rgb}{0.0,0.0,0.2}
\hypersetup{colorlinks,breaklinks,
	linkcolor=darkblue,urlcolor=darkblue,
	anchorcolor=darkblue,citecolor=darkblue}
\usepackage{wrapfig}
\usepackage{subcaption}
\usepackage[colorinlistoftodos,textsize=tiny]{todonotes} % need xargs for below
%\usepackage{accents}
\usepackage{bbm}
\usepackage{xspace}

%\usetikzlibrary{calc}
\newcommand{\Comments}{1}
\newcommand{\mynote}[2]{\ifnum\Comments=1\textcolor{#1}{#2}\fi}
\newcommand{\mytodo}[2]{\ifnum\Comments=1%
	\todo[linecolor=#1!80!black,backgroundcolor=#1,bordercolor=#1!80!black]{#2}\fi}
\newcommand{\raf}[1]{\mynote{green}{[RF: #1]}}
\newcommand{\raft}[1]{\mytodo{green!20!white}{RF: #1}}
\newcommand{\jessie}[1]{\mynote{purple}{[JF: #1]}}
\newcommand{\jessiet}[1]{\mytodo{purple!20!white}{JF: #1}}
\newcommand{\bo}[1]{\mynote{blue}{[Bo: #1]}}
\newcommand{\botodo}[1]{\mytodo{blue!20!white}{[Bo: #1]}}
\newcommand{\btw}[1]{\mytodo{orange!20!white}{BTW: #1}}
\ifnum\Comments=1               % fix margins for todonotes
\setlength{\marginparwidth}{1in}
\fi


\newcommand{\reals}{\mathbb{R}}
\newcommand{\posreals}{\reals_{>0}}%{\reals_{++}}
\newcommand{\dom}{\mathrm{dom}}
\newcommand{\epi}{\text{epi}}
\newcommand{\relint}{\mathrm{relint}}
\newcommand{\prop}[1]{\Gamma[#1]}
\newcommand{\eliccts}{\mathrm{elic}_\mathrm{cts}}
\newcommand{\eliccvx}{\mathrm{elic}_\mathrm{cvx}}
\newcommand{\elicpoly}{\mathrm{elic}_\mathrm{pcvx}}
\newcommand{\elicembed}{\mathrm{elic}_\mathrm{embed}}

\newcommand{\cell}{\mathrm{cell}}

\newcommand{\abstain}[1]{\mathrm{abstain}_{#1}}
\newcommand{\mode}{\mathrm{mode}}

\newcommand{\simplex}{\Delta_\Y}

% alphabetical order, by convention
\newcommand{\C}{\mathcal{C}}
\newcommand{\D}{\mathcal{D}}
\newcommand{\E}{\mathbb{E}}
\newcommand{\F}{\mathcal{F}}
\renewcommand{\H}{\mathcal{H}}
\newcommand{\N}{\mathcal{N}}
\newcommand{\I}{\mathcal{I}}
\newcommand{\R}{\mathcal{R}}
\newcommand{\T}{\mathcal{T}}
\newcommand{\U}{\mathcal{U}}
\newcommand{\V}{\mathcal{V}}
\newcommand{\X}{\mathcal{X}}
\newcommand{\Y}{\mathcal{Y}}
\renewcommand{\P}{\mathcal{P}}

\newcommand{\hinge}{L_{\mathrm{hinge}}}
\newcommand{\ellzo}{\ell_{\text{0-1}}}
\newcommand{\ellabs}[1]{\ell_{#1}}

\newcommand{\risk}[1]{\underline{#1}}
\newcommand{\inprod}[2]{\langle #1, #2 \rangle}%\mathrm{int}(#1)}
\newcommand{\inter}[1]{\mathrm{int}(#1)}%\mathrm{int}(#1)}
%\newcommand{\expectedv}[3]{\overline{#1}(#2,#3)}
\newcommand{\expectedv}[3]{\E_{Y\sim{#3}} {#1}(#2,Y)}
\newcommand{\toto}{\rightrightarrows}
\newcommand{\strip}{\mathrm{strip}}
\newcommand{\trim}{\mathrm{trim}}
\newcommand{\fplc}{finite-piecewise-linear and convex\xspace} %xspace for use in text
\newcommand{\conv}{\mathrm{conv}}
\newcommand{\indopp}{\bar{\mathbbm{1}}}
\newcommand{\ones}{\mathbbm{1}}
\DeclarePairedDelimiter\ceil{\lceil}{\rceil}

\newcommand{\Ind}[1]{\mathbf{1}\{#1\}}

\DeclareMathOperator*{\argmax}{arg\,max}
\DeclareMathOperator*{\argmin}{arg\,min}
\DeclareMathOperator*{\arginf}{arg\,inf}
\DeclareMathOperator*{\sgn}{sgn}

%\newtheorem{theorem}{Theorem}
%\newtheorem{lemma}{Lemma}
%\newtheorem{proposition}{Proposition}
%\newtheorem{definition}{Definition}
%\newtheorem{corollary}{Corollary}
%\newtheorem{conjecture}{Conjecture}
\newtheorem{condition}{Condition}
\newtheorem{claim}{Claim}
% Use \Name{Author Name} to specify the name.
% If the surname contains spaces, enclose the surname
% in braces, e.g. \Name{John {Smith Jones}} similarly
% if the name has a "von" part, e.g \Name{Jane {de Winter}}.
% If the first letter in the forenames is a diacritic
% enclose the diacritic in braces, e.g. \Name{{\'E}louise Smith}

% Two authors with the same address
% \coltauthor{\Name{Author Name1} \Email{abc@sample.com}\and
%  \Name{Author Name2} \Email{xyz@sample.com}\\
%  \addr Address}

% Three or more authors with the same address:
 \coltauthor{\Name{Jessie Finocchiaro} \Email{jefi8453@colorado.edu}\\
  \Name{Rafael Frongillo} \Email{raf@colorado.edu}\\
  \Name{Bo Waggoner} \Email{bwag@colorado.edu}\\
  \addr CU Boulder}

% Authors with different addresses:
%\coltauthor{%
% \Name{Jessie Finocchiaro} \Email{jefi8453@colorado.edu}\\
% \addr CU Boulder
% \AND
% \Name{Rafael Frongillo} \Email{Raf@colorado.edu}\\
% \addr CU Boulder
% \AND
% \Name{Bo Waggoner} \Email{bwag@colorado.edu}\\
% \addr CU Boulder%
%}

\begin{document}

\maketitle

\begin{abstract}%
  Recent work has proposed the notion of designing calibrated surrogate losses for classification-like problems through the lens of \emph{embedding} the original problem into $\reals^d$ and optimizing a polyhedral loss that is calibrated with respect to the original loss.
  In this work, we study the notion of \emph{embedding dimension} for given discrete losses.
  We characterize when a given discrete loss can be embedded into the real line, as well as when a higher-dimension input to the surrogate loss is required.
  Moreover, we give a quadratic feasibility program that yields lower bounds on the embedding dimension of a given discrete loss.
\end{abstract}

\begin{keywords}%
  Calibrated surrogates, convex surrogates, proper scoring rules%
\end{keywords}

\section{Introduction}
To do...

\section{Related work and background}

\subsection{Discrete losses and embeddings}

\jessie{A lot fo this is taken from the NeurIPS paper}
Let $\Y$ be a finite outcome (label) space, and throughout let $n=|\Y|$.
The set of probability distributions on $\Y$ is denoted $\simplex\subseteq\reals_+^{\Y}$, represented as vectors of probabilities.
We write $p_y$ for the probability of outcome $y \in \Y$ drawn from $p \in \simplex$.

As is assumed by~\cite{finocchiaro2019embedding}, we assume the given discrete loss is \emph{non-redundant}, meaning every report is uniquely optimal (minimizes expected loss) for some distribution $p\in\simplex$.

We denote surrogate losses $L:\reals^d\to\reals^\Y_+$, typically with reports written $u\in\reals^d$.
We write the corresponding expected loss when $Y \sim p$ as $\inprod{p}{\ell(r)}$ and $\inprod{p}{L(u)}$ since we are in the finite outcome setting.

For example, 0-1 loss is a discrete loss with $\R = \Y = \{-1,1\}$
given by $\ellzo(r)_y = \Ind{r \neq y}$, with Bayes risk $\risk{\ellzo}(p) = 1-\max_{y\in\Y} p_y$.
Two important surrogates for $\ellzo$ are hinge loss $\hinge(u)_y = (1-yu)_+$, where $(x)_+ = \max(x,0)$, and logistic loss $L(u)_y = \log(1+\exp(-yu))$ for $u\in\reals$.

Most of the surrogates $L$ we consider will be \emph{polyhedral}, meaning piecewise linear and convex; we therefore briefly recall the relevant definitions.
In $\reals^d$, a \emph{polyhedral set} or \emph{polyhedron} is the intersection of a finite number of closed halfspaces.
A \emph{polytope} is a bounded polyhedral set.
A convex function $f:\reals^d\to\reals$ is \emph{polyhedral} if its epigraph is polyhedral, or equivalently, if it can be written as a pointwise maximum of a finite set of affine functions~\citep{rockafellar1997convex}.
%
\begin{definition}[Polyhedral loss]
	A loss $L: \reals^d \to \reals^{\Y}_+$ is \emph{polyhedral} if $L(u)_y$ is a polyhedral (convex) function of $u$ for each $y\in\Y$.
\end{definition}
%
For example, hinge loss is polyhedral, whereas logistic loss is not.

We specifically study polyhedral functions in part because they have high-dimensional subgradient sets at points of nondifferentiability, which will be of interest to us later.

\section{Embedding dimension}

\section{1d-characterization}
\label{sec:1d}


\section{Higher dimensions}

\subsection{Polytope fun}

\subsection{Quadratic feasibility program}


% Acknowledgments---Will not appear in anonymized version
\acks{JF- Need that GRFP acknowledgement}

\bibliography{diss,extra}

\newpage
\appendix

\section{Additional proofs}

\end{document}
